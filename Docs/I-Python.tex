
% Default to the notebook output style

    


% Inherit from the specified cell style.




    
\documentclass{article}

    
    
    \usepackage{graphicx} % Used to insert images
    \usepackage{adjustbox} % Used to constrain images to a maximum size 
    \usepackage{color} % Allow colors to be defined
    \usepackage{enumerate} % Needed for markdown enumerations to work
    \usepackage{geometry} % Used to adjust the document margins
    \usepackage{amsmath} % Equations
    \usepackage{amssymb} % Equations
    \usepackage[mathletters]{ucs} % Extended unicode (utf-8) support
    \usepackage[utf8x]{inputenc} % Allow utf-8 characters in the tex document
    \usepackage{fancyvrb} % verbatim replacement that allows latex
    \usepackage{grffile} % extends the file name processing of package graphics 
                         % to support a larger range 
    % The hyperref package gives us a pdf with properly built
    % internal navigation ('pdf bookmarks' for the table of contents,
    % internal cross-reference links, web links for URLs, etc.)
    \usepackage{hyperref}
    \usepackage{longtable} % longtable support required by pandoc >1.10
    \usepackage{booktabs}  % table support for pandoc > 1.12.2
    

    
    
    \definecolor{orange}{cmyk}{0,0.4,0.8,0.2}
    \definecolor{darkorange}{rgb}{.71,0.21,0.01}
    \definecolor{darkgreen}{rgb}{.12,.54,.11}
    \definecolor{myteal}{rgb}{.26, .44, .56}
    \definecolor{gray}{gray}{0.45}
    \definecolor{lightgray}{gray}{.95}
    \definecolor{mediumgray}{gray}{.8}
    \definecolor{inputbackground}{rgb}{.95, .95, .85}
    \definecolor{outputbackground}{rgb}{.95, .95, .95}
    \definecolor{traceback}{rgb}{1, .95, .95}
    % ansi colors
    \definecolor{red}{rgb}{.6,0,0}
    \definecolor{green}{rgb}{0,.65,0}
    \definecolor{brown}{rgb}{0.6,0.6,0}
    \definecolor{blue}{rgb}{0,.145,.698}
    \definecolor{purple}{rgb}{.698,.145,.698}
    \definecolor{cyan}{rgb}{0,.698,.698}
    \definecolor{lightgray}{gray}{0.5}
    
    % bright ansi colors
    \definecolor{darkgray}{gray}{0.25}
    \definecolor{lightred}{rgb}{1.0,0.39,0.28}
    \definecolor{lightgreen}{rgb}{0.48,0.99,0.0}
    \definecolor{lightblue}{rgb}{0.53,0.81,0.92}
    \definecolor{lightpurple}{rgb}{0.87,0.63,0.87}
    \definecolor{lightcyan}{rgb}{0.5,1.0,0.83}
    
    % commands and environments needed by pandoc snippets
    % extracted from the output of `pandoc -s`
    \DefineVerbatimEnvironment{Highlighting}{Verbatim}{commandchars=\\\{\}}
    % Add ',fontsize=\small' for more characters per line
    \newenvironment{Shaded}{}{}
    \newcommand{\KeywordTok}[1]{\textcolor[rgb]{0.00,0.44,0.13}{\textbf{{#1}}}}
    \newcommand{\DataTypeTok}[1]{\textcolor[rgb]{0.56,0.13,0.00}{{#1}}}
    \newcommand{\DecValTok}[1]{\textcolor[rgb]{0.25,0.63,0.44}{{#1}}}
    \newcommand{\BaseNTok}[1]{\textcolor[rgb]{0.25,0.63,0.44}{{#1}}}
    \newcommand{\FloatTok}[1]{\textcolor[rgb]{0.25,0.63,0.44}{{#1}}}
    \newcommand{\CharTok}[1]{\textcolor[rgb]{0.25,0.44,0.63}{{#1}}}
    \newcommand{\StringTok}[1]{\textcolor[rgb]{0.25,0.44,0.63}{{#1}}}
    \newcommand{\CommentTok}[1]{\textcolor[rgb]{0.38,0.63,0.69}{\textit{{#1}}}}
    \newcommand{\OtherTok}[1]{\textcolor[rgb]{0.00,0.44,0.13}{{#1}}}
    \newcommand{\AlertTok}[1]{\textcolor[rgb]{1.00,0.00,0.00}{\textbf{{#1}}}}
    \newcommand{\FunctionTok}[1]{\textcolor[rgb]{0.02,0.16,0.49}{{#1}}}
    \newcommand{\RegionMarkerTok}[1]{{#1}}
    \newcommand{\ErrorTok}[1]{\textcolor[rgb]{1.00,0.00,0.00}{\textbf{{#1}}}}
    \newcommand{\NormalTok}[1]{{#1}}
    
    % Define a nice break command that doesn't care if a line doesn't already
    % exist.
    \def\br{\hspace*{\fill} \\* }
    % Math Jax compatability definitions
    \def\gt{>}
    \def\lt{<}
    % Document parameters
    \title{I-Python}
    
    
    

    % Pygments definitions
    
\makeatletter
\def\PY@reset{\let\PY@it=\relax \let\PY@bf=\relax%
    \let\PY@ul=\relax \let\PY@tc=\relax%
    \let\PY@bc=\relax \let\PY@ff=\relax}
\def\PY@tok#1{\csname PY@tok@#1\endcsname}
\def\PY@toks#1+{\ifx\relax#1\empty\else%
    \PY@tok{#1}\expandafter\PY@toks\fi}
\def\PY@do#1{\PY@bc{\PY@tc{\PY@ul{%
    \PY@it{\PY@bf{\PY@ff{#1}}}}}}}
\def\PY#1#2{\PY@reset\PY@toks#1+\relax+\PY@do{#2}}

\expandafter\def\csname PY@tok@gd\endcsname{\def\PY@tc##1{\textcolor[rgb]{0.63,0.00,0.00}{##1}}}
\expandafter\def\csname PY@tok@gu\endcsname{\let\PY@bf=\textbf\def\PY@tc##1{\textcolor[rgb]{0.50,0.00,0.50}{##1}}}
\expandafter\def\csname PY@tok@gt\endcsname{\def\PY@tc##1{\textcolor[rgb]{0.00,0.27,0.87}{##1}}}
\expandafter\def\csname PY@tok@gs\endcsname{\let\PY@bf=\textbf}
\expandafter\def\csname PY@tok@gr\endcsname{\def\PY@tc##1{\textcolor[rgb]{1.00,0.00,0.00}{##1}}}
\expandafter\def\csname PY@tok@cm\endcsname{\let\PY@it=\textit\def\PY@tc##1{\textcolor[rgb]{0.25,0.50,0.50}{##1}}}
\expandafter\def\csname PY@tok@vg\endcsname{\def\PY@tc##1{\textcolor[rgb]{0.10,0.09,0.49}{##1}}}
\expandafter\def\csname PY@tok@m\endcsname{\def\PY@tc##1{\textcolor[rgb]{0.40,0.40,0.40}{##1}}}
\expandafter\def\csname PY@tok@mh\endcsname{\def\PY@tc##1{\textcolor[rgb]{0.40,0.40,0.40}{##1}}}
\expandafter\def\csname PY@tok@go\endcsname{\def\PY@tc##1{\textcolor[rgb]{0.53,0.53,0.53}{##1}}}
\expandafter\def\csname PY@tok@ge\endcsname{\let\PY@it=\textit}
\expandafter\def\csname PY@tok@vc\endcsname{\def\PY@tc##1{\textcolor[rgb]{0.10,0.09,0.49}{##1}}}
\expandafter\def\csname PY@tok@il\endcsname{\def\PY@tc##1{\textcolor[rgb]{0.40,0.40,0.40}{##1}}}
\expandafter\def\csname PY@tok@cs\endcsname{\let\PY@it=\textit\def\PY@tc##1{\textcolor[rgb]{0.25,0.50,0.50}{##1}}}
\expandafter\def\csname PY@tok@cp\endcsname{\def\PY@tc##1{\textcolor[rgb]{0.74,0.48,0.00}{##1}}}
\expandafter\def\csname PY@tok@gi\endcsname{\def\PY@tc##1{\textcolor[rgb]{0.00,0.63,0.00}{##1}}}
\expandafter\def\csname PY@tok@gh\endcsname{\let\PY@bf=\textbf\def\PY@tc##1{\textcolor[rgb]{0.00,0.00,0.50}{##1}}}
\expandafter\def\csname PY@tok@ni\endcsname{\let\PY@bf=\textbf\def\PY@tc##1{\textcolor[rgb]{0.60,0.60,0.60}{##1}}}
\expandafter\def\csname PY@tok@nl\endcsname{\def\PY@tc##1{\textcolor[rgb]{0.63,0.63,0.00}{##1}}}
\expandafter\def\csname PY@tok@nn\endcsname{\let\PY@bf=\textbf\def\PY@tc##1{\textcolor[rgb]{0.00,0.00,1.00}{##1}}}
\expandafter\def\csname PY@tok@no\endcsname{\def\PY@tc##1{\textcolor[rgb]{0.53,0.00,0.00}{##1}}}
\expandafter\def\csname PY@tok@na\endcsname{\def\PY@tc##1{\textcolor[rgb]{0.49,0.56,0.16}{##1}}}
\expandafter\def\csname PY@tok@nb\endcsname{\def\PY@tc##1{\textcolor[rgb]{0.00,0.50,0.00}{##1}}}
\expandafter\def\csname PY@tok@nc\endcsname{\let\PY@bf=\textbf\def\PY@tc##1{\textcolor[rgb]{0.00,0.00,1.00}{##1}}}
\expandafter\def\csname PY@tok@nd\endcsname{\def\PY@tc##1{\textcolor[rgb]{0.67,0.13,1.00}{##1}}}
\expandafter\def\csname PY@tok@ne\endcsname{\let\PY@bf=\textbf\def\PY@tc##1{\textcolor[rgb]{0.82,0.25,0.23}{##1}}}
\expandafter\def\csname PY@tok@nf\endcsname{\def\PY@tc##1{\textcolor[rgb]{0.00,0.00,1.00}{##1}}}
\expandafter\def\csname PY@tok@si\endcsname{\let\PY@bf=\textbf\def\PY@tc##1{\textcolor[rgb]{0.73,0.40,0.53}{##1}}}
\expandafter\def\csname PY@tok@s2\endcsname{\def\PY@tc##1{\textcolor[rgb]{0.73,0.13,0.13}{##1}}}
\expandafter\def\csname PY@tok@vi\endcsname{\def\PY@tc##1{\textcolor[rgb]{0.10,0.09,0.49}{##1}}}
\expandafter\def\csname PY@tok@nt\endcsname{\let\PY@bf=\textbf\def\PY@tc##1{\textcolor[rgb]{0.00,0.50,0.00}{##1}}}
\expandafter\def\csname PY@tok@nv\endcsname{\def\PY@tc##1{\textcolor[rgb]{0.10,0.09,0.49}{##1}}}
\expandafter\def\csname PY@tok@s1\endcsname{\def\PY@tc##1{\textcolor[rgb]{0.73,0.13,0.13}{##1}}}
\expandafter\def\csname PY@tok@kd\endcsname{\let\PY@bf=\textbf\def\PY@tc##1{\textcolor[rgb]{0.00,0.50,0.00}{##1}}}
\expandafter\def\csname PY@tok@sh\endcsname{\def\PY@tc##1{\textcolor[rgb]{0.73,0.13,0.13}{##1}}}
\expandafter\def\csname PY@tok@sc\endcsname{\def\PY@tc##1{\textcolor[rgb]{0.73,0.13,0.13}{##1}}}
\expandafter\def\csname PY@tok@sx\endcsname{\def\PY@tc##1{\textcolor[rgb]{0.00,0.50,0.00}{##1}}}
\expandafter\def\csname PY@tok@bp\endcsname{\def\PY@tc##1{\textcolor[rgb]{0.00,0.50,0.00}{##1}}}
\expandafter\def\csname PY@tok@c1\endcsname{\let\PY@it=\textit\def\PY@tc##1{\textcolor[rgb]{0.25,0.50,0.50}{##1}}}
\expandafter\def\csname PY@tok@kc\endcsname{\let\PY@bf=\textbf\def\PY@tc##1{\textcolor[rgb]{0.00,0.50,0.00}{##1}}}
\expandafter\def\csname PY@tok@c\endcsname{\let\PY@it=\textit\def\PY@tc##1{\textcolor[rgb]{0.25,0.50,0.50}{##1}}}
\expandafter\def\csname PY@tok@mf\endcsname{\def\PY@tc##1{\textcolor[rgb]{0.40,0.40,0.40}{##1}}}
\expandafter\def\csname PY@tok@err\endcsname{\def\PY@bc##1{\setlength{\fboxsep}{0pt}\fcolorbox[rgb]{1.00,0.00,0.00}{1,1,1}{\strut ##1}}}
\expandafter\def\csname PY@tok@mb\endcsname{\def\PY@tc##1{\textcolor[rgb]{0.40,0.40,0.40}{##1}}}
\expandafter\def\csname PY@tok@ss\endcsname{\def\PY@tc##1{\textcolor[rgb]{0.10,0.09,0.49}{##1}}}
\expandafter\def\csname PY@tok@sr\endcsname{\def\PY@tc##1{\textcolor[rgb]{0.73,0.40,0.53}{##1}}}
\expandafter\def\csname PY@tok@mo\endcsname{\def\PY@tc##1{\textcolor[rgb]{0.40,0.40,0.40}{##1}}}
\expandafter\def\csname PY@tok@kn\endcsname{\let\PY@bf=\textbf\def\PY@tc##1{\textcolor[rgb]{0.00,0.50,0.00}{##1}}}
\expandafter\def\csname PY@tok@mi\endcsname{\def\PY@tc##1{\textcolor[rgb]{0.40,0.40,0.40}{##1}}}
\expandafter\def\csname PY@tok@gp\endcsname{\let\PY@bf=\textbf\def\PY@tc##1{\textcolor[rgb]{0.00,0.00,0.50}{##1}}}
\expandafter\def\csname PY@tok@o\endcsname{\def\PY@tc##1{\textcolor[rgb]{0.40,0.40,0.40}{##1}}}
\expandafter\def\csname PY@tok@kr\endcsname{\let\PY@bf=\textbf\def\PY@tc##1{\textcolor[rgb]{0.00,0.50,0.00}{##1}}}
\expandafter\def\csname PY@tok@s\endcsname{\def\PY@tc##1{\textcolor[rgb]{0.73,0.13,0.13}{##1}}}
\expandafter\def\csname PY@tok@kp\endcsname{\def\PY@tc##1{\textcolor[rgb]{0.00,0.50,0.00}{##1}}}
\expandafter\def\csname PY@tok@w\endcsname{\def\PY@tc##1{\textcolor[rgb]{0.73,0.73,0.73}{##1}}}
\expandafter\def\csname PY@tok@kt\endcsname{\def\PY@tc##1{\textcolor[rgb]{0.69,0.00,0.25}{##1}}}
\expandafter\def\csname PY@tok@ow\endcsname{\let\PY@bf=\textbf\def\PY@tc##1{\textcolor[rgb]{0.67,0.13,1.00}{##1}}}
\expandafter\def\csname PY@tok@sb\endcsname{\def\PY@tc##1{\textcolor[rgb]{0.73,0.13,0.13}{##1}}}
\expandafter\def\csname PY@tok@k\endcsname{\let\PY@bf=\textbf\def\PY@tc##1{\textcolor[rgb]{0.00,0.50,0.00}{##1}}}
\expandafter\def\csname PY@tok@se\endcsname{\let\PY@bf=\textbf\def\PY@tc##1{\textcolor[rgb]{0.73,0.40,0.13}{##1}}}
\expandafter\def\csname PY@tok@sd\endcsname{\let\PY@it=\textit\def\PY@tc##1{\textcolor[rgb]{0.73,0.13,0.13}{##1}}}

\def\PYZbs{\char`\\}
\def\PYZus{\char`\_}
\def\PYZob{\char`\{}
\def\PYZcb{\char`\}}
\def\PYZca{\char`\^}
\def\PYZam{\char`\&}
\def\PYZlt{\char`\<}
\def\PYZgt{\char`\>}
\def\PYZsh{\char`\#}
\def\PYZpc{\char`\%}
\def\PYZdl{\char`\$}
\def\PYZhy{\char`\-}
\def\PYZsq{\char`\'}
\def\PYZdq{\char`\"}
\def\PYZti{\char`\~}
% for compatibility with earlier versions
\def\PYZat{@}
\def\PYZlb{[}
\def\PYZrb{]}
\makeatother


    % Exact colors from NB
    \definecolor{incolor}{rgb}{0.0, 0.0, 0.5}
    \definecolor{outcolor}{rgb}{0.545, 0.0, 0.0}



    
    % Prevent overflowing lines due to hard-to-break entities
    \sloppy 
    % Setup hyperref package
    \hypersetup{
      breaklinks=true,  % so long urls are correctly broken across lines
      colorlinks=true,
      urlcolor=blue,
      linkcolor=darkorange,
      citecolor=darkgreen,
      }
    % Slightly bigger margins than the latex defaults
    
    \geometry{verbose,tmargin=1in,bmargin=1in,lmargin=1in,rmargin=1in}
    
    

    \begin{document}
    
    
    \maketitle
    
    

    
    

    \section{I-\href{https://www.python.org/}{Python}}\label{i-python}


    \subsection{Index}


    \begin{itemize}
\itemsep1pt\parskip0pt\parsep0pt
\item
  \hyperref[characterux5fencoding]{Character Encoding}
\item
  \hyperref[modules]{Modules}
\item
  \hyperref[variablesux5fandux5ftypes]{Variables and Types}
\item
  \hyperref[fundamentalux5ftypes]{Fundamental Types}
\item
  \hyperref[operatorsux5fandux5fcomparisons]{Operators and Comparisons}
\item
  \hyperref[compoundux5ftypes]{Compound Types}
\item
  \hyperref[controlux5fflow]{Control Flow}
\item
  \hyperref[loops]{Loops}
\item
  \hyperref[functions]{Functions}
\item
  \hyperref[classes]{Classes}
\end{itemize}

    \subsection{Python program files}\label{python-program-files}

\begin{itemize}
\item
  Python code is usually stored in text files with the file ending
  ``\texttt{.py}'':

\begin{verbatim}
myprogram.py
\end{verbatim}
\item
  Every line in a Python program file is assumed to be a Python
  statement.

  \begin{itemize}
  \itemsep1pt\parskip0pt\parsep0pt
  \item
    The only exception is comment lines, which start with the character
    \texttt{\#} (optionally preceded by an arbitrary number of
    white-space characters, i.e., tabs or spaces). Comment lines are
    usually ignored by the Python interpreter.
  \end{itemize}
\item
  To run our Python program from the command line we use:

\begin{verbatim}
$ python myprogram.py
\end{verbatim}
\item
  On UNIX systems it is common to define the path to the interpreter on
  the first line of the program (note that this is a comment line as far
  as the Python interpreter is concerned):

\begin{verbatim}
#!/usr/bin/env python
\end{verbatim}
\end{itemize}

If we do, and if we additionally set the file script to be executable,
we can run the program like this:

\begin{verbatim}
    $ myprogram.py
\end{verbatim}

\paragraph{Example:}\label{example}

    \begin{Verbatim}[commandchars=\\\{\}]
{\color{incolor}In [{\color{incolor}8}]:} \PY{n}{script\PYZus{}dir} \PY{o}{=} \PY{l+s}{\PYZsq{}}\PY{l+s}{../scripts/}\PY{l+s}{\PYZsq{}}
\end{Verbatim}

    \begin{Verbatim}[commandchars=\\\{\}]
{\color{incolor}In [{\color{incolor}22}]:} \PY{n}{ls} \PY{err}{\PYZdl{}}\PY{n}{script\PYZus{}dir}\PY{l+s}{\PYZdq{}}\PY{l+s}{hello\PYZhy{}world}\PY{l+s}{\PYZdq{}}\PY{o}{*}\PY{o}{.}\PY{n}{py}
\end{Verbatim}

    \begin{Verbatim}[commandchars=\\\{\}]
../scripts/hello-world-in-german.py  ../scripts/hello-world.py
    \end{Verbatim}

    \begin{Verbatim}[commandchars=\\\{\}]
{\color{incolor}In [{\color{incolor}18}]:} \PY{n}{cat} \PY{err}{\PYZdl{}}\PY{n}{script\PYZus{}dir}\PY{l+s}{\PYZdq{}}\PY{l+s}{hello\PYZhy{}world.py}\PY{l+s}{\PYZdq{}}
\end{Verbatim}

    \begin{Verbatim}[commandchars=\\\{\}]
\#!/usr/bin/env python

print("Bergische Universitat Wuppertal!")
    \end{Verbatim}

    \begin{Verbatim}[commandchars=\\\{\}]
{\color{incolor}In [{\color{incolor}23}]:} \PY{o}{!}python \PY{n+nv}{\PYZdl{}script\PYZus{}dir}\PY{l+s+s2}{\PYZdq{}hello\PYZhy{}world.py\PYZdq{}}
\end{Verbatim}

    \begin{Verbatim}[commandchars=\\\{\}]
Bergische Universitat Wuppertal!
    \end{Verbatim}

    

    \subsubsection{Character encoding}\label{character-encoding}

The standard character encoding is ASCII, but we can use any other
encoding, for example UTF-8. To specify that UTF-8 is used we include
the special line

\begin{verbatim}
# -*- coding: UTF-8 -*-
\end{verbatim}

at the top of the file.

    \begin{Verbatim}[commandchars=\\\{\}]
{\color{incolor}In [{\color{incolor}24}]:} \PY{n}{cat} \PY{err}{\PYZdl{}}\PY{n}{script\PYZus{}dir}\PY{l+s}{\PYZdq{}}\PY{l+s}{hello\PYZhy{}world\PYZhy{}in\PYZhy{}german.py}\PY{l+s}{\PYZdq{}}
\end{Verbatim}

    \begin{Verbatim}[commandchars=\\\{\}]
\#!/usr/bin/env python
\# -*- coding: UTF-8 -*-

print("Bergische Universität Wuppertal!")
    \end{Verbatim}

    \begin{Verbatim}[commandchars=\\\{\}]
{\color{incolor}In [{\color{incolor}25}]:} \PY{o}{!}python \PY{n+nv}{\PYZdl{}script\PYZus{}dir}\PY{l+s+s2}{\PYZdq{}hello\PYZhy{}world\PYZhy{}in\PYZhy{}german.py\PYZdq{}}
\end{Verbatim}

    \begin{Verbatim}[commandchars=\\\{\}]
Bergische Universität Wuppertal!
    \end{Verbatim}

    Other than these two \emph{optional} lines in the beginning of a Python
code file, no additional code is required for initializing a program.

    

    \subsection{Modules}\label{modules}

Most of the functionality in Python is provided by \emph{modules}.

To use a module in a Python program it first has to be imported. A
module can be imported using the \texttt{import} statement. For example,
to import the module \texttt{math}, which contains many standard
mathematical functions, we can do:

    \begin{Verbatim}[commandchars=\\\{\}]
{\color{incolor}In [{\color{incolor}26}]:} \PY{k+kn}{import} \PY{n+nn}{math}
\end{Verbatim}

    This includes the whole module and makes it available for use later in
the program. For example, we can do:

    \begin{Verbatim}[commandchars=\\\{\}]
{\color{incolor}In [{\color{incolor}27}]:} \PY{k+kn}{import} \PY{n+nn}{math}
         
         \PY{n}{x} \PY{o}{=} \PY{n}{math}\PY{o}{.}\PY{n}{cos}\PY{p}{(}\PY{l+m+mi}{2} \PY{o}{*} \PY{n}{math}\PY{o}{.}\PY{n}{pi}\PY{p}{)}
         
         \PY{k}{print}\PY{p}{(}\PY{n}{x}\PY{p}{)}
\end{Verbatim}

    \begin{Verbatim}[commandchars=\\\{\}]
1.0
    \end{Verbatim}

    Alternatively, we can chose to import all symbols (functions and
variables) in a module to the current namespace (so that we don't need
to use the prefix ``\texttt{math.}'' every time we use something from
the \texttt{math} module:

    \begin{Verbatim}[commandchars=\\\{\}]
{\color{incolor}In [{\color{incolor}28}]:} \PY{k+kn}{from} \PY{n+nn}{math} \PY{k+kn}{import} \PY{o}{*}
         
         \PY{n}{x} \PY{o}{=} \PY{n}{cos}\PY{p}{(}\PY{l+m+mi}{2} \PY{o}{*} \PY{n}{pi}\PY{p}{)}
         
         \PY{k}{print}\PY{p}{(}\PY{n}{x}\PY{p}{)}
\end{Verbatim}

    \begin{Verbatim}[commandchars=\\\{\}]
1.0
    \end{Verbatim}

    As a third alternative, we can chose to import only a few selected
symbols from a module by explicitly listing which ones we want to import
instead of using the wildcard character \texttt{*}:

    \begin{Verbatim}[commandchars=\\\{\}]
{\color{incolor}In [{\color{incolor}29}]:} \PY{k+kn}{from} \PY{n+nn}{math} \PY{k+kn}{import} \PY{n}{cos}\PY{p}{,} \PY{n}{pi}
         
         \PY{n}{x} \PY{o}{=} \PY{n}{cos}\PY{p}{(}\PY{l+m+mi}{2} \PY{o}{*} \PY{n}{pi}\PY{p}{)}
         
         \PY{k}{print}\PY{p}{(}\PY{n}{x}\PY{p}{)}
\end{Verbatim}

    \begin{Verbatim}[commandchars=\\\{\}]
1.0
    \end{Verbatim}

    Although not a very good practice, we can rename the symbols for ease of
comprehension

    \begin{Verbatim}[commandchars=\\\{\}]
{\color{incolor}In [{\color{incolor}30}]:} \PY{k+kn}{from} \PY{n+nn}{numpy.linalg} \PY{k+kn}{import} \PY{n}{inv}
         \PY{k+kn}{from} \PY{n+nn}{scipy.sparse.linalg} \PY{k+kn}{import} \PY{n}{inv} \PY{k}{as} \PY{n}{sparseinv}
\end{Verbatim}

    \subsubsection{Looking at what a module contains, and its
documentation}\label{looking-at-what-a-module-contains-and-its-documentation}

Once a module is imported, we can list the symbols it provides using the
\texttt{dir} function:

    \begin{Verbatim}[commandchars=\\\{\}]
{\color{incolor}In [{\color{incolor}31}]:} \PY{k+kn}{import} \PY{n+nn}{math}
         
         \PY{k}{print}\PY{p}{(}\PY{n+nb}{dir}\PY{p}{(}\PY{n}{math}\PY{p}{)}\PY{p}{)}
\end{Verbatim}

    \begin{Verbatim}[commandchars=\\\{\}]
['\_\_doc\_\_', '\_\_file\_\_', '\_\_name\_\_', '\_\_package\_\_', 'acos', 'acosh', 'asin', 'asinh', 'atan', 'atan2', 'atanh', 'ceil', 'copysign', 'cos', 'cosh', 'degrees', 'e', 'erf', 'erfc', 'exp', 'expm1', 'fabs', 'factorial', 'floor', 'fmod', 'frexp', 'fsum', 'gamma', 'hypot', 'isinf', 'isnan', 'ldexp', 'lgamma', 'log', 'log10', 'log1p', 'modf', 'pi', 'pow', 'radians', 'sin', 'sinh', 'sqrt', 'tan', 'tanh', 'trunc']
    \end{Verbatim}

    And using the function \texttt{help} we can get a description of each
function (almost .. not all functions have docstrings, as they are
technically called, but the vast majority of functions are documented
this way).

    \begin{Verbatim}[commandchars=\\\{\}]
{\color{incolor}In [{\color{incolor}32}]:} \PY{n}{help}\PY{p}{(}\PY{n}{math}\PY{o}{.}\PY{n}{log}\PY{p}{)}
\end{Verbatim}

    \begin{Verbatim}[commandchars=\\\{\}]
Help on built-in function log in module math:

log(\ldots)
    log(x[, base])
    
    Return the logarithm of x to the given base.
    If the base not specified, returns the natural logarithm (base e) of x.
    \end{Verbatim}

    \begin{Verbatim}[commandchars=\\\{\}]
{\color{incolor}In [{\color{incolor}33}]:} \PY{n}{log}\PY{p}{(}\PY{l+m+mi}{10}\PY{p}{)}
\end{Verbatim}

            \begin{Verbatim}[commandchars=\\\{\}]
{\color{outcolor}Out[{\color{outcolor}33}]:} 2.302585092994046
\end{Verbatim}
        
    \begin{Verbatim}[commandchars=\\\{\}]
{\color{incolor}In [{\color{incolor}34}]:} \PY{n}{log}\PY{p}{(}\PY{l+m+mi}{10}\PY{p}{,} \PY{l+m+mi}{2}\PY{p}{)}
\end{Verbatim}

            \begin{Verbatim}[commandchars=\\\{\}]
{\color{outcolor}Out[{\color{outcolor}34}]:} 3.3219280948873626
\end{Verbatim}
        
    We can also use the \texttt{help} function directly on modules: Try

\begin{verbatim}
help(math) 
\end{verbatim}

    \begin{Verbatim}[commandchars=\\\{\}]
{\color{incolor}In [{\color{incolor}35}]:} \PY{n}{help}\PY{p}{(}\PY{n}{math}\PY{p}{)}
\end{Verbatim}

    \begin{Verbatim}[commandchars=\\\{\}]
Help on module math:

NAME
    math

FILE
    /home/jpsilva/anaconda/lib/python2.7/lib-dynload/math.so

MODULE DOCS
    http://docs.python.org/library/math

DESCRIPTION
    This module is always available.  It provides access to the
    mathematical functions defined by the C standard.

FUNCTIONS
    acos(\ldots)
        acos(x)
        
        Return the arc cosine (measured in radians) of x.
    
    acosh(\ldots)
        acosh(x)
        
        Return the hyperbolic arc cosine (measured in radians) of x.
    
    asin(\ldots)
        asin(x)
        
        Return the arc sine (measured in radians) of x.
    
    asinh(\ldots)
        asinh(x)
        
        Return the hyperbolic arc sine (measured in radians) of x.
    
    atan(\ldots)
        atan(x)
        
        Return the arc tangent (measured in radians) of x.
    
    atan2(\ldots)
        atan2(y, x)
        
        Return the arc tangent (measured in radians) of y/x.
        Unlike atan(y/x), the signs of both x and y are considered.
    
    atanh(\ldots)
        atanh(x)
        
        Return the hyperbolic arc tangent (measured in radians) of x.
    
    ceil(\ldots)
        ceil(x)
        
        Return the ceiling of x as a float.
        This is the smallest integral value >= x.
    
    copysign(\ldots)
        copysign(x, y)
        
        Return x with the sign of y.
    
    cos(\ldots)
        cos(x)
        
        Return the cosine of x (measured in radians).
    
    cosh(\ldots)
        cosh(x)
        
        Return the hyperbolic cosine of x.
    
    degrees(\ldots)
        degrees(x)
        
        Convert angle x from radians to degrees.
    
    erf(\ldots)
        erf(x)
        
        Error function at x.
    
    erfc(\ldots)
        erfc(x)
        
        Complementary error function at x.
    
    exp(\ldots)
        exp(x)
        
        Return e raised to the power of x.
    
    expm1(\ldots)
        expm1(x)
        
        Return exp(x)-1.
        This function avoids the loss of precision involved in the direct evaluation of exp(x)-1 for small x.
    
    fabs(\ldots)
        fabs(x)
        
        Return the absolute value of the float x.
    
    factorial(\ldots)
        factorial(x) -> Integral
        
        Find x!. Raise a ValueError if x is negative or non-integral.
    
    floor(\ldots)
        floor(x)
        
        Return the floor of x as a float.
        This is the largest integral value <= x.
    
    fmod(\ldots)
        fmod(x, y)
        
        Return fmod(x, y), according to platform C.  x \% y may differ.
    
    frexp(\ldots)
        frexp(x)
        
        Return the mantissa and exponent of x, as pair (m, e).
        m is a float and e is an int, such that x = m * 2.**e.
        If x is 0, m and e are both 0.  Else 0.5 <= abs(m) < 1.0.
    
    fsum(\ldots)
        fsum(iterable)
        
        Return an accurate floating point sum of values in the iterable.
        Assumes IEEE-754 floating point arithmetic.
    
    gamma(\ldots)
        gamma(x)
        
        Gamma function at x.
    
    hypot(\ldots)
        hypot(x, y)
        
        Return the Euclidean distance, sqrt(x*x + y*y).
    
    isinf(\ldots)
        isinf(x) -> bool
        
        Check if float x is infinite (positive or negative).
    
    isnan(\ldots)
        isnan(x) -> bool
        
        Check if float x is not a number (NaN).
    
    ldexp(\ldots)
        ldexp(x, i)
        
        Return x * (2**i).
    
    lgamma(\ldots)
        lgamma(x)
        
        Natural logarithm of absolute value of Gamma function at x.
    
    log(\ldots)
        log(x[, base])
        
        Return the logarithm of x to the given base.
        If the base not specified, returns the natural logarithm (base e) of x.
    
    log10(\ldots)
        log10(x)
        
        Return the base 10 logarithm of x.
    
    log1p(\ldots)
        log1p(x)
        
        Return the natural logarithm of 1+x (base e).
        The result is computed in a way which is accurate for x near zero.
    
    modf(\ldots)
        modf(x)
        
        Return the fractional and integer parts of x.  Both results carry the sign
        of x and are floats.
    
    pow(\ldots)
        pow(x, y)
        
        Return x**y (x to the power of y).
    
    radians(\ldots)
        radians(x)
        
        Convert angle x from degrees to radians.
    
    sin(\ldots)
        sin(x)
        
        Return the sine of x (measured in radians).
    
    sinh(\ldots)
        sinh(x)
        
        Return the hyperbolic sine of x.
    
    sqrt(\ldots)
        sqrt(x)
        
        Return the square root of x.
    
    tan(\ldots)
        tan(x)
        
        Return the tangent of x (measured in radians).
    
    tanh(\ldots)
        tanh(x)
        
        Return the hyperbolic tangent of x.
    
    trunc(\ldots)
        trunc(x:Real) -> Integral
        
        Truncates x to the nearest Integral toward 0. Uses the \_\_trunc\_\_ magic method.

DATA
    e = 2.718281828459045
    pi = 3.141592653589793
    \end{Verbatim}

    

    \subsection{Variables and types}\label{variables-and-types}

\subsubsection{Symbol names}\label{symbol-names}

Variable names in Python can contain alphanumerical characters
\texttt{a-z}, \texttt{A-Z}, \texttt{0-9} and some special characters
such as \texttt{\_}. Normal variable names must start with a letter.

By convension, variable names start with a lower-case letter, and Class
names start with a capital letter.

In addition, there are a number of Python keywords that cannot be used
as variable names. These keywords are:

\begin{verbatim}
and, as, assert, break, class, continue, def, del, elif, else, except, 
exec, finally, for, from, global, if, import, in, is, lambda, not, or,
pass, print, raise, return, try, while, with, yield
\end{verbatim}

Note: Be aware of the keyword \texttt{lambda}, which could easily be a
natural variable name in a scientific program. But being a keyword, it
cannot be used as a variable name.

\subsubsection{Assignment}\label{assignment}

The assignment operator in Python is \texttt{=}. Python is a dynamically
typed language, so we do not need to specify the type of a variable when
we create one.

Assigning a value to a new variable creates the variable:

    \begin{Verbatim}[commandchars=\\\{\}]
{\color{incolor}In [{\color{incolor}36}]:} \PY{k+kn}{import} \PY{n+nn}{sys}
         \PY{k+kn}{import} \PY{n+nn}{keyword}
         
         \PY{k}{print} \PY{n}{sys}\PY{o}{.}\PY{n}{version}
         \PY{k}{print} \PY{l+s}{\PYZsq{}}\PY{l+s}{\PYZsq{}}
         \PY{k}{print}\PY{p}{(}\PY{n}{keyword}\PY{o}{.}\PY{n}{kwlist}\PY{p}{)}
\end{Verbatim}

    \begin{Verbatim}[commandchars=\\\{\}]
2.7.8 |Anaconda 2.1.0 (64-bit)| (default, Aug 21 2014, 18:22:21) 
[GCC 4.4.7 20120313 (Red Hat 4.4.7-1)]

['and', 'as', 'assert', 'break', 'class', 'continue', 'def', 'del', 'elif', 'else', 'except', 'exec', 'finally', 'for', 'from', 'global', 'if', 'import', 'in', 'is', 'lambda', 'not', 'or', 'pass', 'print', 'raise', 'return', 'try', 'while', 'with', 'yield']
    \end{Verbatim}

    \begin{Verbatim}[commandchars=\\\{\}]
{\color{incolor}In [{\color{incolor}37}]:} \PY{c}{\PYZsh{} variable assignments}
         \PY{n}{x} \PY{o}{=} \PY{l+m+mf}{1.0}
         \PY{n}{my\PYZus{}variable} \PY{o}{=} \PY{l+m+mf}{12.2}
\end{Verbatim}

    Although not explicitly specified, a variable do have a type associated
with it. The type is derived from the value it was assigned.

    \begin{Verbatim}[commandchars=\\\{\}]
{\color{incolor}In [{\color{incolor}38}]:} \PY{n+nb}{type}\PY{p}{(}\PY{n}{x}\PY{p}{)}
\end{Verbatim}

            \begin{Verbatim}[commandchars=\\\{\}]
{\color{outcolor}Out[{\color{outcolor}38}]:} float
\end{Verbatim}
        
    If we assign a new value to a variable, its type can change.

    \begin{Verbatim}[commandchars=\\\{\}]
{\color{incolor}In [{\color{incolor}39}]:} \PY{n}{x} \PY{o}{=} \PY{l+m+mi}{1}
\end{Verbatim}

    \begin{Verbatim}[commandchars=\\\{\}]
{\color{incolor}In [{\color{incolor}40}]:} \PY{n+nb}{type}\PY{p}{(}\PY{n}{x}\PY{p}{)}
\end{Verbatim}

            \begin{Verbatim}[commandchars=\\\{\}]
{\color{outcolor}Out[{\color{outcolor}40}]:} int
\end{Verbatim}
        
    If we try to use a variable that has not yet been defined we get an
\texttt{NameError}:

    \begin{Verbatim}[commandchars=\\\{\}]
{\color{incolor}In [{\color{incolor}41}]:} \PY{k}{print}\PY{p}{(}\PY{n}{y}\PY{p}{)}
\end{Verbatim}

    \begin{Verbatim}[commandchars=\\\{\}]

        ---------------------------------------------------------------------------
    NameError                                 Traceback (most recent call last)

        <ipython-input-41-36b2093251cd> in <module>()
    ----> 1 print(y)
    

        NameError: name 'y' is not defined

    \end{Verbatim}

    

    \subsubsection{Fundamental types}\label{fundamental-types}

    \begin{Verbatim}[commandchars=\\\{\}]
{\color{incolor}In [{\color{incolor}42}]:} \PY{c}{\PYZsh{} integers}
         \PY{n}{x} \PY{o}{=} \PY{l+m+mi}{1}
         \PY{n+nb}{type}\PY{p}{(}\PY{n}{x}\PY{p}{)}
\end{Verbatim}

            \begin{Verbatim}[commandchars=\\\{\}]
{\color{outcolor}Out[{\color{outcolor}42}]:} int
\end{Verbatim}
        
    \begin{Verbatim}[commandchars=\\\{\}]
{\color{incolor}In [{\color{incolor}43}]:} \PY{c}{\PYZsh{} float}
         \PY{n}{x} \PY{o}{=} \PY{l+m+mf}{1.0}
         \PY{n+nb}{type}\PY{p}{(}\PY{n}{x}\PY{p}{)}
\end{Verbatim}

            \begin{Verbatim}[commandchars=\\\{\}]
{\color{outcolor}Out[{\color{outcolor}43}]:} float
\end{Verbatim}
        
    \begin{Verbatim}[commandchars=\\\{\}]
{\color{incolor}In [{\color{incolor}44}]:} \PY{c}{\PYZsh{} boolean}
         \PY{n}{b1} \PY{o}{=} \PY{n+nb+bp}{True}
         \PY{n}{b2} \PY{o}{=} \PY{n+nb+bp}{False}
         
         \PY{n+nb}{type}\PY{p}{(}\PY{n}{b1}\PY{p}{)}
\end{Verbatim}

            \begin{Verbatim}[commandchars=\\\{\}]
{\color{outcolor}Out[{\color{outcolor}44}]:} bool
\end{Verbatim}
        
    \begin{Verbatim}[commandchars=\\\{\}]
{\color{incolor}In [{\color{incolor}45}]:} \PY{c}{\PYZsh{} complex numbers: note the use of `j` to specify the imaginary part}
         \PY{n}{x} \PY{o}{=} \PY{l+m+mf}{1.0} \PY{o}{\PYZhy{}} \PY{l+m+mf}{1.0j}
         \PY{n+nb}{type}\PY{p}{(}\PY{n}{x}\PY{p}{)}
\end{Verbatim}

            \begin{Verbatim}[commandchars=\\\{\}]
{\color{outcolor}Out[{\color{outcolor}45}]:} complex
\end{Verbatim}
        
    \begin{Verbatim}[commandchars=\\\{\}]
{\color{incolor}In [{\color{incolor}46}]:} \PY{k}{print}\PY{p}{(}\PY{n}{x}\PY{p}{)}
\end{Verbatim}

    \begin{Verbatim}[commandchars=\\\{\}]
(1-1j)
    \end{Verbatim}

    \begin{Verbatim}[commandchars=\\\{\}]
{\color{incolor}In [{\color{incolor}47}]:} \PY{k}{print}\PY{p}{(}\PY{n}{x}\PY{o}{.}\PY{n}{real}\PY{p}{,} \PY{n}{x}\PY{o}{.}\PY{n}{imag}\PY{p}{)}
\end{Verbatim}

    \begin{Verbatim}[commandchars=\\\{\}]
(1.0, -1.0)
    \end{Verbatim}

    \begin{Verbatim}[commandchars=\\\{\}]
{\color{incolor}In [{\color{incolor}48}]:} \PY{k+kn}{import} \PY{n+nn}{types}
         
         \PY{c}{\PYZsh{} print all types defined in the `types` module}
         \PY{k}{print}\PY{p}{(}\PY{n+nb}{dir}\PY{p}{(}\PY{n}{types}\PY{p}{)}\PY{p}{)}
\end{Verbatim}

    \begin{Verbatim}[commandchars=\\\{\}]
['BooleanType', 'BufferType', 'BuiltinFunctionType', 'BuiltinMethodType', 'ClassType', 'CodeType', 'ComplexType', 'DictProxyType', 'DictType', 'DictionaryType', 'EllipsisType', 'FileType', 'FloatType', 'FrameType', 'FunctionType', 'GeneratorType', 'GetSetDescriptorType', 'InstanceType', 'IntType', 'LambdaType', 'ListType', 'LongType', 'MemberDescriptorType', 'MethodType', 'ModuleType', 'NoneType', 'NotImplementedType', 'ObjectType', 'SliceType', 'StringType', 'StringTypes', 'TracebackType', 'TupleType', 'TypeType', 'UnboundMethodType', 'UnicodeType', 'XRangeType', '\_\_builtins\_\_', '\_\_doc\_\_', '\_\_file\_\_', '\_\_name\_\_', '\_\_package\_\_']
    \end{Verbatim}

    \begin{Verbatim}[commandchars=\\\{\}]
{\color{incolor}In [{\color{incolor}49}]:} \PY{n}{x} \PY{o}{=} \PY{l+m+mf}{1.0}
         
         \PY{c}{\PYZsh{} check if the variable x is a float}
         \PY{n+nb}{type}\PY{p}{(}\PY{n}{x}\PY{p}{)} \PY{o+ow}{is} \PY{n+nb}{float}
\end{Verbatim}

            \begin{Verbatim}[commandchars=\\\{\}]
{\color{outcolor}Out[{\color{outcolor}49}]:} True
\end{Verbatim}
        
    \begin{Verbatim}[commandchars=\\\{\}]
{\color{incolor}In [{\color{incolor}50}]:} \PY{c}{\PYZsh{} check if the variable x is an int}
         \PY{n+nb}{type}\PY{p}{(}\PY{n}{x}\PY{p}{)} \PY{o+ow}{is} \PY{n+nb}{int}
\end{Verbatim}

            \begin{Verbatim}[commandchars=\\\{\}]
{\color{outcolor}Out[{\color{outcolor}50}]:} False
\end{Verbatim}
        
    We can also use the \texttt{isinstance} method for testing types of
variables:

    \begin{Verbatim}[commandchars=\\\{\}]
{\color{incolor}In [{\color{incolor}51}]:} \PY{n+nb}{isinstance}\PY{p}{(}\PY{n}{x}\PY{p}{,} \PY{n+nb}{float}\PY{p}{)}
\end{Verbatim}

            \begin{Verbatim}[commandchars=\\\{\}]
{\color{outcolor}Out[{\color{outcolor}51}]:} True
\end{Verbatim}
        
    \subsubsection{Type casting}\label{type-casting}

    \begin{Verbatim}[commandchars=\\\{\}]
{\color{incolor}In [{\color{incolor}52}]:} \PY{n}{x} \PY{o}{=} \PY{l+m+mf}{1.5}
         
         \PY{k}{print}\PY{p}{(}\PY{n}{x}\PY{p}{,} \PY{n+nb}{type}\PY{p}{(}\PY{n}{x}\PY{p}{)}\PY{p}{)}
\end{Verbatim}

    \begin{Verbatim}[commandchars=\\\{\}]
(1.5, <type 'float'>)
    \end{Verbatim}

    \begin{Verbatim}[commandchars=\\\{\}]
{\color{incolor}In [{\color{incolor}53}]:} \PY{n}{x} \PY{o}{=} \PY{n+nb}{int}\PY{p}{(}\PY{n}{x}\PY{p}{)}
         
         \PY{k}{print}\PY{p}{(}\PY{n}{x}\PY{p}{,} \PY{n+nb}{type}\PY{p}{(}\PY{n}{x}\PY{p}{)}\PY{p}{)}
\end{Verbatim}

    \begin{Verbatim}[commandchars=\\\{\}]
(1, <type 'int'>)
    \end{Verbatim}

    \begin{Verbatim}[commandchars=\\\{\}]
{\color{incolor}In [{\color{incolor}54}]:} \PY{n}{z} \PY{o}{=} \PY{n+nb}{complex}\PY{p}{(}\PY{n}{x}\PY{p}{)}
         
         \PY{k}{print}\PY{p}{(}\PY{n}{z}\PY{p}{,} \PY{n+nb}{type}\PY{p}{(}\PY{n}{z}\PY{p}{)}\PY{p}{)}
\end{Verbatim}

    \begin{Verbatim}[commandchars=\\\{\}]
((1+0j), <type 'complex'>)
    \end{Verbatim}

    \begin{Verbatim}[commandchars=\\\{\}]
{\color{incolor}In [{\color{incolor}55}]:} \PY{n}{x} \PY{o}{=} \PY{n+nb}{float}\PY{p}{(}\PY{n}{z}\PY{p}{)}
\end{Verbatim}

    \begin{Verbatim}[commandchars=\\\{\}]

        ---------------------------------------------------------------------------
    TypeError                                 Traceback (most recent call last)

        <ipython-input-55-e719cc7b3e96> in <module>()
    ----> 1 x = float(z)
    

        TypeError: can't convert complex to float

    \end{Verbatim}

    Complex variables cannot be cast to floats or integers. We need to use
\texttt{z.real} or \texttt{z.imag} to extract the part of the complex
number we want:

    \begin{Verbatim}[commandchars=\\\{\}]
{\color{incolor}In [{\color{incolor}56}]:} \PY{n}{y} \PY{o}{=} \PY{n+nb}{bool}\PY{p}{(}\PY{n}{z}\PY{o}{.}\PY{n}{real}\PY{p}{)}
         
         \PY{k}{print}\PY{p}{(}\PY{n}{z}\PY{o}{.}\PY{n}{real}\PY{p}{,} \PY{l+s}{\PYZdq{}}\PY{l+s}{ \PYZhy{}\PYZgt{} }\PY{l+s}{\PYZdq{}}\PY{p}{,} \PY{n}{y}\PY{p}{,} \PY{n+nb}{type}\PY{p}{(}\PY{n}{y}\PY{p}{)}\PY{p}{)}
         
         \PY{n}{y} \PY{o}{=} \PY{n+nb}{bool}\PY{p}{(}\PY{n}{z}\PY{o}{.}\PY{n}{imag}\PY{p}{)}
         
         \PY{k}{print}\PY{p}{(}\PY{n}{z}\PY{o}{.}\PY{n}{imag}\PY{p}{,} \PY{l+s}{\PYZdq{}}\PY{l+s}{ \PYZhy{}\PYZgt{} }\PY{l+s}{\PYZdq{}}\PY{p}{,} \PY{n}{y}\PY{p}{,} \PY{n+nb}{type}\PY{p}{(}\PY{n}{y}\PY{p}{)}\PY{p}{)}
\end{Verbatim}

    \begin{Verbatim}[commandchars=\\\{\}]
(1.0, ' -> ', True, <type 'bool'>)
(0.0, ' -> ', False, <type 'bool'>)
    \end{Verbatim}

    

    \subsection{Operators and comparisons}\label{operators-and-comparisons}

Most operators and comparisons in Python work as one would expect:

\begin{itemize}
\itemsep1pt\parskip0pt\parsep0pt
\item
  Arithmetic operators \texttt{+}, \texttt{-}, \texttt{*}, \texttt{/},
  \texttt{//} (integer division), '**' power
\end{itemize}

    \begin{Verbatim}[commandchars=\\\{\}]
{\color{incolor}In [{\color{incolor}57}]:} \PY{l+m+mi}{1} \PY{o}{+} \PY{l+m+mi}{2}\PY{p}{,} \PY{l+m+mi}{1} \PY{o}{\PYZhy{}} \PY{l+m+mi}{2}\PY{p}{,} \PY{l+m+mi}{1} \PY{o}{*} \PY{l+m+mi}{2}\PY{p}{,} \PY{l+m+mi}{1} \PY{o}{/} \PY{l+m+mi}{2}
\end{Verbatim}

            \begin{Verbatim}[commandchars=\\\{\}]
{\color{outcolor}Out[{\color{outcolor}57}]:} (3, -1, 2, 0)
\end{Verbatim}
        
    \begin{Verbatim}[commandchars=\\\{\}]
{\color{incolor}In [{\color{incolor}58}]:} \PY{l+m+mf}{1.0} \PY{o}{+} \PY{l+m+mf}{2.0}\PY{p}{,} \PY{l+m+mf}{1.0} \PY{o}{\PYZhy{}} \PY{l+m+mf}{2.0}\PY{p}{,} \PY{l+m+mf}{1.0} \PY{o}{*} \PY{l+m+mf}{2.0}\PY{p}{,} \PY{l+m+mf}{1.0} \PY{o}{/} \PY{l+m+mf}{2.0}
\end{Verbatim}

            \begin{Verbatim}[commandchars=\\\{\}]
{\color{outcolor}Out[{\color{outcolor}58}]:} (3.0, -1.0, 2.0, 0.5)
\end{Verbatim}
        
    \begin{Verbatim}[commandchars=\\\{\}]
{\color{incolor}In [{\color{incolor}59}]:} \PY{l+m+mf}{3.0} \PY{o}{/}\PY{o}{/} \PY{l+m+mf}{2.0}
\end{Verbatim}

            \begin{Verbatim}[commandchars=\\\{\}]
{\color{outcolor}Out[{\color{outcolor}59}]:} 1.0
\end{Verbatim}
        
    \begin{Verbatim}[commandchars=\\\{\}]
{\color{incolor}In [{\color{incolor}60}]:} \PY{l+m+mi}{2} \PY{o}{*}\PY{o}{*} \PY{l+m+mi}{2}
\end{Verbatim}

            \begin{Verbatim}[commandchars=\\\{\}]
{\color{outcolor}Out[{\color{outcolor}60}]:} 4
\end{Verbatim}
        
    \begin{itemize}
\itemsep1pt\parskip0pt\parsep0pt
\item
  The boolean operators are spelled out as words \texttt{and},
  \texttt{not}, \texttt{or}.
\end{itemize}

    \begin{Verbatim}[commandchars=\\\{\}]
{\color{incolor}In [{\color{incolor}61}]:} \PY{n+nb+bp}{True} \PY{o+ow}{and} \PY{n+nb+bp}{False}
\end{Verbatim}

            \begin{Verbatim}[commandchars=\\\{\}]
{\color{outcolor}Out[{\color{outcolor}61}]:} False
\end{Verbatim}
        
    \begin{Verbatim}[commandchars=\\\{\}]
{\color{incolor}In [{\color{incolor}62}]:} \PY{o+ow}{not} \PY{n+nb+bp}{False}
\end{Verbatim}

            \begin{Verbatim}[commandchars=\\\{\}]
{\color{outcolor}Out[{\color{outcolor}62}]:} True
\end{Verbatim}
        
    \begin{Verbatim}[commandchars=\\\{\}]
{\color{incolor}In [{\color{incolor}63}]:} \PY{n+nb+bp}{True} \PY{o+ow}{or} \PY{n+nb+bp}{False}
\end{Verbatim}

            \begin{Verbatim}[commandchars=\\\{\}]
{\color{outcolor}Out[{\color{outcolor}63}]:} True
\end{Verbatim}
        
    \begin{itemize}
\itemsep1pt\parskip0pt\parsep0pt
\item
  Comparison operators \texttt{\textgreater{}}, \texttt{\textless{}},
  \texttt{\textgreater{}=} (greater or equal), \texttt{\textless{}=}
  (less or equal), \texttt{==} equality, \texttt{is} identical.
\end{itemize}

    \begin{Verbatim}[commandchars=\\\{\}]
{\color{incolor}In [{\color{incolor}64}]:} \PY{l+m+mi}{2} \PY{o}{\PYZgt{}} \PY{l+m+mi}{1}\PY{p}{,} \PY{l+m+mi}{2} \PY{o}{\PYZlt{}} \PY{l+m+mi}{1}
\end{Verbatim}

            \begin{Verbatim}[commandchars=\\\{\}]
{\color{outcolor}Out[{\color{outcolor}64}]:} (True, False)
\end{Verbatim}
        
    \begin{Verbatim}[commandchars=\\\{\}]
{\color{incolor}In [{\color{incolor}65}]:} \PY{l+m+mi}{2} \PY{o}{\PYZgt{}} \PY{l+m+mi}{2}\PY{p}{,} \PY{l+m+mi}{2} \PY{o}{\PYZlt{}} \PY{l+m+mi}{2}
\end{Verbatim}

            \begin{Verbatim}[commandchars=\\\{\}]
{\color{outcolor}Out[{\color{outcolor}65}]:} (False, False)
\end{Verbatim}
        
    \begin{Verbatim}[commandchars=\\\{\}]
{\color{incolor}In [{\color{incolor}66}]:} \PY{l+m+mi}{2} \PY{o}{\PYZgt{}}\PY{o}{=} \PY{l+m+mi}{2}\PY{p}{,} \PY{l+m+mi}{2} \PY{o}{\PYZlt{}}\PY{o}{=} \PY{l+m+mi}{2}
\end{Verbatim}

            \begin{Verbatim}[commandchars=\\\{\}]
{\color{outcolor}Out[{\color{outcolor}66}]:} (True, True)
\end{Verbatim}
        
    \begin{Verbatim}[commandchars=\\\{\}]
{\color{incolor}In [{\color{incolor}67}]:} \PY{c}{\PYZsh{} equality}
         \PY{p}{[}\PY{l+m+mi}{1}\PY{p}{,}\PY{l+m+mi}{2}\PY{p}{]} \PY{o}{==} \PY{p}{[}\PY{l+m+mi}{1}\PY{p}{,}\PY{l+m+mi}{2}\PY{p}{]}
\end{Verbatim}

            \begin{Verbatim}[commandchars=\\\{\}]
{\color{outcolor}Out[{\color{outcolor}67}]:} True
\end{Verbatim}
        
    \begin{Verbatim}[commandchars=\\\{\}]
{\color{incolor}In [{\color{incolor}68}]:} \PY{c}{\PYZsh{} objects identical?}
         \PY{n}{l1} \PY{o}{=} \PY{n}{l2} \PY{o}{=} \PY{p}{[}\PY{l+m+mi}{1}\PY{p}{,}\PY{l+m+mi}{2}\PY{p}{]}
         
         \PY{n}{l1} \PY{o+ow}{is} \PY{n}{l2}
\end{Verbatim}

            \begin{Verbatim}[commandchars=\\\{\}]
{\color{outcolor}Out[{\color{outcolor}68}]:} True
\end{Verbatim}
        
    

    \subsection{Compound types: Strings, List and
dictionaries}\label{compound-types-strings-list-and-dictionaries}

\subsubsection{Strings}\label{strings}

Strings are the variable type that is used for storing text messages.

    \begin{Verbatim}[commandchars=\\\{\}]
{\color{incolor}In [{\color{incolor}69}]:} \PY{n}{s} \PY{o}{=} \PY{l+s}{\PYZdq{}}\PY{l+s}{Hello world}\PY{l+s}{\PYZdq{}}
         \PY{n+nb}{type}\PY{p}{(}\PY{n}{s}\PY{p}{)}
\end{Verbatim}

            \begin{Verbatim}[commandchars=\\\{\}]
{\color{outcolor}Out[{\color{outcolor}69}]:} str
\end{Verbatim}
        
    \begin{Verbatim}[commandchars=\\\{\}]
{\color{incolor}In [{\color{incolor}70}]:} \PY{c}{\PYZsh{} length of the string: the number of characters}
         \PY{n+nb}{len}\PY{p}{(}\PY{n}{s}\PY{p}{)}
\end{Verbatim}

            \begin{Verbatim}[commandchars=\\\{\}]
{\color{outcolor}Out[{\color{outcolor}70}]:} 11
\end{Verbatim}
        
    \begin{Verbatim}[commandchars=\\\{\}]
{\color{incolor}In [{\color{incolor}71}]:} \PY{c}{\PYZsh{} replace a substring in a string with somethign else}
         \PY{n}{s2} \PY{o}{=} \PY{n}{s}\PY{o}{.}\PY{n}{replace}\PY{p}{(}\PY{l+s}{\PYZdq{}}\PY{l+s}{world}\PY{l+s}{\PYZdq{}}\PY{p}{,} \PY{l+s}{\PYZdq{}}\PY{l+s}{test}\PY{l+s}{\PYZdq{}}\PY{p}{)}
         \PY{k}{print}\PY{p}{(}\PY{n}{s2}\PY{p}{)}
\end{Verbatim}

    \begin{Verbatim}[commandchars=\\\{\}]
Hello test
    \end{Verbatim}

    We can index a character in a string using \texttt{{[}{]}}:

    \begin{Verbatim}[commandchars=\\\{\}]
{\color{incolor}In [{\color{incolor}72}]:} \PY{n}{s}\PY{p}{[}\PY{l+m+mi}{0}\PY{p}{]}
\end{Verbatim}

            \begin{Verbatim}[commandchars=\\\{\}]
{\color{outcolor}Out[{\color{outcolor}72}]:} 'H'
\end{Verbatim}
        
    We can extract a part of a string using the syntax
\texttt{{[}start:stop{]}}, which extracts characters between index
\texttt{start} and \texttt{stop}:

    \begin{Verbatim}[commandchars=\\\{\}]
{\color{incolor}In [{\color{incolor}73}]:} \PY{n}{s}\PY{p}{[}\PY{l+m+mi}{0}\PY{p}{:}\PY{l+m+mi}{5}\PY{p}{]}
\end{Verbatim}

            \begin{Verbatim}[commandchars=\\\{\}]
{\color{outcolor}Out[{\color{outcolor}73}]:} 'Hello'
\end{Verbatim}
        
    If we omit either (or both) of \texttt{start} or \texttt{stop} from
\texttt{{[}start:stop{]}}, the default is the beginning and the end of
the string, respectively:

    \begin{Verbatim}[commandchars=\\\{\}]
{\color{incolor}In [{\color{incolor}74}]:} \PY{n}{s}\PY{p}{[}\PY{p}{:}\PY{l+m+mi}{5}\PY{p}{]}
\end{Verbatim}

            \begin{Verbatim}[commandchars=\\\{\}]
{\color{outcolor}Out[{\color{outcolor}74}]:} 'Hello'
\end{Verbatim}
        
    \begin{Verbatim}[commandchars=\\\{\}]
{\color{incolor}In [{\color{incolor}75}]:} \PY{n}{s}\PY{p}{[}\PY{l+m+mi}{6}\PY{p}{:}\PY{p}{]}
\end{Verbatim}

            \begin{Verbatim}[commandchars=\\\{\}]
{\color{outcolor}Out[{\color{outcolor}75}]:} 'world'
\end{Verbatim}
        
    \begin{Verbatim}[commandchars=\\\{\}]
{\color{incolor}In [{\color{incolor}76}]:} \PY{n}{s}\PY{p}{[}\PY{p}{:}\PY{p}{]}
\end{Verbatim}

            \begin{Verbatim}[commandchars=\\\{\}]
{\color{outcolor}Out[{\color{outcolor}76}]:} 'Hello world'
\end{Verbatim}
        
    We can also define the step size using the syntax
\texttt{{[}start:end:step{]}} (the default value for \texttt{step} is 1,
as we saw above):

    \begin{Verbatim}[commandchars=\\\{\}]
{\color{incolor}In [{\color{incolor}77}]:} \PY{n}{s}\PY{p}{[}\PY{p}{:}\PY{p}{:}\PY{l+m+mi}{1}\PY{p}{]}
\end{Verbatim}

            \begin{Verbatim}[commandchars=\\\{\}]
{\color{outcolor}Out[{\color{outcolor}77}]:} 'Hello world'
\end{Verbatim}
        
    \begin{Verbatim}[commandchars=\\\{\}]
{\color{incolor}In [{\color{incolor}78}]:} \PY{n}{s}\PY{p}{[}\PY{p}{:}\PY{p}{:}\PY{l+m+mi}{2}\PY{p}{]}
\end{Verbatim}

            \begin{Verbatim}[commandchars=\\\{\}]
{\color{outcolor}Out[{\color{outcolor}78}]:} 'Hlowrd'
\end{Verbatim}
        
    \paragraph{String formatting examples}\label{string-formatting-examples}

    \begin{Verbatim}[commandchars=\\\{\}]
{\color{incolor}In [{\color{incolor}79}]:} \PY{k}{print}\PY{p}{(}\PY{l+s}{\PYZdq{}}\PY{l+s}{str1}\PY{l+s}{\PYZdq{}} \PY{o}{+} \PY{l+s}{\PYZdq{}}\PY{l+s}{str2}\PY{l+s}{\PYZdq{}} \PY{o}{+} \PY{l+s}{\PYZdq{}}\PY{l+s}{str3}\PY{l+s}{\PYZdq{}}\PY{p}{)} \PY{c}{\PYZsh{} strings added with + are concatenated without space}
\end{Verbatim}

    \begin{Verbatim}[commandchars=\\\{\}]
str1str2str3
    \end{Verbatim}

    \begin{Verbatim}[commandchars=\\\{\}]
{\color{incolor}In [{\color{incolor}80}]:} \PY{k}{print}\PY{p}{(}\PY{l+s}{\PYZdq{}}\PY{l+s}{value = }\PY{l+s+si}{\PYZpc{}f}\PY{l+s}{\PYZdq{}} \PY{o}{\PYZpc{}} \PY{l+m+mf}{1.0}\PY{p}{)}
\end{Verbatim}

    \begin{Verbatim}[commandchars=\\\{\}]
value = 1.000000
    \end{Verbatim}

    \begin{Verbatim}[commandchars=\\\{\}]
{\color{incolor}In [{\color{incolor}81}]:} \PY{n}{s2} \PY{o}{=} \PY{l+s}{\PYZdq{}}\PY{l+s}{value1 = }\PY{l+s+si}{\PYZpc{}.2f}\PY{l+s}{ value2 = }\PY{l+s+si}{\PYZpc{}d}\PY{l+s}{\PYZdq{}} \PY{o}{\PYZpc{}} \PY{p}{(}\PY{l+m+mf}{3.1415}\PY{p}{,} \PY{l+m+mf}{1.5}\PY{p}{)}
         
         \PY{k}{print}\PY{p}{(}\PY{n}{s2}\PY{p}{)}
\end{Verbatim}

    \begin{Verbatim}[commandchars=\\\{\}]
value1 = 3.14 value2 = 1
    \end{Verbatim}

    \begin{Verbatim}[commandchars=\\\{\}]
{\color{incolor}In [{\color{incolor}82}]:} \PY{n}{s3} \PY{o}{=} \PY{l+s}{\PYZsq{}}\PY{l+s}{value1 = \PYZob{}0\PYZcb{}, value2 = \PYZob{}1\PYZcb{}}\PY{l+s}{\PYZsq{}}\PY{o}{.}\PY{n}{format}\PY{p}{(}\PY{l+m+mf}{3.1415}\PY{p}{,} \PY{l+m+mf}{1.5}\PY{p}{)}
         
         \PY{k}{print}\PY{p}{(}\PY{n}{s3}\PY{p}{)}
\end{Verbatim}

    \begin{Verbatim}[commandchars=\\\{\}]
value1 = 3.1415, value2 = 1.5
    \end{Verbatim}

    \subsubsection{List}\label{list}

Lists are very similar to strings, except that each element can be of
any type.

The syntax for creating lists in Python is \texttt{{[}...{]}}:

    \begin{Verbatim}[commandchars=\\\{\}]
{\color{incolor}In [{\color{incolor}83}]:} \PY{n}{l} \PY{o}{=} \PY{p}{[}\PY{l+m+mi}{1}\PY{p}{,}\PY{l+m+mi}{2}\PY{p}{,}\PY{l+m+mi}{3}\PY{p}{,}\PY{l+m+mi}{4}\PY{p}{]}
         
         \PY{k}{print}\PY{p}{(}\PY{n+nb}{type}\PY{p}{(}\PY{n}{l}\PY{p}{)}\PY{p}{)}
         \PY{k}{print}\PY{p}{(}\PY{n}{l}\PY{p}{)}
\end{Verbatim}

    \begin{Verbatim}[commandchars=\\\{\}]
<type 'list'>
[1, 2, 3, 4]
    \end{Verbatim}

    \begin{Verbatim}[commandchars=\\\{\}]
{\color{incolor}In [{\color{incolor}84}]:} \PY{k}{print}\PY{p}{(}\PY{n}{l}\PY{p}{)}
         
         \PY{k}{print}\PY{p}{(}\PY{n}{l}\PY{p}{[}\PY{l+m+mi}{1}\PY{p}{:}\PY{l+m+mi}{3}\PY{p}{]}\PY{p}{)}
         
         \PY{k}{print}\PY{p}{(}\PY{n}{l}\PY{p}{[}\PY{p}{:}\PY{p}{:}\PY{l+m+mi}{2}\PY{p}{]}\PY{p}{)}
\end{Verbatim}

    \begin{Verbatim}[commandchars=\\\{\}]
[1, 2, 3, 4]
[2, 3]
[1, 3]
    \end{Verbatim}

    \begin{Verbatim}[commandchars=\\\{\}]
{\color{incolor}In [{\color{incolor}85}]:} \PY{n}{l}\PY{p}{[}\PY{l+m+mi}{0}\PY{p}{]}
\end{Verbatim}

            \begin{Verbatim}[commandchars=\\\{\}]
{\color{outcolor}Out[{\color{outcolor}85}]:} 1
\end{Verbatim}
        
    \begin{Verbatim}[commandchars=\\\{\}]
{\color{incolor}In [{\color{incolor}86}]:} \PY{n}{l} \PY{o}{=} \PY{p}{[}\PY{l+m+mi}{1}\PY{p}{,} \PY{l+s}{\PYZsq{}}\PY{l+s}{a}\PY{l+s}{\PYZsq{}}\PY{p}{,} \PY{l+m+mf}{1.0}\PY{p}{,} \PY{l+m+mi}{1}\PY{o}{\PYZhy{}}\PY{l+m+mi}{1j}\PY{p}{]}
         
         \PY{k}{print}\PY{p}{(}\PY{n}{l}\PY{p}{)}
\end{Verbatim}

    \begin{Verbatim}[commandchars=\\\{\}]
[1, 'a', 1.0, (1-1j)]
    \end{Verbatim}

    \begin{Verbatim}[commandchars=\\\{\}]
{\color{incolor}In [{\color{incolor}87}]:} \PY{n}{start} \PY{o}{=} \PY{l+m+mi}{10}
         \PY{n}{stop} \PY{o}{=} \PY{l+m+mi}{30}
         \PY{n}{step} \PY{o}{=} \PY{l+m+mi}{2}
         
         \PY{n+nb}{range}\PY{p}{(}\PY{n}{start}\PY{p}{,} \PY{n}{stop}\PY{p}{,} \PY{n}{step}\PY{p}{)}
\end{Verbatim}

            \begin{Verbatim}[commandchars=\\\{\}]
{\color{outcolor}Out[{\color{outcolor}87}]:} [10, 12, 14, 16, 18, 20, 22, 24, 26, 28]
\end{Verbatim}
        
    \begin{Verbatim}[commandchars=\\\{\}]
{\color{incolor}In [{\color{incolor}88}]:} \PY{n+nb}{list}\PY{p}{(}\PY{n+nb}{range}\PY{p}{(}\PY{o}{\PYZhy{}}\PY{l+m+mi}{10}\PY{p}{,} \PY{l+m+mi}{10}\PY{p}{)}\PY{p}{)}
\end{Verbatim}

            \begin{Verbatim}[commandchars=\\\{\}]
{\color{outcolor}Out[{\color{outcolor}88}]:} [-10, -9, -8, -7, -6, -5, -4, -3, -2, -1, 0, 1, 2, 3, 4, 5, 6, 7, 8, 9]
\end{Verbatim}
        
    \begin{Verbatim}[commandchars=\\\{\}]
{\color{incolor}In [{\color{incolor}89}]:} \PY{n}{s}
\end{Verbatim}

            \begin{Verbatim}[commandchars=\\\{\}]
{\color{outcolor}Out[{\color{outcolor}89}]:} 'Hello world'
\end{Verbatim}
        
    \begin{Verbatim}[commandchars=\\\{\}]
{\color{incolor}In [{\color{incolor}90}]:} \PY{c}{\PYZsh{} convert a string to a list:}
         
         \PY{n}{s2} \PY{o}{=} \PY{n+nb}{list}\PY{p}{(}\PY{n}{s}\PY{p}{)}
         
         \PY{n}{s2}
\end{Verbatim}

            \begin{Verbatim}[commandchars=\\\{\}]
{\color{outcolor}Out[{\color{outcolor}90}]:} ['H', 'e', 'l', 'l', 'o', ' ', 'w', 'o', 'r', 'l', 'd']
\end{Verbatim}
        
    \begin{Verbatim}[commandchars=\\\{\}]
{\color{incolor}In [{\color{incolor}91}]:} \PY{c}{\PYZsh{} sorting lists}
         \PY{n}{s2}\PY{o}{.}\PY{n}{sort}\PY{p}{(}\PY{p}{)}
         
         \PY{k}{print}\PY{p}{(}\PY{n}{s2}\PY{p}{)}
\end{Verbatim}

    \begin{Verbatim}[commandchars=\\\{\}]
[' ', 'H', 'd', 'e', 'l', 'l', 'l', 'o', 'o', 'r', 'w']
    \end{Verbatim}

    \paragraph{Adding, inserting, modifying, and removing elements from
lists}\label{adding-inserting-modifying-and-removing-elements-from-lists}

    \begin{Verbatim}[commandchars=\\\{\}]
{\color{incolor}In [{\color{incolor}94}]:} \PY{c}{\PYZsh{} create a new empty list}
         \PY{n}{l} \PY{o}{=} \PY{p}{[}\PY{p}{]}
         
         \PY{c}{\PYZsh{} add an elements using `append`}
         \PY{n}{l}\PY{o}{.}\PY{n}{append}\PY{p}{(}\PY{l+s}{\PYZdq{}}\PY{l+s}{A}\PY{l+s}{\PYZdq{}}\PY{p}{)}
         \PY{n}{l}\PY{o}{.}\PY{n}{append}\PY{p}{(}\PY{l+s}{\PYZdq{}}\PY{l+s}{d}\PY{l+s}{\PYZdq{}}\PY{p}{)}
         \PY{n}{l}\PY{o}{.}\PY{n}{append}\PY{p}{(}\PY{l+s}{\PYZdq{}}\PY{l+s}{d}\PY{l+s}{\PYZdq{}}\PY{p}{)}
         
         \PY{k}{print}\PY{p}{(}\PY{n}{l}\PY{p}{)}
\end{Verbatim}

    \begin{Verbatim}[commandchars=\\\{\}]
['A', 'd', 'd']
    \end{Verbatim}

    \begin{Verbatim}[commandchars=\\\{\}]
{\color{incolor}In [{\color{incolor}95}]:} \PY{n}{l}\PY{p}{[}\PY{l+m+mi}{1}\PY{p}{]} \PY{o}{=} \PY{l+s}{\PYZdq{}}\PY{l+s}{p}\PY{l+s}{\PYZdq{}}
         \PY{n}{l}\PY{p}{[}\PY{l+m+mi}{2}\PY{p}{]} \PY{o}{=} \PY{l+s}{\PYZdq{}}\PY{l+s}{p}\PY{l+s}{\PYZdq{}}
         
         \PY{k}{print}\PY{p}{(}\PY{n}{l}\PY{p}{)}
\end{Verbatim}

    \begin{Verbatim}[commandchars=\\\{\}]
['A', 'p', 'p']
    \end{Verbatim}

    \begin{Verbatim}[commandchars=\\\{\}]
{\color{incolor}In [{\color{incolor}96}]:} \PY{n}{l}\PY{p}{[}\PY{l+m+mi}{1}\PY{p}{:}\PY{l+m+mi}{3}\PY{p}{]} \PY{o}{=} \PY{p}{[}\PY{l+s}{\PYZdq{}}\PY{l+s}{d}\PY{l+s}{\PYZdq{}}\PY{p}{,} \PY{l+s}{\PYZdq{}}\PY{l+s}{d}\PY{l+s}{\PYZdq{}}\PY{p}{]}
         
         \PY{k}{print}\PY{p}{(}\PY{n}{l}\PY{p}{)}
\end{Verbatim}

    \begin{Verbatim}[commandchars=\\\{\}]
['A', 'd', 'd']
    \end{Verbatim}

    \begin{Verbatim}[commandchars=\\\{\}]
{\color{incolor}In [{\color{incolor}97}]:} \PY{n}{l}\PY{o}{.}\PY{n}{insert}\PY{p}{(}\PY{l+m+mi}{0}\PY{p}{,} \PY{l+s}{\PYZdq{}}\PY{l+s}{i}\PY{l+s}{\PYZdq{}}\PY{p}{)}
         \PY{n}{l}\PY{o}{.}\PY{n}{insert}\PY{p}{(}\PY{l+m+mi}{1}\PY{p}{,} \PY{l+s}{\PYZdq{}}\PY{l+s}{n}\PY{l+s}{\PYZdq{}}\PY{p}{)}
         \PY{n}{l}\PY{o}{.}\PY{n}{insert}\PY{p}{(}\PY{l+m+mi}{2}\PY{p}{,} \PY{l+s}{\PYZdq{}}\PY{l+s}{s}\PY{l+s}{\PYZdq{}}\PY{p}{)}
         \PY{n}{l}\PY{o}{.}\PY{n}{insert}\PY{p}{(}\PY{l+m+mi}{3}\PY{p}{,} \PY{l+s}{\PYZdq{}}\PY{l+s}{e}\PY{l+s}{\PYZdq{}}\PY{p}{)}
         \PY{n}{l}\PY{o}{.}\PY{n}{insert}\PY{p}{(}\PY{l+m+mi}{4}\PY{p}{,} \PY{l+s}{\PYZdq{}}\PY{l+s}{r}\PY{l+s}{\PYZdq{}}\PY{p}{)}
         \PY{n}{l}\PY{o}{.}\PY{n}{insert}\PY{p}{(}\PY{l+m+mi}{5}\PY{p}{,} \PY{l+s}{\PYZdq{}}\PY{l+s}{t}\PY{l+s}{\PYZdq{}}\PY{p}{)}
         
         \PY{k}{print}\PY{p}{(}\PY{n}{l}\PY{p}{)}
\end{Verbatim}

    \begin{Verbatim}[commandchars=\\\{\}]
['i', 'n', 's', 'e', 'r', 't', 'A', 'd', 'd']
    \end{Verbatim}

    \begin{Verbatim}[commandchars=\\\{\}]
{\color{incolor}In [{\color{incolor}98}]:} \PY{n}{l}\PY{o}{.}\PY{n}{remove}\PY{p}{(}\PY{l+s}{\PYZdq{}}\PY{l+s}{A}\PY{l+s}{\PYZdq{}}\PY{p}{)}
         
         \PY{k}{print}\PY{p}{(}\PY{n}{l}\PY{p}{)}
\end{Verbatim}

    \begin{Verbatim}[commandchars=\\\{\}]
['i', 'n', 's', 'e', 'r', 't', 'd', 'd']
    \end{Verbatim}

    \begin{Verbatim}[commandchars=\\\{\}]
{\color{incolor}In [{\color{incolor}99}]:} \PY{k}{del} \PY{n}{l}\PY{p}{[}\PY{l+m+mi}{7}\PY{p}{]}
         \PY{k}{del} \PY{n}{l}\PY{p}{[}\PY{l+m+mi}{6}\PY{p}{]}
         
         \PY{k}{print}\PY{p}{(}\PY{n}{l}\PY{p}{)}
\end{Verbatim}

    \begin{Verbatim}[commandchars=\\\{\}]
['i', 'n', 's', 'e', 'r', 't']
    \end{Verbatim}

    \subsubsection{Tuples}\label{tuples}

Tuples are like lists, except that they cannot be modified once created,
that is they are \emph{immutable}.

In Python, tuples are created using the syntax \texttt{(..., ..., ...)},
or even \texttt{..., ...}:

    \begin{Verbatim}[commandchars=\\\{\}]
{\color{incolor}In [{\color{incolor}100}]:} \PY{n}{point} \PY{o}{=} \PY{p}{(}\PY{l+m+mi}{10}\PY{p}{,} \PY{l+m+mi}{20}\PY{p}{)}
          
          \PY{k}{print}\PY{p}{(}\PY{n}{point}\PY{p}{,} \PY{n+nb}{type}\PY{p}{(}\PY{n}{point}\PY{p}{)}\PY{p}{)}
\end{Verbatim}

    \begin{Verbatim}[commandchars=\\\{\}]
((10, 20), <type 'tuple'>)
    \end{Verbatim}

    \begin{Verbatim}[commandchars=\\\{\}]
{\color{incolor}In [{\color{incolor}101}]:} \PY{n}{point} \PY{o}{=} \PY{l+m+mi}{10}\PY{p}{,} \PY{l+m+mi}{20}
          
          \PY{k}{print}\PY{p}{(}\PY{n}{point}\PY{p}{,} \PY{n+nb}{type}\PY{p}{(}\PY{n}{point}\PY{p}{)}\PY{p}{)}
\end{Verbatim}

    \begin{Verbatim}[commandchars=\\\{\}]
((10, 20), <type 'tuple'>)
    \end{Verbatim}

    We can unpack a tuple by assigning it to a comma-separated list of
variables:

    \begin{Verbatim}[commandchars=\\\{\}]
{\color{incolor}In [{\color{incolor}102}]:} \PY{n}{x}\PY{p}{,} \PY{n}{y} \PY{o}{=} \PY{n}{point}
          
          \PY{k}{print}\PY{p}{(}\PY{l+s}{\PYZdq{}}\PY{l+s}{x =}\PY{l+s}{\PYZdq{}}\PY{p}{,} \PY{n}{x}\PY{p}{)}
          \PY{k}{print}\PY{p}{(}\PY{l+s}{\PYZdq{}}\PY{l+s}{y =}\PY{l+s}{\PYZdq{}}\PY{p}{,} \PY{n}{y}\PY{p}{)}
\end{Verbatim}

    \begin{Verbatim}[commandchars=\\\{\}]
('x =', 10)
('y =', 20)
    \end{Verbatim}

    If we try to assign a new value to an element in a tuple we get an
error:

    \begin{Verbatim}[commandchars=\\\{\}]
{\color{incolor}In [{\color{incolor}103}]:} \PY{n}{point}\PY{p}{[}\PY{l+m+mi}{0}\PY{p}{]} \PY{o}{=} \PY{l+m+mi}{20}
\end{Verbatim}

    \begin{Verbatim}[commandchars=\\\{\}]

        ---------------------------------------------------------------------------
    TypeError                                 Traceback (most recent call last)

        <ipython-input-103-ac1c641a5dca> in <module>()
    ----> 1 point[0] = 20
    

        TypeError: 'tuple' object does not support item assignment

    \end{Verbatim}

    \subsubsection{Dictionaries}\label{dictionaries}

Dictionaries are also like lists, except that each element is a
key-value pair. The syntax for dictionaries is
\texttt{\{key1 : value1, ...\}}:

    \begin{Verbatim}[commandchars=\\\{\}]
{\color{incolor}In [{\color{incolor}104}]:} \PY{n}{params} \PY{o}{=} \PY{p}{\PYZob{}}\PY{l+s}{\PYZdq{}}\PY{l+s}{parameter1}\PY{l+s}{\PYZdq{}} \PY{p}{:} \PY{l+m+mf}{1.0}\PY{p}{,}
                    \PY{l+s}{\PYZdq{}}\PY{l+s}{parameter2}\PY{l+s}{\PYZdq{}} \PY{p}{:} \PY{l+m+mf}{2.0}\PY{p}{,}
                    \PY{l+s}{\PYZdq{}}\PY{l+s}{parameter3}\PY{l+s}{\PYZdq{}} \PY{p}{:} \PY{l+m+mf}{3.0}\PY{p}{,}\PY{p}{\PYZcb{}}
          
          \PY{k}{print}\PY{p}{(}\PY{n+nb}{type}\PY{p}{(}\PY{n}{params}\PY{p}{)}\PY{p}{)}
          \PY{k}{print}\PY{p}{(}\PY{n}{params}\PY{p}{)}
\end{Verbatim}

    \begin{Verbatim}[commandchars=\\\{\}]
<type 'dict'>
\{'parameter1': 1.0, 'parameter3': 3.0, 'parameter2': 2.0\}
    \end{Verbatim}

    \begin{Verbatim}[commandchars=\\\{\}]
{\color{incolor}In [{\color{incolor}105}]:} \PY{k}{print}\PY{p}{(}\PY{l+s}{\PYZdq{}}\PY{l+s}{parameter1 = }\PY{l+s}{\PYZdq{}} \PY{o}{+} \PY{n+nb}{str}\PY{p}{(}\PY{n}{params}\PY{p}{[}\PY{l+s}{\PYZdq{}}\PY{l+s}{parameter1}\PY{l+s}{\PYZdq{}}\PY{p}{]}\PY{p}{)}\PY{p}{)}
          \PY{k}{print}\PY{p}{(}\PY{l+s}{\PYZdq{}}\PY{l+s}{parameter2 = }\PY{l+s}{\PYZdq{}} \PY{o}{+} \PY{n+nb}{str}\PY{p}{(}\PY{n}{params}\PY{p}{[}\PY{l+s}{\PYZdq{}}\PY{l+s}{parameter2}\PY{l+s}{\PYZdq{}}\PY{p}{]}\PY{p}{)}\PY{p}{)}
          \PY{k}{print}\PY{p}{(}\PY{l+s}{\PYZdq{}}\PY{l+s}{parameter3 = }\PY{l+s}{\PYZdq{}} \PY{o}{+} \PY{n+nb}{str}\PY{p}{(}\PY{n}{params}\PY{p}{[}\PY{l+s}{\PYZdq{}}\PY{l+s}{parameter3}\PY{l+s}{\PYZdq{}}\PY{p}{]}\PY{p}{)}\PY{p}{)}
\end{Verbatim}

    \begin{Verbatim}[commandchars=\\\{\}]
parameter1 = 1.0
parameter2 = 2.0
parameter3 = 3.0
    \end{Verbatim}

    \begin{Verbatim}[commandchars=\\\{\}]
{\color{incolor}In [{\color{incolor}106}]:} \PY{n}{params}\PY{p}{[}\PY{l+s}{\PYZdq{}}\PY{l+s}{parameter1}\PY{l+s}{\PYZdq{}}\PY{p}{]} \PY{o}{=} \PY{l+s}{\PYZdq{}}\PY{l+s}{A}\PY{l+s}{\PYZdq{}}
          \PY{n}{params}\PY{p}{[}\PY{l+s}{\PYZdq{}}\PY{l+s}{parameter2}\PY{l+s}{\PYZdq{}}\PY{p}{]} \PY{o}{=} \PY{l+s}{\PYZdq{}}\PY{l+s}{B}\PY{l+s}{\PYZdq{}}
          
          \PY{c}{\PYZsh{} add a new entry}
          \PY{n}{params}\PY{p}{[}\PY{l+s}{\PYZdq{}}\PY{l+s}{parameter4}\PY{l+s}{\PYZdq{}}\PY{p}{]} \PY{o}{=} \PY{l+s}{\PYZdq{}}\PY{l+s}{D}\PY{l+s}{\PYZdq{}}
          
          \PY{k}{print}\PY{p}{(}\PY{l+s}{\PYZdq{}}\PY{l+s}{parameter1 = }\PY{l+s}{\PYZdq{}} \PY{o}{+} \PY{n+nb}{str}\PY{p}{(}\PY{n}{params}\PY{p}{[}\PY{l+s}{\PYZdq{}}\PY{l+s}{parameter1}\PY{l+s}{\PYZdq{}}\PY{p}{]}\PY{p}{)}\PY{p}{)}
          \PY{k}{print}\PY{p}{(}\PY{l+s}{\PYZdq{}}\PY{l+s}{parameter2 = }\PY{l+s}{\PYZdq{}} \PY{o}{+} \PY{n+nb}{str}\PY{p}{(}\PY{n}{params}\PY{p}{[}\PY{l+s}{\PYZdq{}}\PY{l+s}{parameter2}\PY{l+s}{\PYZdq{}}\PY{p}{]}\PY{p}{)}\PY{p}{)}
          \PY{k}{print}\PY{p}{(}\PY{l+s}{\PYZdq{}}\PY{l+s}{parameter3 = }\PY{l+s}{\PYZdq{}} \PY{o}{+} \PY{n+nb}{str}\PY{p}{(}\PY{n}{params}\PY{p}{[}\PY{l+s}{\PYZdq{}}\PY{l+s}{parameter3}\PY{l+s}{\PYZdq{}}\PY{p}{]}\PY{p}{)}\PY{p}{)}
          \PY{k}{print}\PY{p}{(}\PY{l+s}{\PYZdq{}}\PY{l+s}{parameter4 = }\PY{l+s}{\PYZdq{}} \PY{o}{+} \PY{n+nb}{str}\PY{p}{(}\PY{n}{params}\PY{p}{[}\PY{l+s}{\PYZdq{}}\PY{l+s}{parameter4}\PY{l+s}{\PYZdq{}}\PY{p}{]}\PY{p}{)}\PY{p}{)}
\end{Verbatim}

    \begin{Verbatim}[commandchars=\\\{\}]
parameter1 = A
parameter2 = B
parameter3 = 3.0
parameter4 = D
    \end{Verbatim}

    

    \subsection{Control Flow}\label{control-flow}

    \subsubsection{Conditional statements: if, elif,
else}\label{conditional-statements-if-elif-else}

The Python syntax for conditional execution of code use the keywords
\texttt{if}, \texttt{elif} (else if), \texttt{else}:

    \begin{Verbatim}[commandchars=\\\{\}]
{\color{incolor}In [{\color{incolor}107}]:} \PY{n}{statement1} \PY{o}{=} \PY{n+nb+bp}{False}
          \PY{n}{statement2} \PY{o}{=} \PY{n+nb+bp}{False}
          
          \PY{k}{if} \PY{n}{statement1}\PY{p}{:}
              \PY{k}{print}\PY{p}{(}\PY{l+s}{\PYZdq{}}\PY{l+s}{statement1 is True}\PY{l+s}{\PYZdq{}}\PY{p}{)}
              
          \PY{k}{elif} \PY{n}{statement2}\PY{p}{:}
              \PY{k}{print}\PY{p}{(}\PY{l+s}{\PYZdq{}}\PY{l+s}{statement2 is True}\PY{l+s}{\PYZdq{}}\PY{p}{)}
              
          \PY{k}{else}\PY{p}{:}
              \PY{k}{print}\PY{p}{(}\PY{l+s}{\PYZdq{}}\PY{l+s}{statement1 and statement2 are False}\PY{l+s}{\PYZdq{}}\PY{p}{)}
\end{Verbatim}

    \begin{Verbatim}[commandchars=\\\{\}]
statement1 and statement2 are False
    \end{Verbatim}

    In Python, the extent of a code block is defined by the indentation
level (usually a tab or say four white spaces). This means that we have
to be careful to indent our code correctly, or else we will get syntax
errors.

\textbf{Examples:}

    \begin{Verbatim}[commandchars=\\\{\}]
{\color{incolor}In [{\color{incolor}108}]:} \PY{n}{statement1} \PY{o}{=} \PY{n}{statement2} \PY{o}{=} \PY{n+nb+bp}{True}
          
          \PY{k}{if} \PY{n}{statement1}\PY{p}{:}
              \PY{k}{if} \PY{n}{statement2}\PY{p}{:}
                  \PY{k}{print}\PY{p}{(}\PY{l+s}{\PYZdq{}}\PY{l+s}{both statement1 and statement2 are True}\PY{l+s}{\PYZdq{}}\PY{p}{)}
\end{Verbatim}

    \begin{Verbatim}[commandchars=\\\{\}]
both statement1 and statement2 are True
    \end{Verbatim}

    \begin{Verbatim}[commandchars=\\\{\}]
{\color{incolor}In [{\color{incolor}109}]:} \PY{c}{\PYZsh{} Bad indentation!}
          \PY{k}{if} \PY{n}{statement1}\PY{p}{:}
              \PY{k}{if} \PY{n}{statement2}\PY{p}{:}
                  \PY{k}{print}\PY{p}{(}\PY{l+s}{\PYZdq{}}\PY{l+s}{both statement1 and statement2 are True}\PY{l+s}{\PYZdq{}}\PY{p}{)}  \PY{c}{\PYZsh{} this line is not properly indented}
\end{Verbatim}

    \begin{Verbatim}[commandchars=\\\{\}]
both statement1 and statement2 are True
    \end{Verbatim}

    \begin{Verbatim}[commandchars=\\\{\}]
{\color{incolor}In [{\color{incolor}110}]:} \PY{n}{statement1} \PY{o}{=} \PY{n+nb+bp}{False} 
          
          \PY{k}{if} \PY{n}{statement1}\PY{p}{:}
              \PY{k}{print}\PY{p}{(}\PY{l+s}{\PYZdq{}}\PY{l+s}{printed if statement1 is True}\PY{l+s}{\PYZdq{}}\PY{p}{)}
              
              \PY{k}{print}\PY{p}{(}\PY{l+s}{\PYZdq{}}\PY{l+s}{still inside the if block}\PY{l+s}{\PYZdq{}}\PY{p}{)}
\end{Verbatim}

    \begin{Verbatim}[commandchars=\\\{\}]
{\color{incolor}In [{\color{incolor}111}]:} \PY{k}{if} \PY{n}{statement1}\PY{p}{:}
              \PY{k}{print}\PY{p}{(}\PY{l+s}{\PYZdq{}}\PY{l+s}{printed if statement1 is True}\PY{l+s}{\PYZdq{}}\PY{p}{)}
              
          \PY{k}{print}\PY{p}{(}\PY{l+s}{\PYZdq{}}\PY{l+s}{now outside the if block}\PY{l+s}{\PYZdq{}}\PY{p}{)}
\end{Verbatim}

    \begin{Verbatim}[commandchars=\\\{\}]
now outside the if block
    \end{Verbatim}

    

    \subsection{Loops}\label{loops}

In Python, loops can be programmed in a number of different ways. The
most common is the \texttt{for} loop, which is used together with
iterable objects, such as lists. The basic syntax is:

\textbf{\texttt{for} loops}:

    \begin{Verbatim}[commandchars=\\\{\}]
{\color{incolor}In [{\color{incolor}112}]:} \PY{k}{for} \PY{n}{x} \PY{o+ow}{in} \PY{p}{[}\PY{l+m+mi}{1}\PY{p}{,}\PY{l+m+mi}{2}\PY{p}{,}\PY{l+m+mi}{3}\PY{p}{]}\PY{p}{:}
              \PY{k}{print}\PY{p}{(}\PY{n}{x}\PY{p}{)}
\end{Verbatim}

    \begin{Verbatim}[commandchars=\\\{\}]
1
2
3
    \end{Verbatim}

    The \texttt{for} loop iterates over the elements of the supplied list,
and executes the containing block once for each element. Any kind of
list can be used in the \texttt{for} loop. For example:

    \begin{Verbatim}[commandchars=\\\{\}]
{\color{incolor}In [{\color{incolor}113}]:} \PY{k}{for} \PY{n}{x} \PY{o+ow}{in} \PY{n+nb}{range}\PY{p}{(}\PY{l+m+mi}{4}\PY{p}{)}\PY{p}{:} \PY{c}{\PYZsh{} by default range start at 0}
              \PY{k}{print}\PY{p}{(}\PY{n}{x}\PY{p}{)}
\end{Verbatim}

    \begin{Verbatim}[commandchars=\\\{\}]
0
1
2
3
    \end{Verbatim}

    Note: \texttt{range(4)} does not include 4 !

    \begin{Verbatim}[commandchars=\\\{\}]
{\color{incolor}In [{\color{incolor}114}]:} \PY{k}{for} \PY{n}{x} \PY{o+ow}{in} \PY{n+nb}{range}\PY{p}{(}\PY{o}{\PYZhy{}}\PY{l+m+mi}{3}\PY{p}{,}\PY{l+m+mi}{3}\PY{p}{)}\PY{p}{:}
              \PY{k}{print}\PY{p}{(}\PY{n}{x}\PY{p}{)}
\end{Verbatim}

    \begin{Verbatim}[commandchars=\\\{\}]
-3
-2
-1
0
1
2
    \end{Verbatim}

    \begin{Verbatim}[commandchars=\\\{\}]
{\color{incolor}In [{\color{incolor}115}]:} \PY{k}{for} \PY{n}{word} \PY{o+ow}{in} \PY{p}{[}\PY{l+s}{\PYZdq{}}\PY{l+s}{scientific}\PY{l+s}{\PYZdq{}}\PY{p}{,} \PY{l+s}{\PYZdq{}}\PY{l+s}{computing}\PY{l+s}{\PYZdq{}}\PY{p}{,} \PY{l+s}{\PYZdq{}}\PY{l+s}{with}\PY{l+s}{\PYZdq{}}\PY{p}{,} \PY{l+s}{\PYZdq{}}\PY{l+s}{python}\PY{l+s}{\PYZdq{}}\PY{p}{]}\PY{p}{:}
              \PY{k}{print}\PY{p}{(}\PY{n}{word}\PY{p}{)}
\end{Verbatim}

    \begin{Verbatim}[commandchars=\\\{\}]
scientific
computing
with
python
    \end{Verbatim}

    To iterate over key-value pairs of a dictionary:

    \begin{Verbatim}[commandchars=\\\{\}]
{\color{incolor}In [{\color{incolor}116}]:} \PY{k}{for} \PY{n}{key}\PY{p}{,} \PY{n}{value} \PY{o+ow}{in} \PY{n}{params}\PY{o}{.}\PY{n}{items}\PY{p}{(}\PY{p}{)}\PY{p}{:}
              \PY{k}{print}\PY{p}{(}\PY{n}{key} \PY{o}{+} \PY{l+s}{\PYZdq{}}\PY{l+s}{ = }\PY{l+s}{\PYZdq{}} \PY{o}{+} \PY{n+nb}{str}\PY{p}{(}\PY{n}{value}\PY{p}{)}\PY{p}{)}
\end{Verbatim}

    \begin{Verbatim}[commandchars=\\\{\}]
parameter4 = D
parameter1 = A
parameter3 = 3.0
parameter2 = B
    \end{Verbatim}

    Sometimes it is useful to have access to the indices of the values when
iterating over a list. We can use the \texttt{enumerate} function for
this:

    \begin{Verbatim}[commandchars=\\\{\}]
{\color{incolor}In [{\color{incolor}117}]:} \PY{k}{for} \PY{n}{idx}\PY{p}{,} \PY{n}{x} \PY{o+ow}{in} \PY{n+nb}{enumerate}\PY{p}{(}\PY{n+nb}{range}\PY{p}{(}\PY{o}{\PYZhy{}}\PY{l+m+mi}{3}\PY{p}{,}\PY{l+m+mi}{3}\PY{p}{)}\PY{p}{)}\PY{p}{:}
              \PY{k}{print}\PY{p}{(}\PY{n}{idx}\PY{p}{,} \PY{n}{x}\PY{p}{)}
\end{Verbatim}

    \begin{Verbatim}[commandchars=\\\{\}]
(0, -3)
(1, -2)
(2, -1)
(3, 0)
(4, 1)
(5, 2)
    \end{Verbatim}

    \textbf{List comprehensions: Creating lists using \texttt{for} loops}:

    \begin{Verbatim}[commandchars=\\\{\}]
{\color{incolor}In [{\color{incolor}118}]:} \PY{n}{l1} \PY{o}{=} \PY{p}{[}\PY{n}{x}\PY{o}{*}\PY{o}{*}\PY{l+m+mi}{2} \PY{k}{for} \PY{n}{x} \PY{o+ow}{in} \PY{n+nb}{range}\PY{p}{(}\PY{l+m+mi}{0}\PY{p}{,}\PY{l+m+mi}{5}\PY{p}{)}\PY{p}{]}
          
          \PY{k}{print}\PY{p}{(}\PY{n}{l1}\PY{p}{)}
\end{Verbatim}

    \begin{Verbatim}[commandchars=\\\{\}]
[0, 1, 4, 9, 16]
    \end{Verbatim}

    \textbf{\texttt{while} loops}:

    \begin{Verbatim}[commandchars=\\\{\}]
{\color{incolor}In [{\color{incolor}119}]:} \PY{n}{i} \PY{o}{=} \PY{l+m+mi}{0}
          
          \PY{k}{while} \PY{n}{i} \PY{o}{\PYZlt{}} \PY{l+m+mi}{5}\PY{p}{:}
              \PY{k}{print}\PY{p}{(}\PY{n}{i}\PY{p}{)}
              
              \PY{n}{i} \PY{o}{=} \PY{n}{i} \PY{o}{+} \PY{l+m+mi}{1}
              
          \PY{k}{print}\PY{p}{(}\PY{l+s}{\PYZdq{}}\PY{l+s}{done}\PY{l+s}{\PYZdq{}}\PY{p}{)}
\end{Verbatim}

    \begin{Verbatim}[commandchars=\\\{\}]
0
1
2
3
4
done
    \end{Verbatim}

    

    \subsection{Functions}\label{functions}

A function in Python is defined using the keyword \texttt{def}, followed
by a function name, a signature within parenthises \texttt{()}, and a
colon \texttt{:}. The following code, with one additional level of
indentation, is the function body.

    \begin{Verbatim}[commandchars=\\\{\}]
{\color{incolor}In [{\color{incolor}120}]:} \PY{k}{def} \PY{n+nf}{func0}\PY{p}{(}\PY{p}{)}\PY{p}{:}   
              \PY{k}{print}\PY{p}{(}\PY{l+s}{\PYZdq{}}\PY{l+s}{test}\PY{l+s}{\PYZdq{}}\PY{p}{)}
\end{Verbatim}

    \begin{Verbatim}[commandchars=\\\{\}]
{\color{incolor}In [{\color{incolor}121}]:} \PY{n}{func0}\PY{p}{(}\PY{p}{)}
\end{Verbatim}

    \begin{Verbatim}[commandchars=\\\{\}]
test
    \end{Verbatim}

    Optionally, but highly recommended, we can define a so called
``docstring'', which is a description of the functions purpose and
behavior. The docstring should be located after the function definition
and before the code in the function body.

    \begin{Verbatim}[commandchars=\\\{\}]
{\color{incolor}In [{\color{incolor}122}]:} \PY{k}{def} \PY{n+nf}{func1}\PY{p}{(}\PY{n}{s}\PY{p}{)}\PY{p}{:}
              \PY{l+s+sd}{\PYZdq{}\PYZdq{}\PYZdq{}}
          \PY{l+s+sd}{    Print a string \PYZsq{}s\PYZsq{} and tell how many characters it has    }
          \PY{l+s+sd}{    \PYZdq{}\PYZdq{}\PYZdq{}}
              
              \PY{k}{print}\PY{p}{(}\PY{n}{s} \PY{o}{+} \PY{l+s}{\PYZdq{}}\PY{l+s}{ has }\PY{l+s}{\PYZdq{}} \PY{o}{+} \PY{n+nb}{str}\PY{p}{(}\PY{n+nb}{len}\PY{p}{(}\PY{n}{s}\PY{p}{)}\PY{p}{)} \PY{o}{+} \PY{l+s}{\PYZdq{}}\PY{l+s}{ characters}\PY{l+s}{\PYZdq{}}\PY{p}{)}
\end{Verbatim}

    \begin{Verbatim}[commandchars=\\\{\}]
{\color{incolor}In [{\color{incolor}123}]:} \PY{n}{help}\PY{p}{(}\PY{n}{func1}\PY{p}{)}
\end{Verbatim}

    \begin{Verbatim}[commandchars=\\\{\}]
Help on function func1 in module \_\_main\_\_:

func1(s)
    Print a string 's' and tell how many characters it has
    \end{Verbatim}

    \begin{Verbatim}[commandchars=\\\{\}]
{\color{incolor}In [{\color{incolor}124}]:} \PY{n}{func1}\PY{p}{(}\PY{l+s}{\PYZdq{}}\PY{l+s}{test}\PY{l+s}{\PYZdq{}}\PY{p}{)}
\end{Verbatim}

    \begin{Verbatim}[commandchars=\\\{\}]
test has 4 characters
    \end{Verbatim}

    Functions that returns a value use the \texttt{return} keyword:

    \begin{Verbatim}[commandchars=\\\{\}]
{\color{incolor}In [{\color{incolor}125}]:} \PY{k}{def} \PY{n+nf}{square}\PY{p}{(}\PY{n}{x}\PY{p}{)}\PY{p}{:}
              \PY{l+s+sd}{\PYZdq{}\PYZdq{}\PYZdq{}}
          \PY{l+s+sd}{    Return the square of x.}
          \PY{l+s+sd}{    \PYZdq{}\PYZdq{}\PYZdq{}}
              \PY{k}{return} \PY{n}{x} \PY{o}{*}\PY{o}{*} \PY{l+m+mi}{2}
\end{Verbatim}

    \begin{Verbatim}[commandchars=\\\{\}]
{\color{incolor}In [{\color{incolor}126}]:} \PY{n}{square}\PY{p}{(}\PY{l+m+mi}{4}\PY{p}{)}
\end{Verbatim}

            \begin{Verbatim}[commandchars=\\\{\}]
{\color{outcolor}Out[{\color{outcolor}126}]:} 16
\end{Verbatim}
        
    We can return multiple values from a function using tuples (see above):

    \begin{Verbatim}[commandchars=\\\{\}]
{\color{incolor}In [{\color{incolor}127}]:} \PY{k}{def} \PY{n+nf}{powers}\PY{p}{(}\PY{n}{x}\PY{p}{)}\PY{p}{:}
              \PY{l+s+sd}{\PYZdq{}\PYZdq{}\PYZdq{}}
          \PY{l+s+sd}{    Return a few powers of x.}
          \PY{l+s+sd}{    \PYZdq{}\PYZdq{}\PYZdq{}}
              \PY{k}{return} \PY{n}{x} \PY{o}{*}\PY{o}{*} \PY{l+m+mi}{2}\PY{p}{,} \PY{n}{x} \PY{o}{*}\PY{o}{*} \PY{l+m+mi}{3}\PY{p}{,} \PY{n}{x} \PY{o}{*}\PY{o}{*} \PY{l+m+mi}{4}
\end{Verbatim}

    \begin{Verbatim}[commandchars=\\\{\}]
{\color{incolor}In [{\color{incolor}128}]:} \PY{n}{powers}\PY{p}{(}\PY{l+m+mi}{3}\PY{p}{)}
\end{Verbatim}

            \begin{Verbatim}[commandchars=\\\{\}]
{\color{outcolor}Out[{\color{outcolor}128}]:} (9, 27, 81)
\end{Verbatim}
        
    \begin{Verbatim}[commandchars=\\\{\}]
{\color{incolor}In [{\color{incolor}129}]:} \PY{n}{x2}\PY{p}{,} \PY{n}{x3}\PY{p}{,} \PY{n}{x4} \PY{o}{=} \PY{n}{powers}\PY{p}{(}\PY{l+m+mi}{3}\PY{p}{)}
          
          \PY{k}{print}\PY{p}{(}\PY{n}{x3}\PY{p}{)}
\end{Verbatim}

    \begin{Verbatim}[commandchars=\\\{\}]
27
    \end{Verbatim}

    \subsubsection{Default argument and keyword
arguments}\label{default-argument-and-keyword-arguments}

    \begin{Verbatim}[commandchars=\\\{\}]
{\color{incolor}In [{\color{incolor}130}]:} \PY{k}{def} \PY{n+nf}{myfunc}\PY{p}{(}\PY{n}{x}\PY{p}{,} \PY{n}{p}\PY{o}{=}\PY{l+m+mi}{2}\PY{p}{,} \PY{n}{debug}\PY{o}{=}\PY{n+nb+bp}{False}\PY{p}{)}\PY{p}{:}
              \PY{k}{if} \PY{n}{debug}\PY{p}{:}
                  \PY{k}{print}\PY{p}{(}\PY{l+s}{\PYZdq{}}\PY{l+s}{evaluating myfunc for x = }\PY{l+s}{\PYZdq{}} \PY{o}{+} \PY{n+nb}{str}\PY{p}{(}\PY{n}{x}\PY{p}{)} \PY{o}{+} \PY{l+s}{\PYZdq{}}\PY{l+s}{ using exponent p = }\PY{l+s}{\PYZdq{}} \PY{o}{+} \PY{n+nb}{str}\PY{p}{(}\PY{n}{p}\PY{p}{)}\PY{p}{)}
              \PY{k}{return} \PY{n}{x}\PY{o}{*}\PY{o}{*}\PY{n}{p}
\end{Verbatim}

    \begin{Verbatim}[commandchars=\\\{\}]
{\color{incolor}In [{\color{incolor}131}]:} \PY{n}{myfunc}\PY{p}{(}\PY{l+m+mi}{5}\PY{p}{)}
\end{Verbatim}

            \begin{Verbatim}[commandchars=\\\{\}]
{\color{outcolor}Out[{\color{outcolor}131}]:} 25
\end{Verbatim}
        
    \begin{Verbatim}[commandchars=\\\{\}]
{\color{incolor}In [{\color{incolor}132}]:} \PY{n}{myfunc}\PY{p}{(}\PY{l+m+mi}{5}\PY{p}{,} \PY{n}{debug}\PY{o}{=}\PY{n+nb+bp}{True}\PY{p}{)}
\end{Verbatim}

    \begin{Verbatim}[commandchars=\\\{\}]
evaluating myfunc for x = 5 using exponent p = 2
    \end{Verbatim}

            \begin{Verbatim}[commandchars=\\\{\}]
{\color{outcolor}Out[{\color{outcolor}132}]:} 25
\end{Verbatim}
        
    \begin{Verbatim}[commandchars=\\\{\}]
{\color{incolor}In [{\color{incolor}133}]:} \PY{n}{myfunc}\PY{p}{(}\PY{n}{p}\PY{o}{=}\PY{l+m+mi}{3}\PY{p}{,} \PY{n}{debug}\PY{o}{=}\PY{n+nb+bp}{True}\PY{p}{,} \PY{n}{x}\PY{o}{=}\PY{l+m+mi}{7}\PY{p}{)}
\end{Verbatim}

    \begin{Verbatim}[commandchars=\\\{\}]
evaluating myfunc for x = 7 using exponent p = 3
    \end{Verbatim}

            \begin{Verbatim}[commandchars=\\\{\}]
{\color{outcolor}Out[{\color{outcolor}133}]:} 343
\end{Verbatim}
        
    \subsubsection{Unnamed functions (lambda
function)}\label{unnamed-functions-lambda-function}

In Python we can also create unnamed functions, using the
\texttt{lambda} keyword:

    \begin{Verbatim}[commandchars=\\\{\}]
{\color{incolor}In [{\color{incolor}134}]:} \PY{n}{f1} \PY{o}{=} \PY{k}{lambda} \PY{n}{x}\PY{p}{:} \PY{n}{x}\PY{o}{*}\PY{o}{*}\PY{l+m+mi}{2}
              
          \PY{c}{\PYZsh{} is equivalent to }
          
          \PY{k}{def} \PY{n+nf}{f2}\PY{p}{(}\PY{n}{x}\PY{p}{)}\PY{p}{:}
              \PY{k}{return} \PY{n}{x}\PY{o}{*}\PY{o}{*}\PY{l+m+mi}{2}
\end{Verbatim}

    \begin{Verbatim}[commandchars=\\\{\}]
{\color{incolor}In [{\color{incolor}135}]:} \PY{n}{f1}\PY{p}{(}\PY{l+m+mi}{2}\PY{p}{)}\PY{p}{,} \PY{n}{f2}\PY{p}{(}\PY{l+m+mi}{2}\PY{p}{)}
\end{Verbatim}

            \begin{Verbatim}[commandchars=\\\{\}]
{\color{outcolor}Out[{\color{outcolor}135}]:} (4, 4)
\end{Verbatim}
        
    This technique is useful for exmample when we want to pass a simple
function as an argument to another function, like this:

    \begin{Verbatim}[commandchars=\\\{\}]
{\color{incolor}In [{\color{incolor}136}]:} \PY{c}{\PYZsh{} map is a built\PYZhy{}in python function}
          \PY{n+nb}{map}\PY{p}{(}\PY{k}{lambda} \PY{n}{x}\PY{p}{:} \PY{n}{x}\PY{o}{*}\PY{o}{*}\PY{l+m+mi}{2}\PY{p}{,} \PY{n+nb}{range}\PY{p}{(}\PY{o}{\PYZhy{}}\PY{l+m+mi}{3}\PY{p}{,}\PY{l+m+mi}{4}\PY{p}{)}\PY{p}{)}
\end{Verbatim}

            \begin{Verbatim}[commandchars=\\\{\}]
{\color{outcolor}Out[{\color{outcolor}136}]:} [9, 4, 1, 0, 1, 4, 9]
\end{Verbatim}
        
    

    \subsection{Classes}\label{classes}

Classes are the key features of object-oriented programming. A class is
a structure for representing an object and the operations that can be
performed on the object.

In Python a class can contain \emph{attributes} (variables) and
\emph{methods} (functions).

In python a class is defined almost like a function, but using the
\texttt{class} keyword, and the class definition usually contains a
number of class method definitions (a function in a class).

\begin{itemize}
\item
  Each class method should have an argument \texttt{self} as it first
  argument. This object is a self-reference.
\item
  Some class method names have special meaning, for example:
\item
  \texttt{\_\_init\_\_}: The name of the method that is invoked when the
  object is first created.
\item
  \texttt{\_\_str\_\_} : A method that is invoked when a simple string
  representation of the class is needed, as for example when printed.
\end{itemize}

    \begin{Verbatim}[commandchars=\\\{\}]
{\color{incolor}In [{\color{incolor}137}]:} \PY{k}{class} \PY{n+nc}{Point}\PY{p}{:}
              \PY{l+s+sd}{\PYZdq{}\PYZdq{}\PYZdq{}}
          \PY{l+s+sd}{    Simple class for representing a point in a Cartesian coordinate system.}
          \PY{l+s+sd}{    \PYZdq{}\PYZdq{}\PYZdq{}}
              
              \PY{k}{def} \PY{n+nf}{\PYZus{}\PYZus{}init\PYZus{}\PYZus{}}\PY{p}{(}\PY{n+nb+bp}{self}\PY{p}{,} \PY{n}{x}\PY{p}{,} \PY{n}{y}\PY{p}{)}\PY{p}{:}
                  \PY{l+s+sd}{\PYZdq{}\PYZdq{}\PYZdq{}}
          \PY{l+s+sd}{        Create a new Point at x, y.}
          \PY{l+s+sd}{        \PYZdq{}\PYZdq{}\PYZdq{}}
                  \PY{n+nb+bp}{self}\PY{o}{.}\PY{n}{x} \PY{o}{=} \PY{n}{x}
                  \PY{n+nb+bp}{self}\PY{o}{.}\PY{n}{y} \PY{o}{=} \PY{n}{y}
                  
              \PY{k}{def} \PY{n+nf}{translate}\PY{p}{(}\PY{n+nb+bp}{self}\PY{p}{,} \PY{n}{dx}\PY{p}{,} \PY{n}{dy}\PY{p}{)}\PY{p}{:}
                  \PY{l+s+sd}{\PYZdq{}\PYZdq{}\PYZdq{}}
          \PY{l+s+sd}{        Translate the point by dx and dy in the x and y direction.}
          \PY{l+s+sd}{        \PYZdq{}\PYZdq{}\PYZdq{}}
                  \PY{n+nb+bp}{self}\PY{o}{.}\PY{n}{x} \PY{o}{+}\PY{o}{=} \PY{n}{dx}
                  \PY{n+nb+bp}{self}\PY{o}{.}\PY{n}{y} \PY{o}{+}\PY{o}{=} \PY{n}{dy}
                  
              \PY{k}{def} \PY{n+nf}{\PYZus{}\PYZus{}str\PYZus{}\PYZus{}}\PY{p}{(}\PY{n+nb+bp}{self}\PY{p}{)}\PY{p}{:}
                  \PY{k}{return}\PY{p}{(}\PY{l+s}{\PYZdq{}}\PY{l+s}{Point at [}\PY{l+s+si}{\PYZpc{}f}\PY{l+s}{, }\PY{l+s+si}{\PYZpc{}f}\PY{l+s}{]}\PY{l+s}{\PYZdq{}} \PY{o}{\PYZpc{}} \PY{p}{(}\PY{n+nb+bp}{self}\PY{o}{.}\PY{n}{x}\PY{p}{,} \PY{n+nb+bp}{self}\PY{o}{.}\PY{n}{y}\PY{p}{)}\PY{p}{)}
\end{Verbatim}

    To create a new instance of a class:

    \begin{Verbatim}[commandchars=\\\{\}]
{\color{incolor}In [{\color{incolor}138}]:} \PY{n}{p1} \PY{o}{=} \PY{n}{Point}\PY{p}{(}\PY{l+m+mi}{0}\PY{p}{,} \PY{l+m+mi}{0}\PY{p}{)} \PY{c}{\PYZsh{} this will invoke the \PYZus{}\PYZus{}init\PYZus{}\PYZus{} method in the Point class}
          
          \PY{k}{print}\PY{p}{(}\PY{n}{p1}\PY{p}{)}         \PY{c}{\PYZsh{} this will invoke the \PYZus{}\PYZus{}str\PYZus{}\PYZus{} method}
\end{Verbatim}

    \begin{Verbatim}[commandchars=\\\{\}]
Point at [0.000000, 0.000000]
    \end{Verbatim}

    \begin{Verbatim}[commandchars=\\\{\}]
{\color{incolor}In [{\color{incolor}139}]:} \PY{n}{p2} \PY{o}{=} \PY{n}{Point}\PY{p}{(}\PY{l+m+mi}{1}\PY{p}{,} \PY{l+m+mi}{1}\PY{p}{)}
          
          \PY{n}{p1}\PY{o}{.}\PY{n}{translate}\PY{p}{(}\PY{l+m+mf}{0.25}\PY{p}{,} \PY{l+m+mf}{1.5}\PY{p}{)}
          
          \PY{k}{print}\PY{p}{(}\PY{n}{p1}\PY{p}{)}
          \PY{k}{print}\PY{p}{(}\PY{n}{p2}\PY{p}{)}
\end{Verbatim}

    \begin{Verbatim}[commandchars=\\\{\}]
Point at [0.250000, 1.500000]
Point at [1.000000, 1.000000]
    \end{Verbatim}

    \subsection{Modules}\label{modules}

One of the most important concepts in good programming is to reuse code
and avoid repetitions.

Consider the following example: the file \texttt{mymodule.py} contains
simple example implementations of a variable, function and a class:

    \begin{Verbatim}[commandchars=\\\{\}]
{\color{incolor}In [{\color{incolor}140}]:} \PY{o}{\PYZpc{}\PYZpc{}}\PY{k}{file} \PY{n}{mymodule}\PY{o}{.}\PY{n}{py}
          \PY{l+s+sd}{\PYZdq{}\PYZdq{}\PYZdq{}}
          \PY{l+s+sd}{Example of a python module. Contains a variable called my\PYZus{}variable,}
          \PY{l+s+sd}{a function called my\PYZus{}function, and a class called MyClass.}
          \PY{l+s+sd}{\PYZdq{}\PYZdq{}\PYZdq{}}
          
          \PY{n}{my\PYZus{}variable} \PY{o}{=} \PY{l+m+mi}{0}
          
          \PY{k}{def} \PY{n+nf}{my\PYZus{}function}\PY{p}{(}\PY{p}{)}\PY{p}{:}
              \PY{l+s+sd}{\PYZdq{}\PYZdq{}\PYZdq{}}
          \PY{l+s+sd}{    Example function}
          \PY{l+s+sd}{    \PYZdq{}\PYZdq{}\PYZdq{}}
              \PY{k}{return} \PY{n}{my\PYZus{}variable}
              
          \PY{k}{class} \PY{n+nc}{MyClass}\PY{p}{:}
              \PY{l+s+sd}{\PYZdq{}\PYZdq{}\PYZdq{}}
          \PY{l+s+sd}{    Example class.}
          \PY{l+s+sd}{    \PYZdq{}\PYZdq{}\PYZdq{}}
          
              \PY{k}{def} \PY{n+nf}{\PYZus{}\PYZus{}init\PYZus{}\PYZus{}}\PY{p}{(}\PY{n+nb+bp}{self}\PY{p}{)}\PY{p}{:}
                  \PY{n+nb+bp}{self}\PY{o}{.}\PY{n}{variable} \PY{o}{=} \PY{n}{my\PYZus{}variable}
                  
              \PY{k}{def} \PY{n+nf}{set\PYZus{}variable}\PY{p}{(}\PY{n+nb+bp}{self}\PY{p}{,} \PY{n}{new\PYZus{}value}\PY{p}{)}\PY{p}{:}
                  \PY{l+s+sd}{\PYZdq{}\PYZdq{}\PYZdq{}}
          \PY{l+s+sd}{        Set self.variable to a new value}
          \PY{l+s+sd}{        \PYZdq{}\PYZdq{}\PYZdq{}}
                  \PY{n+nb+bp}{self}\PY{o}{.}\PY{n}{variable} \PY{o}{=} \PY{n}{new\PYZus{}value}
                  
              \PY{k}{def} \PY{n+nf}{get\PYZus{}variable}\PY{p}{(}\PY{n+nb+bp}{self}\PY{p}{)}\PY{p}{:}
                  \PY{k}{return} \PY{n+nb+bp}{self}\PY{o}{.}\PY{n}{variable}
\end{Verbatim}

    \begin{Verbatim}[commandchars=\\\{\}]
Writing mymodule.py
    \end{Verbatim}

    We can import the module \texttt{mymodule} into our Python program using
\texttt{import}:

    \begin{Verbatim}[commandchars=\\\{\}]
{\color{incolor}In [{\color{incolor}141}]:} \PY{k+kn}{import} \PY{n+nn}{mymodule}
\end{Verbatim}

    Use \texttt{help(module)} to get a summary of what the module provides:

    \begin{Verbatim}[commandchars=\\\{\}]
{\color{incolor}In [{\color{incolor}142}]:} \PY{n}{help}\PY{p}{(}\PY{n}{mymodule}\PY{p}{)}
\end{Verbatim}

    \begin{Verbatim}[commandchars=\\\{\}]
Help on module mymodule:

NAME
    mymodule

FILE
    /home/jpsilva/Lisbon1214/notebooks/mymodule.py

DESCRIPTION
    Example of a python module. Contains a variable called my\_variable,
    a function called my\_function, and a class called MyClass.

CLASSES
    MyClass
    
    class MyClass
     |  Example class.
     |  
     |  Methods defined here:
     |  
     |  \_\_init\_\_(self)
     |  
     |  get\_variable(self)
     |  
     |  set\_variable(self, new\_value)
     |      Set self.variable to a new value

FUNCTIONS
    my\_function()
        Example function

DATA
    my\_variable = 0
    \end{Verbatim}

    \begin{Verbatim}[commandchars=\\\{\}]
{\color{incolor}In [{\color{incolor}143}]:} \PY{n}{mymodule}\PY{o}{.}\PY{n}{my\PYZus{}variable}
\end{Verbatim}

            \begin{Verbatim}[commandchars=\\\{\}]
{\color{outcolor}Out[{\color{outcolor}143}]:} 0
\end{Verbatim}
        
    \begin{Verbatim}[commandchars=\\\{\}]
{\color{incolor}In [{\color{incolor}144}]:} \PY{n}{mymodule}\PY{o}{.}\PY{n}{my\PYZus{}function}\PY{p}{(}\PY{p}{)} 
\end{Verbatim}

            \begin{Verbatim}[commandchars=\\\{\}]
{\color{outcolor}Out[{\color{outcolor}144}]:} 0
\end{Verbatim}
        
    \begin{Verbatim}[commandchars=\\\{\}]
{\color{incolor}In [{\color{incolor}145}]:} \PY{n}{my\PYZus{}class} \PY{o}{=} \PY{n}{mymodule}\PY{o}{.}\PY{n}{MyClass}\PY{p}{(}\PY{p}{)} 
          \PY{n}{my\PYZus{}class}\PY{o}{.}\PY{n}{set\PYZus{}variable}\PY{p}{(}\PY{l+m+mi}{10}\PY{p}{)}
          \PY{n}{my\PYZus{}class}\PY{o}{.}\PY{n}{get\PYZus{}variable}\PY{p}{(}\PY{p}{)}
\end{Verbatim}

            \begin{Verbatim}[commandchars=\\\{\}]
{\color{outcolor}Out[{\color{outcolor}145}]:} 10
\end{Verbatim}
        
    If we make changes to the code in \texttt{mymodule.py}, we need to
reload it using \texttt{reload}:

    \begin{Verbatim}[commandchars=\\\{\}]
{\color{incolor}In [{\color{incolor}146}]:} \PY{n+nb}{reload}\PY{p}{(}\PY{n}{mymodule}\PY{p}{)}
\end{Verbatim}

            \begin{Verbatim}[commandchars=\\\{\}]
{\color{outcolor}Out[{\color{outcolor}146}]:} <module 'mymodule' from 'mymodule.pyc'>
\end{Verbatim}
        
    \subsubsection{Versions}\label{versions}

    \begin{Verbatim}[commandchars=\\\{\}]
{\color{incolor}In [{\color{incolor}147}]:} \PY{o}{\PYZpc{}}\PY{k}{reload\PYZus{}ext} \PY{n}{version\PYZus{}information}
          
          \PY{o}{\PYZpc{}}\PY{k}{version\PYZus{}information} \PY{n}{math}
\end{Verbatim}
\texttt{\color{outcolor}Out[{\color{outcolor}147}]:}
    
    \begin{tabular}{|l|l|}\hline
{\bf Software} & {\bf Version} \\ \hline\hline
Python & 2.7.8 |Anaconda 2.1.0 (64-bit)| (default, Aug 21 2014, 18:22:21) [GCC 4.4.7 20120313 (Red Hat 4.4.7-1)] \\ \hline
IPython & 2.3.0 \\ \hline
OS & posix [linux2] \\ \hline
math & 'module' object has no attribute '__version__' \\ \hline
\hline \multicolumn{2}{|l|}{Fri Dec 05 09:59:20 2014 CET} \\ \hline
\end{tabular}


    

    \begin{Verbatim}[commandchars=\\\{\}]
{\color{incolor}In [{\color{incolor}148}]:} \PY{o}{\PYZpc{}}\PY{k}{who}
\end{Verbatim}

    \begin{Verbatim}[commandchars=\\\{\}]
HTML	 Point	 acos	 acosh	 asin	 asinh	 atan	 atan2	 atanh	 
b1	 b2	 ceil	 copysign	 cos	 cosh	 css\_styling	 degrees	 e	 
erf	 erfc	 exp	 expm1	 f1	 f2	 fabs	 factorial	 floor	 
fmod	 frexp	 fsum	 func0	 func1	 gamma	 hypot	 i	 idx	 
inv	 isinf	 isnan	 key	 keyword	 l	 l1	 l2	 ldexp	 
lgamma	 log	 log10	 log1p	 math	 modf	 my\_class	 my\_variable	 myfunc	 
mymodule	 p1	 p2	 params	 pi	 point	 pow	 powers	 radians	 
s	 s2	 s3	 script\_dir	 sin	 sinh	 sparseinv	 sqrt	 square	 
start	 statement1	 statement2	 step	 stop	 sys	 tan	 tanh	 trunc	 
types	 value	 word	 x	 x2	 x3	 x4	 y	 z
    \end{Verbatim}

    We can find more information about IPython capabilities in
\href{https://github.com/ipython/ipython/tree/1.x/examples/notebooks}{IPython
example notebooks}

    \emph{The full notebook can be downloaded}
\href{https://raw.github.com/PoeticCapybara/Python-Introduction-Zittau/master/Lecture-1-Introduction-to-Python.ipynb}{\emph{here}},
\emph{or viewed statically on}
\href{http://nbviewer.ipython.org/urls/raw.github.com/PoeticCapybara/Python-Introduction-Zittau/master/Lecture-1-Introduction-to-Python.ipynb}{\emph{nbviewer}}

    \begin{Verbatim}[commandchars=\\\{\}]
{\color{incolor}In [{\color{incolor}149}]:} \PY{k+kn}{from} \PY{n+nn}{IPython.core.display} \PY{k+kn}{import} \PY{n}{HTML}
          \PY{k}{def} \PY{n+nf}{css\PYZus{}styling}\PY{p}{(}\PY{p}{)}\PY{p}{:}
              \PY{n}{styles} \PY{o}{=} \PY{n+nb}{open}\PY{p}{(}\PY{l+s}{\PYZdq{}}\PY{l+s}{./styles/custom.css}\PY{l+s}{\PYZdq{}}\PY{p}{,} \PY{l+s}{\PYZdq{}}\PY{l+s}{r}\PY{l+s}{\PYZdq{}}\PY{p}{)}\PY{o}{.}\PY{n}{read}\PY{p}{(}\PY{p}{)}
              \PY{k}{return} \PY{n}{HTML}\PY{p}{(}\PY{n}{styles}\PY{p}{)}
          \PY{n}{css\PYZus{}styling}\PY{p}{(}\PY{p}{)}
\end{Verbatim}

            \begin{Verbatim}[commandchars=\\\{\}]
{\color{outcolor}Out[{\color{outcolor}149}]:} <IPython.core.display.HTML at 0x7faefcf17990>
\end{Verbatim}
        
    \hyperref[Index]{Back to top}

    \begin{Verbatim}[commandchars=\\\{\}]
{\color{incolor}In [{\color{incolor}}]:} 
\end{Verbatim}


    % Add a bibliography block to the postdoc
    
    
    
    \end{document}
