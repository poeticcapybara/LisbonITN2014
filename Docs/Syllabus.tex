






    
    \section{\href{http://financialmathematics.pt/python2014/}{Workshop
Python for Finance}}\label{workshop-python-for-finance}

\section{- an Introduction}\label{an-introduction}

\emph{José Pedro Silva}
\href{http://www-num.math.uni-wuppertal.de/en/amna/people/jose-pedro-silva.html}{1}
- silva@math.uni-wuppertal.de

\subsubsection{Date}\label{date}

9-11 Dec 2014

\subsubsection{Location:}\label{location}

ISEG Rua do Quelhas, 6 1200-781 Lisboa, PORTUGAL

    \subsection{Brief description}\label{brief-description}

This workshop is aimed at users with few or no proficiency in Python who
want to apply numerical methods in Finance. A short introduction to the
language and its syntax will be given, as well as a brief description of
the most important modules (NumPy, SciPy, Matplotlib, Pandas, IPython)
for scientific and financial analysis purposes. The workshop will
consist of interactive lectures with several examples followed by
practical sessions.

    \begin{itemize}
\item
  \subsection{Day1}\label{day1}

  \begin{itemize}
  \itemsep1pt\parskip0pt\parsep0pt
  \item
    \href{I-Python.ipynb}{Python}
  \item
    \href{II-IPython.ipynb}{IPython}
  \item
    \href{III-Numpy.ipynb}{Numpy}
  \item
    \href{V-Matplotlib.ipynb}{Matplotlib}
  \end{itemize}
\item
  \subsection{Day2}\label{day2}

  \begin{itemize}
  \itemsep1pt\parskip0pt\parsep0pt
  \item
    \href{IV-SciPy.ipynb}{SciPy}
  \item
    \href{VI-Pandas.ipynb}{Pandas}
  \item
    \href{VII-Tips-and-Tricks.ipynb}{Tips and Tricks}
  \item
    \href{VIII-Benchmarks.ipynb}{Benchmarks}
  \item
    \href{IX-Randomness.ipynb}{Randomness}
  \item
    \href{X-FFT.ipynb}{FFT}
  \item
    \href{XI-OptionPricing.ipynb}{Option Pricing}
  \end{itemize}
\end{itemize}


    \chapter{More information}


    \begin{itemize}
\itemsep1pt\parskip0pt\parsep0pt
\item
  \href{http://forpythonquants.com/}{For Python Quants Conference}
\item
  \href{http://www.quantopian.com}{Quantopian}
\item
  \href{https://store.continuum.io/cshop/python-for-finance/}{ContinuumIO
  Python for Finance course}
\end{itemize}

    % Add contents below.

{\par%
\vspace{-1\baselineskip}%
\needspace{4\baselineskip}}%
\begin{notebookcell}[1]%
\begin{addmargin}[\cellleftmargin]{0em}% left, right
{\smaller%
\par%
%
\vspace{-1\smallerfontscale}%
\begin{Verbatim}[commandchars=\\\{\}]
\PY{o}{\PYZpc{}}\PY{k}{load\PYZus{}ext} \PY{n}{version\PYZus{}information}
\PY{o}{\PYZpc{}}\PY{k}{version\PYZus{}information} \PY{n}{numpy}\PY{p}{,} \PY{n}{matplotlib}\PY{p}{,} \PY{n}{scipy}\PY{p}{,} \PY{n}{pandas}
\end{Verbatim}
%
\par%
\vspace{-1\smallerfontscale}}%
\end{addmargin}
\end{notebookcell}

\par\vspace{1\smallerfontscale}%
    \needspace{4\baselineskip}%
    % Only render the prompt if the cell is pyout.  Note, the outputs prompt 
    % block isn't used since we need to check each indiviual output and only
    % add prompts to the pyout ones.
    
        {\par%
        \vspace{-1\smallerfontscale}%
        \noindent%
        \begin{minipage}{\cellleftmargin}%
    \hfill%
    {\smaller%
    \tt%
    \color{nbframe-out-prompt}%
    Out[1]:}%
    \hspace{\inputpadding}%
    \hspace{0em}%
    \hspace{3pt}%
    \end{minipage}%%
        }%
    %
    %
    \begin{addmargin}[\cellleftmargin]{0em}% left, right
    {\smaller%
    \vspace{-1\smallerfontscale}%
    
    
    \begin{tabular}{|l|l|}\hline
{\bf Software} & {\bf Version} \\ \hline\hline
Python & 2.7.8 |Anaconda 2.1.0 (64-bit)| (default, Aug 21 2014, 18:22:21) [GCC 4.4.7 20120313 (Red Hat 4.4.7-1)] \\ \hline
IPython & 2.3.1 \\ \hline
OS & posix [linux2] \\ \hline
numpy & 1.9.1 \\ \hline
matplotlib & 1.4.2 \\ \hline
scipy & 0.14.0 \\ \hline
pandas & 0.15.1 \\ \hline
\hline \multicolumn{2}{|l|}{Fri Dec 05 10:35:46 2014 CET} \\ \hline
\end{tabular}


    
}%
    \end{addmargin}%
    \emph{All notebooks will be available from the official ITN-Strike
repository soon}

    % Add contents below.

{\par%
\vspace{-1\baselineskip}%
\needspace{4\baselineskip}}%
\begin{notebookcell}[2]%
\begin{addmargin}[\cellleftmargin]{0em}% left, right
{\smaller%
\par%
%
\vspace{-1\smallerfontscale}%
\begin{Verbatim}[commandchars=\\\{\}]
\PY{k+kn}{from} \PY{n+nn}{IPython.core.display} \PY{k+kn}{import} \PY{n}{HTML}
\PY{k}{def} \PY{n+nf}{css\PYZus{}styling}\PY{p}{(}\PY{p}{)}\PY{p}{:}
    \PY{n}{styles} \PY{o}{=} \PY{n+nb}{open}\PY{p}{(}\PY{l+s}{\PYZdq{}}\PY{l+s}{./styles/custom.css}\PY{l+s}{\PYZdq{}}\PY{p}{,} \PY{l+s}{\PYZdq{}}\PY{l+s}{r}\PY{l+s}{\PYZdq{}}\PY{p}{)}\PY{o}{.}\PY{n}{read}\PY{p}{(}\PY{p}{)}
    \PY{k}{return} \PY{n}{HTML}\PY{p}{(}\PY{n}{styles}\PY{p}{)}
\PY{n}{css\PYZus{}styling}\PY{p}{(}\PY{p}{)}
\end{Verbatim}
%
\par%
\vspace{-1\smallerfontscale}}%
\end{addmargin}
\end{notebookcell}

\par\vspace{1\smallerfontscale}%
    \needspace{4\baselineskip}%
    % Only render the prompt if the cell is pyout.  Note, the outputs prompt 
    % block isn't used since we need to check each indiviual output and only
    % add prompts to the pyout ones.
    
        {\par%
        \vspace{-1\smallerfontscale}%
        \noindent%
        \begin{minipage}{\cellleftmargin}%
    \hfill%
    {\smaller%
    \tt%
    \color{nbframe-out-prompt}%
    Out[2]:}%
    \hspace{\inputpadding}%
    \hspace{0em}%
    \hspace{3pt}%
    \end{minipage}%%
        }%
    %
    %
    \begin{addmargin}[\cellleftmargin]{0em}% left, right
    {\smaller%
    \vspace{-1\smallerfontscale}%
    
    
    
    \begin{verbatim}
<IPython.core.display.HTML at 0x7f2fa3d8abd0>
    \end{verbatim}

    
}%
    \end{addmargin}%
    % Add contents below.

{\par%
\vspace{-1\baselineskip}%
\needspace{4\baselineskip}}%
\begin{notebookcell}[]%
\begin{addmargin}[\cellleftmargin]{0em}% left, right
{\smaller%
\par%
%
\vspace{-1\smallerfontscale}%
\begin{Verbatim}[commandchars=\\\{\}]

\end{Verbatim}
%
\par%
\vspace{-1\smallerfontscale}}%
\end{addmargin}
\end{notebookcell}


