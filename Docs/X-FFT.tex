
% Default to the notebook output style

    


% Inherit from the specified cell style.




    
\documentclass{article}

    
    
    \usepackage{graphicx} % Used to insert images
    \usepackage{adjustbox} % Used to constrain images to a maximum size 
    \usepackage{color} % Allow colors to be defined
    \usepackage{enumerate} % Needed for markdown enumerations to work
    \usepackage{geometry} % Used to adjust the document margins
    \usepackage{amsmath} % Equations
    \usepackage{amssymb} % Equations
    \usepackage[mathletters]{ucs} % Extended unicode (utf-8) support
    \usepackage[utf8x]{inputenc} % Allow utf-8 characters in the tex document
    \usepackage{fancyvrb} % verbatim replacement that allows latex
    \usepackage{grffile} % extends the file name processing of package graphics 
                         % to support a larger range 
    % The hyperref package gives us a pdf with properly built
    % internal navigation ('pdf bookmarks' for the table of contents,
    % internal cross-reference links, web links for URLs, etc.)
    \usepackage{hyperref}
    \usepackage{longtable} % longtable support required by pandoc >1.10
    \usepackage{booktabs}  % table support for pandoc > 1.12.2
    

    
    
    \definecolor{orange}{cmyk}{0,0.4,0.8,0.2}
    \definecolor{darkorange}{rgb}{.71,0.21,0.01}
    \definecolor{darkgreen}{rgb}{.12,.54,.11}
    \definecolor{myteal}{rgb}{.26, .44, .56}
    \definecolor{gray}{gray}{0.45}
    \definecolor{lightgray}{gray}{.95}
    \definecolor{mediumgray}{gray}{.8}
    \definecolor{inputbackground}{rgb}{.95, .95, .85}
    \definecolor{outputbackground}{rgb}{.95, .95, .95}
    \definecolor{traceback}{rgb}{1, .95, .95}
    % ansi colors
    \definecolor{red}{rgb}{.6,0,0}
    \definecolor{green}{rgb}{0,.65,0}
    \definecolor{brown}{rgb}{0.6,0.6,0}
    \definecolor{blue}{rgb}{0,.145,.698}
    \definecolor{purple}{rgb}{.698,.145,.698}
    \definecolor{cyan}{rgb}{0,.698,.698}
    \definecolor{lightgray}{gray}{0.5}
    
    % bright ansi colors
    \definecolor{darkgray}{gray}{0.25}
    \definecolor{lightred}{rgb}{1.0,0.39,0.28}
    \definecolor{lightgreen}{rgb}{0.48,0.99,0.0}
    \definecolor{lightblue}{rgb}{0.53,0.81,0.92}
    \definecolor{lightpurple}{rgb}{0.87,0.63,0.87}
    \definecolor{lightcyan}{rgb}{0.5,1.0,0.83}
    
    % commands and environments needed by pandoc snippets
    % extracted from the output of `pandoc -s`
    \DefineVerbatimEnvironment{Highlighting}{Verbatim}{commandchars=\\\{\}}
    % Add ',fontsize=\small' for more characters per line
    \newenvironment{Shaded}{}{}
    \newcommand{\KeywordTok}[1]{\textcolor[rgb]{0.00,0.44,0.13}{\textbf{{#1}}}}
    \newcommand{\DataTypeTok}[1]{\textcolor[rgb]{0.56,0.13,0.00}{{#1}}}
    \newcommand{\DecValTok}[1]{\textcolor[rgb]{0.25,0.63,0.44}{{#1}}}
    \newcommand{\BaseNTok}[1]{\textcolor[rgb]{0.25,0.63,0.44}{{#1}}}
    \newcommand{\FloatTok}[1]{\textcolor[rgb]{0.25,0.63,0.44}{{#1}}}
    \newcommand{\CharTok}[1]{\textcolor[rgb]{0.25,0.44,0.63}{{#1}}}
    \newcommand{\StringTok}[1]{\textcolor[rgb]{0.25,0.44,0.63}{{#1}}}
    \newcommand{\CommentTok}[1]{\textcolor[rgb]{0.38,0.63,0.69}{\textit{{#1}}}}
    \newcommand{\OtherTok}[1]{\textcolor[rgb]{0.00,0.44,0.13}{{#1}}}
    \newcommand{\AlertTok}[1]{\textcolor[rgb]{1.00,0.00,0.00}{\textbf{{#1}}}}
    \newcommand{\FunctionTok}[1]{\textcolor[rgb]{0.02,0.16,0.49}{{#1}}}
    \newcommand{\RegionMarkerTok}[1]{{#1}}
    \newcommand{\ErrorTok}[1]{\textcolor[rgb]{1.00,0.00,0.00}{\textbf{{#1}}}}
    \newcommand{\NormalTok}[1]{{#1}}
    
    % Define a nice break command that doesn't care if a line doesn't already
    % exist.
    \def\br{\hspace*{\fill} \\* }
    % Math Jax compatability definitions
    \def\gt{>}
    \def\lt{<}
    % Document parameters
    \title{X-FFT}
    
    
    

    % Pygments definitions
    
\makeatletter
\def\PY@reset{\let\PY@it=\relax \let\PY@bf=\relax%
    \let\PY@ul=\relax \let\PY@tc=\relax%
    \let\PY@bc=\relax \let\PY@ff=\relax}
\def\PY@tok#1{\csname PY@tok@#1\endcsname}
\def\PY@toks#1+{\ifx\relax#1\empty\else%
    \PY@tok{#1}\expandafter\PY@toks\fi}
\def\PY@do#1{\PY@bc{\PY@tc{\PY@ul{%
    \PY@it{\PY@bf{\PY@ff{#1}}}}}}}
\def\PY#1#2{\PY@reset\PY@toks#1+\relax+\PY@do{#2}}

\expandafter\def\csname PY@tok@gd\endcsname{\def\PY@tc##1{\textcolor[rgb]{0.63,0.00,0.00}{##1}}}
\expandafter\def\csname PY@tok@gu\endcsname{\let\PY@bf=\textbf\def\PY@tc##1{\textcolor[rgb]{0.50,0.00,0.50}{##1}}}
\expandafter\def\csname PY@tok@gt\endcsname{\def\PY@tc##1{\textcolor[rgb]{0.00,0.27,0.87}{##1}}}
\expandafter\def\csname PY@tok@gs\endcsname{\let\PY@bf=\textbf}
\expandafter\def\csname PY@tok@gr\endcsname{\def\PY@tc##1{\textcolor[rgb]{1.00,0.00,0.00}{##1}}}
\expandafter\def\csname PY@tok@cm\endcsname{\let\PY@it=\textit\def\PY@tc##1{\textcolor[rgb]{0.25,0.50,0.50}{##1}}}
\expandafter\def\csname PY@tok@vg\endcsname{\def\PY@tc##1{\textcolor[rgb]{0.10,0.09,0.49}{##1}}}
\expandafter\def\csname PY@tok@m\endcsname{\def\PY@tc##1{\textcolor[rgb]{0.40,0.40,0.40}{##1}}}
\expandafter\def\csname PY@tok@mh\endcsname{\def\PY@tc##1{\textcolor[rgb]{0.40,0.40,0.40}{##1}}}
\expandafter\def\csname PY@tok@go\endcsname{\def\PY@tc##1{\textcolor[rgb]{0.53,0.53,0.53}{##1}}}
\expandafter\def\csname PY@tok@ge\endcsname{\let\PY@it=\textit}
\expandafter\def\csname PY@tok@vc\endcsname{\def\PY@tc##1{\textcolor[rgb]{0.10,0.09,0.49}{##1}}}
\expandafter\def\csname PY@tok@il\endcsname{\def\PY@tc##1{\textcolor[rgb]{0.40,0.40,0.40}{##1}}}
\expandafter\def\csname PY@tok@cs\endcsname{\let\PY@it=\textit\def\PY@tc##1{\textcolor[rgb]{0.25,0.50,0.50}{##1}}}
\expandafter\def\csname PY@tok@cp\endcsname{\def\PY@tc##1{\textcolor[rgb]{0.74,0.48,0.00}{##1}}}
\expandafter\def\csname PY@tok@gi\endcsname{\def\PY@tc##1{\textcolor[rgb]{0.00,0.63,0.00}{##1}}}
\expandafter\def\csname PY@tok@gh\endcsname{\let\PY@bf=\textbf\def\PY@tc##1{\textcolor[rgb]{0.00,0.00,0.50}{##1}}}
\expandafter\def\csname PY@tok@ni\endcsname{\let\PY@bf=\textbf\def\PY@tc##1{\textcolor[rgb]{0.60,0.60,0.60}{##1}}}
\expandafter\def\csname PY@tok@nl\endcsname{\def\PY@tc##1{\textcolor[rgb]{0.63,0.63,0.00}{##1}}}
\expandafter\def\csname PY@tok@nn\endcsname{\let\PY@bf=\textbf\def\PY@tc##1{\textcolor[rgb]{0.00,0.00,1.00}{##1}}}
\expandafter\def\csname PY@tok@no\endcsname{\def\PY@tc##1{\textcolor[rgb]{0.53,0.00,0.00}{##1}}}
\expandafter\def\csname PY@tok@na\endcsname{\def\PY@tc##1{\textcolor[rgb]{0.49,0.56,0.16}{##1}}}
\expandafter\def\csname PY@tok@nb\endcsname{\def\PY@tc##1{\textcolor[rgb]{0.00,0.50,0.00}{##1}}}
\expandafter\def\csname PY@tok@nc\endcsname{\let\PY@bf=\textbf\def\PY@tc##1{\textcolor[rgb]{0.00,0.00,1.00}{##1}}}
\expandafter\def\csname PY@tok@nd\endcsname{\def\PY@tc##1{\textcolor[rgb]{0.67,0.13,1.00}{##1}}}
\expandafter\def\csname PY@tok@ne\endcsname{\let\PY@bf=\textbf\def\PY@tc##1{\textcolor[rgb]{0.82,0.25,0.23}{##1}}}
\expandafter\def\csname PY@tok@nf\endcsname{\def\PY@tc##1{\textcolor[rgb]{0.00,0.00,1.00}{##1}}}
\expandafter\def\csname PY@tok@si\endcsname{\let\PY@bf=\textbf\def\PY@tc##1{\textcolor[rgb]{0.73,0.40,0.53}{##1}}}
\expandafter\def\csname PY@tok@s2\endcsname{\def\PY@tc##1{\textcolor[rgb]{0.73,0.13,0.13}{##1}}}
\expandafter\def\csname PY@tok@vi\endcsname{\def\PY@tc##1{\textcolor[rgb]{0.10,0.09,0.49}{##1}}}
\expandafter\def\csname PY@tok@nt\endcsname{\let\PY@bf=\textbf\def\PY@tc##1{\textcolor[rgb]{0.00,0.50,0.00}{##1}}}
\expandafter\def\csname PY@tok@nv\endcsname{\def\PY@tc##1{\textcolor[rgb]{0.10,0.09,0.49}{##1}}}
\expandafter\def\csname PY@tok@s1\endcsname{\def\PY@tc##1{\textcolor[rgb]{0.73,0.13,0.13}{##1}}}
\expandafter\def\csname PY@tok@kd\endcsname{\let\PY@bf=\textbf\def\PY@tc##1{\textcolor[rgb]{0.00,0.50,0.00}{##1}}}
\expandafter\def\csname PY@tok@sh\endcsname{\def\PY@tc##1{\textcolor[rgb]{0.73,0.13,0.13}{##1}}}
\expandafter\def\csname PY@tok@sc\endcsname{\def\PY@tc##1{\textcolor[rgb]{0.73,0.13,0.13}{##1}}}
\expandafter\def\csname PY@tok@sx\endcsname{\def\PY@tc##1{\textcolor[rgb]{0.00,0.50,0.00}{##1}}}
\expandafter\def\csname PY@tok@bp\endcsname{\def\PY@tc##1{\textcolor[rgb]{0.00,0.50,0.00}{##1}}}
\expandafter\def\csname PY@tok@c1\endcsname{\let\PY@it=\textit\def\PY@tc##1{\textcolor[rgb]{0.25,0.50,0.50}{##1}}}
\expandafter\def\csname PY@tok@kc\endcsname{\let\PY@bf=\textbf\def\PY@tc##1{\textcolor[rgb]{0.00,0.50,0.00}{##1}}}
\expandafter\def\csname PY@tok@c\endcsname{\let\PY@it=\textit\def\PY@tc##1{\textcolor[rgb]{0.25,0.50,0.50}{##1}}}
\expandafter\def\csname PY@tok@mf\endcsname{\def\PY@tc##1{\textcolor[rgb]{0.40,0.40,0.40}{##1}}}
\expandafter\def\csname PY@tok@err\endcsname{\def\PY@bc##1{\setlength{\fboxsep}{0pt}\fcolorbox[rgb]{1.00,0.00,0.00}{1,1,1}{\strut ##1}}}
\expandafter\def\csname PY@tok@mb\endcsname{\def\PY@tc##1{\textcolor[rgb]{0.40,0.40,0.40}{##1}}}
\expandafter\def\csname PY@tok@ss\endcsname{\def\PY@tc##1{\textcolor[rgb]{0.10,0.09,0.49}{##1}}}
\expandafter\def\csname PY@tok@sr\endcsname{\def\PY@tc##1{\textcolor[rgb]{0.73,0.40,0.53}{##1}}}
\expandafter\def\csname PY@tok@mo\endcsname{\def\PY@tc##1{\textcolor[rgb]{0.40,0.40,0.40}{##1}}}
\expandafter\def\csname PY@tok@kn\endcsname{\let\PY@bf=\textbf\def\PY@tc##1{\textcolor[rgb]{0.00,0.50,0.00}{##1}}}
\expandafter\def\csname PY@tok@mi\endcsname{\def\PY@tc##1{\textcolor[rgb]{0.40,0.40,0.40}{##1}}}
\expandafter\def\csname PY@tok@gp\endcsname{\let\PY@bf=\textbf\def\PY@tc##1{\textcolor[rgb]{0.00,0.00,0.50}{##1}}}
\expandafter\def\csname PY@tok@o\endcsname{\def\PY@tc##1{\textcolor[rgb]{0.40,0.40,0.40}{##1}}}
\expandafter\def\csname PY@tok@kr\endcsname{\let\PY@bf=\textbf\def\PY@tc##1{\textcolor[rgb]{0.00,0.50,0.00}{##1}}}
\expandafter\def\csname PY@tok@s\endcsname{\def\PY@tc##1{\textcolor[rgb]{0.73,0.13,0.13}{##1}}}
\expandafter\def\csname PY@tok@kp\endcsname{\def\PY@tc##1{\textcolor[rgb]{0.00,0.50,0.00}{##1}}}
\expandafter\def\csname PY@tok@w\endcsname{\def\PY@tc##1{\textcolor[rgb]{0.73,0.73,0.73}{##1}}}
\expandafter\def\csname PY@tok@kt\endcsname{\def\PY@tc##1{\textcolor[rgb]{0.69,0.00,0.25}{##1}}}
\expandafter\def\csname PY@tok@ow\endcsname{\let\PY@bf=\textbf\def\PY@tc##1{\textcolor[rgb]{0.67,0.13,1.00}{##1}}}
\expandafter\def\csname PY@tok@sb\endcsname{\def\PY@tc##1{\textcolor[rgb]{0.73,0.13,0.13}{##1}}}
\expandafter\def\csname PY@tok@k\endcsname{\let\PY@bf=\textbf\def\PY@tc##1{\textcolor[rgb]{0.00,0.50,0.00}{##1}}}
\expandafter\def\csname PY@tok@se\endcsname{\let\PY@bf=\textbf\def\PY@tc##1{\textcolor[rgb]{0.73,0.40,0.13}{##1}}}
\expandafter\def\csname PY@tok@sd\endcsname{\let\PY@it=\textit\def\PY@tc##1{\textcolor[rgb]{0.73,0.13,0.13}{##1}}}

\def\PYZbs{\char`\\}
\def\PYZus{\char`\_}
\def\PYZob{\char`\{}
\def\PYZcb{\char`\}}
\def\PYZca{\char`\^}
\def\PYZam{\char`\&}
\def\PYZlt{\char`\<}
\def\PYZgt{\char`\>}
\def\PYZsh{\char`\#}
\def\PYZpc{\char`\%}
\def\PYZdl{\char`\$}
\def\PYZhy{\char`\-}
\def\PYZsq{\char`\'}
\def\PYZdq{\char`\"}
\def\PYZti{\char`\~}
% for compatibility with earlier versions
\def\PYZat{@}
\def\PYZlb{[}
\def\PYZrb{]}
\makeatother


    % Exact colors from NB
    \definecolor{incolor}{rgb}{0.0, 0.0, 0.5}
    \definecolor{outcolor}{rgb}{0.545, 0.0, 0.0}



    
    % Prevent overflowing lines due to hard-to-break entities
    \sloppy 
    % Setup hyperref package
    \hypersetup{
      breaklinks=true,  % so long urls are correctly broken across lines
      colorlinks=true,
      urlcolor=blue,
      linkcolor=darkorange,
      citecolor=darkgreen,
      }
    % Slightly bigger margins than the latex defaults
    
    \geometry{verbose,tmargin=1in,bmargin=1in,lmargin=1in,rmargin=1in}
    
    

    \begin{document}
    
    
    \maketitle
    
    

    
    \section{X-\href{http://www.ams.org/journals/mcom/1965-19-090/S0025-5718-1965-0178586-1/}{FFT}
- Fast Fourier Transform}\label{x-fft---fast-fourier-transform}

    \begin{Verbatim}[commandchars=\\\{\}]
{\color{incolor}In [{\color{incolor}8}]:} \PY{k+kn}{from} \PY{n+nn}{IPython.display} \PY{k+kn}{import} \PY{n}{HTML}
\end{Verbatim}

    \begin{Verbatim}[commandchars=\\\{\}]
{\color{incolor}In [{\color{incolor}9}]:} \PY{n}{HTML}\PY{p}{(}\PY{l+s}{\PYZsq{}}\PY{l+s}{http://www.ams.org/journals/mcom/1965\PYZhy{}19\PYZhy{}090/S0025\PYZhy{}5718\PYZhy{}1965\PYZhy{}0178586\PYZhy{}1/}\PY{l+s}{\PYZsq{}}\PY{p}{)}
\end{Verbatim}

            \begin{Verbatim}[commandchars=\\\{\}]
{\color{outcolor}Out[{\color{outcolor}9}]:} <IPython.core.display.HTML at 0x7f27b419fa90>
\end{Verbatim}
        
    There are several FFT implementations in python. We will use the NumPy
and SciPy ones. The fastest is
\href{https://pypi.python.org/pypi/pyFFTW}{PyFFTW}.

    \begin{Verbatim}[commandchars=\\\{\}]
{\color{incolor}In [{\color{incolor}9}]:} \PY{k+kn}{import} \PY{n+nn}{numpy} \PY{k+kn}{as} \PY{n+nn}{np}
        \PY{k+kn}{import} \PY{n+nn}{scipy} \PY{k+kn}{as} \PY{n+nn}{scp}
        \PY{o}{\PYZpc{}}\PY{k}{pylab}
\end{Verbatim}

    \begin{Verbatim}[commandchars=\\\{\}]
Using matplotlib backend: Qt4Agg
Populating the interactive namespace from numpy and matplotlib
    \end{Verbatim}

    \begin{Verbatim}[commandchars=\\\{\}]
{\color{incolor}In [{\color{incolor}2}]:} \PY{n}{help}\PY{p}{(}\PY{n}{np}\PY{o}{.}\PY{n}{fft}\PY{p}{)}
\end{Verbatim}

    \begin{Verbatim}[commandchars=\\\{\}]
Help on package numpy.fft in numpy:

NAME
    numpy.fft

FILE
    /home/jpsilva/anaconda/lib/python2.7/site-packages/numpy/fft/\_\_init\_\_.py

DESCRIPTION
    Discrete Fourier Transform (:mod:`numpy.fft`)
    =============================================
    
    .. currentmodule:: numpy.fft
    
    Standard FFTs
    -------------
    
    .. autosummary::
       :toctree: generated/
    
       fft       Discrete Fourier transform.
       ifft      Inverse discrete Fourier transform.
       fft2      Discrete Fourier transform in two dimensions.
       ifft2     Inverse discrete Fourier transform in two dimensions.
       fftn      Discrete Fourier transform in N-dimensions.
       ifftn     Inverse discrete Fourier transform in N dimensions.
    
    Real FFTs
    ---------
    
    .. autosummary::
       :toctree: generated/
    
       rfft      Real discrete Fourier transform.
       irfft     Inverse real discrete Fourier transform.
       rfft2     Real discrete Fourier transform in two dimensions.
       irfft2    Inverse real discrete Fourier transform in two dimensions.
       rfftn     Real discrete Fourier transform in N dimensions.
       irfftn    Inverse real discrete Fourier transform in N dimensions.
    
    Hermitian FFTs
    --------------
    
    .. autosummary::
       :toctree: generated/
    
       hfft      Hermitian discrete Fourier transform.
       ihfft     Inverse Hermitian discrete Fourier transform.
    
    Helper routines
    ---------------
    
    .. autosummary::
       :toctree: generated/
    
       fftfreq   Discrete Fourier Transform sample frequencies.
       rfftfreq  DFT sample frequencies (for usage with rfft, irfft).
       fftshift  Shift zero-frequency component to center of spectrum.
       ifftshift Inverse of fftshift.
    
    
    Background information
    ----------------------
    
    Fourier analysis is fundamentally a method for expressing a function as a
    sum of periodic components, and for recovering the function from those
    components.  When both the function and its Fourier transform are
    replaced with discretized counterparts, it is called the discrete Fourier
    transform (DFT).  The DFT has become a mainstay of numerical computing in
    part because of a very fast algorithm for computing it, called the Fast
    Fourier Transform (FFT), which was known to Gauss (1805) and was brought
    to light in its current form by Cooley and Tukey [CT]\_.  Press et al. [NR]\_
    provide an accessible introduction to Fourier analysis and its
    applications.
    
    Because the discrete Fourier transform separates its input into
    components that contribute at discrete frequencies, it has a great number
    of applications in digital signal processing, e.g., for filtering, and in
    this context the discretized input to the transform is customarily
    referred to as a *signal*, which exists in the *time domain*.  The output
    is called a *spectrum* or *transform* and exists in the *frequency
    domain*.
    
    Implementation details
    ----------------------
    
    There are many ways to define the DFT, varying in the sign of the
    exponent, normalization, etc.  In this implementation, the DFT is defined
    as
    
    .. math::
       A\_k =  \textbackslash{}sum\_\{m=0\}\^{}\{n-1\} a\_m \textbackslash{}exp\textbackslash{}left\textbackslash{}\{-2\textbackslash{}pi i\{mk \textbackslash{}over n\}\textbackslash{}right\textbackslash{}\}
       \textbackslash{}qquad k = 0,\textbackslash{}ldots,n-1.
    
    The DFT is in general defined for complex inputs and outputs, and a
    single-frequency component at linear frequency :math:`f` is
    represented by a complex exponential
    :math:`a\_m = \textbackslash{}exp\textbackslash{}\{2\textbackslash{}pi i\textbackslash{},f m\textbackslash{}Delta t\textbackslash{}\}`, where :math:`\textbackslash{}Delta t`
    is the sampling interval.
    
    The values in the result follow so-called "standard" order: If ``A =
    fft(a, n)``, then ``A[0]`` contains the zero-frequency term (the mean of
    the signal), which is always purely real for real inputs. Then ``A[1:n/2]``
    contains the positive-frequency terms, and ``A[n/2+1:]`` contains the
    negative-frequency terms, in order of decreasingly negative frequency.
    For an even number of input points, ``A[n/2]`` represents both positive and
    negative Nyquist frequency, and is also purely real for real input.  For
    an odd number of input points, ``A[(n-1)/2]`` contains the largest positive
    frequency, while ``A[(n+1)/2]`` contains the largest negative frequency.
    The routine ``np.fft.fftfreq(n)`` returns an array giving the frequencies
    of corresponding elements in the output.  The routine
    ``np.fft.fftshift(A)`` shifts transforms and their frequencies to put the
    zero-frequency components in the middle, and ``np.fft.ifftshift(A)`` undoes
    that shift.
    
    When the input `a` is a time-domain signal and ``A = fft(a)``, ``np.abs(A)``
    is its amplitude spectrum and ``np.abs(A)**2`` is its power spectrum.
    The phase spectrum is obtained by ``np.angle(A)``.
    
    The inverse DFT is defined as
    
    .. math::
       a\_m = \textbackslash{}frac\{1\}\{n\}\textbackslash{}sum\_\{k=0\}\^{}\{n-1\}A\_k\textbackslash{}exp\textbackslash{}left\textbackslash{}\{2\textbackslash{}pi i\{mk\textbackslash{}over n\}\textbackslash{}right\textbackslash{}\}
       \textbackslash{}qquad m = 0,\textbackslash{}ldots,n-1.
    
    It differs from the forward transform by the sign of the exponential
    argument and the normalization by :math:`1/n`.
    
    Real and Hermitian transforms
    -----------------------------
    
    When the input is purely real, its transform is Hermitian, i.e., the
    component at frequency :math:`f\_k` is the complex conjugate of the
    component at frequency :math:`-f\_k`, which means that for real
    inputs there is no information in the negative frequency components that
    is not already available from the positive frequency components.
    The family of `rfft` functions is
    designed to operate on real inputs, and exploits this symmetry by
    computing only the positive frequency components, up to and including the
    Nyquist frequency.  Thus, ``n`` input points produce ``n/2+1`` complex
    output points.  The inverses of this family assumes the same symmetry of
    its input, and for an output of ``n`` points uses ``n/2+1`` input points.
    
    Correspondingly, when the spectrum is purely real, the signal is
    Hermitian.  The `hfft` family of functions exploits this symmetry by
    using ``n/2+1`` complex points in the input (time) domain for ``n`` real
    points in the frequency domain.
    
    In higher dimensions, FFTs are used, e.g., for image analysis and
    filtering.  The computational efficiency of the FFT means that it can
    also be a faster way to compute large convolutions, using the property
    that a convolution in the time domain is equivalent to a point-by-point
    multiplication in the frequency domain.
    
    Higher dimensions
    -----------------
    
    In two dimensions, the DFT is defined as
    
    .. math::
       A\_\{kl\} =  \textbackslash{}sum\_\{m=0\}\^{}\{M-1\} \textbackslash{}sum\_\{n=0\}\^{}\{N-1\}
       a\_\{mn\}\textbackslash{}exp\textbackslash{}left\textbackslash{}\{-2\textbackslash{}pi i \textbackslash{}left(\{mk\textbackslash{}over M\}+\{nl\textbackslash{}over N\}\textbackslash{}right)\textbackslash{}right\textbackslash{}\}
       \textbackslash{}qquad k = 0, \textbackslash{}ldots, M-1;\textbackslash{}quad l = 0, \textbackslash{}ldots, N-1,
    
    which extends in the obvious way to higher dimensions, and the inverses
    in higher dimensions also extend in the same way.
    
    References
    ----------
    
    .. [CT] Cooley, James W., and John W. Tukey, 1965, "An algorithm for the
            machine calculation of complex Fourier series," *Math. Comput.*
            19: 297-301.
    
    .. [NR] Press, W., Teukolsky, S., Vetterline, W.T., and Flannery, B.P.,
            2007, *Numerical Recipes: The Art of Scientific Computing*, ch.
            12-13.  Cambridge Univ. Press, Cambridge, UK.
    
    Examples
    --------
    
    For examples, see the various functions.

PACKAGE CONTENTS
    fftpack
    fftpack\_lite
    helper
    info
    setup

DATA
    absolute\_import = \_Feature((2, 5, 0, 'alpha', 1), (3, 0, 0, 'alpha', 0\ldots
    division = \_Feature((2, 2, 0, 'alpha', 2), (3, 0, 0, 'alpha', 0), 8192\ldots
    print\_function = \_Feature((2, 6, 0, 'alpha', 2), (3, 0, 0, 'alpha', 0)\ldots
    using\_mklfft = True
    \end{Verbatim}

    \begin{Verbatim}[commandchars=\\\{\}]
{\color{incolor}In [{\color{incolor}3}]:} \PY{n}{help}\PY{p}{(}\PY{n}{scp}\PY{o}{.}\PY{n}{fft}\PY{p}{)}
\end{Verbatim}

    \begin{Verbatim}[commandchars=\\\{\}]
Help on function fft in module mklfft.fftpack:

fft(a, n=None, axis=-1)
    Compute the one-dimensional discrete Fourier Transform.
    
    This function computes the one-dimensional *n*-point discrete Fourier
    Transform (DFT) with the efficient Fast Fourier Transform (FFT)
    algorithm [CT].
    
    Parameters
    ----------
    a : array\_like
        Input array, can be complex.
    n : int, optional
        Length of the transformed axis of the output.
        If `n` is smaller than the length of the input, the input is cropped.
        If it is larger, the input is padded with zeros.  If `n` is not given,
        the length of the input along the axis specified by `axis` is used.
    axis : int, optional
        Axis over which to compute the FFT.  If not given, the last axis is
        used.
    
    Returns
    -------
    out : complex ndarray
        The truncated or zero-padded input, transformed along the axis
        indicated by `axis`, or the last one if `axis` is not specified.
    
    Raises
    ------
    IndexError
        if `axes` is larger than the last axis of `a`.
    
    See Also
    --------
    numpy.fft : for definition of the DFT and conventions used.
    ifft : The inverse of `fft`.
    fft2 : The two-dimensional FFT.
    fftn : The *n*-dimensional FFT.
    rfftn : The *n*-dimensional FFT of real input.
    fftfreq : Frequency bins for given FFT parameters.
    
    Notes
    -----
    FFT (Fast Fourier Transform) refers to a way the discrete Fourier
    Transform (DFT) can be calculated efficiently, by using symmetries in the
    calculated terms.  The symmetry is highest when `n` is a power of 2, and
    the transform is therefore most efficient for these sizes.
    
    The DFT is defined, with the conventions used in this implementation, in
    the documentation for the `numpy.fft` module.
    
    References
    ----------
    .. [CT] Cooley, James W., and John W. Tukey, 1965, "An algorithm for the
            machine calculation of complex Fourier series," *Math. Comput.*
            19: 297-301.
    
    Examples
    --------
    >>> np.fft.fft(np.exp(2j * np.pi * np.arange(8) / 8))
    array([ -3.44505240e-16 +1.14383329e-17j,
             8.00000000e+00 -5.71092652e-15j,
             2.33482938e-16 +1.22460635e-16j,
             1.64863782e-15 +1.77635684e-15j,
             9.95839695e-17 +2.33482938e-16j,
             0.00000000e+00 +1.66837030e-15j,
             1.14383329e-17 +1.22460635e-16j,
             -1.64863782e-15 +1.77635684e-15j])
    
    >>> import matplotlib.pyplot as plt
    >>> t = np.arange(256)
    >>> sp = np.fft.fft(np.sin(t))
    >>> freq = np.fft.fftfreq(t.shape[-1])
    >>> plt.plot(freq, sp.real, freq, sp.imag)
    [<matplotlib.lines.Line2D object at 0x\ldots>, <matplotlib.lines.Line2D object at 0x\ldots>]
    >>> plt.show()
    
    In this example, real input has an FFT which is Hermitian, i.e., symmetric
    in the real part and anti-symmetric in the imaginary part, as described in
    the `numpy.fft` documentation.
    \end{Verbatim}

    Let's obtain the FFT of \[e^{\frac{i2\pi k}{8}}\] for $k=1,...,8$

    \begin{Verbatim}[commandchars=\\\{\}]
{\color{incolor}In [{\color{incolor}15}]:} \PY{n}{np}\PY{o}{.}\PY{n}{fft}\PY{o}{.}\PY{n}{fft}\PY{p}{(}\PY{n}{np}\PY{o}{.}\PY{n}{exp}\PY{p}{(}\PY{l+m+mi}{2j} \PY{o}{*} \PY{n}{np}\PY{o}{.}\PY{n}{pi} \PY{o}{*} \PY{n}{np}\PY{o}{.}\PY{n}{arange}\PY{p}{(}\PY{l+m+mi}{8}\PY{p}{)} \PY{o}{/} \PY{l+m+mi}{8}\PY{p}{)}\PY{p}{)}
\end{Verbatim}

            \begin{Verbatim}[commandchars=\\\{\}]
{\color{outcolor}Out[{\color{outcolor}15}]:} array([ -3.44505240e-16 +1.14383329e-17j,
                  8.00000000e+00 -8.52057261e-16j,
                  2.33482938e-16 +1.22460635e-16j,
                  0.00000000e+00 +1.22460635e-16j,
                  9.95839695e-17 +2.33482938e-16j,
                 -8.88178420e-16 +1.17293449e-16j,
                  1.14383329e-17 +1.22460635e-16j,   0.00000000e+00 +1.22460635e-16j])
\end{Verbatim}
        
    \begin{Verbatim}[commandchars=\\\{\}]
{\color{incolor}In [{\color{incolor}16}]:} \PY{n}{x} \PY{o}{=} \PY{n}{np}\PY{o}{.}\PY{n}{exp}\PY{p}{(}\PY{l+m+mi}{2j} \PY{o}{*} \PY{n}{np}\PY{o}{.}\PY{n}{pi} \PY{o}{*} \PY{n}{np}\PY{o}{.}\PY{n}{arange}\PY{p}{(}\PY{l+m+mi}{9}\PY{p}{)} \PY{o}{/} \PY{l+m+mi}{8}\PY{p}{)}
         \PY{n}{xt} \PY{o}{=} \PY{n}{np}\PY{o}{.}\PY{n}{fft}\PY{o}{.}\PY{n}{fft}\PY{p}{(}\PY{n}{x}\PY{p}{)}
\end{Verbatim}

    \begin{Verbatim}[commandchars=\\\{\}]
{\color{incolor}In [{\color{incolor}17}]:} \PY{n}{plot}\PY{p}{(}\PY{n}{xt}\PY{p}{)}
\end{Verbatim}

            \begin{Verbatim}[commandchars=\\\{\}]
{\color{outcolor}Out[{\color{outcolor}17}]:} [<matplotlib.lines.Line2D at 0x7f4897a5dc50>]
\end{Verbatim}
        
    \begin{Verbatim}[commandchars=\\\{\}]
{\color{incolor}In [{\color{incolor}}]:} 
\end{Verbatim}

    \begin{Verbatim}[commandchars=\\\{\}]
{\color{incolor}In [{\color{incolor}}]:} 
\end{Verbatim}

    \begin{Verbatim}[commandchars=\\\{\}]
{\color{incolor}In [{\color{incolor}}]:} 
\end{Verbatim}

    \begin{Verbatim}[commandchars=\\\{\}]
{\color{incolor}In [{\color{incolor}}]:} 
\end{Verbatim}

    \begin{Verbatim}[commandchars=\\\{\}]
{\color{incolor}In [{\color{incolor}}]:} 
\end{Verbatim}

    \begin{Verbatim}[commandchars=\\\{\}]
{\color{incolor}In [{\color{incolor}}]:} 
\end{Verbatim}

    \begin{Verbatim}[commandchars=\\\{\}]
{\color{incolor}In [{\color{incolor}}]:} 
\end{Verbatim}

    \begin{Verbatim}[commandchars=\\\{\}]
{\color{incolor}In [{\color{incolor}}]:} 
\end{Verbatim}

    \begin{Verbatim}[commandchars=\\\{\}]
{\color{incolor}In [{\color{incolor}}]:} 
\end{Verbatim}

    \begin{Verbatim}[commandchars=\\\{\}]
{\color{incolor}In [{\color{incolor}}]:} 
\end{Verbatim}

    \begin{Verbatim}[commandchars=\\\{\}]
{\color{incolor}In [{\color{incolor}}]:} 
\end{Verbatim}

    \begin{Verbatim}[commandchars=\\\{\}]
{\color{incolor}In [{\color{incolor}}]:} 
\end{Verbatim}

    \begin{Verbatim}[commandchars=\\\{\}]
{\color{incolor}In [{\color{incolor}}]:} 
\end{Verbatim}

    \begin{Verbatim}[commandchars=\\\{\}]
{\color{incolor}In [{\color{incolor}}]:} 
\end{Verbatim}

    \begin{Verbatim}[commandchars=\\\{\}]
{\color{incolor}In [{\color{incolor}}]:} 
\end{Verbatim}

    \begin{Verbatim}[commandchars=\\\{\}]
{\color{incolor}In [{\color{incolor}}]:} 
\end{Verbatim}

    \begin{Verbatim}[commandchars=\\\{\}]
{\color{incolor}In [{\color{incolor}}]:} 
\end{Verbatim}

    \begin{Verbatim}[commandchars=\\\{\}]
{\color{incolor}In [{\color{incolor}}]:} 
\end{Verbatim}

    \begin{Verbatim}[commandchars=\\\{\}]
{\color{incolor}In [{\color{incolor}}]:} 
\end{Verbatim}

    \begin{Verbatim}[commandchars=\\\{\}]
{\color{incolor}In [{\color{incolor}}]:} 
\end{Verbatim}


    \subsubsection{Optional}


    \begin{itemize}
\item
  Install \href{http://www.fftw.org/download.html}{FFTW} sudo apt-get
  install fftw3 libfftw3-dev
\item
  Install \href{https://pypi.python.org/pypi/pyFFTW}{PyFFTW} pip install
  PyFFTW
\end{itemize}

    \begin{Verbatim}[commandchars=\\\{\}]
{\color{incolor}In [{\color{incolor}1}]:} \PY{k+kn}{import} \PY{n+nn}{pyfftw}
\end{Verbatim}

    \begin{Verbatim}[commandchars=\\\{\}]
{\color{incolor}In [{\color{incolor}}]:} 
\end{Verbatim}

    \begin{Verbatim}[commandchars=\\\{\}]
{\color{incolor}In [{\color{incolor}}]:} 
\end{Verbatim}

    \begin{Verbatim}[commandchars=\\\{\}]
{\color{incolor}In [{\color{incolor}7}]:} \PY{o}{\PYZpc{}}\PY{k}{load\PYZus{}ext} \PY{n}{version\PYZus{}information}
        \PY{o}{\PYZpc{}}\PY{k}{version\PYZus{}information} \PY{n}{numpy}\PY{p}{,}\PY{n}{scipy}\PY{p}{,}\PY{n}{pyfftw}
\end{Verbatim}

    \begin{Verbatim}[commandchars=\\\{\}]
The version\_information extension is already loaded. To reload it, use:
  \%reload\_ext version\_information
    \end{Verbatim}
\texttt{\color{outcolor}Out[{\color{outcolor}7}]:}
    
    \begin{tabular}{|l|l|}\hline
{\bf Software} & {\bf Version} \\ \hline\hline
Python & 2.7.8 |Anaconda 2.1.0 (64-bit)| (default, Aug 21 2014, 18:22:21) [GCC 4.4.7 20120313 (Red Hat 4.4.7-1)] \\ \hline
IPython & 2.3.1 \\ \hline
OS & posix [linux2] \\ \hline
numpy & 1.9.1 \\ \hline
scipy & 0.14.0 \\ \hline
pyfftw & 0.9.2 \\ \hline
\hline \multicolumn{2}{|l|}{Fri Dec 05 14:20:07 2014 CET} \\ \hline
\end{tabular}


    

    \begin{Verbatim}[commandchars=\\\{\}]
{\color{incolor}In [{\color{incolor}2}]:} \PY{k+kn}{from} \PY{n+nn}{IPython.core.display} \PY{k+kn}{import} \PY{n}{HTML}
        \PY{k}{def} \PY{n+nf}{css\PYZus{}styling}\PY{p}{(}\PY{p}{)}\PY{p}{:}
            \PY{n}{styles} \PY{o}{=} \PY{n+nb}{open}\PY{p}{(}\PY{l+s}{\PYZdq{}}\PY{l+s}{./styles/custom.css}\PY{l+s}{\PYZdq{}}\PY{p}{,} \PY{l+s}{\PYZdq{}}\PY{l+s}{r}\PY{l+s}{\PYZdq{}}\PY{p}{)}\PY{o}{.}\PY{n}{read}\PY{p}{(}\PY{p}{)}
            \PY{k}{return} \PY{n}{HTML}\PY{p}{(}\PY{n}{styles}\PY{p}{)}
        \PY{n}{css\PYZus{}styling}\PY{p}{(}\PY{p}{)}
\end{Verbatim}

            \begin{Verbatim}[commandchars=\\\{\}]
{\color{outcolor}Out[{\color{outcolor}2}]:} <IPython.core.display.HTML at 0x7fcfb8ffd550>
\end{Verbatim}
        
    \begin{Verbatim}[commandchars=\\\{\}]
{\color{incolor}In [{\color{incolor}}]:} 
\end{Verbatim}


    % Add a bibliography block to the postdoc
    
    
    
    \end{document}
