






    
    \section{\href{https://www.python.org/}{Python}}\label{python}

    % Add contents below.

{\par%
\vspace{-1\baselineskip}%
\needspace{4\baselineskip}}%
\begin{notebookcell}[1]%
\begin{addmargin}[\cellleftmargin]{0em}% left, right
{\smaller%
\par%
%
\vspace{-1\smallerfontscale}%
\begin{Verbatim}[commandchars=\\\{\}]
\PY{k+kn}{import} \PY{n+nn}{antigravity}
\end{Verbatim}
%
\par%
\vspace{-1\smallerfontscale}}%
\end{addmargin}
\end{notebookcell}


    % Add contents below.

{\par%
\vspace{-1\baselineskip}%
\needspace{4\baselineskip}}%
\begin{notebookcell}[1]%
\begin{addmargin}[\cellleftmargin]{0em}% left, right
{\smaller%
\par%
%
\vspace{-1\smallerfontscale}%
\begin{Verbatim}[commandchars=\\\{\}]
\PY{k+kn}{import} \PY{n+nn}{this}
\end{Verbatim}
%
\par%
\vspace{-1\smallerfontscale}}%
\end{addmargin}
\end{notebookcell}



    \subsection{\begin{enumerate}
\def\labelenumi{\arabic{enumi}.}
\setcounter{enumi}{-1}
\itemsep1pt\parskip0pt\parsep0pt
\item
  Installation checkup
\end{enumerate}}


    The easiest way to setup a working scientific python environment is
through \href{http://continuum.io/downloads}{anaconda} from ContinuumIO.
You should be able to download it from the following links directly:

\begin{itemize}
\itemsep1pt\parskip0pt\parsep0pt
\item
  \href{http://09c8d0b2229f813c1b93-c95ac804525aac4b6dba79b00b39d1d3.r79.cf1.rackcdn.com/Anaconda-2.1.0-Linux-x86_64.sh}{Linux
  64-bit Python 2.7}
  \href{http://docs.continuum.io/anaconda/install.html\#linux-install}{Installation
  Instructions}
\item
  \href{http://09c8d0b2229f813c1b93-c95ac804525aac4b6dba79b00b39d1d3.r79.cf1.rackcdn.com/Anaconda-2.1.0-Linux-x86.sh}{Linux
  32-bit Python 2.7}
  \href{http://docs.continuum.io/anaconda/install.html\#linux-install}{Installation
  Instructions}
\item
  \href{http://09c8d0b2229f813c1b93-c95ac804525aac4b6dba79b00b39d1d3.r79.cf1.rackcdn.com/Anaconda-2.1.0-Windows-x86_64.exe}{Windows
  64-bit Python 2.7}
  \href{http://docs.continuum.io/anaconda/install.html\#windows-install}{Installation
  Instructions}
\item
  \href{http://09c8d0b2229f813c1b93-c95ac804525aac4b6dba79b00b39d1d3.r79.cf1.rackcdn.com/Anaconda-2.1.0-Windows-x86.exe}{Windows
  32-bit Python 2.7}
  \href{http://docs.continuum.io/anaconda/install.html\#windows-install}{Installation
  Instructions}
\item
  \href{http://09c8d0b2229f813c1b93-c95ac804525aac4b6dba79b00b39d1d3.r79.cf1.rackcdn.com/Anaconda-2.1.0-MacOSX-x86_64.sh}{Mac
  OS X 64-bit Python 2.7}
  \href{http://docs.continuum.io/anaconda/install.html\#mac-install}{Installation
  Instructions}
\end{itemize}

After installing run the following commands (Linux and Mac OS X):

\begin{verbatim}
conda update conda
conda update numpy scipy matplotlib pandas ipython-notebook pip
pip install seaborn
\end{verbatim}

    Run ipython and install the following extension by running

\begin{verbatim}
%install_ext http://raw.github.com/jrjohansson/version_information/master/version_information.py
\end{verbatim}

    % Add contents below.

{\par%
\vspace{-1\baselineskip}%
\needspace{4\baselineskip}}%
\begin{notebookcell}[2]%
\begin{addmargin}[\cellleftmargin]{0em}% left, right
{\smaller%
\par%
%
\vspace{-1\smallerfontscale}%
\begin{Verbatim}[commandchars=\\\{\}]
\PY{o}{\PYZpc{}}\PY{k}{load\PYZus{}ext} \PY{n}{version\PYZus{}information}
\end{Verbatim}
%
\par%
\vspace{-1\smallerfontscale}}%
\end{addmargin}
\end{notebookcell}


    % Add contents below.

{\par%
\vspace{-1\baselineskip}%
\needspace{4\baselineskip}}%
\begin{notebookcell}[6]%
\begin{addmargin}[\cellleftmargin]{0em}% left, right
{\smaller%
\par%
%
\vspace{-1\smallerfontscale}%
\begin{Verbatim}[commandchars=\\\{\}]
\PY{o}{\PYZpc{}}\PY{k}{version\PYZus{}information} \PY{n}{numpy}\PY{p}{,}\PY{n}{scipy}\PY{p}{,}\PY{n}{matplotlib}\PY{p}{,}\PY{n}{pandas}\PY{p}{,}\PY{n}{seaborn}
\end{Verbatim}
%
\par%
\vspace{-1\smallerfontscale}}%
\end{addmargin}
\end{notebookcell}

\par\vspace{1\smallerfontscale}%
    \needspace{4\baselineskip}%
    % Only render the prompt if the cell is pyout.  Note, the outputs prompt 
    % block isn't used since we need to check each indiviual output and only
    % add prompts to the pyout ones.
    
        {\par%
        \vspace{-1\smallerfontscale}%
        \noindent%
        \begin{minipage}{\cellleftmargin}%
    \hfill%
    {\smaller%
    \tt%
    \color{nbframe-out-prompt}%
    Out[6]:}%
    \hspace{\inputpadding}%
    \hspace{0em}%
    \hspace{3pt}%
    \end{minipage}%%
        }%
    %
    %
    \begin{addmargin}[\cellleftmargin]{0em}% left, right
    {\smaller%
    \vspace{-1\smallerfontscale}%
    
    
    \begin{tabular}{|l|l|}\hline
{\bf Software} & {\bf Version} \\ \hline\hline
Python & 2.7.8 |Anaconda 2.1.0 (64-bit)| (default, Aug 21 2014, 18:22:21) [GCC 4.4.7 20120313 (Red Hat 4.4.7-1)] \\ \hline
IPython & 2.3.0 \\ \hline
OS & posix [linux2] \\ \hline
numpy & 1.9.1 \\ \hline
scipy & 0.14.0 \\ \hline
matplotlib & 1.4.2 \\ \hline
pandas & 0.15.0 \\ \hline
seaborn & 0.3.1 \\ \hline
\hline \multicolumn{2}{|l|}{Thu Dec 04 14:45:40 2014 CET} \\ \hline
\end{tabular}


    
}%
    \end{addmargin}%
    % Add contents below.

{\par%
\vspace{-1\baselineskip}%
\needspace{4\baselineskip}}%
\begin{notebookcell}[2]%
\begin{addmargin}[\cellleftmargin]{0em}% left, right
{\smaller%
\par%
%
\vspace{-1\smallerfontscale}%
\begin{Verbatim}[commandchars=\\\{\}]
\PY{k+kn}{from} \PY{n+nn}{IPython.core.display} \PY{k+kn}{import} \PY{n}{HTML}
\PY{k}{def} \PY{n+nf}{css\PYZus{}styling}\PY{p}{(}\PY{p}{)}\PY{p}{:}
    \PY{n}{styles} \PY{o}{=} \PY{n+nb}{open}\PY{p}{(}\PY{l+s}{\PYZdq{}}\PY{l+s}{../styles/custom.css}\PY{l+s}{\PYZdq{}}\PY{p}{,} \PY{l+s}{\PYZdq{}}\PY{l+s}{r}\PY{l+s}{\PYZdq{}}\PY{p}{)}\PY{o}{.}\PY{n}{read}\PY{p}{(}\PY{p}{)}
    \PY{k}{return} \PY{n}{HTML}\PY{p}{(}\PY{n}{styles}\PY{p}{)}
\PY{n}{css\PYZus{}styling}\PY{p}{(}\PY{p}{)}
\end{Verbatim}
%
\par%
\vspace{-1\smallerfontscale}}%
\end{addmargin}
\end{notebookcell}

\par\vspace{1\smallerfontscale}%
    \needspace{4\baselineskip}%
    % Only render the prompt if the cell is pyout.  Note, the outputs prompt 
    % block isn't used since we need to check each indiviual output and only
    % add prompts to the pyout ones.
    
        {\par%
        \vspace{-1\smallerfontscale}%
        \noindent%
        \begin{minipage}{\cellleftmargin}%
    \hfill%
    {\smaller%
    \tt%
    \color{nbframe-out-prompt}%
    Out[2]:}%
    \hspace{\inputpadding}%
    \hspace{0em}%
    \hspace{3pt}%
    \end{minipage}%%
        }%
    %
    %
    \begin{addmargin}[\cellleftmargin]{0em}% left, right
    {\smaller%
    \vspace{-1\smallerfontscale}%
    
    
    
    \begin{verbatim}
<IPython.core.display.HTML at 0x7fe0c41c8410>
    \end{verbatim}

    
}%
    \end{addmargin}%
