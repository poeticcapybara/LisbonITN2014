% Original file by Robert Johansson http://jrjohansson.github.io/

\documentclass{report}    
\usepackage{graphicx}  % Used to insert images
\usepackage{adjustbox} % Used to constrain images to a maximum size 
\usepackage{color}     % Allow colors to be defined
\usepackage{enumerate} % Needed for markdown enumerations to work
\usepackage{geometry}  % Used to adjust the document margins
\usepackage{amsmath}   % Equations
\usepackage{amssymb}   % Equations
\usepackage[mathletters]{ucs}
\usepackage[utf8x]{inputenc}
\usepackage{fancyvrb}
\usepackage{grffile}
\usepackage{hyperref}
\usepackage{longtable}
\usepackage{booktabs}
\usepackage{tikz}
\usepackage{scrextend}
\usepackage{needspace}
\usepackage{framed}

\definecolor{orange}{cmyk}{0,0.4,0.8,0.2}
\definecolor{darkorange}{rgb}{.71,0.21,0.01}
\definecolor{darkgreen}{rgb}{.12,.54,.11}
\definecolor{myteal}{rgb}{.26, .44, .56}
\definecolor{gray}{gray}{0.45}
\definecolor{lightgray}{gray}{.95}
\definecolor{mediumgray}{gray}{.8}
\definecolor{inputbackground}{rgb}{.95, .95, .85}
\definecolor{outputbackground}{rgb}{.95, .95, .95}
\definecolor{traceback}{rgb}{1, .95, .95}
% ansi colors
\definecolor{red}{rgb}{.6,0,0}
\definecolor{green}{rgb}{0,.65,0}
\definecolor{brown}{rgb}{0.6,0.6,0}
\definecolor{blue}{rgb}{0,.145,.698}
\definecolor{purple}{rgb}{.698,.145,.698}
\definecolor{cyan}{rgb}{0,.698,.698}
\definecolor{lightgray}{gray}{0.5}

% bright ansi colors
\definecolor{darkgray}{gray}{0.25}
\definecolor{lightred}{rgb}{1.0,0.39,0.28}
\definecolor{lightgreen}{rgb}{0.48,0.99,0.0}
\definecolor{lightblue}{rgb}{0.53,0.81,0.92}
\definecolor{lightpurple}{rgb}{0.87,0.63,0.87}
\definecolor{lightcyan}{rgb}{0.5,1.0,0.83}

% commands and environments needed by pandoc snippets
% extracted from the output of `pandoc -s`
\DefineVerbatimEnvironment{Highlighting}{Verbatim}{commandchars=\\\{\}}
% Add ',fontsize=\small' for more characters per line
\newenvironment{Shaded}{}{}
\newcommand{\KeywordTok}[1]{\textcolor[rgb]{0.00,0.44,0.13}{\textbf{{#1}}}}
\newcommand{\DataTypeTok}[1]{\textcolor[rgb]{0.56,0.13,0.00}{{#1}}}
\newcommand{\DecValTok}[1]{\textcolor[rgb]{0.25,0.63,0.44}{{#1}}}
\newcommand{\BaseNTok}[1]{\textcolor[rgb]{0.25,0.63,0.44}{{#1}}}
\newcommand{\FloatTok}[1]{\textcolor[rgb]{0.25,0.63,0.44}{{#1}}}
\newcommand{\CharTok}[1]{\textcolor[rgb]{0.25,0.44,0.63}{{#1}}}
\newcommand{\StringTok}[1]{\textcolor[rgb]{0.25,0.44,0.63}{{#1}}}
\newcommand{\CommentTok}[1]{\textcolor[rgb]{0.38,0.63,0.69}{\textit{{#1}}}}
\newcommand{\OtherTok}[1]{\textcolor[rgb]{0.00,0.44,0.13}{{#1}}}
\newcommand{\AlertTok}[1]{\textcolor[rgb]{1.00,0.00,0.00}{\textbf{{#1}}}}
\newcommand{\FunctionTok}[1]{\textcolor[rgb]{0.02,0.16,0.49}{{#1}}}
\newcommand{\RegionMarkerTok}[1]{{#1}}
\newcommand{\ErrorTok}[1]{\textcolor[rgb]{1.00,0.00,0.00}{\textbf{{#1}}}}
\newcommand{\NormalTok}[1]{{#1}}

\def\br{\hspace*{\fill} \\* }
\def\gt{>}
\def\lt{<}
\makeatletter
\def\PY@reset{\let\PY@it=\relax \let\PY@bf=\relax%
\let\PY@ul=\relax \let\PY@tc=\relax%
\let\PY@bc=\relax \let\PY@ff=\relax}
\def\PY@tok#1{\csname PY@tok@#1\endcsname}
\def\PY@toks#1+{\ifx\relax#1\empty\else%
\PY@tok{#1}\expandafter\PY@toks\fi}
\def\PY@do#1{\PY@bc{\PY@tc{\PY@ul{%
\PY@it{\PY@bf{\PY@ff{#1}}}}}}}
\def\PY#1#2{\PY@reset\PY@toks#1+\relax+\PY@do{#2}}

\expandafter\def\csname PY@tok@bp\endcsname{\def\PY@tc##1{\textcolor[rgb]{0.00,0.50,0.00}{##1}}}
\expandafter\def\csname PY@tok@nd\endcsname{\def\PY@tc##1{\textcolor[rgb]{0.67,0.13,1.00}{##1}}}
\expandafter\def\csname PY@tok@no\endcsname{\def\PY@tc##1{\textcolor[rgb]{0.53,0.00,0.00}{##1}}}
\expandafter\def\csname PY@tok@nv\endcsname{\def\PY@tc##1{\textcolor[rgb]{0.10,0.09,0.49}{##1}}}
\expandafter\def\csname PY@tok@err\endcsname{\def\PY@bc##1{\setlength{\fboxsep}{0pt}\fcolorbox[rgb]{1.00,0.00,0.00}{1,1,1}{\strut ##1}}}
\expandafter\def\csname PY@tok@gt\endcsname{\def\PY@tc##1{\textcolor[rgb]{0.00,0.27,0.87}{##1}}}
\expandafter\def\csname PY@tok@kp\endcsname{\def\PY@tc##1{\textcolor[rgb]{0.00,0.50,0.00}{##1}}}
\expandafter\def\csname PY@tok@s\endcsname{\def\PY@tc##1{\textcolor[rgb]{0.73,0.13,0.13}{##1}}}
\expandafter\def\csname PY@tok@gi\endcsname{\def\PY@tc##1{\textcolor[rgb]{0.00,0.63,0.00}{##1}}}
\expandafter\def\csname PY@tok@si\endcsname{\let\PY@bf=\textbf\def\PY@tc##1{\textcolor[rgb]{0.73,0.40,0.53}{##1}}}
\expandafter\def\csname PY@tok@cs\endcsname{\let\PY@it=\textit\def\PY@tc##1{\textcolor[rgb]{0.25,0.50,0.50}{##1}}}
\expandafter\def\csname PY@tok@m\endcsname{\def\PY@tc##1{\textcolor[rgb]{0.40,0.40,0.40}{##1}}}
\expandafter\def\csname PY@tok@vg\endcsname{\def\PY@tc##1{\textcolor[rgb]{0.10,0.09,0.49}{##1}}}
\expandafter\def\csname PY@tok@nn\endcsname{\let\PY@bf=\textbf\def\PY@tc##1{\textcolor[rgb]{0.00,0.00,1.00}{##1}}}
\expandafter\def\csname PY@tok@sh\endcsname{\def\PY@tc##1{\textcolor[rgb]{0.73,0.13,0.13}{##1}}}
\expandafter\def\csname PY@tok@vi\endcsname{\def\PY@tc##1{\textcolor[rgb]{0.10,0.09,0.49}{##1}}}
\expandafter\def\csname PY@tok@ne\endcsname{\let\PY@bf=\textbf\def\PY@tc##1{\textcolor[rgb]{0.82,0.25,0.23}{##1}}}
\expandafter\def\csname PY@tok@gu\endcsname{\let\PY@bf=\textbf\def\PY@tc##1{\textcolor[rgb]{0.50,0.00,0.50}{##1}}}
\expandafter\def\csname PY@tok@nc\endcsname{\let\PY@bf=\textbf\def\PY@tc##1{\textcolor[rgb]{0.00,0.00,1.00}{##1}}}
\expandafter\def\csname PY@tok@sb\endcsname{\def\PY@tc##1{\textcolor[rgb]{0.73,0.13,0.13}{##1}}}
\expandafter\def\csname PY@tok@k\endcsname{\let\PY@bf=\textbf\def\PY@tc##1{\textcolor[rgb]{0.00,0.50,0.00}{##1}}}
\expandafter\def\csname PY@tok@na\endcsname{\def\PY@tc##1{\textcolor[rgb]{0.49,0.56,0.16}{##1}}}
\expandafter\def\csname PY@tok@kc\endcsname{\let\PY@bf=\textbf\def\PY@tc##1{\textcolor[rgb]{0.00,0.50,0.00}{##1}}}
\expandafter\def\csname PY@tok@sx\endcsname{\def\PY@tc##1{\textcolor[rgb]{0.00,0.50,0.00}{##1}}}
\expandafter\def\csname PY@tok@se\endcsname{\let\PY@bf=\textbf\def\PY@tc##1{\textcolor[rgb]{0.73,0.40,0.13}{##1}}}
\expandafter\def\csname PY@tok@vc\endcsname{\def\PY@tc##1{\textcolor[rgb]{0.10,0.09,0.49}{##1}}}
\expandafter\def\csname PY@tok@s1\endcsname{\def\PY@tc##1{\textcolor[rgb]{0.73,0.13,0.13}{##1}}}
\expandafter\def\csname PY@tok@kd\endcsname{\let\PY@bf=\textbf\def\PY@tc##1{\textcolor[rgb]{0.00,0.50,0.00}{##1}}}
\expandafter\def\csname PY@tok@go\endcsname{\def\PY@tc##1{\textcolor[rgb]{0.53,0.53,0.53}{##1}}}
\expandafter\def\csname PY@tok@ow\endcsname{\let\PY@bf=\textbf\def\PY@tc##1{\textcolor[rgb]{0.67,0.13,1.00}{##1}}}
\expandafter\def\csname PY@tok@nf\endcsname{\def\PY@tc##1{\textcolor[rgb]{0.00,0.00,1.00}{##1}}}
\expandafter\def\csname PY@tok@gs\endcsname{\let\PY@bf=\textbf}
\expandafter\def\csname PY@tok@cm\endcsname{\let\PY@it=\textit\def\PY@tc##1{\textcolor[rgb]{0.25,0.50,0.50}{##1}}}
\expandafter\def\csname PY@tok@il\endcsname{\def\PY@tc##1{\textcolor[rgb]{0.40,0.40,0.40}{##1}}}
\expandafter\def\csname PY@tok@kt\endcsname{\def\PY@tc##1{\textcolor[rgb]{0.69,0.00,0.25}{##1}}}
\expandafter\def\csname PY@tok@c1\endcsname{\let\PY@it=\textit\def\PY@tc##1{\textcolor[rgb]{0.25,0.50,0.50}{##1}}}
\expandafter\def\csname PY@tok@mi\endcsname{\def\PY@tc##1{\textcolor[rgb]{0.40,0.40,0.40}{##1}}}
\expandafter\def\csname PY@tok@ni\endcsname{\let\PY@bf=\textbf\def\PY@tc##1{\textcolor[rgb]{0.60,0.60,0.60}{##1}}}
\expandafter\def\csname PY@tok@gd\endcsname{\def\PY@tc##1{\textcolor[rgb]{0.63,0.00,0.00}{##1}}}
\expandafter\def\csname PY@tok@ge\endcsname{\let\PY@it=\textit}
\expandafter\def\csname PY@tok@nt\endcsname{\let\PY@bf=\textbf\def\PY@tc##1{\textcolor[rgb]{0.00,0.50,0.00}{##1}}}
\expandafter\def\csname PY@tok@s2\endcsname{\def\PY@tc##1{\textcolor[rgb]{0.73,0.13,0.13}{##1}}}
\expandafter\def\csname PY@tok@w\endcsname{\def\PY@tc##1{\textcolor[rgb]{0.73,0.73,0.73}{##1}}}
\expandafter\def\csname PY@tok@cp\endcsname{\def\PY@tc##1{\textcolor[rgb]{0.74,0.48,0.00}{##1}}}
\expandafter\def\csname PY@tok@nl\endcsname{\def\PY@tc##1{\textcolor[rgb]{0.63,0.63,0.00}{##1}}}
\expandafter\def\csname PY@tok@ss\endcsname{\def\PY@tc##1{\textcolor[rgb]{0.10,0.09,0.49}{##1}}}
\expandafter\def\csname PY@tok@o\endcsname{\def\PY@tc##1{\textcolor[rgb]{0.40,0.40,0.40}{##1}}}
\expandafter\def\csname PY@tok@mf\endcsname{\def\PY@tc##1{\textcolor[rgb]{0.40,0.40,0.40}{##1}}}
\expandafter\def\csname PY@tok@gp\endcsname{\let\PY@bf=\textbf\def\PY@tc##1{\textcolor[rgb]{0.00,0.00,0.50}{##1}}}
\expandafter\def\csname PY@tok@kn\endcsname{\let\PY@bf=\textbf\def\PY@tc##1{\textcolor[rgb]{0.00,0.50,0.00}{##1}}}
\expandafter\def\csname PY@tok@sc\endcsname{\def\PY@tc##1{\textcolor[rgb]{0.73,0.13,0.13}{##1}}}
\expandafter\def\csname PY@tok@mo\endcsname{\def\PY@tc##1{\textcolor[rgb]{0.40,0.40,0.40}{##1}}}
\expandafter\def\csname PY@tok@sd\endcsname{\let\PY@it=\textit\def\PY@tc##1{\textcolor[rgb]{0.73,0.13,0.13}{##1}}}
\expandafter\def\csname PY@tok@nb\endcsname{\def\PY@tc##1{\textcolor[rgb]{0.00,0.50,0.00}{##1}}}
\expandafter\def\csname PY@tok@gr\endcsname{\def\PY@tc##1{\textcolor[rgb]{1.00,0.00,0.00}{##1}}}
\expandafter\def\csname PY@tok@kr\endcsname{\let\PY@bf=\textbf\def\PY@tc##1{\textcolor[rgb]{0.00,0.50,0.00}{##1}}}
\expandafter\def\csname PY@tok@gh\endcsname{\let\PY@bf=\textbf\def\PY@tc##1{\textcolor[rgb]{0.00,0.00,0.50}{##1}}}
\expandafter\def\csname PY@tok@c\endcsname{\let\PY@it=\textit\def\PY@tc##1{\textcolor[rgb]{0.25,0.50,0.50}{##1}}}
\expandafter\def\csname PY@tok@mh\endcsname{\def\PY@tc##1{\textcolor[rgb]{0.40,0.40,0.40}{##1}}}
\expandafter\def\csname PY@tok@sr\endcsname{\def\PY@tc##1{\textcolor[rgb]{0.73,0.40,0.53}{##1}}}

\def\PYZbs{\char`\\}
\def\PYZus{\char`\_}
\def\PYZob{\char`\{}
\def\PYZcb{\char`\}}
\def\PYZca{\char`\^}
\def\PYZam{\char`\&}
\def\PYZlt{\char`\<}
\def\PYZgt{\char`\>}
\def\PYZsh{\char`\#}
\def\PYZpc{\char`\%}
\def\PYZdl{\char`\$}
\def\PYZhy{\char`\-}
\def\PYZsq{\char`\'}
\def\PYZdq{\char`\"}
\def\PYZti{\char`\~}
\def\PYZat{@}
\def\PYZlb{[}
\def\PYZrb{]}
\makeatother


% NB prompt colors
\definecolor{nbframe-border}{rgb}{0.867,0.867,0.867}
\definecolor{nbframe-bg}{rgb}{0.969,0.969,0.969}
\definecolor{nbframe-in-prompt}{rgb}{0.0,0.0,0.502}
\definecolor{nbframe-out-prompt}{rgb}{0.545,0.0,0.0}

% NB prompt lengths
\newlength{\inputpadding}
\setlength{\inputpadding}{0.5em}
\newlength{\cellleftmargin}
\setlength{\cellleftmargin}{0.15\linewidth}
\newlength{\borderthickness}
\setlength{\borderthickness}{0.4pt}
\newlength{\smallerfontscale}
\setlength{\smallerfontscale}{9.5pt}

% NB prompt font size
\def\smaller{\fontsize{\smallerfontscale}{\smallerfontscale}\selectfont}

% Define a background layer, in which the nb prompt shape is drawn
\pgfdeclarelayer{background}
\pgfsetlayers{background,main}
\usetikzlibrary{calc}

% define styles for the normal border and the torn border
\tikzset{
  normal border/.style={draw=nbframe-border, fill=nbframe-bg,
    rectangle, rounded corners=2.5pt, line width=\borderthickness},
  torn border/.style={draw=white, fill=white, line width=\borderthickness}}

% Macro to draw the shape behind the text, when it fits completly in the
% page
\def\notebookcellframe#1{%
\tikz{%
  \node[inner sep=\inputpadding] (A) {#1};% Draw the text of the node
  \begin{pgfonlayer}{background}% Draw the shape behind
  \fill[normal border]%
        (A.south east) -- ($(A.south west)+(\cellleftmargin,0)$) -- 
        ($(A.north west)+(\cellleftmargin,0)$) -- (A.north east) -- cycle;
  \end{pgfonlayer}}}%

% Macro to draw the shape, when the text will continue in next page
\def\notebookcellframetop#1{%
\tikz{%
  \node[inner sep=\inputpadding] (A) {#1};    % Draw the text of the node
  \begin{pgfonlayer}{background}    
  \fill[normal border]              % Draw the ``complete shape'' behind
        (A.south east) -- ($(A.south west)+(\cellleftmargin,0)$) -- 
        ($(A.north west)+(\cellleftmargin,0)$) -- (A.north east) -- cycle;
  \fill[torn border]                % Add the torn lower border
        ($(A.south east)-(0,.1)$) -- ($(A.south west)+(\cellleftmargin,-.1)$) -- 
        ($(A.south west)+(\cellleftmargin,.1)$) -- ($(A.south east)+(0,.1)$) -- cycle;
  \end{pgfonlayer}}}

% Macro to draw the shape, when the text continues from previous page
\def\notebookcellframebottom#1{%
\tikz{%
  \node[inner sep=\inputpadding] (A) {#1};   % Draw the text of the node
  \begin{pgfonlayer}{background}   
  \fill[normal border]             % Draw the ``complete shape'' behind
        (A.south east) -- ($(A.south west)+(\cellleftmargin,0)$) -- 
        ($(A.north west)+(\cellleftmargin,0)$) -- (A.north east) -- cycle;
  \fill[torn border]               % Add the torn upper border
        ($(A.north east)-(0,.1)$) -- ($(A.north west)+(\cellleftmargin,-.1)$) -- 
        ($(A.north west)+(\cellleftmargin,.1)$) -- ($(A.north east)+(0,.1)$) -- cycle;
  \end{pgfonlayer}}}

% Macro to draw the shape, when both the text continues from previous page
% and it will continue in next page
\def\notebookcellframemiddle#1{%
\tikz{%
  \node[inner sep=\inputpadding] (A) {#1};   % Draw the text of the node
  \begin{pgfonlayer}{background}   
  \fill[normal border]             % Draw the ``complete shape'' behind
        (A.south east) -- ($(A.south west)+(\cellleftmargin,0)$) -- 
        ($(A.north west)+(\cellleftmargin,0)$) -- (A.north east) -- cycle;
  \fill[torn border]               % Add the torn lower border
        ($(A.south east)-(0,.1)$) -- ($(A.south west)+(\cellleftmargin,-.1)$) -- 
        ($(A.south west)+(\cellleftmargin,.1)$) -- ($(A.south east)+(0,.1)$) -- cycle;
  \fill[torn border]               % Add the torn upper border
        ($(A.north east)-(0,.1)$) -- ($(A.north west)+(\cellleftmargin,-.1)$) -- 
        ($(A.north west)+(\cellleftmargin,.1)$) -- ($(A.north east)+(0,.1)$) -- cycle;
  \end{pgfonlayer}}}

% Define the environment which puts the frame
% In this case, the environment also accepts an argument with an optional
% title (which defaults to ``Example'', which is typeset in a box overlaid
% on the top border
\newenvironment{notebookcell}[1][0]{%
  \def\FrameCommand{\notebookcellframe}%
  \def\FirstFrameCommand{\notebookcellframetop}%
  \def\LastFrameCommand{\notebookcellframebottom}%
  \def\MidFrameCommand{\notebookcellframemiddle}%
  \par\vspace{1\baselineskip}%
  \MakeFramed {\FrameRestore}%
  \noindent\tikz\node[inner sep=0em] at ($(A.north west)-(0,0)$) {%
  \begin{minipage}{\cellleftmargin}%
\hfill%
{\smaller%
\tt%
\color{nbframe-in-prompt}%
In[#1]:}%
\hspace{\inputpadding}%
\hspace{2pt}%
\hspace{3pt}%
\end{minipage}%%
  }; \par}%
{\endMakeFramed}
\sloppy 
\hypersetup{
  breaklinks=true,  % so long urls are correctly broken across lines
  colorlinks=true,
  urlcolor=blue,
  linkcolor=darkorange,
  citecolor=darkgreen,
  }
\geometry{verbose,tmargin=1in,bmargin=1in,lmargin=1in,rmargin=1in}

\title{Python for Finance - an Introduction}
\author{José Pedro Silva}

\begin{document}
\maketitle

\tableofcontents

%\newpage
%






    
    \section{\href{https://www.python.org/}{Python}}\label{python}

    % Add contents below.

{\par%
\vspace{-1\baselineskip}%
\needspace{4\baselineskip}}%
\begin{notebookcell}[1]%
\begin{addmargin}[\cellleftmargin]{0em}% left, right
{\smaller%
\par%
%
\vspace{-1\smallerfontscale}%
\begin{Verbatim}[commandchars=\\\{\}]
\PY{k+kn}{import} \PY{n+nn}{antigravity}
\end{Verbatim}
%
\par%
\vspace{-1\smallerfontscale}}%
\end{addmargin}
\end{notebookcell}


    % Add contents below.

{\par%
\vspace{-1\baselineskip}%
\needspace{4\baselineskip}}%
\begin{notebookcell}[1]%
\begin{addmargin}[\cellleftmargin]{0em}% left, right
{\smaller%
\par%
%
\vspace{-1\smallerfontscale}%
\begin{Verbatim}[commandchars=\\\{\}]
\PY{k+kn}{import} \PY{n+nn}{this}
\end{Verbatim}
%
\par%
\vspace{-1\smallerfontscale}}%
\end{addmargin}
\end{notebookcell}



    \subsection{\begin{enumerate}
\def\labelenumi{\arabic{enumi}.}
\setcounter{enumi}{-1}
\itemsep1pt\parskip0pt\parsep0pt
\item
  Installation checkup
\end{enumerate}}


    The easiest way to setup a working scientific python environment is
through \href{http://continuum.io/downloads}{anaconda} from ContinuumIO.
You should be able to download it from the following links directly:

\begin{itemize}
\itemsep1pt\parskip0pt\parsep0pt
\item
  \href{http://09c8d0b2229f813c1b93-c95ac804525aac4b6dba79b00b39d1d3.r79.cf1.rackcdn.com/Anaconda-2.1.0-Linux-x86_64.sh}{Linux
  64-bit Python 2.7}
  \href{http://docs.continuum.io/anaconda/install.html\#linux-install}{Installation
  Instructions}
\item
  \href{http://09c8d0b2229f813c1b93-c95ac804525aac4b6dba79b00b39d1d3.r79.cf1.rackcdn.com/Anaconda-2.1.0-Linux-x86.sh}{Linux
  32-bit Python 2.7}
  \href{http://docs.continuum.io/anaconda/install.html\#linux-install}{Installation
  Instructions}
\item
  \href{http://09c8d0b2229f813c1b93-c95ac804525aac4b6dba79b00b39d1d3.r79.cf1.rackcdn.com/Anaconda-2.1.0-Windows-x86_64.exe}{Windows
  64-bit Python 2.7}
  \href{http://docs.continuum.io/anaconda/install.html\#windows-install}{Installation
  Instructions}
\item
  \href{http://09c8d0b2229f813c1b93-c95ac804525aac4b6dba79b00b39d1d3.r79.cf1.rackcdn.com/Anaconda-2.1.0-Windows-x86.exe}{Windows
  32-bit Python 2.7}
  \href{http://docs.continuum.io/anaconda/install.html\#windows-install}{Installation
  Instructions}
\item
  \href{http://09c8d0b2229f813c1b93-c95ac804525aac4b6dba79b00b39d1d3.r79.cf1.rackcdn.com/Anaconda-2.1.0-MacOSX-x86_64.sh}{Mac
  OS X 64-bit Python 2.7}
  \href{http://docs.continuum.io/anaconda/install.html\#mac-install}{Installation
  Instructions}
\end{itemize}

After installing run the following commands (Linux and Mac OS X):

\begin{verbatim}
conda update conda
conda update numpy scipy matplotlib pandas ipython-notebook pip
pip install seaborn
\end{verbatim}

    Run ipython and install the following extension by running

\begin{verbatim}
%install_ext http://raw.github.com/jrjohansson/version_information/master/version_information.py
\end{verbatim}

    % Add contents below.

{\par%
\vspace{-1\baselineskip}%
\needspace{4\baselineskip}}%
\begin{notebookcell}[2]%
\begin{addmargin}[\cellleftmargin]{0em}% left, right
{\smaller%
\par%
%
\vspace{-1\smallerfontscale}%
\begin{Verbatim}[commandchars=\\\{\}]
\PY{o}{\PYZpc{}}\PY{k}{load\PYZus{}ext} \PY{n}{version\PYZus{}information}
\end{Verbatim}
%
\par%
\vspace{-1\smallerfontscale}}%
\end{addmargin}
\end{notebookcell}


    % Add contents below.

{\par%
\vspace{-1\baselineskip}%
\needspace{4\baselineskip}}%
\begin{notebookcell}[6]%
\begin{addmargin}[\cellleftmargin]{0em}% left, right
{\smaller%
\par%
%
\vspace{-1\smallerfontscale}%
\begin{Verbatim}[commandchars=\\\{\}]
\PY{o}{\PYZpc{}}\PY{k}{version\PYZus{}information} \PY{n}{numpy}\PY{p}{,}\PY{n}{scipy}\PY{p}{,}\PY{n}{matplotlib}\PY{p}{,}\PY{n}{pandas}\PY{p}{,}\PY{n}{seaborn}
\end{Verbatim}
%
\par%
\vspace{-1\smallerfontscale}}%
\end{addmargin}
\end{notebookcell}

\par\vspace{1\smallerfontscale}%
    \needspace{4\baselineskip}%
    % Only render the prompt if the cell is pyout.  Note, the outputs prompt 
    % block isn't used since we need to check each indiviual output and only
    % add prompts to the pyout ones.
    
        {\par%
        \vspace{-1\smallerfontscale}%
        \noindent%
        \begin{minipage}{\cellleftmargin}%
    \hfill%
    {\smaller%
    \tt%
    \color{nbframe-out-prompt}%
    Out[6]:}%
    \hspace{\inputpadding}%
    \hspace{0em}%
    \hspace{3pt}%
    \end{minipage}%%
        }%
    %
    %
    \begin{addmargin}[\cellleftmargin]{0em}% left, right
    {\smaller%
    \vspace{-1\smallerfontscale}%
    
    
    \begin{tabular}{|l|l|}\hline
{\bf Software} & {\bf Version} \\ \hline\hline
Python & 2.7.8 |Anaconda 2.1.0 (64-bit)| (default, Aug 21 2014, 18:22:21) [GCC 4.4.7 20120313 (Red Hat 4.4.7-1)] \\ \hline
IPython & 2.3.0 \\ \hline
OS & posix [linux2] \\ \hline
numpy & 1.9.1 \\ \hline
scipy & 0.14.0 \\ \hline
matplotlib & 1.4.2 \\ \hline
pandas & 0.15.0 \\ \hline
seaborn & 0.3.1 \\ \hline
\hline \multicolumn{2}{|l|}{Thu Dec 04 14:45:40 2014 CET} \\ \hline
\end{tabular}


    
}%
    \end{addmargin}%
    % Add contents below.

{\par%
\vspace{-1\baselineskip}%
\needspace{4\baselineskip}}%
\begin{notebookcell}[2]%
\begin{addmargin}[\cellleftmargin]{0em}% left, right
{\smaller%
\par%
%
\vspace{-1\smallerfontscale}%
\begin{Verbatim}[commandchars=\\\{\}]
\PY{k+kn}{from} \PY{n+nn}{IPython.core.display} \PY{k+kn}{import} \PY{n}{HTML}
\PY{k}{def} \PY{n+nf}{css\PYZus{}styling}\PY{p}{(}\PY{p}{)}\PY{p}{:}
    \PY{n}{styles} \PY{o}{=} \PY{n+nb}{open}\PY{p}{(}\PY{l+s}{\PYZdq{}}\PY{l+s}{../styles/custom.css}\PY{l+s}{\PYZdq{}}\PY{p}{,} \PY{l+s}{\PYZdq{}}\PY{l+s}{r}\PY{l+s}{\PYZdq{}}\PY{p}{)}\PY{o}{.}\PY{n}{read}\PY{p}{(}\PY{p}{)}
    \PY{k}{return} \PY{n}{HTML}\PY{p}{(}\PY{n}{styles}\PY{p}{)}
\PY{n}{css\PYZus{}styling}\PY{p}{(}\PY{p}{)}
\end{Verbatim}
%
\par%
\vspace{-1\smallerfontscale}}%
\end{addmargin}
\end{notebookcell}

\par\vspace{1\smallerfontscale}%
    \needspace{4\baselineskip}%
    % Only render the prompt if the cell is pyout.  Note, the outputs prompt 
    % block isn't used since we need to check each indiviual output and only
    % add prompts to the pyout ones.
    
        {\par%
        \vspace{-1\smallerfontscale}%
        \noindent%
        \begin{minipage}{\cellleftmargin}%
    \hfill%
    {\smaller%
    \tt%
    \color{nbframe-out-prompt}%
    Out[2]:}%
    \hspace{\inputpadding}%
    \hspace{0em}%
    \hspace{3pt}%
    \end{minipage}%%
        }%
    %
    %
    \begin{addmargin}[\cellleftmargin]{0em}% left, right
    {\smaller%
    \vspace{-1\smallerfontscale}%
    
    
    
    \begin{verbatim}
<IPython.core.display.HTML at 0x7fe0c41c8410>
    \end{verbatim}

    
}%
    \end{addmargin}%


\newpage







    
    \section{\href{http://financialmathematics.pt/python2014/}{Workshop
Python for Finance}}\label{workshop-python-for-finance}

\section{- an Introduction}\label{an-introduction}

\emph{José Pedro Silva}
\href{http://www-num.math.uni-wuppertal.de/en/amna/people/jose-pedro-silva.html}{1}
- silva@math.uni-wuppertal.de

\subsubsection{Date}\label{date}

9-11 Dec 2014

\subsubsection{Location:}\label{location}

ISEG Rua do Quelhas, 6 1200-781 Lisboa, PORTUGAL

    \subsection{Brief description}\label{brief-description}

This workshop is aimed at users with few or no proficiency in Python who
want to apply numerical methods in Finance. A short introduction to the
language and its syntax will be given, as well as a brief description of
the most important modules (NumPy, SciPy, Matplotlib, Pandas, IPython)
for scientific and financial analysis purposes. The workshop will
consist of interactive lectures with several examples followed by
practical sessions.

    \begin{itemize}
\item
  \subsection{Day1}\label{day1}

  \begin{itemize}
  \itemsep1pt\parskip0pt\parsep0pt
  \item
    \href{I-Python.ipynb}{Python}
  \item
    \href{II-IPython.ipynb}{IPython}
  \item
    \href{III-Numpy.ipynb}{Numpy}
  \item
    \href{V-Matplotlib.ipynb}{Matplotlib}
  \end{itemize}
\item
  \subsection{Day2}\label{day2}

  \begin{itemize}
  \itemsep1pt\parskip0pt\parsep0pt
  \item
    \href{IV-SciPy.ipynb}{SciPy}
  \item
    \href{VI-Pandas.ipynb}{Pandas}
  \item
    \href{VII-Tips-and-Tricks.ipynb}{Tips and Tricks}
  \item
    \href{VIII-Benchmarks.ipynb}{Benchmarks}
  \item
    \href{IX-Randomness.ipynb}{Randomness}
  \item
    \href{X-FFT.ipynb}{FFT}
  \item
    \href{XI-OptionPricing.ipynb}{Option Pricing}
  \end{itemize}
\end{itemize}


    \chapter{More information}


    \begin{itemize}
\itemsep1pt\parskip0pt\parsep0pt
\item
  \href{http://forpythonquants.com/}{For Python Quants Conference}
\item
  \href{http://www.quantopian.com}{Quantopian}
\item
  \href{https://store.continuum.io/cshop/python-for-finance/}{ContinuumIO
  Python for Finance course}
\end{itemize}

    % Add contents below.

{\par%
\vspace{-1\baselineskip}%
\needspace{4\baselineskip}}%
\begin{notebookcell}[1]%
\begin{addmargin}[\cellleftmargin]{0em}% left, right
{\smaller%
\par%
%
\vspace{-1\smallerfontscale}%
\begin{Verbatim}[commandchars=\\\{\}]
\PY{o}{\PYZpc{}}\PY{k}{load\PYZus{}ext} \PY{n}{version\PYZus{}information}
\PY{o}{\PYZpc{}}\PY{k}{version\PYZus{}information} \PY{n}{numpy}\PY{p}{,} \PY{n}{matplotlib}\PY{p}{,} \PY{n}{scipy}\PY{p}{,} \PY{n}{pandas}
\end{Verbatim}
%
\par%
\vspace{-1\smallerfontscale}}%
\end{addmargin}
\end{notebookcell}

\par\vspace{1\smallerfontscale}%
    \needspace{4\baselineskip}%
    % Only render the prompt if the cell is pyout.  Note, the outputs prompt 
    % block isn't used since we need to check each indiviual output and only
    % add prompts to the pyout ones.
    
        {\par%
        \vspace{-1\smallerfontscale}%
        \noindent%
        \begin{minipage}{\cellleftmargin}%
    \hfill%
    {\smaller%
    \tt%
    \color{nbframe-out-prompt}%
    Out[1]:}%
    \hspace{\inputpadding}%
    \hspace{0em}%
    \hspace{3pt}%
    \end{minipage}%%
        }%
    %
    %
    \begin{addmargin}[\cellleftmargin]{0em}% left, right
    {\smaller%
    \vspace{-1\smallerfontscale}%
    
    
    \begin{tabular}{|l|l|}\hline
{\bf Software} & {\bf Version} \\ \hline\hline
Python & 2.7.8 |Anaconda 2.1.0 (64-bit)| (default, Aug 21 2014, 18:22:21) [GCC 4.4.7 20120313 (Red Hat 4.4.7-1)] \\ \hline
IPython & 2.3.1 \\ \hline
OS & posix [linux2] \\ \hline
numpy & 1.9.1 \\ \hline
matplotlib & 1.4.2 \\ \hline
scipy & 0.14.0 \\ \hline
pandas & 0.15.1 \\ \hline
\hline \multicolumn{2}{|l|}{Fri Dec 05 10:35:46 2014 CET} \\ \hline
\end{tabular}


    
}%
    \end{addmargin}%
    \emph{All notebooks will be available from the official ITN-Strike
repository soon}

    % Add contents below.

{\par%
\vspace{-1\baselineskip}%
\needspace{4\baselineskip}}%
\begin{notebookcell}[2]%
\begin{addmargin}[\cellleftmargin]{0em}% left, right
{\smaller%
\par%
%
\vspace{-1\smallerfontscale}%
\begin{Verbatim}[commandchars=\\\{\}]
\PY{k+kn}{from} \PY{n+nn}{IPython.core.display} \PY{k+kn}{import} \PY{n}{HTML}
\PY{k}{def} \PY{n+nf}{css\PYZus{}styling}\PY{p}{(}\PY{p}{)}\PY{p}{:}
    \PY{n}{styles} \PY{o}{=} \PY{n+nb}{open}\PY{p}{(}\PY{l+s}{\PYZdq{}}\PY{l+s}{./styles/custom.css}\PY{l+s}{\PYZdq{}}\PY{p}{,} \PY{l+s}{\PYZdq{}}\PY{l+s}{r}\PY{l+s}{\PYZdq{}}\PY{p}{)}\PY{o}{.}\PY{n}{read}\PY{p}{(}\PY{p}{)}
    \PY{k}{return} \PY{n}{HTML}\PY{p}{(}\PY{n}{styles}\PY{p}{)}
\PY{n}{css\PYZus{}styling}\PY{p}{(}\PY{p}{)}
\end{Verbatim}
%
\par%
\vspace{-1\smallerfontscale}}%
\end{addmargin}
\end{notebookcell}

\par\vspace{1\smallerfontscale}%
    \needspace{4\baselineskip}%
    % Only render the prompt if the cell is pyout.  Note, the outputs prompt 
    % block isn't used since we need to check each indiviual output and only
    % add prompts to the pyout ones.
    
        {\par%
        \vspace{-1\smallerfontscale}%
        \noindent%
        \begin{minipage}{\cellleftmargin}%
    \hfill%
    {\smaller%
    \tt%
    \color{nbframe-out-prompt}%
    Out[2]:}%
    \hspace{\inputpadding}%
    \hspace{0em}%
    \hspace{3pt}%
    \end{minipage}%%
        }%
    %
    %
    \begin{addmargin}[\cellleftmargin]{0em}% left, right
    {\smaller%
    \vspace{-1\smallerfontscale}%
    
    
    
    \begin{verbatim}
<IPython.core.display.HTML at 0x7f2fa3d8abd0>
    \end{verbatim}

    
}%
    \end{addmargin}%
    % Add contents below.

{\par%
\vspace{-1\baselineskip}%
\needspace{4\baselineskip}}%
\begin{notebookcell}[]%
\begin{addmargin}[\cellleftmargin]{0em}% left, right
{\smaller%
\par%
%
\vspace{-1\smallerfontscale}%
\begin{Verbatim}[commandchars=\\\{\}]

\end{Verbatim}
%
\par%
\vspace{-1\smallerfontscale}}%
\end{addmargin}
\end{notebookcell}




%\newpage
%
% Default to the notebook output style

    


% Inherit from the specified cell style.




    
\documentclass{article}

    
    
    \usepackage{graphicx} % Used to insert images
    \usepackage{adjustbox} % Used to constrain images to a maximum size 
    \usepackage{color} % Allow colors to be defined
    \usepackage{enumerate} % Needed for markdown enumerations to work
    \usepackage{geometry} % Used to adjust the document margins
    \usepackage{amsmath} % Equations
    \usepackage{amssymb} % Equations
    \usepackage[mathletters]{ucs} % Extended unicode (utf-8) support
    \usepackage[utf8x]{inputenc} % Allow utf-8 characters in the tex document
    \usepackage{fancyvrb} % verbatim replacement that allows latex
    \usepackage{grffile} % extends the file name processing of package graphics 
                         % to support a larger range 
    % The hyperref package gives us a pdf with properly built
    % internal navigation ('pdf bookmarks' for the table of contents,
    % internal cross-reference links, web links for URLs, etc.)
    \usepackage{hyperref}
    \usepackage{longtable} % longtable support required by pandoc >1.10
    \usepackage{booktabs}  % table support for pandoc > 1.12.2
    

    
    
    \definecolor{orange}{cmyk}{0,0.4,0.8,0.2}
    \definecolor{darkorange}{rgb}{.71,0.21,0.01}
    \definecolor{darkgreen}{rgb}{.12,.54,.11}
    \definecolor{myteal}{rgb}{.26, .44, .56}
    \definecolor{gray}{gray}{0.45}
    \definecolor{lightgray}{gray}{.95}
    \definecolor{mediumgray}{gray}{.8}
    \definecolor{inputbackground}{rgb}{.95, .95, .85}
    \definecolor{outputbackground}{rgb}{.95, .95, .95}
    \definecolor{traceback}{rgb}{1, .95, .95}
    % ansi colors
    \definecolor{red}{rgb}{.6,0,0}
    \definecolor{green}{rgb}{0,.65,0}
    \definecolor{brown}{rgb}{0.6,0.6,0}
    \definecolor{blue}{rgb}{0,.145,.698}
    \definecolor{purple}{rgb}{.698,.145,.698}
    \definecolor{cyan}{rgb}{0,.698,.698}
    \definecolor{lightgray}{gray}{0.5}
    
    % bright ansi colors
    \definecolor{darkgray}{gray}{0.25}
    \definecolor{lightred}{rgb}{1.0,0.39,0.28}
    \definecolor{lightgreen}{rgb}{0.48,0.99,0.0}
    \definecolor{lightblue}{rgb}{0.53,0.81,0.92}
    \definecolor{lightpurple}{rgb}{0.87,0.63,0.87}
    \definecolor{lightcyan}{rgb}{0.5,1.0,0.83}
    
    % commands and environments needed by pandoc snippets
    % extracted from the output of `pandoc -s`
    \DefineVerbatimEnvironment{Highlighting}{Verbatim}{commandchars=\\\{\}}
    % Add ',fontsize=\small' for more characters per line
    \newenvironment{Shaded}{}{}
    \newcommand{\KeywordTok}[1]{\textcolor[rgb]{0.00,0.44,0.13}{\textbf{{#1}}}}
    \newcommand{\DataTypeTok}[1]{\textcolor[rgb]{0.56,0.13,0.00}{{#1}}}
    \newcommand{\DecValTok}[1]{\textcolor[rgb]{0.25,0.63,0.44}{{#1}}}
    \newcommand{\BaseNTok}[1]{\textcolor[rgb]{0.25,0.63,0.44}{{#1}}}
    \newcommand{\FloatTok}[1]{\textcolor[rgb]{0.25,0.63,0.44}{{#1}}}
    \newcommand{\CharTok}[1]{\textcolor[rgb]{0.25,0.44,0.63}{{#1}}}
    \newcommand{\StringTok}[1]{\textcolor[rgb]{0.25,0.44,0.63}{{#1}}}
    \newcommand{\CommentTok}[1]{\textcolor[rgb]{0.38,0.63,0.69}{\textit{{#1}}}}
    \newcommand{\OtherTok}[1]{\textcolor[rgb]{0.00,0.44,0.13}{{#1}}}
    \newcommand{\AlertTok}[1]{\textcolor[rgb]{1.00,0.00,0.00}{\textbf{{#1}}}}
    \newcommand{\FunctionTok}[1]{\textcolor[rgb]{0.02,0.16,0.49}{{#1}}}
    \newcommand{\RegionMarkerTok}[1]{{#1}}
    \newcommand{\ErrorTok}[1]{\textcolor[rgb]{1.00,0.00,0.00}{\textbf{{#1}}}}
    \newcommand{\NormalTok}[1]{{#1}}
    
    % Define a nice break command that doesn't care if a line doesn't already
    % exist.
    \def\br{\hspace*{\fill} \\* }
    % Math Jax compatability definitions
    \def\gt{>}
    \def\lt{<}
    % Document parameters
    \title{I-Python}
    
    
    

    % Pygments definitions
    
\makeatletter
\def\PY@reset{\let\PY@it=\relax \let\PY@bf=\relax%
    \let\PY@ul=\relax \let\PY@tc=\relax%
    \let\PY@bc=\relax \let\PY@ff=\relax}
\def\PY@tok#1{\csname PY@tok@#1\endcsname}
\def\PY@toks#1+{\ifx\relax#1\empty\else%
    \PY@tok{#1}\expandafter\PY@toks\fi}
\def\PY@do#1{\PY@bc{\PY@tc{\PY@ul{%
    \PY@it{\PY@bf{\PY@ff{#1}}}}}}}
\def\PY#1#2{\PY@reset\PY@toks#1+\relax+\PY@do{#2}}

\expandafter\def\csname PY@tok@gd\endcsname{\def\PY@tc##1{\textcolor[rgb]{0.63,0.00,0.00}{##1}}}
\expandafter\def\csname PY@tok@gu\endcsname{\let\PY@bf=\textbf\def\PY@tc##1{\textcolor[rgb]{0.50,0.00,0.50}{##1}}}
\expandafter\def\csname PY@tok@gt\endcsname{\def\PY@tc##1{\textcolor[rgb]{0.00,0.27,0.87}{##1}}}
\expandafter\def\csname PY@tok@gs\endcsname{\let\PY@bf=\textbf}
\expandafter\def\csname PY@tok@gr\endcsname{\def\PY@tc##1{\textcolor[rgb]{1.00,0.00,0.00}{##1}}}
\expandafter\def\csname PY@tok@cm\endcsname{\let\PY@it=\textit\def\PY@tc##1{\textcolor[rgb]{0.25,0.50,0.50}{##1}}}
\expandafter\def\csname PY@tok@vg\endcsname{\def\PY@tc##1{\textcolor[rgb]{0.10,0.09,0.49}{##1}}}
\expandafter\def\csname PY@tok@m\endcsname{\def\PY@tc##1{\textcolor[rgb]{0.40,0.40,0.40}{##1}}}
\expandafter\def\csname PY@tok@mh\endcsname{\def\PY@tc##1{\textcolor[rgb]{0.40,0.40,0.40}{##1}}}
\expandafter\def\csname PY@tok@go\endcsname{\def\PY@tc##1{\textcolor[rgb]{0.53,0.53,0.53}{##1}}}
\expandafter\def\csname PY@tok@ge\endcsname{\let\PY@it=\textit}
\expandafter\def\csname PY@tok@vc\endcsname{\def\PY@tc##1{\textcolor[rgb]{0.10,0.09,0.49}{##1}}}
\expandafter\def\csname PY@tok@il\endcsname{\def\PY@tc##1{\textcolor[rgb]{0.40,0.40,0.40}{##1}}}
\expandafter\def\csname PY@tok@cs\endcsname{\let\PY@it=\textit\def\PY@tc##1{\textcolor[rgb]{0.25,0.50,0.50}{##1}}}
\expandafter\def\csname PY@tok@cp\endcsname{\def\PY@tc##1{\textcolor[rgb]{0.74,0.48,0.00}{##1}}}
\expandafter\def\csname PY@tok@gi\endcsname{\def\PY@tc##1{\textcolor[rgb]{0.00,0.63,0.00}{##1}}}
\expandafter\def\csname PY@tok@gh\endcsname{\let\PY@bf=\textbf\def\PY@tc##1{\textcolor[rgb]{0.00,0.00,0.50}{##1}}}
\expandafter\def\csname PY@tok@ni\endcsname{\let\PY@bf=\textbf\def\PY@tc##1{\textcolor[rgb]{0.60,0.60,0.60}{##1}}}
\expandafter\def\csname PY@tok@nl\endcsname{\def\PY@tc##1{\textcolor[rgb]{0.63,0.63,0.00}{##1}}}
\expandafter\def\csname PY@tok@nn\endcsname{\let\PY@bf=\textbf\def\PY@tc##1{\textcolor[rgb]{0.00,0.00,1.00}{##1}}}
\expandafter\def\csname PY@tok@no\endcsname{\def\PY@tc##1{\textcolor[rgb]{0.53,0.00,0.00}{##1}}}
\expandafter\def\csname PY@tok@na\endcsname{\def\PY@tc##1{\textcolor[rgb]{0.49,0.56,0.16}{##1}}}
\expandafter\def\csname PY@tok@nb\endcsname{\def\PY@tc##1{\textcolor[rgb]{0.00,0.50,0.00}{##1}}}
\expandafter\def\csname PY@tok@nc\endcsname{\let\PY@bf=\textbf\def\PY@tc##1{\textcolor[rgb]{0.00,0.00,1.00}{##1}}}
\expandafter\def\csname PY@tok@nd\endcsname{\def\PY@tc##1{\textcolor[rgb]{0.67,0.13,1.00}{##1}}}
\expandafter\def\csname PY@tok@ne\endcsname{\let\PY@bf=\textbf\def\PY@tc##1{\textcolor[rgb]{0.82,0.25,0.23}{##1}}}
\expandafter\def\csname PY@tok@nf\endcsname{\def\PY@tc##1{\textcolor[rgb]{0.00,0.00,1.00}{##1}}}
\expandafter\def\csname PY@tok@si\endcsname{\let\PY@bf=\textbf\def\PY@tc##1{\textcolor[rgb]{0.73,0.40,0.53}{##1}}}
\expandafter\def\csname PY@tok@s2\endcsname{\def\PY@tc##1{\textcolor[rgb]{0.73,0.13,0.13}{##1}}}
\expandafter\def\csname PY@tok@vi\endcsname{\def\PY@tc##1{\textcolor[rgb]{0.10,0.09,0.49}{##1}}}
\expandafter\def\csname PY@tok@nt\endcsname{\let\PY@bf=\textbf\def\PY@tc##1{\textcolor[rgb]{0.00,0.50,0.00}{##1}}}
\expandafter\def\csname PY@tok@nv\endcsname{\def\PY@tc##1{\textcolor[rgb]{0.10,0.09,0.49}{##1}}}
\expandafter\def\csname PY@tok@s1\endcsname{\def\PY@tc##1{\textcolor[rgb]{0.73,0.13,0.13}{##1}}}
\expandafter\def\csname PY@tok@kd\endcsname{\let\PY@bf=\textbf\def\PY@tc##1{\textcolor[rgb]{0.00,0.50,0.00}{##1}}}
\expandafter\def\csname PY@tok@sh\endcsname{\def\PY@tc##1{\textcolor[rgb]{0.73,0.13,0.13}{##1}}}
\expandafter\def\csname PY@tok@sc\endcsname{\def\PY@tc##1{\textcolor[rgb]{0.73,0.13,0.13}{##1}}}
\expandafter\def\csname PY@tok@sx\endcsname{\def\PY@tc##1{\textcolor[rgb]{0.00,0.50,0.00}{##1}}}
\expandafter\def\csname PY@tok@bp\endcsname{\def\PY@tc##1{\textcolor[rgb]{0.00,0.50,0.00}{##1}}}
\expandafter\def\csname PY@tok@c1\endcsname{\let\PY@it=\textit\def\PY@tc##1{\textcolor[rgb]{0.25,0.50,0.50}{##1}}}
\expandafter\def\csname PY@tok@kc\endcsname{\let\PY@bf=\textbf\def\PY@tc##1{\textcolor[rgb]{0.00,0.50,0.00}{##1}}}
\expandafter\def\csname PY@tok@c\endcsname{\let\PY@it=\textit\def\PY@tc##1{\textcolor[rgb]{0.25,0.50,0.50}{##1}}}
\expandafter\def\csname PY@tok@mf\endcsname{\def\PY@tc##1{\textcolor[rgb]{0.40,0.40,0.40}{##1}}}
\expandafter\def\csname PY@tok@err\endcsname{\def\PY@bc##1{\setlength{\fboxsep}{0pt}\fcolorbox[rgb]{1.00,0.00,0.00}{1,1,1}{\strut ##1}}}
\expandafter\def\csname PY@tok@mb\endcsname{\def\PY@tc##1{\textcolor[rgb]{0.40,0.40,0.40}{##1}}}
\expandafter\def\csname PY@tok@ss\endcsname{\def\PY@tc##1{\textcolor[rgb]{0.10,0.09,0.49}{##1}}}
\expandafter\def\csname PY@tok@sr\endcsname{\def\PY@tc##1{\textcolor[rgb]{0.73,0.40,0.53}{##1}}}
\expandafter\def\csname PY@tok@mo\endcsname{\def\PY@tc##1{\textcolor[rgb]{0.40,0.40,0.40}{##1}}}
\expandafter\def\csname PY@tok@kn\endcsname{\let\PY@bf=\textbf\def\PY@tc##1{\textcolor[rgb]{0.00,0.50,0.00}{##1}}}
\expandafter\def\csname PY@tok@mi\endcsname{\def\PY@tc##1{\textcolor[rgb]{0.40,0.40,0.40}{##1}}}
\expandafter\def\csname PY@tok@gp\endcsname{\let\PY@bf=\textbf\def\PY@tc##1{\textcolor[rgb]{0.00,0.00,0.50}{##1}}}
\expandafter\def\csname PY@tok@o\endcsname{\def\PY@tc##1{\textcolor[rgb]{0.40,0.40,0.40}{##1}}}
\expandafter\def\csname PY@tok@kr\endcsname{\let\PY@bf=\textbf\def\PY@tc##1{\textcolor[rgb]{0.00,0.50,0.00}{##1}}}
\expandafter\def\csname PY@tok@s\endcsname{\def\PY@tc##1{\textcolor[rgb]{0.73,0.13,0.13}{##1}}}
\expandafter\def\csname PY@tok@kp\endcsname{\def\PY@tc##1{\textcolor[rgb]{0.00,0.50,0.00}{##1}}}
\expandafter\def\csname PY@tok@w\endcsname{\def\PY@tc##1{\textcolor[rgb]{0.73,0.73,0.73}{##1}}}
\expandafter\def\csname PY@tok@kt\endcsname{\def\PY@tc##1{\textcolor[rgb]{0.69,0.00,0.25}{##1}}}
\expandafter\def\csname PY@tok@ow\endcsname{\let\PY@bf=\textbf\def\PY@tc##1{\textcolor[rgb]{0.67,0.13,1.00}{##1}}}
\expandafter\def\csname PY@tok@sb\endcsname{\def\PY@tc##1{\textcolor[rgb]{0.73,0.13,0.13}{##1}}}
\expandafter\def\csname PY@tok@k\endcsname{\let\PY@bf=\textbf\def\PY@tc##1{\textcolor[rgb]{0.00,0.50,0.00}{##1}}}
\expandafter\def\csname PY@tok@se\endcsname{\let\PY@bf=\textbf\def\PY@tc##1{\textcolor[rgb]{0.73,0.40,0.13}{##1}}}
\expandafter\def\csname PY@tok@sd\endcsname{\let\PY@it=\textit\def\PY@tc##1{\textcolor[rgb]{0.73,0.13,0.13}{##1}}}

\def\PYZbs{\char`\\}
\def\PYZus{\char`\_}
\def\PYZob{\char`\{}
\def\PYZcb{\char`\}}
\def\PYZca{\char`\^}
\def\PYZam{\char`\&}
\def\PYZlt{\char`\<}
\def\PYZgt{\char`\>}
\def\PYZsh{\char`\#}
\def\PYZpc{\char`\%}
\def\PYZdl{\char`\$}
\def\PYZhy{\char`\-}
\def\PYZsq{\char`\'}
\def\PYZdq{\char`\"}
\def\PYZti{\char`\~}
% for compatibility with earlier versions
\def\PYZat{@}
\def\PYZlb{[}
\def\PYZrb{]}
\makeatother


    % Exact colors from NB
    \definecolor{incolor}{rgb}{0.0, 0.0, 0.5}
    \definecolor{outcolor}{rgb}{0.545, 0.0, 0.0}



    
    % Prevent overflowing lines due to hard-to-break entities
    \sloppy 
    % Setup hyperref package
    \hypersetup{
      breaklinks=true,  % so long urls are correctly broken across lines
      colorlinks=true,
      urlcolor=blue,
      linkcolor=darkorange,
      citecolor=darkgreen,
      }
    % Slightly bigger margins than the latex defaults
    
    \geometry{verbose,tmargin=1in,bmargin=1in,lmargin=1in,rmargin=1in}
    
    

    \begin{document}
    
    
    \maketitle
    
    

    
    

    \section{I-\href{https://www.python.org/}{Python}}\label{i-python}


    \subsection{Index}


    \begin{itemize}
\itemsep1pt\parskip0pt\parsep0pt
\item
  \hyperref[characterux5fencoding]{Character Encoding}
\item
  \hyperref[modules]{Modules}
\item
  \hyperref[variablesux5fandux5ftypes]{Variables and Types}
\item
  \hyperref[fundamentalux5ftypes]{Fundamental Types}
\item
  \hyperref[operatorsux5fandux5fcomparisons]{Operators and Comparisons}
\item
  \hyperref[compoundux5ftypes]{Compound Types}
\item
  \hyperref[controlux5fflow]{Control Flow}
\item
  \hyperref[loops]{Loops}
\item
  \hyperref[functions]{Functions}
\item
  \hyperref[classes]{Classes}
\end{itemize}

    \subsection{Python program files}\label{python-program-files}

\begin{itemize}
\item
  Python code is usually stored in text files with the file ending
  ``\texttt{.py}'':

\begin{verbatim}
myprogram.py
\end{verbatim}
\item
  Every line in a Python program file is assumed to be a Python
  statement.

  \begin{itemize}
  \itemsep1pt\parskip0pt\parsep0pt
  \item
    The only exception is comment lines, which start with the character
    \texttt{\#} (optionally preceded by an arbitrary number of
    white-space characters, i.e., tabs or spaces). Comment lines are
    usually ignored by the Python interpreter.
  \end{itemize}
\item
  To run our Python program from the command line we use:

\begin{verbatim}
$ python myprogram.py
\end{verbatim}
\item
  On UNIX systems it is common to define the path to the interpreter on
  the first line of the program (note that this is a comment line as far
  as the Python interpreter is concerned):

\begin{verbatim}
#!/usr/bin/env python
\end{verbatim}
\end{itemize}

If we do, and if we additionally set the file script to be executable,
we can run the program like this:

\begin{verbatim}
    $ myprogram.py
\end{verbatim}

\paragraph{Example:}\label{example}

    \begin{Verbatim}[commandchars=\\\{\}]
{\color{incolor}In [{\color{incolor}8}]:} \PY{n}{script\PYZus{}dir} \PY{o}{=} \PY{l+s}{\PYZsq{}}\PY{l+s}{../scripts/}\PY{l+s}{\PYZsq{}}
\end{Verbatim}

    \begin{Verbatim}[commandchars=\\\{\}]
{\color{incolor}In [{\color{incolor}22}]:} \PY{n}{ls} \PY{err}{\PYZdl{}}\PY{n}{script\PYZus{}dir}\PY{l+s}{\PYZdq{}}\PY{l+s}{hello\PYZhy{}world}\PY{l+s}{\PYZdq{}}\PY{o}{*}\PY{o}{.}\PY{n}{py}
\end{Verbatim}

    \begin{Verbatim}[commandchars=\\\{\}]
../scripts/hello-world-in-german.py  ../scripts/hello-world.py
    \end{Verbatim}

    \begin{Verbatim}[commandchars=\\\{\}]
{\color{incolor}In [{\color{incolor}18}]:} \PY{n}{cat} \PY{err}{\PYZdl{}}\PY{n}{script\PYZus{}dir}\PY{l+s}{\PYZdq{}}\PY{l+s}{hello\PYZhy{}world.py}\PY{l+s}{\PYZdq{}}
\end{Verbatim}

    \begin{Verbatim}[commandchars=\\\{\}]
\#!/usr/bin/env python

print("Bergische Universitat Wuppertal!")
    \end{Verbatim}

    \begin{Verbatim}[commandchars=\\\{\}]
{\color{incolor}In [{\color{incolor}23}]:} \PY{o}{!}python \PY{n+nv}{\PYZdl{}script\PYZus{}dir}\PY{l+s+s2}{\PYZdq{}hello\PYZhy{}world.py\PYZdq{}}
\end{Verbatim}

    \begin{Verbatim}[commandchars=\\\{\}]
Bergische Universitat Wuppertal!
    \end{Verbatim}

    

    \subsubsection{Character encoding}\label{character-encoding}

The standard character encoding is ASCII, but we can use any other
encoding, for example UTF-8. To specify that UTF-8 is used we include
the special line

\begin{verbatim}
# -*- coding: UTF-8 -*-
\end{verbatim}

at the top of the file.

    \begin{Verbatim}[commandchars=\\\{\}]
{\color{incolor}In [{\color{incolor}24}]:} \PY{n}{cat} \PY{err}{\PYZdl{}}\PY{n}{script\PYZus{}dir}\PY{l+s}{\PYZdq{}}\PY{l+s}{hello\PYZhy{}world\PYZhy{}in\PYZhy{}german.py}\PY{l+s}{\PYZdq{}}
\end{Verbatim}

    \begin{Verbatim}[commandchars=\\\{\}]
\#!/usr/bin/env python
\# -*- coding: UTF-8 -*-

print("Bergische Universität Wuppertal!")
    \end{Verbatim}

    \begin{Verbatim}[commandchars=\\\{\}]
{\color{incolor}In [{\color{incolor}25}]:} \PY{o}{!}python \PY{n+nv}{\PYZdl{}script\PYZus{}dir}\PY{l+s+s2}{\PYZdq{}hello\PYZhy{}world\PYZhy{}in\PYZhy{}german.py\PYZdq{}}
\end{Verbatim}

    \begin{Verbatim}[commandchars=\\\{\}]
Bergische Universität Wuppertal!
    \end{Verbatim}

    Other than these two \emph{optional} lines in the beginning of a Python
code file, no additional code is required for initializing a program.

    

    \subsection{Modules}\label{modules}

Most of the functionality in Python is provided by \emph{modules}.

To use a module in a Python program it first has to be imported. A
module can be imported using the \texttt{import} statement. For example,
to import the module \texttt{math}, which contains many standard
mathematical functions, we can do:

    \begin{Verbatim}[commandchars=\\\{\}]
{\color{incolor}In [{\color{incolor}26}]:} \PY{k+kn}{import} \PY{n+nn}{math}
\end{Verbatim}

    This includes the whole module and makes it available for use later in
the program. For example, we can do:

    \begin{Verbatim}[commandchars=\\\{\}]
{\color{incolor}In [{\color{incolor}27}]:} \PY{k+kn}{import} \PY{n+nn}{math}
         
         \PY{n}{x} \PY{o}{=} \PY{n}{math}\PY{o}{.}\PY{n}{cos}\PY{p}{(}\PY{l+m+mi}{2} \PY{o}{*} \PY{n}{math}\PY{o}{.}\PY{n}{pi}\PY{p}{)}
         
         \PY{k}{print}\PY{p}{(}\PY{n}{x}\PY{p}{)}
\end{Verbatim}

    \begin{Verbatim}[commandchars=\\\{\}]
1.0
    \end{Verbatim}

    Alternatively, we can chose to import all symbols (functions and
variables) in a module to the current namespace (so that we don't need
to use the prefix ``\texttt{math.}'' every time we use something from
the \texttt{math} module:

    \begin{Verbatim}[commandchars=\\\{\}]
{\color{incolor}In [{\color{incolor}28}]:} \PY{k+kn}{from} \PY{n+nn}{math} \PY{k+kn}{import} \PY{o}{*}
         
         \PY{n}{x} \PY{o}{=} \PY{n}{cos}\PY{p}{(}\PY{l+m+mi}{2} \PY{o}{*} \PY{n}{pi}\PY{p}{)}
         
         \PY{k}{print}\PY{p}{(}\PY{n}{x}\PY{p}{)}
\end{Verbatim}

    \begin{Verbatim}[commandchars=\\\{\}]
1.0
    \end{Verbatim}

    As a third alternative, we can chose to import only a few selected
symbols from a module by explicitly listing which ones we want to import
instead of using the wildcard character \texttt{*}:

    \begin{Verbatim}[commandchars=\\\{\}]
{\color{incolor}In [{\color{incolor}29}]:} \PY{k+kn}{from} \PY{n+nn}{math} \PY{k+kn}{import} \PY{n}{cos}\PY{p}{,} \PY{n}{pi}
         
         \PY{n}{x} \PY{o}{=} \PY{n}{cos}\PY{p}{(}\PY{l+m+mi}{2} \PY{o}{*} \PY{n}{pi}\PY{p}{)}
         
         \PY{k}{print}\PY{p}{(}\PY{n}{x}\PY{p}{)}
\end{Verbatim}

    \begin{Verbatim}[commandchars=\\\{\}]
1.0
    \end{Verbatim}

    Although not a very good practice, we can rename the symbols for ease of
comprehension

    \begin{Verbatim}[commandchars=\\\{\}]
{\color{incolor}In [{\color{incolor}30}]:} \PY{k+kn}{from} \PY{n+nn}{numpy.linalg} \PY{k+kn}{import} \PY{n}{inv}
         \PY{k+kn}{from} \PY{n+nn}{scipy.sparse.linalg} \PY{k+kn}{import} \PY{n}{inv} \PY{k}{as} \PY{n}{sparseinv}
\end{Verbatim}

    \subsubsection{Looking at what a module contains, and its
documentation}\label{looking-at-what-a-module-contains-and-its-documentation}

Once a module is imported, we can list the symbols it provides using the
\texttt{dir} function:

    \begin{Verbatim}[commandchars=\\\{\}]
{\color{incolor}In [{\color{incolor}31}]:} \PY{k+kn}{import} \PY{n+nn}{math}
         
         \PY{k}{print}\PY{p}{(}\PY{n+nb}{dir}\PY{p}{(}\PY{n}{math}\PY{p}{)}\PY{p}{)}
\end{Verbatim}

    \begin{Verbatim}[commandchars=\\\{\}]
['\_\_doc\_\_', '\_\_file\_\_', '\_\_name\_\_', '\_\_package\_\_', 'acos', 'acosh', 'asin', 'asinh', 'atan', 'atan2', 'atanh', 'ceil', 'copysign', 'cos', 'cosh', 'degrees', 'e', 'erf', 'erfc', 'exp', 'expm1', 'fabs', 'factorial', 'floor', 'fmod', 'frexp', 'fsum', 'gamma', 'hypot', 'isinf', 'isnan', 'ldexp', 'lgamma', 'log', 'log10', 'log1p', 'modf', 'pi', 'pow', 'radians', 'sin', 'sinh', 'sqrt', 'tan', 'tanh', 'trunc']
    \end{Verbatim}

    And using the function \texttt{help} we can get a description of each
function (almost .. not all functions have docstrings, as they are
technically called, but the vast majority of functions are documented
this way).

    \begin{Verbatim}[commandchars=\\\{\}]
{\color{incolor}In [{\color{incolor}32}]:} \PY{n}{help}\PY{p}{(}\PY{n}{math}\PY{o}{.}\PY{n}{log}\PY{p}{)}
\end{Verbatim}

    \begin{Verbatim}[commandchars=\\\{\}]
Help on built-in function log in module math:

log(\ldots)
    log(x[, base])
    
    Return the logarithm of x to the given base.
    If the base not specified, returns the natural logarithm (base e) of x.
    \end{Verbatim}

    \begin{Verbatim}[commandchars=\\\{\}]
{\color{incolor}In [{\color{incolor}33}]:} \PY{n}{log}\PY{p}{(}\PY{l+m+mi}{10}\PY{p}{)}
\end{Verbatim}

            \begin{Verbatim}[commandchars=\\\{\}]
{\color{outcolor}Out[{\color{outcolor}33}]:} 2.302585092994046
\end{Verbatim}
        
    \begin{Verbatim}[commandchars=\\\{\}]
{\color{incolor}In [{\color{incolor}34}]:} \PY{n}{log}\PY{p}{(}\PY{l+m+mi}{10}\PY{p}{,} \PY{l+m+mi}{2}\PY{p}{)}
\end{Verbatim}

            \begin{Verbatim}[commandchars=\\\{\}]
{\color{outcolor}Out[{\color{outcolor}34}]:} 3.3219280948873626
\end{Verbatim}
        
    We can also use the \texttt{help} function directly on modules: Try

\begin{verbatim}
help(math) 
\end{verbatim}

    \begin{Verbatim}[commandchars=\\\{\}]
{\color{incolor}In [{\color{incolor}35}]:} \PY{n}{help}\PY{p}{(}\PY{n}{math}\PY{p}{)}
\end{Verbatim}

    \begin{Verbatim}[commandchars=\\\{\}]
Help on module math:

NAME
    math

FILE
    /home/jpsilva/anaconda/lib/python2.7/lib-dynload/math.so

MODULE DOCS
    http://docs.python.org/library/math

DESCRIPTION
    This module is always available.  It provides access to the
    mathematical functions defined by the C standard.

FUNCTIONS
    acos(\ldots)
        acos(x)
        
        Return the arc cosine (measured in radians) of x.
    
    acosh(\ldots)
        acosh(x)
        
        Return the hyperbolic arc cosine (measured in radians) of x.
    
    asin(\ldots)
        asin(x)
        
        Return the arc sine (measured in radians) of x.
    
    asinh(\ldots)
        asinh(x)
        
        Return the hyperbolic arc sine (measured in radians) of x.
    
    atan(\ldots)
        atan(x)
        
        Return the arc tangent (measured in radians) of x.
    
    atan2(\ldots)
        atan2(y, x)
        
        Return the arc tangent (measured in radians) of y/x.
        Unlike atan(y/x), the signs of both x and y are considered.
    
    atanh(\ldots)
        atanh(x)
        
        Return the hyperbolic arc tangent (measured in radians) of x.
    
    ceil(\ldots)
        ceil(x)
        
        Return the ceiling of x as a float.
        This is the smallest integral value >= x.
    
    copysign(\ldots)
        copysign(x, y)
        
        Return x with the sign of y.
    
    cos(\ldots)
        cos(x)
        
        Return the cosine of x (measured in radians).
    
    cosh(\ldots)
        cosh(x)
        
        Return the hyperbolic cosine of x.
    
    degrees(\ldots)
        degrees(x)
        
        Convert angle x from radians to degrees.
    
    erf(\ldots)
        erf(x)
        
        Error function at x.
    
    erfc(\ldots)
        erfc(x)
        
        Complementary error function at x.
    
    exp(\ldots)
        exp(x)
        
        Return e raised to the power of x.
    
    expm1(\ldots)
        expm1(x)
        
        Return exp(x)-1.
        This function avoids the loss of precision involved in the direct evaluation of exp(x)-1 for small x.
    
    fabs(\ldots)
        fabs(x)
        
        Return the absolute value of the float x.
    
    factorial(\ldots)
        factorial(x) -> Integral
        
        Find x!. Raise a ValueError if x is negative or non-integral.
    
    floor(\ldots)
        floor(x)
        
        Return the floor of x as a float.
        This is the largest integral value <= x.
    
    fmod(\ldots)
        fmod(x, y)
        
        Return fmod(x, y), according to platform C.  x \% y may differ.
    
    frexp(\ldots)
        frexp(x)
        
        Return the mantissa and exponent of x, as pair (m, e).
        m is a float and e is an int, such that x = m * 2.**e.
        If x is 0, m and e are both 0.  Else 0.5 <= abs(m) < 1.0.
    
    fsum(\ldots)
        fsum(iterable)
        
        Return an accurate floating point sum of values in the iterable.
        Assumes IEEE-754 floating point arithmetic.
    
    gamma(\ldots)
        gamma(x)
        
        Gamma function at x.
    
    hypot(\ldots)
        hypot(x, y)
        
        Return the Euclidean distance, sqrt(x*x + y*y).
    
    isinf(\ldots)
        isinf(x) -> bool
        
        Check if float x is infinite (positive or negative).
    
    isnan(\ldots)
        isnan(x) -> bool
        
        Check if float x is not a number (NaN).
    
    ldexp(\ldots)
        ldexp(x, i)
        
        Return x * (2**i).
    
    lgamma(\ldots)
        lgamma(x)
        
        Natural logarithm of absolute value of Gamma function at x.
    
    log(\ldots)
        log(x[, base])
        
        Return the logarithm of x to the given base.
        If the base not specified, returns the natural logarithm (base e) of x.
    
    log10(\ldots)
        log10(x)
        
        Return the base 10 logarithm of x.
    
    log1p(\ldots)
        log1p(x)
        
        Return the natural logarithm of 1+x (base e).
        The result is computed in a way which is accurate for x near zero.
    
    modf(\ldots)
        modf(x)
        
        Return the fractional and integer parts of x.  Both results carry the sign
        of x and are floats.
    
    pow(\ldots)
        pow(x, y)
        
        Return x**y (x to the power of y).
    
    radians(\ldots)
        radians(x)
        
        Convert angle x from degrees to radians.
    
    sin(\ldots)
        sin(x)
        
        Return the sine of x (measured in radians).
    
    sinh(\ldots)
        sinh(x)
        
        Return the hyperbolic sine of x.
    
    sqrt(\ldots)
        sqrt(x)
        
        Return the square root of x.
    
    tan(\ldots)
        tan(x)
        
        Return the tangent of x (measured in radians).
    
    tanh(\ldots)
        tanh(x)
        
        Return the hyperbolic tangent of x.
    
    trunc(\ldots)
        trunc(x:Real) -> Integral
        
        Truncates x to the nearest Integral toward 0. Uses the \_\_trunc\_\_ magic method.

DATA
    e = 2.718281828459045
    pi = 3.141592653589793
    \end{Verbatim}

    

    \subsection{Variables and types}\label{variables-and-types}

\subsubsection{Symbol names}\label{symbol-names}

Variable names in Python can contain alphanumerical characters
\texttt{a-z}, \texttt{A-Z}, \texttt{0-9} and some special characters
such as \texttt{\_}. Normal variable names must start with a letter.

By convension, variable names start with a lower-case letter, and Class
names start with a capital letter.

In addition, there are a number of Python keywords that cannot be used
as variable names. These keywords are:

\begin{verbatim}
and, as, assert, break, class, continue, def, del, elif, else, except, 
exec, finally, for, from, global, if, import, in, is, lambda, not, or,
pass, print, raise, return, try, while, with, yield
\end{verbatim}

Note: Be aware of the keyword \texttt{lambda}, which could easily be a
natural variable name in a scientific program. But being a keyword, it
cannot be used as a variable name.

\subsubsection{Assignment}\label{assignment}

The assignment operator in Python is \texttt{=}. Python is a dynamically
typed language, so we do not need to specify the type of a variable when
we create one.

Assigning a value to a new variable creates the variable:

    \begin{Verbatim}[commandchars=\\\{\}]
{\color{incolor}In [{\color{incolor}36}]:} \PY{k+kn}{import} \PY{n+nn}{sys}
         \PY{k+kn}{import} \PY{n+nn}{keyword}
         
         \PY{k}{print} \PY{n}{sys}\PY{o}{.}\PY{n}{version}
         \PY{k}{print} \PY{l+s}{\PYZsq{}}\PY{l+s}{\PYZsq{}}
         \PY{k}{print}\PY{p}{(}\PY{n}{keyword}\PY{o}{.}\PY{n}{kwlist}\PY{p}{)}
\end{Verbatim}

    \begin{Verbatim}[commandchars=\\\{\}]
2.7.8 |Anaconda 2.1.0 (64-bit)| (default, Aug 21 2014, 18:22:21) 
[GCC 4.4.7 20120313 (Red Hat 4.4.7-1)]

['and', 'as', 'assert', 'break', 'class', 'continue', 'def', 'del', 'elif', 'else', 'except', 'exec', 'finally', 'for', 'from', 'global', 'if', 'import', 'in', 'is', 'lambda', 'not', 'or', 'pass', 'print', 'raise', 'return', 'try', 'while', 'with', 'yield']
    \end{Verbatim}

    \begin{Verbatim}[commandchars=\\\{\}]
{\color{incolor}In [{\color{incolor}37}]:} \PY{c}{\PYZsh{} variable assignments}
         \PY{n}{x} \PY{o}{=} \PY{l+m+mf}{1.0}
         \PY{n}{my\PYZus{}variable} \PY{o}{=} \PY{l+m+mf}{12.2}
\end{Verbatim}

    Although not explicitly specified, a variable do have a type associated
with it. The type is derived from the value it was assigned.

    \begin{Verbatim}[commandchars=\\\{\}]
{\color{incolor}In [{\color{incolor}38}]:} \PY{n+nb}{type}\PY{p}{(}\PY{n}{x}\PY{p}{)}
\end{Verbatim}

            \begin{Verbatim}[commandchars=\\\{\}]
{\color{outcolor}Out[{\color{outcolor}38}]:} float
\end{Verbatim}
        
    If we assign a new value to a variable, its type can change.

    \begin{Verbatim}[commandchars=\\\{\}]
{\color{incolor}In [{\color{incolor}39}]:} \PY{n}{x} \PY{o}{=} \PY{l+m+mi}{1}
\end{Verbatim}

    \begin{Verbatim}[commandchars=\\\{\}]
{\color{incolor}In [{\color{incolor}40}]:} \PY{n+nb}{type}\PY{p}{(}\PY{n}{x}\PY{p}{)}
\end{Verbatim}

            \begin{Verbatim}[commandchars=\\\{\}]
{\color{outcolor}Out[{\color{outcolor}40}]:} int
\end{Verbatim}
        
    If we try to use a variable that has not yet been defined we get an
\texttt{NameError}:

    \begin{Verbatim}[commandchars=\\\{\}]
{\color{incolor}In [{\color{incolor}41}]:} \PY{k}{print}\PY{p}{(}\PY{n}{y}\PY{p}{)}
\end{Verbatim}

    \begin{Verbatim}[commandchars=\\\{\}]

        ---------------------------------------------------------------------------
    NameError                                 Traceback (most recent call last)

        <ipython-input-41-36b2093251cd> in <module>()
    ----> 1 print(y)
    

        NameError: name 'y' is not defined

    \end{Verbatim}

    

    \subsubsection{Fundamental types}\label{fundamental-types}

    \begin{Verbatim}[commandchars=\\\{\}]
{\color{incolor}In [{\color{incolor}42}]:} \PY{c}{\PYZsh{} integers}
         \PY{n}{x} \PY{o}{=} \PY{l+m+mi}{1}
         \PY{n+nb}{type}\PY{p}{(}\PY{n}{x}\PY{p}{)}
\end{Verbatim}

            \begin{Verbatim}[commandchars=\\\{\}]
{\color{outcolor}Out[{\color{outcolor}42}]:} int
\end{Verbatim}
        
    \begin{Verbatim}[commandchars=\\\{\}]
{\color{incolor}In [{\color{incolor}43}]:} \PY{c}{\PYZsh{} float}
         \PY{n}{x} \PY{o}{=} \PY{l+m+mf}{1.0}
         \PY{n+nb}{type}\PY{p}{(}\PY{n}{x}\PY{p}{)}
\end{Verbatim}

            \begin{Verbatim}[commandchars=\\\{\}]
{\color{outcolor}Out[{\color{outcolor}43}]:} float
\end{Verbatim}
        
    \begin{Verbatim}[commandchars=\\\{\}]
{\color{incolor}In [{\color{incolor}44}]:} \PY{c}{\PYZsh{} boolean}
         \PY{n}{b1} \PY{o}{=} \PY{n+nb+bp}{True}
         \PY{n}{b2} \PY{o}{=} \PY{n+nb+bp}{False}
         
         \PY{n+nb}{type}\PY{p}{(}\PY{n}{b1}\PY{p}{)}
\end{Verbatim}

            \begin{Verbatim}[commandchars=\\\{\}]
{\color{outcolor}Out[{\color{outcolor}44}]:} bool
\end{Verbatim}
        
    \begin{Verbatim}[commandchars=\\\{\}]
{\color{incolor}In [{\color{incolor}45}]:} \PY{c}{\PYZsh{} complex numbers: note the use of `j` to specify the imaginary part}
         \PY{n}{x} \PY{o}{=} \PY{l+m+mf}{1.0} \PY{o}{\PYZhy{}} \PY{l+m+mf}{1.0j}
         \PY{n+nb}{type}\PY{p}{(}\PY{n}{x}\PY{p}{)}
\end{Verbatim}

            \begin{Verbatim}[commandchars=\\\{\}]
{\color{outcolor}Out[{\color{outcolor}45}]:} complex
\end{Verbatim}
        
    \begin{Verbatim}[commandchars=\\\{\}]
{\color{incolor}In [{\color{incolor}46}]:} \PY{k}{print}\PY{p}{(}\PY{n}{x}\PY{p}{)}
\end{Verbatim}

    \begin{Verbatim}[commandchars=\\\{\}]
(1-1j)
    \end{Verbatim}

    \begin{Verbatim}[commandchars=\\\{\}]
{\color{incolor}In [{\color{incolor}47}]:} \PY{k}{print}\PY{p}{(}\PY{n}{x}\PY{o}{.}\PY{n}{real}\PY{p}{,} \PY{n}{x}\PY{o}{.}\PY{n}{imag}\PY{p}{)}
\end{Verbatim}

    \begin{Verbatim}[commandchars=\\\{\}]
(1.0, -1.0)
    \end{Verbatim}

    \begin{Verbatim}[commandchars=\\\{\}]
{\color{incolor}In [{\color{incolor}48}]:} \PY{k+kn}{import} \PY{n+nn}{types}
         
         \PY{c}{\PYZsh{} print all types defined in the `types` module}
         \PY{k}{print}\PY{p}{(}\PY{n+nb}{dir}\PY{p}{(}\PY{n}{types}\PY{p}{)}\PY{p}{)}
\end{Verbatim}

    \begin{Verbatim}[commandchars=\\\{\}]
['BooleanType', 'BufferType', 'BuiltinFunctionType', 'BuiltinMethodType', 'ClassType', 'CodeType', 'ComplexType', 'DictProxyType', 'DictType', 'DictionaryType', 'EllipsisType', 'FileType', 'FloatType', 'FrameType', 'FunctionType', 'GeneratorType', 'GetSetDescriptorType', 'InstanceType', 'IntType', 'LambdaType', 'ListType', 'LongType', 'MemberDescriptorType', 'MethodType', 'ModuleType', 'NoneType', 'NotImplementedType', 'ObjectType', 'SliceType', 'StringType', 'StringTypes', 'TracebackType', 'TupleType', 'TypeType', 'UnboundMethodType', 'UnicodeType', 'XRangeType', '\_\_builtins\_\_', '\_\_doc\_\_', '\_\_file\_\_', '\_\_name\_\_', '\_\_package\_\_']
    \end{Verbatim}

    \begin{Verbatim}[commandchars=\\\{\}]
{\color{incolor}In [{\color{incolor}49}]:} \PY{n}{x} \PY{o}{=} \PY{l+m+mf}{1.0}
         
         \PY{c}{\PYZsh{} check if the variable x is a float}
         \PY{n+nb}{type}\PY{p}{(}\PY{n}{x}\PY{p}{)} \PY{o+ow}{is} \PY{n+nb}{float}
\end{Verbatim}

            \begin{Verbatim}[commandchars=\\\{\}]
{\color{outcolor}Out[{\color{outcolor}49}]:} True
\end{Verbatim}
        
    \begin{Verbatim}[commandchars=\\\{\}]
{\color{incolor}In [{\color{incolor}50}]:} \PY{c}{\PYZsh{} check if the variable x is an int}
         \PY{n+nb}{type}\PY{p}{(}\PY{n}{x}\PY{p}{)} \PY{o+ow}{is} \PY{n+nb}{int}
\end{Verbatim}

            \begin{Verbatim}[commandchars=\\\{\}]
{\color{outcolor}Out[{\color{outcolor}50}]:} False
\end{Verbatim}
        
    We can also use the \texttt{isinstance} method for testing types of
variables:

    \begin{Verbatim}[commandchars=\\\{\}]
{\color{incolor}In [{\color{incolor}51}]:} \PY{n+nb}{isinstance}\PY{p}{(}\PY{n}{x}\PY{p}{,} \PY{n+nb}{float}\PY{p}{)}
\end{Verbatim}

            \begin{Verbatim}[commandchars=\\\{\}]
{\color{outcolor}Out[{\color{outcolor}51}]:} True
\end{Verbatim}
        
    \subsubsection{Type casting}\label{type-casting}

    \begin{Verbatim}[commandchars=\\\{\}]
{\color{incolor}In [{\color{incolor}52}]:} \PY{n}{x} \PY{o}{=} \PY{l+m+mf}{1.5}
         
         \PY{k}{print}\PY{p}{(}\PY{n}{x}\PY{p}{,} \PY{n+nb}{type}\PY{p}{(}\PY{n}{x}\PY{p}{)}\PY{p}{)}
\end{Verbatim}

    \begin{Verbatim}[commandchars=\\\{\}]
(1.5, <type 'float'>)
    \end{Verbatim}

    \begin{Verbatim}[commandchars=\\\{\}]
{\color{incolor}In [{\color{incolor}53}]:} \PY{n}{x} \PY{o}{=} \PY{n+nb}{int}\PY{p}{(}\PY{n}{x}\PY{p}{)}
         
         \PY{k}{print}\PY{p}{(}\PY{n}{x}\PY{p}{,} \PY{n+nb}{type}\PY{p}{(}\PY{n}{x}\PY{p}{)}\PY{p}{)}
\end{Verbatim}

    \begin{Verbatim}[commandchars=\\\{\}]
(1, <type 'int'>)
    \end{Verbatim}

    \begin{Verbatim}[commandchars=\\\{\}]
{\color{incolor}In [{\color{incolor}54}]:} \PY{n}{z} \PY{o}{=} \PY{n+nb}{complex}\PY{p}{(}\PY{n}{x}\PY{p}{)}
         
         \PY{k}{print}\PY{p}{(}\PY{n}{z}\PY{p}{,} \PY{n+nb}{type}\PY{p}{(}\PY{n}{z}\PY{p}{)}\PY{p}{)}
\end{Verbatim}

    \begin{Verbatim}[commandchars=\\\{\}]
((1+0j), <type 'complex'>)
    \end{Verbatim}

    \begin{Verbatim}[commandchars=\\\{\}]
{\color{incolor}In [{\color{incolor}55}]:} \PY{n}{x} \PY{o}{=} \PY{n+nb}{float}\PY{p}{(}\PY{n}{z}\PY{p}{)}
\end{Verbatim}

    \begin{Verbatim}[commandchars=\\\{\}]

        ---------------------------------------------------------------------------
    TypeError                                 Traceback (most recent call last)

        <ipython-input-55-e719cc7b3e96> in <module>()
    ----> 1 x = float(z)
    

        TypeError: can't convert complex to float

    \end{Verbatim}

    Complex variables cannot be cast to floats or integers. We need to use
\texttt{z.real} or \texttt{z.imag} to extract the part of the complex
number we want:

    \begin{Verbatim}[commandchars=\\\{\}]
{\color{incolor}In [{\color{incolor}56}]:} \PY{n}{y} \PY{o}{=} \PY{n+nb}{bool}\PY{p}{(}\PY{n}{z}\PY{o}{.}\PY{n}{real}\PY{p}{)}
         
         \PY{k}{print}\PY{p}{(}\PY{n}{z}\PY{o}{.}\PY{n}{real}\PY{p}{,} \PY{l+s}{\PYZdq{}}\PY{l+s}{ \PYZhy{}\PYZgt{} }\PY{l+s}{\PYZdq{}}\PY{p}{,} \PY{n}{y}\PY{p}{,} \PY{n+nb}{type}\PY{p}{(}\PY{n}{y}\PY{p}{)}\PY{p}{)}
         
         \PY{n}{y} \PY{o}{=} \PY{n+nb}{bool}\PY{p}{(}\PY{n}{z}\PY{o}{.}\PY{n}{imag}\PY{p}{)}
         
         \PY{k}{print}\PY{p}{(}\PY{n}{z}\PY{o}{.}\PY{n}{imag}\PY{p}{,} \PY{l+s}{\PYZdq{}}\PY{l+s}{ \PYZhy{}\PYZgt{} }\PY{l+s}{\PYZdq{}}\PY{p}{,} \PY{n}{y}\PY{p}{,} \PY{n+nb}{type}\PY{p}{(}\PY{n}{y}\PY{p}{)}\PY{p}{)}
\end{Verbatim}

    \begin{Verbatim}[commandchars=\\\{\}]
(1.0, ' -> ', True, <type 'bool'>)
(0.0, ' -> ', False, <type 'bool'>)
    \end{Verbatim}

    

    \subsection{Operators and comparisons}\label{operators-and-comparisons}

Most operators and comparisons in Python work as one would expect:

\begin{itemize}
\itemsep1pt\parskip0pt\parsep0pt
\item
  Arithmetic operators \texttt{+}, \texttt{-}, \texttt{*}, \texttt{/},
  \texttt{//} (integer division), '**' power
\end{itemize}

    \begin{Verbatim}[commandchars=\\\{\}]
{\color{incolor}In [{\color{incolor}57}]:} \PY{l+m+mi}{1} \PY{o}{+} \PY{l+m+mi}{2}\PY{p}{,} \PY{l+m+mi}{1} \PY{o}{\PYZhy{}} \PY{l+m+mi}{2}\PY{p}{,} \PY{l+m+mi}{1} \PY{o}{*} \PY{l+m+mi}{2}\PY{p}{,} \PY{l+m+mi}{1} \PY{o}{/} \PY{l+m+mi}{2}
\end{Verbatim}

            \begin{Verbatim}[commandchars=\\\{\}]
{\color{outcolor}Out[{\color{outcolor}57}]:} (3, -1, 2, 0)
\end{Verbatim}
        
    \begin{Verbatim}[commandchars=\\\{\}]
{\color{incolor}In [{\color{incolor}58}]:} \PY{l+m+mf}{1.0} \PY{o}{+} \PY{l+m+mf}{2.0}\PY{p}{,} \PY{l+m+mf}{1.0} \PY{o}{\PYZhy{}} \PY{l+m+mf}{2.0}\PY{p}{,} \PY{l+m+mf}{1.0} \PY{o}{*} \PY{l+m+mf}{2.0}\PY{p}{,} \PY{l+m+mf}{1.0} \PY{o}{/} \PY{l+m+mf}{2.0}
\end{Verbatim}

            \begin{Verbatim}[commandchars=\\\{\}]
{\color{outcolor}Out[{\color{outcolor}58}]:} (3.0, -1.0, 2.0, 0.5)
\end{Verbatim}
        
    \begin{Verbatim}[commandchars=\\\{\}]
{\color{incolor}In [{\color{incolor}59}]:} \PY{l+m+mf}{3.0} \PY{o}{/}\PY{o}{/} \PY{l+m+mf}{2.0}
\end{Verbatim}

            \begin{Verbatim}[commandchars=\\\{\}]
{\color{outcolor}Out[{\color{outcolor}59}]:} 1.0
\end{Verbatim}
        
    \begin{Verbatim}[commandchars=\\\{\}]
{\color{incolor}In [{\color{incolor}60}]:} \PY{l+m+mi}{2} \PY{o}{*}\PY{o}{*} \PY{l+m+mi}{2}
\end{Verbatim}

            \begin{Verbatim}[commandchars=\\\{\}]
{\color{outcolor}Out[{\color{outcolor}60}]:} 4
\end{Verbatim}
        
    \begin{itemize}
\itemsep1pt\parskip0pt\parsep0pt
\item
  The boolean operators are spelled out as words \texttt{and},
  \texttt{not}, \texttt{or}.
\end{itemize}

    \begin{Verbatim}[commandchars=\\\{\}]
{\color{incolor}In [{\color{incolor}61}]:} \PY{n+nb+bp}{True} \PY{o+ow}{and} \PY{n+nb+bp}{False}
\end{Verbatim}

            \begin{Verbatim}[commandchars=\\\{\}]
{\color{outcolor}Out[{\color{outcolor}61}]:} False
\end{Verbatim}
        
    \begin{Verbatim}[commandchars=\\\{\}]
{\color{incolor}In [{\color{incolor}62}]:} \PY{o+ow}{not} \PY{n+nb+bp}{False}
\end{Verbatim}

            \begin{Verbatim}[commandchars=\\\{\}]
{\color{outcolor}Out[{\color{outcolor}62}]:} True
\end{Verbatim}
        
    \begin{Verbatim}[commandchars=\\\{\}]
{\color{incolor}In [{\color{incolor}63}]:} \PY{n+nb+bp}{True} \PY{o+ow}{or} \PY{n+nb+bp}{False}
\end{Verbatim}

            \begin{Verbatim}[commandchars=\\\{\}]
{\color{outcolor}Out[{\color{outcolor}63}]:} True
\end{Verbatim}
        
    \begin{itemize}
\itemsep1pt\parskip0pt\parsep0pt
\item
  Comparison operators \texttt{\textgreater{}}, \texttt{\textless{}},
  \texttt{\textgreater{}=} (greater or equal), \texttt{\textless{}=}
  (less or equal), \texttt{==} equality, \texttt{is} identical.
\end{itemize}

    \begin{Verbatim}[commandchars=\\\{\}]
{\color{incolor}In [{\color{incolor}64}]:} \PY{l+m+mi}{2} \PY{o}{\PYZgt{}} \PY{l+m+mi}{1}\PY{p}{,} \PY{l+m+mi}{2} \PY{o}{\PYZlt{}} \PY{l+m+mi}{1}
\end{Verbatim}

            \begin{Verbatim}[commandchars=\\\{\}]
{\color{outcolor}Out[{\color{outcolor}64}]:} (True, False)
\end{Verbatim}
        
    \begin{Verbatim}[commandchars=\\\{\}]
{\color{incolor}In [{\color{incolor}65}]:} \PY{l+m+mi}{2} \PY{o}{\PYZgt{}} \PY{l+m+mi}{2}\PY{p}{,} \PY{l+m+mi}{2} \PY{o}{\PYZlt{}} \PY{l+m+mi}{2}
\end{Verbatim}

            \begin{Verbatim}[commandchars=\\\{\}]
{\color{outcolor}Out[{\color{outcolor}65}]:} (False, False)
\end{Verbatim}
        
    \begin{Verbatim}[commandchars=\\\{\}]
{\color{incolor}In [{\color{incolor}66}]:} \PY{l+m+mi}{2} \PY{o}{\PYZgt{}}\PY{o}{=} \PY{l+m+mi}{2}\PY{p}{,} \PY{l+m+mi}{2} \PY{o}{\PYZlt{}}\PY{o}{=} \PY{l+m+mi}{2}
\end{Verbatim}

            \begin{Verbatim}[commandchars=\\\{\}]
{\color{outcolor}Out[{\color{outcolor}66}]:} (True, True)
\end{Verbatim}
        
    \begin{Verbatim}[commandchars=\\\{\}]
{\color{incolor}In [{\color{incolor}67}]:} \PY{c}{\PYZsh{} equality}
         \PY{p}{[}\PY{l+m+mi}{1}\PY{p}{,}\PY{l+m+mi}{2}\PY{p}{]} \PY{o}{==} \PY{p}{[}\PY{l+m+mi}{1}\PY{p}{,}\PY{l+m+mi}{2}\PY{p}{]}
\end{Verbatim}

            \begin{Verbatim}[commandchars=\\\{\}]
{\color{outcolor}Out[{\color{outcolor}67}]:} True
\end{Verbatim}
        
    \begin{Verbatim}[commandchars=\\\{\}]
{\color{incolor}In [{\color{incolor}68}]:} \PY{c}{\PYZsh{} objects identical?}
         \PY{n}{l1} \PY{o}{=} \PY{n}{l2} \PY{o}{=} \PY{p}{[}\PY{l+m+mi}{1}\PY{p}{,}\PY{l+m+mi}{2}\PY{p}{]}
         
         \PY{n}{l1} \PY{o+ow}{is} \PY{n}{l2}
\end{Verbatim}

            \begin{Verbatim}[commandchars=\\\{\}]
{\color{outcolor}Out[{\color{outcolor}68}]:} True
\end{Verbatim}
        
    

    \subsection{Compound types: Strings, List and
dictionaries}\label{compound-types-strings-list-and-dictionaries}

\subsubsection{Strings}\label{strings}

Strings are the variable type that is used for storing text messages.

    \begin{Verbatim}[commandchars=\\\{\}]
{\color{incolor}In [{\color{incolor}69}]:} \PY{n}{s} \PY{o}{=} \PY{l+s}{\PYZdq{}}\PY{l+s}{Hello world}\PY{l+s}{\PYZdq{}}
         \PY{n+nb}{type}\PY{p}{(}\PY{n}{s}\PY{p}{)}
\end{Verbatim}

            \begin{Verbatim}[commandchars=\\\{\}]
{\color{outcolor}Out[{\color{outcolor}69}]:} str
\end{Verbatim}
        
    \begin{Verbatim}[commandchars=\\\{\}]
{\color{incolor}In [{\color{incolor}70}]:} \PY{c}{\PYZsh{} length of the string: the number of characters}
         \PY{n+nb}{len}\PY{p}{(}\PY{n}{s}\PY{p}{)}
\end{Verbatim}

            \begin{Verbatim}[commandchars=\\\{\}]
{\color{outcolor}Out[{\color{outcolor}70}]:} 11
\end{Verbatim}
        
    \begin{Verbatim}[commandchars=\\\{\}]
{\color{incolor}In [{\color{incolor}71}]:} \PY{c}{\PYZsh{} replace a substring in a string with somethign else}
         \PY{n}{s2} \PY{o}{=} \PY{n}{s}\PY{o}{.}\PY{n}{replace}\PY{p}{(}\PY{l+s}{\PYZdq{}}\PY{l+s}{world}\PY{l+s}{\PYZdq{}}\PY{p}{,} \PY{l+s}{\PYZdq{}}\PY{l+s}{test}\PY{l+s}{\PYZdq{}}\PY{p}{)}
         \PY{k}{print}\PY{p}{(}\PY{n}{s2}\PY{p}{)}
\end{Verbatim}

    \begin{Verbatim}[commandchars=\\\{\}]
Hello test
    \end{Verbatim}

    We can index a character in a string using \texttt{{[}{]}}:

    \begin{Verbatim}[commandchars=\\\{\}]
{\color{incolor}In [{\color{incolor}72}]:} \PY{n}{s}\PY{p}{[}\PY{l+m+mi}{0}\PY{p}{]}
\end{Verbatim}

            \begin{Verbatim}[commandchars=\\\{\}]
{\color{outcolor}Out[{\color{outcolor}72}]:} 'H'
\end{Verbatim}
        
    We can extract a part of a string using the syntax
\texttt{{[}start:stop{]}}, which extracts characters between index
\texttt{start} and \texttt{stop}:

    \begin{Verbatim}[commandchars=\\\{\}]
{\color{incolor}In [{\color{incolor}73}]:} \PY{n}{s}\PY{p}{[}\PY{l+m+mi}{0}\PY{p}{:}\PY{l+m+mi}{5}\PY{p}{]}
\end{Verbatim}

            \begin{Verbatim}[commandchars=\\\{\}]
{\color{outcolor}Out[{\color{outcolor}73}]:} 'Hello'
\end{Verbatim}
        
    If we omit either (or both) of \texttt{start} or \texttt{stop} from
\texttt{{[}start:stop{]}}, the default is the beginning and the end of
the string, respectively:

    \begin{Verbatim}[commandchars=\\\{\}]
{\color{incolor}In [{\color{incolor}74}]:} \PY{n}{s}\PY{p}{[}\PY{p}{:}\PY{l+m+mi}{5}\PY{p}{]}
\end{Verbatim}

            \begin{Verbatim}[commandchars=\\\{\}]
{\color{outcolor}Out[{\color{outcolor}74}]:} 'Hello'
\end{Verbatim}
        
    \begin{Verbatim}[commandchars=\\\{\}]
{\color{incolor}In [{\color{incolor}75}]:} \PY{n}{s}\PY{p}{[}\PY{l+m+mi}{6}\PY{p}{:}\PY{p}{]}
\end{Verbatim}

            \begin{Verbatim}[commandchars=\\\{\}]
{\color{outcolor}Out[{\color{outcolor}75}]:} 'world'
\end{Verbatim}
        
    \begin{Verbatim}[commandchars=\\\{\}]
{\color{incolor}In [{\color{incolor}76}]:} \PY{n}{s}\PY{p}{[}\PY{p}{:}\PY{p}{]}
\end{Verbatim}

            \begin{Verbatim}[commandchars=\\\{\}]
{\color{outcolor}Out[{\color{outcolor}76}]:} 'Hello world'
\end{Verbatim}
        
    We can also define the step size using the syntax
\texttt{{[}start:end:step{]}} (the default value for \texttt{step} is 1,
as we saw above):

    \begin{Verbatim}[commandchars=\\\{\}]
{\color{incolor}In [{\color{incolor}77}]:} \PY{n}{s}\PY{p}{[}\PY{p}{:}\PY{p}{:}\PY{l+m+mi}{1}\PY{p}{]}
\end{Verbatim}

            \begin{Verbatim}[commandchars=\\\{\}]
{\color{outcolor}Out[{\color{outcolor}77}]:} 'Hello world'
\end{Verbatim}
        
    \begin{Verbatim}[commandchars=\\\{\}]
{\color{incolor}In [{\color{incolor}78}]:} \PY{n}{s}\PY{p}{[}\PY{p}{:}\PY{p}{:}\PY{l+m+mi}{2}\PY{p}{]}
\end{Verbatim}

            \begin{Verbatim}[commandchars=\\\{\}]
{\color{outcolor}Out[{\color{outcolor}78}]:} 'Hlowrd'
\end{Verbatim}
        
    \paragraph{String formatting examples}\label{string-formatting-examples}

    \begin{Verbatim}[commandchars=\\\{\}]
{\color{incolor}In [{\color{incolor}79}]:} \PY{k}{print}\PY{p}{(}\PY{l+s}{\PYZdq{}}\PY{l+s}{str1}\PY{l+s}{\PYZdq{}} \PY{o}{+} \PY{l+s}{\PYZdq{}}\PY{l+s}{str2}\PY{l+s}{\PYZdq{}} \PY{o}{+} \PY{l+s}{\PYZdq{}}\PY{l+s}{str3}\PY{l+s}{\PYZdq{}}\PY{p}{)} \PY{c}{\PYZsh{} strings added with + are concatenated without space}
\end{Verbatim}

    \begin{Verbatim}[commandchars=\\\{\}]
str1str2str3
    \end{Verbatim}

    \begin{Verbatim}[commandchars=\\\{\}]
{\color{incolor}In [{\color{incolor}80}]:} \PY{k}{print}\PY{p}{(}\PY{l+s}{\PYZdq{}}\PY{l+s}{value = }\PY{l+s+si}{\PYZpc{}f}\PY{l+s}{\PYZdq{}} \PY{o}{\PYZpc{}} \PY{l+m+mf}{1.0}\PY{p}{)}
\end{Verbatim}

    \begin{Verbatim}[commandchars=\\\{\}]
value = 1.000000
    \end{Verbatim}

    \begin{Verbatim}[commandchars=\\\{\}]
{\color{incolor}In [{\color{incolor}81}]:} \PY{n}{s2} \PY{o}{=} \PY{l+s}{\PYZdq{}}\PY{l+s}{value1 = }\PY{l+s+si}{\PYZpc{}.2f}\PY{l+s}{ value2 = }\PY{l+s+si}{\PYZpc{}d}\PY{l+s}{\PYZdq{}} \PY{o}{\PYZpc{}} \PY{p}{(}\PY{l+m+mf}{3.1415}\PY{p}{,} \PY{l+m+mf}{1.5}\PY{p}{)}
         
         \PY{k}{print}\PY{p}{(}\PY{n}{s2}\PY{p}{)}
\end{Verbatim}

    \begin{Verbatim}[commandchars=\\\{\}]
value1 = 3.14 value2 = 1
    \end{Verbatim}

    \begin{Verbatim}[commandchars=\\\{\}]
{\color{incolor}In [{\color{incolor}82}]:} \PY{n}{s3} \PY{o}{=} \PY{l+s}{\PYZsq{}}\PY{l+s}{value1 = \PYZob{}0\PYZcb{}, value2 = \PYZob{}1\PYZcb{}}\PY{l+s}{\PYZsq{}}\PY{o}{.}\PY{n}{format}\PY{p}{(}\PY{l+m+mf}{3.1415}\PY{p}{,} \PY{l+m+mf}{1.5}\PY{p}{)}
         
         \PY{k}{print}\PY{p}{(}\PY{n}{s3}\PY{p}{)}
\end{Verbatim}

    \begin{Verbatim}[commandchars=\\\{\}]
value1 = 3.1415, value2 = 1.5
    \end{Verbatim}

    \subsubsection{List}\label{list}

Lists are very similar to strings, except that each element can be of
any type.

The syntax for creating lists in Python is \texttt{{[}...{]}}:

    \begin{Verbatim}[commandchars=\\\{\}]
{\color{incolor}In [{\color{incolor}83}]:} \PY{n}{l} \PY{o}{=} \PY{p}{[}\PY{l+m+mi}{1}\PY{p}{,}\PY{l+m+mi}{2}\PY{p}{,}\PY{l+m+mi}{3}\PY{p}{,}\PY{l+m+mi}{4}\PY{p}{]}
         
         \PY{k}{print}\PY{p}{(}\PY{n+nb}{type}\PY{p}{(}\PY{n}{l}\PY{p}{)}\PY{p}{)}
         \PY{k}{print}\PY{p}{(}\PY{n}{l}\PY{p}{)}
\end{Verbatim}

    \begin{Verbatim}[commandchars=\\\{\}]
<type 'list'>
[1, 2, 3, 4]
    \end{Verbatim}

    \begin{Verbatim}[commandchars=\\\{\}]
{\color{incolor}In [{\color{incolor}84}]:} \PY{k}{print}\PY{p}{(}\PY{n}{l}\PY{p}{)}
         
         \PY{k}{print}\PY{p}{(}\PY{n}{l}\PY{p}{[}\PY{l+m+mi}{1}\PY{p}{:}\PY{l+m+mi}{3}\PY{p}{]}\PY{p}{)}
         
         \PY{k}{print}\PY{p}{(}\PY{n}{l}\PY{p}{[}\PY{p}{:}\PY{p}{:}\PY{l+m+mi}{2}\PY{p}{]}\PY{p}{)}
\end{Verbatim}

    \begin{Verbatim}[commandchars=\\\{\}]
[1, 2, 3, 4]
[2, 3]
[1, 3]
    \end{Verbatim}

    \begin{Verbatim}[commandchars=\\\{\}]
{\color{incolor}In [{\color{incolor}85}]:} \PY{n}{l}\PY{p}{[}\PY{l+m+mi}{0}\PY{p}{]}
\end{Verbatim}

            \begin{Verbatim}[commandchars=\\\{\}]
{\color{outcolor}Out[{\color{outcolor}85}]:} 1
\end{Verbatim}
        
    \begin{Verbatim}[commandchars=\\\{\}]
{\color{incolor}In [{\color{incolor}86}]:} \PY{n}{l} \PY{o}{=} \PY{p}{[}\PY{l+m+mi}{1}\PY{p}{,} \PY{l+s}{\PYZsq{}}\PY{l+s}{a}\PY{l+s}{\PYZsq{}}\PY{p}{,} \PY{l+m+mf}{1.0}\PY{p}{,} \PY{l+m+mi}{1}\PY{o}{\PYZhy{}}\PY{l+m+mi}{1j}\PY{p}{]}
         
         \PY{k}{print}\PY{p}{(}\PY{n}{l}\PY{p}{)}
\end{Verbatim}

    \begin{Verbatim}[commandchars=\\\{\}]
[1, 'a', 1.0, (1-1j)]
    \end{Verbatim}

    \begin{Verbatim}[commandchars=\\\{\}]
{\color{incolor}In [{\color{incolor}87}]:} \PY{n}{start} \PY{o}{=} \PY{l+m+mi}{10}
         \PY{n}{stop} \PY{o}{=} \PY{l+m+mi}{30}
         \PY{n}{step} \PY{o}{=} \PY{l+m+mi}{2}
         
         \PY{n+nb}{range}\PY{p}{(}\PY{n}{start}\PY{p}{,} \PY{n}{stop}\PY{p}{,} \PY{n}{step}\PY{p}{)}
\end{Verbatim}

            \begin{Verbatim}[commandchars=\\\{\}]
{\color{outcolor}Out[{\color{outcolor}87}]:} [10, 12, 14, 16, 18, 20, 22, 24, 26, 28]
\end{Verbatim}
        
    \begin{Verbatim}[commandchars=\\\{\}]
{\color{incolor}In [{\color{incolor}88}]:} \PY{n+nb}{list}\PY{p}{(}\PY{n+nb}{range}\PY{p}{(}\PY{o}{\PYZhy{}}\PY{l+m+mi}{10}\PY{p}{,} \PY{l+m+mi}{10}\PY{p}{)}\PY{p}{)}
\end{Verbatim}

            \begin{Verbatim}[commandchars=\\\{\}]
{\color{outcolor}Out[{\color{outcolor}88}]:} [-10, -9, -8, -7, -6, -5, -4, -3, -2, -1, 0, 1, 2, 3, 4, 5, 6, 7, 8, 9]
\end{Verbatim}
        
    \begin{Verbatim}[commandchars=\\\{\}]
{\color{incolor}In [{\color{incolor}89}]:} \PY{n}{s}
\end{Verbatim}

            \begin{Verbatim}[commandchars=\\\{\}]
{\color{outcolor}Out[{\color{outcolor}89}]:} 'Hello world'
\end{Verbatim}
        
    \begin{Verbatim}[commandchars=\\\{\}]
{\color{incolor}In [{\color{incolor}90}]:} \PY{c}{\PYZsh{} convert a string to a list:}
         
         \PY{n}{s2} \PY{o}{=} \PY{n+nb}{list}\PY{p}{(}\PY{n}{s}\PY{p}{)}
         
         \PY{n}{s2}
\end{Verbatim}

            \begin{Verbatim}[commandchars=\\\{\}]
{\color{outcolor}Out[{\color{outcolor}90}]:} ['H', 'e', 'l', 'l', 'o', ' ', 'w', 'o', 'r', 'l', 'd']
\end{Verbatim}
        
    \begin{Verbatim}[commandchars=\\\{\}]
{\color{incolor}In [{\color{incolor}91}]:} \PY{c}{\PYZsh{} sorting lists}
         \PY{n}{s2}\PY{o}{.}\PY{n}{sort}\PY{p}{(}\PY{p}{)}
         
         \PY{k}{print}\PY{p}{(}\PY{n}{s2}\PY{p}{)}
\end{Verbatim}

    \begin{Verbatim}[commandchars=\\\{\}]
[' ', 'H', 'd', 'e', 'l', 'l', 'l', 'o', 'o', 'r', 'w']
    \end{Verbatim}

    \paragraph{Adding, inserting, modifying, and removing elements from
lists}\label{adding-inserting-modifying-and-removing-elements-from-lists}

    \begin{Verbatim}[commandchars=\\\{\}]
{\color{incolor}In [{\color{incolor}94}]:} \PY{c}{\PYZsh{} create a new empty list}
         \PY{n}{l} \PY{o}{=} \PY{p}{[}\PY{p}{]}
         
         \PY{c}{\PYZsh{} add an elements using `append`}
         \PY{n}{l}\PY{o}{.}\PY{n}{append}\PY{p}{(}\PY{l+s}{\PYZdq{}}\PY{l+s}{A}\PY{l+s}{\PYZdq{}}\PY{p}{)}
         \PY{n}{l}\PY{o}{.}\PY{n}{append}\PY{p}{(}\PY{l+s}{\PYZdq{}}\PY{l+s}{d}\PY{l+s}{\PYZdq{}}\PY{p}{)}
         \PY{n}{l}\PY{o}{.}\PY{n}{append}\PY{p}{(}\PY{l+s}{\PYZdq{}}\PY{l+s}{d}\PY{l+s}{\PYZdq{}}\PY{p}{)}
         
         \PY{k}{print}\PY{p}{(}\PY{n}{l}\PY{p}{)}
\end{Verbatim}

    \begin{Verbatim}[commandchars=\\\{\}]
['A', 'd', 'd']
    \end{Verbatim}

    \begin{Verbatim}[commandchars=\\\{\}]
{\color{incolor}In [{\color{incolor}95}]:} \PY{n}{l}\PY{p}{[}\PY{l+m+mi}{1}\PY{p}{]} \PY{o}{=} \PY{l+s}{\PYZdq{}}\PY{l+s}{p}\PY{l+s}{\PYZdq{}}
         \PY{n}{l}\PY{p}{[}\PY{l+m+mi}{2}\PY{p}{]} \PY{o}{=} \PY{l+s}{\PYZdq{}}\PY{l+s}{p}\PY{l+s}{\PYZdq{}}
         
         \PY{k}{print}\PY{p}{(}\PY{n}{l}\PY{p}{)}
\end{Verbatim}

    \begin{Verbatim}[commandchars=\\\{\}]
['A', 'p', 'p']
    \end{Verbatim}

    \begin{Verbatim}[commandchars=\\\{\}]
{\color{incolor}In [{\color{incolor}96}]:} \PY{n}{l}\PY{p}{[}\PY{l+m+mi}{1}\PY{p}{:}\PY{l+m+mi}{3}\PY{p}{]} \PY{o}{=} \PY{p}{[}\PY{l+s}{\PYZdq{}}\PY{l+s}{d}\PY{l+s}{\PYZdq{}}\PY{p}{,} \PY{l+s}{\PYZdq{}}\PY{l+s}{d}\PY{l+s}{\PYZdq{}}\PY{p}{]}
         
         \PY{k}{print}\PY{p}{(}\PY{n}{l}\PY{p}{)}
\end{Verbatim}

    \begin{Verbatim}[commandchars=\\\{\}]
['A', 'd', 'd']
    \end{Verbatim}

    \begin{Verbatim}[commandchars=\\\{\}]
{\color{incolor}In [{\color{incolor}97}]:} \PY{n}{l}\PY{o}{.}\PY{n}{insert}\PY{p}{(}\PY{l+m+mi}{0}\PY{p}{,} \PY{l+s}{\PYZdq{}}\PY{l+s}{i}\PY{l+s}{\PYZdq{}}\PY{p}{)}
         \PY{n}{l}\PY{o}{.}\PY{n}{insert}\PY{p}{(}\PY{l+m+mi}{1}\PY{p}{,} \PY{l+s}{\PYZdq{}}\PY{l+s}{n}\PY{l+s}{\PYZdq{}}\PY{p}{)}
         \PY{n}{l}\PY{o}{.}\PY{n}{insert}\PY{p}{(}\PY{l+m+mi}{2}\PY{p}{,} \PY{l+s}{\PYZdq{}}\PY{l+s}{s}\PY{l+s}{\PYZdq{}}\PY{p}{)}
         \PY{n}{l}\PY{o}{.}\PY{n}{insert}\PY{p}{(}\PY{l+m+mi}{3}\PY{p}{,} \PY{l+s}{\PYZdq{}}\PY{l+s}{e}\PY{l+s}{\PYZdq{}}\PY{p}{)}
         \PY{n}{l}\PY{o}{.}\PY{n}{insert}\PY{p}{(}\PY{l+m+mi}{4}\PY{p}{,} \PY{l+s}{\PYZdq{}}\PY{l+s}{r}\PY{l+s}{\PYZdq{}}\PY{p}{)}
         \PY{n}{l}\PY{o}{.}\PY{n}{insert}\PY{p}{(}\PY{l+m+mi}{5}\PY{p}{,} \PY{l+s}{\PYZdq{}}\PY{l+s}{t}\PY{l+s}{\PYZdq{}}\PY{p}{)}
         
         \PY{k}{print}\PY{p}{(}\PY{n}{l}\PY{p}{)}
\end{Verbatim}

    \begin{Verbatim}[commandchars=\\\{\}]
['i', 'n', 's', 'e', 'r', 't', 'A', 'd', 'd']
    \end{Verbatim}

    \begin{Verbatim}[commandchars=\\\{\}]
{\color{incolor}In [{\color{incolor}98}]:} \PY{n}{l}\PY{o}{.}\PY{n}{remove}\PY{p}{(}\PY{l+s}{\PYZdq{}}\PY{l+s}{A}\PY{l+s}{\PYZdq{}}\PY{p}{)}
         
         \PY{k}{print}\PY{p}{(}\PY{n}{l}\PY{p}{)}
\end{Verbatim}

    \begin{Verbatim}[commandchars=\\\{\}]
['i', 'n', 's', 'e', 'r', 't', 'd', 'd']
    \end{Verbatim}

    \begin{Verbatim}[commandchars=\\\{\}]
{\color{incolor}In [{\color{incolor}99}]:} \PY{k}{del} \PY{n}{l}\PY{p}{[}\PY{l+m+mi}{7}\PY{p}{]}
         \PY{k}{del} \PY{n}{l}\PY{p}{[}\PY{l+m+mi}{6}\PY{p}{]}
         
         \PY{k}{print}\PY{p}{(}\PY{n}{l}\PY{p}{)}
\end{Verbatim}

    \begin{Verbatim}[commandchars=\\\{\}]
['i', 'n', 's', 'e', 'r', 't']
    \end{Verbatim}

    \subsubsection{Tuples}\label{tuples}

Tuples are like lists, except that they cannot be modified once created,
that is they are \emph{immutable}.

In Python, tuples are created using the syntax \texttt{(..., ..., ...)},
or even \texttt{..., ...}:

    \begin{Verbatim}[commandchars=\\\{\}]
{\color{incolor}In [{\color{incolor}100}]:} \PY{n}{point} \PY{o}{=} \PY{p}{(}\PY{l+m+mi}{10}\PY{p}{,} \PY{l+m+mi}{20}\PY{p}{)}
          
          \PY{k}{print}\PY{p}{(}\PY{n}{point}\PY{p}{,} \PY{n+nb}{type}\PY{p}{(}\PY{n}{point}\PY{p}{)}\PY{p}{)}
\end{Verbatim}

    \begin{Verbatim}[commandchars=\\\{\}]
((10, 20), <type 'tuple'>)
    \end{Verbatim}

    \begin{Verbatim}[commandchars=\\\{\}]
{\color{incolor}In [{\color{incolor}101}]:} \PY{n}{point} \PY{o}{=} \PY{l+m+mi}{10}\PY{p}{,} \PY{l+m+mi}{20}
          
          \PY{k}{print}\PY{p}{(}\PY{n}{point}\PY{p}{,} \PY{n+nb}{type}\PY{p}{(}\PY{n}{point}\PY{p}{)}\PY{p}{)}
\end{Verbatim}

    \begin{Verbatim}[commandchars=\\\{\}]
((10, 20), <type 'tuple'>)
    \end{Verbatim}

    We can unpack a tuple by assigning it to a comma-separated list of
variables:

    \begin{Verbatim}[commandchars=\\\{\}]
{\color{incolor}In [{\color{incolor}102}]:} \PY{n}{x}\PY{p}{,} \PY{n}{y} \PY{o}{=} \PY{n}{point}
          
          \PY{k}{print}\PY{p}{(}\PY{l+s}{\PYZdq{}}\PY{l+s}{x =}\PY{l+s}{\PYZdq{}}\PY{p}{,} \PY{n}{x}\PY{p}{)}
          \PY{k}{print}\PY{p}{(}\PY{l+s}{\PYZdq{}}\PY{l+s}{y =}\PY{l+s}{\PYZdq{}}\PY{p}{,} \PY{n}{y}\PY{p}{)}
\end{Verbatim}

    \begin{Verbatim}[commandchars=\\\{\}]
('x =', 10)
('y =', 20)
    \end{Verbatim}

    If we try to assign a new value to an element in a tuple we get an
error:

    \begin{Verbatim}[commandchars=\\\{\}]
{\color{incolor}In [{\color{incolor}103}]:} \PY{n}{point}\PY{p}{[}\PY{l+m+mi}{0}\PY{p}{]} \PY{o}{=} \PY{l+m+mi}{20}
\end{Verbatim}

    \begin{Verbatim}[commandchars=\\\{\}]

        ---------------------------------------------------------------------------
    TypeError                                 Traceback (most recent call last)

        <ipython-input-103-ac1c641a5dca> in <module>()
    ----> 1 point[0] = 20
    

        TypeError: 'tuple' object does not support item assignment

    \end{Verbatim}

    \subsubsection{Dictionaries}\label{dictionaries}

Dictionaries are also like lists, except that each element is a
key-value pair. The syntax for dictionaries is
\texttt{\{key1 : value1, ...\}}:

    \begin{Verbatim}[commandchars=\\\{\}]
{\color{incolor}In [{\color{incolor}104}]:} \PY{n}{params} \PY{o}{=} \PY{p}{\PYZob{}}\PY{l+s}{\PYZdq{}}\PY{l+s}{parameter1}\PY{l+s}{\PYZdq{}} \PY{p}{:} \PY{l+m+mf}{1.0}\PY{p}{,}
                    \PY{l+s}{\PYZdq{}}\PY{l+s}{parameter2}\PY{l+s}{\PYZdq{}} \PY{p}{:} \PY{l+m+mf}{2.0}\PY{p}{,}
                    \PY{l+s}{\PYZdq{}}\PY{l+s}{parameter3}\PY{l+s}{\PYZdq{}} \PY{p}{:} \PY{l+m+mf}{3.0}\PY{p}{,}\PY{p}{\PYZcb{}}
          
          \PY{k}{print}\PY{p}{(}\PY{n+nb}{type}\PY{p}{(}\PY{n}{params}\PY{p}{)}\PY{p}{)}
          \PY{k}{print}\PY{p}{(}\PY{n}{params}\PY{p}{)}
\end{Verbatim}

    \begin{Verbatim}[commandchars=\\\{\}]
<type 'dict'>
\{'parameter1': 1.0, 'parameter3': 3.0, 'parameter2': 2.0\}
    \end{Verbatim}

    \begin{Verbatim}[commandchars=\\\{\}]
{\color{incolor}In [{\color{incolor}105}]:} \PY{k}{print}\PY{p}{(}\PY{l+s}{\PYZdq{}}\PY{l+s}{parameter1 = }\PY{l+s}{\PYZdq{}} \PY{o}{+} \PY{n+nb}{str}\PY{p}{(}\PY{n}{params}\PY{p}{[}\PY{l+s}{\PYZdq{}}\PY{l+s}{parameter1}\PY{l+s}{\PYZdq{}}\PY{p}{]}\PY{p}{)}\PY{p}{)}
          \PY{k}{print}\PY{p}{(}\PY{l+s}{\PYZdq{}}\PY{l+s}{parameter2 = }\PY{l+s}{\PYZdq{}} \PY{o}{+} \PY{n+nb}{str}\PY{p}{(}\PY{n}{params}\PY{p}{[}\PY{l+s}{\PYZdq{}}\PY{l+s}{parameter2}\PY{l+s}{\PYZdq{}}\PY{p}{]}\PY{p}{)}\PY{p}{)}
          \PY{k}{print}\PY{p}{(}\PY{l+s}{\PYZdq{}}\PY{l+s}{parameter3 = }\PY{l+s}{\PYZdq{}} \PY{o}{+} \PY{n+nb}{str}\PY{p}{(}\PY{n}{params}\PY{p}{[}\PY{l+s}{\PYZdq{}}\PY{l+s}{parameter3}\PY{l+s}{\PYZdq{}}\PY{p}{]}\PY{p}{)}\PY{p}{)}
\end{Verbatim}

    \begin{Verbatim}[commandchars=\\\{\}]
parameter1 = 1.0
parameter2 = 2.0
parameter3 = 3.0
    \end{Verbatim}

    \begin{Verbatim}[commandchars=\\\{\}]
{\color{incolor}In [{\color{incolor}106}]:} \PY{n}{params}\PY{p}{[}\PY{l+s}{\PYZdq{}}\PY{l+s}{parameter1}\PY{l+s}{\PYZdq{}}\PY{p}{]} \PY{o}{=} \PY{l+s}{\PYZdq{}}\PY{l+s}{A}\PY{l+s}{\PYZdq{}}
          \PY{n}{params}\PY{p}{[}\PY{l+s}{\PYZdq{}}\PY{l+s}{parameter2}\PY{l+s}{\PYZdq{}}\PY{p}{]} \PY{o}{=} \PY{l+s}{\PYZdq{}}\PY{l+s}{B}\PY{l+s}{\PYZdq{}}
          
          \PY{c}{\PYZsh{} add a new entry}
          \PY{n}{params}\PY{p}{[}\PY{l+s}{\PYZdq{}}\PY{l+s}{parameter4}\PY{l+s}{\PYZdq{}}\PY{p}{]} \PY{o}{=} \PY{l+s}{\PYZdq{}}\PY{l+s}{D}\PY{l+s}{\PYZdq{}}
          
          \PY{k}{print}\PY{p}{(}\PY{l+s}{\PYZdq{}}\PY{l+s}{parameter1 = }\PY{l+s}{\PYZdq{}} \PY{o}{+} \PY{n+nb}{str}\PY{p}{(}\PY{n}{params}\PY{p}{[}\PY{l+s}{\PYZdq{}}\PY{l+s}{parameter1}\PY{l+s}{\PYZdq{}}\PY{p}{]}\PY{p}{)}\PY{p}{)}
          \PY{k}{print}\PY{p}{(}\PY{l+s}{\PYZdq{}}\PY{l+s}{parameter2 = }\PY{l+s}{\PYZdq{}} \PY{o}{+} \PY{n+nb}{str}\PY{p}{(}\PY{n}{params}\PY{p}{[}\PY{l+s}{\PYZdq{}}\PY{l+s}{parameter2}\PY{l+s}{\PYZdq{}}\PY{p}{]}\PY{p}{)}\PY{p}{)}
          \PY{k}{print}\PY{p}{(}\PY{l+s}{\PYZdq{}}\PY{l+s}{parameter3 = }\PY{l+s}{\PYZdq{}} \PY{o}{+} \PY{n+nb}{str}\PY{p}{(}\PY{n}{params}\PY{p}{[}\PY{l+s}{\PYZdq{}}\PY{l+s}{parameter3}\PY{l+s}{\PYZdq{}}\PY{p}{]}\PY{p}{)}\PY{p}{)}
          \PY{k}{print}\PY{p}{(}\PY{l+s}{\PYZdq{}}\PY{l+s}{parameter4 = }\PY{l+s}{\PYZdq{}} \PY{o}{+} \PY{n+nb}{str}\PY{p}{(}\PY{n}{params}\PY{p}{[}\PY{l+s}{\PYZdq{}}\PY{l+s}{parameter4}\PY{l+s}{\PYZdq{}}\PY{p}{]}\PY{p}{)}\PY{p}{)}
\end{Verbatim}

    \begin{Verbatim}[commandchars=\\\{\}]
parameter1 = A
parameter2 = B
parameter3 = 3.0
parameter4 = D
    \end{Verbatim}

    

    \subsection{Control Flow}\label{control-flow}

    \subsubsection{Conditional statements: if, elif,
else}\label{conditional-statements-if-elif-else}

The Python syntax for conditional execution of code use the keywords
\texttt{if}, \texttt{elif} (else if), \texttt{else}:

    \begin{Verbatim}[commandchars=\\\{\}]
{\color{incolor}In [{\color{incolor}107}]:} \PY{n}{statement1} \PY{o}{=} \PY{n+nb+bp}{False}
          \PY{n}{statement2} \PY{o}{=} \PY{n+nb+bp}{False}
          
          \PY{k}{if} \PY{n}{statement1}\PY{p}{:}
              \PY{k}{print}\PY{p}{(}\PY{l+s}{\PYZdq{}}\PY{l+s}{statement1 is True}\PY{l+s}{\PYZdq{}}\PY{p}{)}
              
          \PY{k}{elif} \PY{n}{statement2}\PY{p}{:}
              \PY{k}{print}\PY{p}{(}\PY{l+s}{\PYZdq{}}\PY{l+s}{statement2 is True}\PY{l+s}{\PYZdq{}}\PY{p}{)}
              
          \PY{k}{else}\PY{p}{:}
              \PY{k}{print}\PY{p}{(}\PY{l+s}{\PYZdq{}}\PY{l+s}{statement1 and statement2 are False}\PY{l+s}{\PYZdq{}}\PY{p}{)}
\end{Verbatim}

    \begin{Verbatim}[commandchars=\\\{\}]
statement1 and statement2 are False
    \end{Verbatim}

    In Python, the extent of a code block is defined by the indentation
level (usually a tab or say four white spaces). This means that we have
to be careful to indent our code correctly, or else we will get syntax
errors.

\textbf{Examples:}

    \begin{Verbatim}[commandchars=\\\{\}]
{\color{incolor}In [{\color{incolor}108}]:} \PY{n}{statement1} \PY{o}{=} \PY{n}{statement2} \PY{o}{=} \PY{n+nb+bp}{True}
          
          \PY{k}{if} \PY{n}{statement1}\PY{p}{:}
              \PY{k}{if} \PY{n}{statement2}\PY{p}{:}
                  \PY{k}{print}\PY{p}{(}\PY{l+s}{\PYZdq{}}\PY{l+s}{both statement1 and statement2 are True}\PY{l+s}{\PYZdq{}}\PY{p}{)}
\end{Verbatim}

    \begin{Verbatim}[commandchars=\\\{\}]
both statement1 and statement2 are True
    \end{Verbatim}

    \begin{Verbatim}[commandchars=\\\{\}]
{\color{incolor}In [{\color{incolor}109}]:} \PY{c}{\PYZsh{} Bad indentation!}
          \PY{k}{if} \PY{n}{statement1}\PY{p}{:}
              \PY{k}{if} \PY{n}{statement2}\PY{p}{:}
                  \PY{k}{print}\PY{p}{(}\PY{l+s}{\PYZdq{}}\PY{l+s}{both statement1 and statement2 are True}\PY{l+s}{\PYZdq{}}\PY{p}{)}  \PY{c}{\PYZsh{} this line is not properly indented}
\end{Verbatim}

    \begin{Verbatim}[commandchars=\\\{\}]
both statement1 and statement2 are True
    \end{Verbatim}

    \begin{Verbatim}[commandchars=\\\{\}]
{\color{incolor}In [{\color{incolor}110}]:} \PY{n}{statement1} \PY{o}{=} \PY{n+nb+bp}{False} 
          
          \PY{k}{if} \PY{n}{statement1}\PY{p}{:}
              \PY{k}{print}\PY{p}{(}\PY{l+s}{\PYZdq{}}\PY{l+s}{printed if statement1 is True}\PY{l+s}{\PYZdq{}}\PY{p}{)}
              
              \PY{k}{print}\PY{p}{(}\PY{l+s}{\PYZdq{}}\PY{l+s}{still inside the if block}\PY{l+s}{\PYZdq{}}\PY{p}{)}
\end{Verbatim}

    \begin{Verbatim}[commandchars=\\\{\}]
{\color{incolor}In [{\color{incolor}111}]:} \PY{k}{if} \PY{n}{statement1}\PY{p}{:}
              \PY{k}{print}\PY{p}{(}\PY{l+s}{\PYZdq{}}\PY{l+s}{printed if statement1 is True}\PY{l+s}{\PYZdq{}}\PY{p}{)}
              
          \PY{k}{print}\PY{p}{(}\PY{l+s}{\PYZdq{}}\PY{l+s}{now outside the if block}\PY{l+s}{\PYZdq{}}\PY{p}{)}
\end{Verbatim}

    \begin{Verbatim}[commandchars=\\\{\}]
now outside the if block
    \end{Verbatim}

    

    \subsection{Loops}\label{loops}

In Python, loops can be programmed in a number of different ways. The
most common is the \texttt{for} loop, which is used together with
iterable objects, such as lists. The basic syntax is:

\textbf{\texttt{for} loops}:

    \begin{Verbatim}[commandchars=\\\{\}]
{\color{incolor}In [{\color{incolor}112}]:} \PY{k}{for} \PY{n}{x} \PY{o+ow}{in} \PY{p}{[}\PY{l+m+mi}{1}\PY{p}{,}\PY{l+m+mi}{2}\PY{p}{,}\PY{l+m+mi}{3}\PY{p}{]}\PY{p}{:}
              \PY{k}{print}\PY{p}{(}\PY{n}{x}\PY{p}{)}
\end{Verbatim}

    \begin{Verbatim}[commandchars=\\\{\}]
1
2
3
    \end{Verbatim}

    The \texttt{for} loop iterates over the elements of the supplied list,
and executes the containing block once for each element. Any kind of
list can be used in the \texttt{for} loop. For example:

    \begin{Verbatim}[commandchars=\\\{\}]
{\color{incolor}In [{\color{incolor}113}]:} \PY{k}{for} \PY{n}{x} \PY{o+ow}{in} \PY{n+nb}{range}\PY{p}{(}\PY{l+m+mi}{4}\PY{p}{)}\PY{p}{:} \PY{c}{\PYZsh{} by default range start at 0}
              \PY{k}{print}\PY{p}{(}\PY{n}{x}\PY{p}{)}
\end{Verbatim}

    \begin{Verbatim}[commandchars=\\\{\}]
0
1
2
3
    \end{Verbatim}

    Note: \texttt{range(4)} does not include 4 !

    \begin{Verbatim}[commandchars=\\\{\}]
{\color{incolor}In [{\color{incolor}114}]:} \PY{k}{for} \PY{n}{x} \PY{o+ow}{in} \PY{n+nb}{range}\PY{p}{(}\PY{o}{\PYZhy{}}\PY{l+m+mi}{3}\PY{p}{,}\PY{l+m+mi}{3}\PY{p}{)}\PY{p}{:}
              \PY{k}{print}\PY{p}{(}\PY{n}{x}\PY{p}{)}
\end{Verbatim}

    \begin{Verbatim}[commandchars=\\\{\}]
-3
-2
-1
0
1
2
    \end{Verbatim}

    \begin{Verbatim}[commandchars=\\\{\}]
{\color{incolor}In [{\color{incolor}115}]:} \PY{k}{for} \PY{n}{word} \PY{o+ow}{in} \PY{p}{[}\PY{l+s}{\PYZdq{}}\PY{l+s}{scientific}\PY{l+s}{\PYZdq{}}\PY{p}{,} \PY{l+s}{\PYZdq{}}\PY{l+s}{computing}\PY{l+s}{\PYZdq{}}\PY{p}{,} \PY{l+s}{\PYZdq{}}\PY{l+s}{with}\PY{l+s}{\PYZdq{}}\PY{p}{,} \PY{l+s}{\PYZdq{}}\PY{l+s}{python}\PY{l+s}{\PYZdq{}}\PY{p}{]}\PY{p}{:}
              \PY{k}{print}\PY{p}{(}\PY{n}{word}\PY{p}{)}
\end{Verbatim}

    \begin{Verbatim}[commandchars=\\\{\}]
scientific
computing
with
python
    \end{Verbatim}

    To iterate over key-value pairs of a dictionary:

    \begin{Verbatim}[commandchars=\\\{\}]
{\color{incolor}In [{\color{incolor}116}]:} \PY{k}{for} \PY{n}{key}\PY{p}{,} \PY{n}{value} \PY{o+ow}{in} \PY{n}{params}\PY{o}{.}\PY{n}{items}\PY{p}{(}\PY{p}{)}\PY{p}{:}
              \PY{k}{print}\PY{p}{(}\PY{n}{key} \PY{o}{+} \PY{l+s}{\PYZdq{}}\PY{l+s}{ = }\PY{l+s}{\PYZdq{}} \PY{o}{+} \PY{n+nb}{str}\PY{p}{(}\PY{n}{value}\PY{p}{)}\PY{p}{)}
\end{Verbatim}

    \begin{Verbatim}[commandchars=\\\{\}]
parameter4 = D
parameter1 = A
parameter3 = 3.0
parameter2 = B
    \end{Verbatim}

    Sometimes it is useful to have access to the indices of the values when
iterating over a list. We can use the \texttt{enumerate} function for
this:

    \begin{Verbatim}[commandchars=\\\{\}]
{\color{incolor}In [{\color{incolor}117}]:} \PY{k}{for} \PY{n}{idx}\PY{p}{,} \PY{n}{x} \PY{o+ow}{in} \PY{n+nb}{enumerate}\PY{p}{(}\PY{n+nb}{range}\PY{p}{(}\PY{o}{\PYZhy{}}\PY{l+m+mi}{3}\PY{p}{,}\PY{l+m+mi}{3}\PY{p}{)}\PY{p}{)}\PY{p}{:}
              \PY{k}{print}\PY{p}{(}\PY{n}{idx}\PY{p}{,} \PY{n}{x}\PY{p}{)}
\end{Verbatim}

    \begin{Verbatim}[commandchars=\\\{\}]
(0, -3)
(1, -2)
(2, -1)
(3, 0)
(4, 1)
(5, 2)
    \end{Verbatim}

    \textbf{List comprehensions: Creating lists using \texttt{for} loops}:

    \begin{Verbatim}[commandchars=\\\{\}]
{\color{incolor}In [{\color{incolor}118}]:} \PY{n}{l1} \PY{o}{=} \PY{p}{[}\PY{n}{x}\PY{o}{*}\PY{o}{*}\PY{l+m+mi}{2} \PY{k}{for} \PY{n}{x} \PY{o+ow}{in} \PY{n+nb}{range}\PY{p}{(}\PY{l+m+mi}{0}\PY{p}{,}\PY{l+m+mi}{5}\PY{p}{)}\PY{p}{]}
          
          \PY{k}{print}\PY{p}{(}\PY{n}{l1}\PY{p}{)}
\end{Verbatim}

    \begin{Verbatim}[commandchars=\\\{\}]
[0, 1, 4, 9, 16]
    \end{Verbatim}

    \textbf{\texttt{while} loops}:

    \begin{Verbatim}[commandchars=\\\{\}]
{\color{incolor}In [{\color{incolor}119}]:} \PY{n}{i} \PY{o}{=} \PY{l+m+mi}{0}
          
          \PY{k}{while} \PY{n}{i} \PY{o}{\PYZlt{}} \PY{l+m+mi}{5}\PY{p}{:}
              \PY{k}{print}\PY{p}{(}\PY{n}{i}\PY{p}{)}
              
              \PY{n}{i} \PY{o}{=} \PY{n}{i} \PY{o}{+} \PY{l+m+mi}{1}
              
          \PY{k}{print}\PY{p}{(}\PY{l+s}{\PYZdq{}}\PY{l+s}{done}\PY{l+s}{\PYZdq{}}\PY{p}{)}
\end{Verbatim}

    \begin{Verbatim}[commandchars=\\\{\}]
0
1
2
3
4
done
    \end{Verbatim}

    

    \subsection{Functions}\label{functions}

A function in Python is defined using the keyword \texttt{def}, followed
by a function name, a signature within parenthises \texttt{()}, and a
colon \texttt{:}. The following code, with one additional level of
indentation, is the function body.

    \begin{Verbatim}[commandchars=\\\{\}]
{\color{incolor}In [{\color{incolor}120}]:} \PY{k}{def} \PY{n+nf}{func0}\PY{p}{(}\PY{p}{)}\PY{p}{:}   
              \PY{k}{print}\PY{p}{(}\PY{l+s}{\PYZdq{}}\PY{l+s}{test}\PY{l+s}{\PYZdq{}}\PY{p}{)}
\end{Verbatim}

    \begin{Verbatim}[commandchars=\\\{\}]
{\color{incolor}In [{\color{incolor}121}]:} \PY{n}{func0}\PY{p}{(}\PY{p}{)}
\end{Verbatim}

    \begin{Verbatim}[commandchars=\\\{\}]
test
    \end{Verbatim}

    Optionally, but highly recommended, we can define a so called
``docstring'', which is a description of the functions purpose and
behavior. The docstring should be located after the function definition
and before the code in the function body.

    \begin{Verbatim}[commandchars=\\\{\}]
{\color{incolor}In [{\color{incolor}122}]:} \PY{k}{def} \PY{n+nf}{func1}\PY{p}{(}\PY{n}{s}\PY{p}{)}\PY{p}{:}
              \PY{l+s+sd}{\PYZdq{}\PYZdq{}\PYZdq{}}
          \PY{l+s+sd}{    Print a string \PYZsq{}s\PYZsq{} and tell how many characters it has    }
          \PY{l+s+sd}{    \PYZdq{}\PYZdq{}\PYZdq{}}
              
              \PY{k}{print}\PY{p}{(}\PY{n}{s} \PY{o}{+} \PY{l+s}{\PYZdq{}}\PY{l+s}{ has }\PY{l+s}{\PYZdq{}} \PY{o}{+} \PY{n+nb}{str}\PY{p}{(}\PY{n+nb}{len}\PY{p}{(}\PY{n}{s}\PY{p}{)}\PY{p}{)} \PY{o}{+} \PY{l+s}{\PYZdq{}}\PY{l+s}{ characters}\PY{l+s}{\PYZdq{}}\PY{p}{)}
\end{Verbatim}

    \begin{Verbatim}[commandchars=\\\{\}]
{\color{incolor}In [{\color{incolor}123}]:} \PY{n}{help}\PY{p}{(}\PY{n}{func1}\PY{p}{)}
\end{Verbatim}

    \begin{Verbatim}[commandchars=\\\{\}]
Help on function func1 in module \_\_main\_\_:

func1(s)
    Print a string 's' and tell how many characters it has
    \end{Verbatim}

    \begin{Verbatim}[commandchars=\\\{\}]
{\color{incolor}In [{\color{incolor}124}]:} \PY{n}{func1}\PY{p}{(}\PY{l+s}{\PYZdq{}}\PY{l+s}{test}\PY{l+s}{\PYZdq{}}\PY{p}{)}
\end{Verbatim}

    \begin{Verbatim}[commandchars=\\\{\}]
test has 4 characters
    \end{Verbatim}

    Functions that returns a value use the \texttt{return} keyword:

    \begin{Verbatim}[commandchars=\\\{\}]
{\color{incolor}In [{\color{incolor}125}]:} \PY{k}{def} \PY{n+nf}{square}\PY{p}{(}\PY{n}{x}\PY{p}{)}\PY{p}{:}
              \PY{l+s+sd}{\PYZdq{}\PYZdq{}\PYZdq{}}
          \PY{l+s+sd}{    Return the square of x.}
          \PY{l+s+sd}{    \PYZdq{}\PYZdq{}\PYZdq{}}
              \PY{k}{return} \PY{n}{x} \PY{o}{*}\PY{o}{*} \PY{l+m+mi}{2}
\end{Verbatim}

    \begin{Verbatim}[commandchars=\\\{\}]
{\color{incolor}In [{\color{incolor}126}]:} \PY{n}{square}\PY{p}{(}\PY{l+m+mi}{4}\PY{p}{)}
\end{Verbatim}

            \begin{Verbatim}[commandchars=\\\{\}]
{\color{outcolor}Out[{\color{outcolor}126}]:} 16
\end{Verbatim}
        
    We can return multiple values from a function using tuples (see above):

    \begin{Verbatim}[commandchars=\\\{\}]
{\color{incolor}In [{\color{incolor}127}]:} \PY{k}{def} \PY{n+nf}{powers}\PY{p}{(}\PY{n}{x}\PY{p}{)}\PY{p}{:}
              \PY{l+s+sd}{\PYZdq{}\PYZdq{}\PYZdq{}}
          \PY{l+s+sd}{    Return a few powers of x.}
          \PY{l+s+sd}{    \PYZdq{}\PYZdq{}\PYZdq{}}
              \PY{k}{return} \PY{n}{x} \PY{o}{*}\PY{o}{*} \PY{l+m+mi}{2}\PY{p}{,} \PY{n}{x} \PY{o}{*}\PY{o}{*} \PY{l+m+mi}{3}\PY{p}{,} \PY{n}{x} \PY{o}{*}\PY{o}{*} \PY{l+m+mi}{4}
\end{Verbatim}

    \begin{Verbatim}[commandchars=\\\{\}]
{\color{incolor}In [{\color{incolor}128}]:} \PY{n}{powers}\PY{p}{(}\PY{l+m+mi}{3}\PY{p}{)}
\end{Verbatim}

            \begin{Verbatim}[commandchars=\\\{\}]
{\color{outcolor}Out[{\color{outcolor}128}]:} (9, 27, 81)
\end{Verbatim}
        
    \begin{Verbatim}[commandchars=\\\{\}]
{\color{incolor}In [{\color{incolor}129}]:} \PY{n}{x2}\PY{p}{,} \PY{n}{x3}\PY{p}{,} \PY{n}{x4} \PY{o}{=} \PY{n}{powers}\PY{p}{(}\PY{l+m+mi}{3}\PY{p}{)}
          
          \PY{k}{print}\PY{p}{(}\PY{n}{x3}\PY{p}{)}
\end{Verbatim}

    \begin{Verbatim}[commandchars=\\\{\}]
27
    \end{Verbatim}

    \subsubsection{Default argument and keyword
arguments}\label{default-argument-and-keyword-arguments}

    \begin{Verbatim}[commandchars=\\\{\}]
{\color{incolor}In [{\color{incolor}130}]:} \PY{k}{def} \PY{n+nf}{myfunc}\PY{p}{(}\PY{n}{x}\PY{p}{,} \PY{n}{p}\PY{o}{=}\PY{l+m+mi}{2}\PY{p}{,} \PY{n}{debug}\PY{o}{=}\PY{n+nb+bp}{False}\PY{p}{)}\PY{p}{:}
              \PY{k}{if} \PY{n}{debug}\PY{p}{:}
                  \PY{k}{print}\PY{p}{(}\PY{l+s}{\PYZdq{}}\PY{l+s}{evaluating myfunc for x = }\PY{l+s}{\PYZdq{}} \PY{o}{+} \PY{n+nb}{str}\PY{p}{(}\PY{n}{x}\PY{p}{)} \PY{o}{+} \PY{l+s}{\PYZdq{}}\PY{l+s}{ using exponent p = }\PY{l+s}{\PYZdq{}} \PY{o}{+} \PY{n+nb}{str}\PY{p}{(}\PY{n}{p}\PY{p}{)}\PY{p}{)}
              \PY{k}{return} \PY{n}{x}\PY{o}{*}\PY{o}{*}\PY{n}{p}
\end{Verbatim}

    \begin{Verbatim}[commandchars=\\\{\}]
{\color{incolor}In [{\color{incolor}131}]:} \PY{n}{myfunc}\PY{p}{(}\PY{l+m+mi}{5}\PY{p}{)}
\end{Verbatim}

            \begin{Verbatim}[commandchars=\\\{\}]
{\color{outcolor}Out[{\color{outcolor}131}]:} 25
\end{Verbatim}
        
    \begin{Verbatim}[commandchars=\\\{\}]
{\color{incolor}In [{\color{incolor}132}]:} \PY{n}{myfunc}\PY{p}{(}\PY{l+m+mi}{5}\PY{p}{,} \PY{n}{debug}\PY{o}{=}\PY{n+nb+bp}{True}\PY{p}{)}
\end{Verbatim}

    \begin{Verbatim}[commandchars=\\\{\}]
evaluating myfunc for x = 5 using exponent p = 2
    \end{Verbatim}

            \begin{Verbatim}[commandchars=\\\{\}]
{\color{outcolor}Out[{\color{outcolor}132}]:} 25
\end{Verbatim}
        
    \begin{Verbatim}[commandchars=\\\{\}]
{\color{incolor}In [{\color{incolor}133}]:} \PY{n}{myfunc}\PY{p}{(}\PY{n}{p}\PY{o}{=}\PY{l+m+mi}{3}\PY{p}{,} \PY{n}{debug}\PY{o}{=}\PY{n+nb+bp}{True}\PY{p}{,} \PY{n}{x}\PY{o}{=}\PY{l+m+mi}{7}\PY{p}{)}
\end{Verbatim}

    \begin{Verbatim}[commandchars=\\\{\}]
evaluating myfunc for x = 7 using exponent p = 3
    \end{Verbatim}

            \begin{Verbatim}[commandchars=\\\{\}]
{\color{outcolor}Out[{\color{outcolor}133}]:} 343
\end{Verbatim}
        
    \subsubsection{Unnamed functions (lambda
function)}\label{unnamed-functions-lambda-function}

In Python we can also create unnamed functions, using the
\texttt{lambda} keyword:

    \begin{Verbatim}[commandchars=\\\{\}]
{\color{incolor}In [{\color{incolor}134}]:} \PY{n}{f1} \PY{o}{=} \PY{k}{lambda} \PY{n}{x}\PY{p}{:} \PY{n}{x}\PY{o}{*}\PY{o}{*}\PY{l+m+mi}{2}
              
          \PY{c}{\PYZsh{} is equivalent to }
          
          \PY{k}{def} \PY{n+nf}{f2}\PY{p}{(}\PY{n}{x}\PY{p}{)}\PY{p}{:}
              \PY{k}{return} \PY{n}{x}\PY{o}{*}\PY{o}{*}\PY{l+m+mi}{2}
\end{Verbatim}

    \begin{Verbatim}[commandchars=\\\{\}]
{\color{incolor}In [{\color{incolor}135}]:} \PY{n}{f1}\PY{p}{(}\PY{l+m+mi}{2}\PY{p}{)}\PY{p}{,} \PY{n}{f2}\PY{p}{(}\PY{l+m+mi}{2}\PY{p}{)}
\end{Verbatim}

            \begin{Verbatim}[commandchars=\\\{\}]
{\color{outcolor}Out[{\color{outcolor}135}]:} (4, 4)
\end{Verbatim}
        
    This technique is useful for exmample when we want to pass a simple
function as an argument to another function, like this:

    \begin{Verbatim}[commandchars=\\\{\}]
{\color{incolor}In [{\color{incolor}136}]:} \PY{c}{\PYZsh{} map is a built\PYZhy{}in python function}
          \PY{n+nb}{map}\PY{p}{(}\PY{k}{lambda} \PY{n}{x}\PY{p}{:} \PY{n}{x}\PY{o}{*}\PY{o}{*}\PY{l+m+mi}{2}\PY{p}{,} \PY{n+nb}{range}\PY{p}{(}\PY{o}{\PYZhy{}}\PY{l+m+mi}{3}\PY{p}{,}\PY{l+m+mi}{4}\PY{p}{)}\PY{p}{)}
\end{Verbatim}

            \begin{Verbatim}[commandchars=\\\{\}]
{\color{outcolor}Out[{\color{outcolor}136}]:} [9, 4, 1, 0, 1, 4, 9]
\end{Verbatim}
        
    

    \subsection{Classes}\label{classes}

Classes are the key features of object-oriented programming. A class is
a structure for representing an object and the operations that can be
performed on the object.

In Python a class can contain \emph{attributes} (variables) and
\emph{methods} (functions).

In python a class is defined almost like a function, but using the
\texttt{class} keyword, and the class definition usually contains a
number of class method definitions (a function in a class).

\begin{itemize}
\item
  Each class method should have an argument \texttt{self} as it first
  argument. This object is a self-reference.
\item
  Some class method names have special meaning, for example:
\item
  \texttt{\_\_init\_\_}: The name of the method that is invoked when the
  object is first created.
\item
  \texttt{\_\_str\_\_} : A method that is invoked when a simple string
  representation of the class is needed, as for example when printed.
\end{itemize}

    \begin{Verbatim}[commandchars=\\\{\}]
{\color{incolor}In [{\color{incolor}137}]:} \PY{k}{class} \PY{n+nc}{Point}\PY{p}{:}
              \PY{l+s+sd}{\PYZdq{}\PYZdq{}\PYZdq{}}
          \PY{l+s+sd}{    Simple class for representing a point in a Cartesian coordinate system.}
          \PY{l+s+sd}{    \PYZdq{}\PYZdq{}\PYZdq{}}
              
              \PY{k}{def} \PY{n+nf}{\PYZus{}\PYZus{}init\PYZus{}\PYZus{}}\PY{p}{(}\PY{n+nb+bp}{self}\PY{p}{,} \PY{n}{x}\PY{p}{,} \PY{n}{y}\PY{p}{)}\PY{p}{:}
                  \PY{l+s+sd}{\PYZdq{}\PYZdq{}\PYZdq{}}
          \PY{l+s+sd}{        Create a new Point at x, y.}
          \PY{l+s+sd}{        \PYZdq{}\PYZdq{}\PYZdq{}}
                  \PY{n+nb+bp}{self}\PY{o}{.}\PY{n}{x} \PY{o}{=} \PY{n}{x}
                  \PY{n+nb+bp}{self}\PY{o}{.}\PY{n}{y} \PY{o}{=} \PY{n}{y}
                  
              \PY{k}{def} \PY{n+nf}{translate}\PY{p}{(}\PY{n+nb+bp}{self}\PY{p}{,} \PY{n}{dx}\PY{p}{,} \PY{n}{dy}\PY{p}{)}\PY{p}{:}
                  \PY{l+s+sd}{\PYZdq{}\PYZdq{}\PYZdq{}}
          \PY{l+s+sd}{        Translate the point by dx and dy in the x and y direction.}
          \PY{l+s+sd}{        \PYZdq{}\PYZdq{}\PYZdq{}}
                  \PY{n+nb+bp}{self}\PY{o}{.}\PY{n}{x} \PY{o}{+}\PY{o}{=} \PY{n}{dx}
                  \PY{n+nb+bp}{self}\PY{o}{.}\PY{n}{y} \PY{o}{+}\PY{o}{=} \PY{n}{dy}
                  
              \PY{k}{def} \PY{n+nf}{\PYZus{}\PYZus{}str\PYZus{}\PYZus{}}\PY{p}{(}\PY{n+nb+bp}{self}\PY{p}{)}\PY{p}{:}
                  \PY{k}{return}\PY{p}{(}\PY{l+s}{\PYZdq{}}\PY{l+s}{Point at [}\PY{l+s+si}{\PYZpc{}f}\PY{l+s}{, }\PY{l+s+si}{\PYZpc{}f}\PY{l+s}{]}\PY{l+s}{\PYZdq{}} \PY{o}{\PYZpc{}} \PY{p}{(}\PY{n+nb+bp}{self}\PY{o}{.}\PY{n}{x}\PY{p}{,} \PY{n+nb+bp}{self}\PY{o}{.}\PY{n}{y}\PY{p}{)}\PY{p}{)}
\end{Verbatim}

    To create a new instance of a class:

    \begin{Verbatim}[commandchars=\\\{\}]
{\color{incolor}In [{\color{incolor}138}]:} \PY{n}{p1} \PY{o}{=} \PY{n}{Point}\PY{p}{(}\PY{l+m+mi}{0}\PY{p}{,} \PY{l+m+mi}{0}\PY{p}{)} \PY{c}{\PYZsh{} this will invoke the \PYZus{}\PYZus{}init\PYZus{}\PYZus{} method in the Point class}
          
          \PY{k}{print}\PY{p}{(}\PY{n}{p1}\PY{p}{)}         \PY{c}{\PYZsh{} this will invoke the \PYZus{}\PYZus{}str\PYZus{}\PYZus{} method}
\end{Verbatim}

    \begin{Verbatim}[commandchars=\\\{\}]
Point at [0.000000, 0.000000]
    \end{Verbatim}

    \begin{Verbatim}[commandchars=\\\{\}]
{\color{incolor}In [{\color{incolor}139}]:} \PY{n}{p2} \PY{o}{=} \PY{n}{Point}\PY{p}{(}\PY{l+m+mi}{1}\PY{p}{,} \PY{l+m+mi}{1}\PY{p}{)}
          
          \PY{n}{p1}\PY{o}{.}\PY{n}{translate}\PY{p}{(}\PY{l+m+mf}{0.25}\PY{p}{,} \PY{l+m+mf}{1.5}\PY{p}{)}
          
          \PY{k}{print}\PY{p}{(}\PY{n}{p1}\PY{p}{)}
          \PY{k}{print}\PY{p}{(}\PY{n}{p2}\PY{p}{)}
\end{Verbatim}

    \begin{Verbatim}[commandchars=\\\{\}]
Point at [0.250000, 1.500000]
Point at [1.000000, 1.000000]
    \end{Verbatim}

    \subsection{Modules}\label{modules}

One of the most important concepts in good programming is to reuse code
and avoid repetitions.

Consider the following example: the file \texttt{mymodule.py} contains
simple example implementations of a variable, function and a class:

    \begin{Verbatim}[commandchars=\\\{\}]
{\color{incolor}In [{\color{incolor}140}]:} \PY{o}{\PYZpc{}\PYZpc{}}\PY{k}{file} \PY{n}{mymodule}\PY{o}{.}\PY{n}{py}
          \PY{l+s+sd}{\PYZdq{}\PYZdq{}\PYZdq{}}
          \PY{l+s+sd}{Example of a python module. Contains a variable called my\PYZus{}variable,}
          \PY{l+s+sd}{a function called my\PYZus{}function, and a class called MyClass.}
          \PY{l+s+sd}{\PYZdq{}\PYZdq{}\PYZdq{}}
          
          \PY{n}{my\PYZus{}variable} \PY{o}{=} \PY{l+m+mi}{0}
          
          \PY{k}{def} \PY{n+nf}{my\PYZus{}function}\PY{p}{(}\PY{p}{)}\PY{p}{:}
              \PY{l+s+sd}{\PYZdq{}\PYZdq{}\PYZdq{}}
          \PY{l+s+sd}{    Example function}
          \PY{l+s+sd}{    \PYZdq{}\PYZdq{}\PYZdq{}}
              \PY{k}{return} \PY{n}{my\PYZus{}variable}
              
          \PY{k}{class} \PY{n+nc}{MyClass}\PY{p}{:}
              \PY{l+s+sd}{\PYZdq{}\PYZdq{}\PYZdq{}}
          \PY{l+s+sd}{    Example class.}
          \PY{l+s+sd}{    \PYZdq{}\PYZdq{}\PYZdq{}}
          
              \PY{k}{def} \PY{n+nf}{\PYZus{}\PYZus{}init\PYZus{}\PYZus{}}\PY{p}{(}\PY{n+nb+bp}{self}\PY{p}{)}\PY{p}{:}
                  \PY{n+nb+bp}{self}\PY{o}{.}\PY{n}{variable} \PY{o}{=} \PY{n}{my\PYZus{}variable}
                  
              \PY{k}{def} \PY{n+nf}{set\PYZus{}variable}\PY{p}{(}\PY{n+nb+bp}{self}\PY{p}{,} \PY{n}{new\PYZus{}value}\PY{p}{)}\PY{p}{:}
                  \PY{l+s+sd}{\PYZdq{}\PYZdq{}\PYZdq{}}
          \PY{l+s+sd}{        Set self.variable to a new value}
          \PY{l+s+sd}{        \PYZdq{}\PYZdq{}\PYZdq{}}
                  \PY{n+nb+bp}{self}\PY{o}{.}\PY{n}{variable} \PY{o}{=} \PY{n}{new\PYZus{}value}
                  
              \PY{k}{def} \PY{n+nf}{get\PYZus{}variable}\PY{p}{(}\PY{n+nb+bp}{self}\PY{p}{)}\PY{p}{:}
                  \PY{k}{return} \PY{n+nb+bp}{self}\PY{o}{.}\PY{n}{variable}
\end{Verbatim}

    \begin{Verbatim}[commandchars=\\\{\}]
Writing mymodule.py
    \end{Verbatim}

    We can import the module \texttt{mymodule} into our Python program using
\texttt{import}:

    \begin{Verbatim}[commandchars=\\\{\}]
{\color{incolor}In [{\color{incolor}141}]:} \PY{k+kn}{import} \PY{n+nn}{mymodule}
\end{Verbatim}

    Use \texttt{help(module)} to get a summary of what the module provides:

    \begin{Verbatim}[commandchars=\\\{\}]
{\color{incolor}In [{\color{incolor}142}]:} \PY{n}{help}\PY{p}{(}\PY{n}{mymodule}\PY{p}{)}
\end{Verbatim}

    \begin{Verbatim}[commandchars=\\\{\}]
Help on module mymodule:

NAME
    mymodule

FILE
    /home/jpsilva/Lisbon1214/notebooks/mymodule.py

DESCRIPTION
    Example of a python module. Contains a variable called my\_variable,
    a function called my\_function, and a class called MyClass.

CLASSES
    MyClass
    
    class MyClass
     |  Example class.
     |  
     |  Methods defined here:
     |  
     |  \_\_init\_\_(self)
     |  
     |  get\_variable(self)
     |  
     |  set\_variable(self, new\_value)
     |      Set self.variable to a new value

FUNCTIONS
    my\_function()
        Example function

DATA
    my\_variable = 0
    \end{Verbatim}

    \begin{Verbatim}[commandchars=\\\{\}]
{\color{incolor}In [{\color{incolor}143}]:} \PY{n}{mymodule}\PY{o}{.}\PY{n}{my\PYZus{}variable}
\end{Verbatim}

            \begin{Verbatim}[commandchars=\\\{\}]
{\color{outcolor}Out[{\color{outcolor}143}]:} 0
\end{Verbatim}
        
    \begin{Verbatim}[commandchars=\\\{\}]
{\color{incolor}In [{\color{incolor}144}]:} \PY{n}{mymodule}\PY{o}{.}\PY{n}{my\PYZus{}function}\PY{p}{(}\PY{p}{)} 
\end{Verbatim}

            \begin{Verbatim}[commandchars=\\\{\}]
{\color{outcolor}Out[{\color{outcolor}144}]:} 0
\end{Verbatim}
        
    \begin{Verbatim}[commandchars=\\\{\}]
{\color{incolor}In [{\color{incolor}145}]:} \PY{n}{my\PYZus{}class} \PY{o}{=} \PY{n}{mymodule}\PY{o}{.}\PY{n}{MyClass}\PY{p}{(}\PY{p}{)} 
          \PY{n}{my\PYZus{}class}\PY{o}{.}\PY{n}{set\PYZus{}variable}\PY{p}{(}\PY{l+m+mi}{10}\PY{p}{)}
          \PY{n}{my\PYZus{}class}\PY{o}{.}\PY{n}{get\PYZus{}variable}\PY{p}{(}\PY{p}{)}
\end{Verbatim}

            \begin{Verbatim}[commandchars=\\\{\}]
{\color{outcolor}Out[{\color{outcolor}145}]:} 10
\end{Verbatim}
        
    If we make changes to the code in \texttt{mymodule.py}, we need to
reload it using \texttt{reload}:

    \begin{Verbatim}[commandchars=\\\{\}]
{\color{incolor}In [{\color{incolor}146}]:} \PY{n+nb}{reload}\PY{p}{(}\PY{n}{mymodule}\PY{p}{)}
\end{Verbatim}

            \begin{Verbatim}[commandchars=\\\{\}]
{\color{outcolor}Out[{\color{outcolor}146}]:} <module 'mymodule' from 'mymodule.pyc'>
\end{Verbatim}
        
    \subsubsection{Versions}\label{versions}

    \begin{Verbatim}[commandchars=\\\{\}]
{\color{incolor}In [{\color{incolor}147}]:} \PY{o}{\PYZpc{}}\PY{k}{reload\PYZus{}ext} \PY{n}{version\PYZus{}information}
          
          \PY{o}{\PYZpc{}}\PY{k}{version\PYZus{}information} \PY{n}{math}
\end{Verbatim}
\texttt{\color{outcolor}Out[{\color{outcolor}147}]:}
    
    \begin{tabular}{|l|l|}\hline
{\bf Software} & {\bf Version} \\ \hline\hline
Python & 2.7.8 |Anaconda 2.1.0 (64-bit)| (default, Aug 21 2014, 18:22:21) [GCC 4.4.7 20120313 (Red Hat 4.4.7-1)] \\ \hline
IPython & 2.3.0 \\ \hline
OS & posix [linux2] \\ \hline
math & 'module' object has no attribute '__version__' \\ \hline
\hline \multicolumn{2}{|l|}{Fri Dec 05 09:59:20 2014 CET} \\ \hline
\end{tabular}


    

    \begin{Verbatim}[commandchars=\\\{\}]
{\color{incolor}In [{\color{incolor}148}]:} \PY{o}{\PYZpc{}}\PY{k}{who}
\end{Verbatim}

    \begin{Verbatim}[commandchars=\\\{\}]
HTML	 Point	 acos	 acosh	 asin	 asinh	 atan	 atan2	 atanh	 
b1	 b2	 ceil	 copysign	 cos	 cosh	 css\_styling	 degrees	 e	 
erf	 erfc	 exp	 expm1	 f1	 f2	 fabs	 factorial	 floor	 
fmod	 frexp	 fsum	 func0	 func1	 gamma	 hypot	 i	 idx	 
inv	 isinf	 isnan	 key	 keyword	 l	 l1	 l2	 ldexp	 
lgamma	 log	 log10	 log1p	 math	 modf	 my\_class	 my\_variable	 myfunc	 
mymodule	 p1	 p2	 params	 pi	 point	 pow	 powers	 radians	 
s	 s2	 s3	 script\_dir	 sin	 sinh	 sparseinv	 sqrt	 square	 
start	 statement1	 statement2	 step	 stop	 sys	 tan	 tanh	 trunc	 
types	 value	 word	 x	 x2	 x3	 x4	 y	 z
    \end{Verbatim}

    We can find more information about IPython capabilities in
\href{https://github.com/ipython/ipython/tree/1.x/examples/notebooks}{IPython
example notebooks}

    \emph{The full notebook can be downloaded}
\href{https://raw.github.com/PoeticCapybara/Python-Introduction-Zittau/master/Lecture-1-Introduction-to-Python.ipynb}{\emph{here}},
\emph{or viewed statically on}
\href{http://nbviewer.ipython.org/urls/raw.github.com/PoeticCapybara/Python-Introduction-Zittau/master/Lecture-1-Introduction-to-Python.ipynb}{\emph{nbviewer}}

    \begin{Verbatim}[commandchars=\\\{\}]
{\color{incolor}In [{\color{incolor}149}]:} \PY{k+kn}{from} \PY{n+nn}{IPython.core.display} \PY{k+kn}{import} \PY{n}{HTML}
          \PY{k}{def} \PY{n+nf}{css\PYZus{}styling}\PY{p}{(}\PY{p}{)}\PY{p}{:}
              \PY{n}{styles} \PY{o}{=} \PY{n+nb}{open}\PY{p}{(}\PY{l+s}{\PYZdq{}}\PY{l+s}{./styles/custom.css}\PY{l+s}{\PYZdq{}}\PY{p}{,} \PY{l+s}{\PYZdq{}}\PY{l+s}{r}\PY{l+s}{\PYZdq{}}\PY{p}{)}\PY{o}{.}\PY{n}{read}\PY{p}{(}\PY{p}{)}
              \PY{k}{return} \PY{n}{HTML}\PY{p}{(}\PY{n}{styles}\PY{p}{)}
          \PY{n}{css\PYZus{}styling}\PY{p}{(}\PY{p}{)}
\end{Verbatim}

            \begin{Verbatim}[commandchars=\\\{\}]
{\color{outcolor}Out[{\color{outcolor}149}]:} <IPython.core.display.HTML at 0x7faefcf17990>
\end{Verbatim}
        
    \hyperref[Index]{Back to top}

    \begin{Verbatim}[commandchars=\\\{\}]
{\color{incolor}In [{\color{incolor}}]:} 
\end{Verbatim}


    % Add a bibliography block to the postdoc
    
    
    
    \end{document}


%\newpage
%






    
    \section{I.II-Iterables}\label{i.ii-iterables}

    An iterable is a python container with two special methods:
\textbf{iter} and \textbf{next}. There are some built-in iterables like
lists, tuples, dicts, strings, files, etc\ldots{}

As the name hints, an iterable allows iteration over its items, usually
with a for statement:

\begin{verbatim}
for item in container:
     print item    #or do sth else with item
     
\end{verbatim}

Iterating is the process of accessing and returning each item from a
container


    \subparagraph{\textbf{List}}


    % Add contents below.

{\par%
\vspace{-1\baselineskip}%
\needspace{4\baselineskip}}%
\begin{notebookcell}[4]%
\begin{addmargin}[\cellleftmargin]{0em}% left, right
{\smaller%
\par%
%
\vspace{-1\smallerfontscale}%
\begin{Verbatim}[commandchars=\\\{\}]
\PY{n}{my\PYZus{}list} \PY{o}{=} \PY{p}{[}\PY{l+m+mi}{1}\PY{p}{,}\PY{l+m+mi}{2}\PY{p}{,}\PY{l+m+mi}{3}\PY{p}{]}
\PY{k}{for} \PY{n}{i} \PY{o+ow}{in} \PY{n}{my\PYZus{}list}\PY{p}{:}
    \PY{k}{print} \PY{n}{i}
\end{Verbatim}
%
\par%
\vspace{-1\smallerfontscale}}%
\end{addmargin}
\end{notebookcell}

\par\vspace{1\smallerfontscale}%
    \needspace{4\baselineskip}%
    % Only render the prompt if the cell is pyout.  Note, the outputs prompt 
    % block isn't used since we need to check each indiviual output and only
    % add prompts to the pyout ones.
    %
    %
    \begin{addmargin}[\cellleftmargin]{0em}% left, right
    {\smaller%
    \vspace{-1\smallerfontscale}%
    
    \begin{Verbatim}[commandchars=\\\{\}]
1
2
3
    \end{Verbatim}
}%
    \end{addmargin}%
    % Add contents below.

{\par%
\vspace{-1\baselineskip}%
\needspace{4\baselineskip}}%
\begin{notebookcell}[5]%
\begin{addmargin}[\cellleftmargin]{0em}% left, right
{\smaller%
\par%
%
\vspace{-1\smallerfontscale}%
\begin{Verbatim}[commandchars=\\\{\}]
\PY{l+s}{\PYZsq{}}\PY{l+s}{\PYZus{}\PYZus{}iter\PYZus{}\PYZus{}}\PY{l+s}{\PYZsq{}} \PY{o+ow}{in} \PY{n+nb}{dir}\PY{p}{(}\PY{n}{my\PYZus{}list}\PY{p}{)}
\end{Verbatim}
%
\par%
\vspace{-1\smallerfontscale}}%
\end{addmargin}
\end{notebookcell}

\par\vspace{1\smallerfontscale}%
    \needspace{4\baselineskip}%
    % Only render the prompt if the cell is pyout.  Note, the outputs prompt 
    % block isn't used since we need to check each indiviual output and only
    % add prompts to the pyout ones.
    
        {\par%
        \vspace{-1\smallerfontscale}%
        \noindent%
        \begin{minipage}{\cellleftmargin}%
    \hfill%
    {\smaller%
    \tt%
    \color{nbframe-out-prompt}%
    Out[5]:}%
    \hspace{\inputpadding}%
    \hspace{0em}%
    \hspace{3pt}%
    \end{minipage}%%
        }%
    %
    %
    \begin{addmargin}[\cellleftmargin]{0em}% left, right
    {\smaller%
    \vspace{-1\smallerfontscale}%
    
    
    
    \begin{verbatim}
True
    \end{verbatim}

    
}%
    \end{addmargin}%
    % Add contents below.

{\par%
\vspace{-1\baselineskip}%
\needspace{4\baselineskip}}%
\begin{notebookcell}[4]%
\begin{addmargin}[\cellleftmargin]{0em}% left, right
{\smaller%
\par%
%
\vspace{-1\smallerfontscale}%
\begin{Verbatim}[commandchars=\\\{\}]
\PY{n}{my\PYZus{}list} \PY{o}{=} \PY{p}{[}\PY{n}{j} \PY{k}{for} \PY{n}{j} \PY{o+ow}{in} \PY{n+nb}{range}\PY{p}{(}\PY{l+m+mi}{5}\PY{p}{)}\PY{p}{]}
\PY{k}{for} \PY{n}{i} \PY{o+ow}{in} \PY{n}{my\PYZus{}list}\PY{p}{:}
    \PY{k}{print} \PY{n}{i}
\end{Verbatim}
%
\par%
\vspace{-1\smallerfontscale}}%
\end{addmargin}
\end{notebookcell}

\par\vspace{1\smallerfontscale}%
    \needspace{4\baselineskip}%
    % Only render the prompt if the cell is pyout.  Note, the outputs prompt 
    % block isn't used since we need to check each indiviual output and only
    % add prompts to the pyout ones.
    %
    %
    \begin{addmargin}[\cellleftmargin]{0em}% left, right
    {\smaller%
    \vspace{-1\smallerfontscale}%
    
    \begin{Verbatim}[commandchars=\\\{\}]
0
1
2
3
4
    \end{Verbatim}
}%
    \end{addmargin}%

    \subparagraph{Tuples}


    % Add contents below.

{\par%
\vspace{-1\baselineskip}%
\needspace{4\baselineskip}}%
\begin{notebookcell}[9]%
\begin{addmargin}[\cellleftmargin]{0em}% left, right
{\smaller%
\par%
%
\vspace{-1\smallerfontscale}%
\begin{Verbatim}[commandchars=\\\{\}]
\PY{n}{a} \PY{o}{=} \PY{p}{(}\PY{l+m+mi}{1}\PY{p}{,}\PY{l+m+mi}{2}\PY{p}{,}\PY{l+m+mi}{3}\PY{p}{)}
\PY{n}{a}
\end{Verbatim}
%
\par%
\vspace{-1\smallerfontscale}}%
\end{addmargin}
\end{notebookcell}

\par\vspace{1\smallerfontscale}%
    \needspace{4\baselineskip}%
    % Only render the prompt if the cell is pyout.  Note, the outputs prompt 
    % block isn't used since we need to check each indiviual output and only
    % add prompts to the pyout ones.
    
        {\par%
        \vspace{-1\smallerfontscale}%
        \noindent%
        \begin{minipage}{\cellleftmargin}%
    \hfill%
    {\smaller%
    \tt%
    \color{nbframe-out-prompt}%
    Out[9]:}%
    \hspace{\inputpadding}%
    \hspace{0em}%
    \hspace{3pt}%
    \end{minipage}%%
        }%
    %
    %
    \begin{addmargin}[\cellleftmargin]{0em}% left, right
    {\smaller%
    \vspace{-1\smallerfontscale}%
    
    
    
    \begin{verbatim}
(1, 2, 3)
    \end{verbatim}

    
}%
    \end{addmargin}%
    % Add contents below.

{\par%
\vspace{-1\baselineskip}%
\needspace{4\baselineskip}}%
\begin{notebookcell}[11]%
\begin{addmargin}[\cellleftmargin]{0em}% left, right
{\smaller%
\par%
%
\vspace{-1\smallerfontscale}%
\begin{Verbatim}[commandchars=\\\{\}]
\PY{k}{for} \PY{n}{item} \PY{o+ow}{in} \PY{n}{a}\PY{p}{:}
    \PY{k}{print} \PY{n}{item}
\end{Verbatim}
%
\par%
\vspace{-1\smallerfontscale}}%
\end{addmargin}
\end{notebookcell}

\par\vspace{1\smallerfontscale}%
    \needspace{4\baselineskip}%
    % Only render the prompt if the cell is pyout.  Note, the outputs prompt 
    % block isn't used since we need to check each indiviual output and only
    % add prompts to the pyout ones.
    %
    %
    \begin{addmargin}[\cellleftmargin]{0em}% left, right
    {\smaller%
    \vspace{-1\smallerfontscale}%
    
    \begin{Verbatim}[commandchars=\\\{\}]
1
2
3
    \end{Verbatim}
}%
    \end{addmargin}%

    \subparagraph{Dicts}


    % Add contents below.

{\par%
\vspace{-1\baselineskip}%
\needspace{4\baselineskip}}%
\begin{notebookcell}[13]%
\begin{addmargin}[\cellleftmargin]{0em}% left, right
{\smaller%
\par%
%
\vspace{-1\smallerfontscale}%
\begin{Verbatim}[commandchars=\\\{\}]
\PY{k+kn}{from} \PY{n+nn}{numpy.random} \PY{k+kn}{import} \PY{n}{randn}
\PY{n}{a} \PY{o}{=} \PY{n+nb}{dict}\PY{p}{(}\PY{n+nb}{zip}\PY{p}{(}\PY{n+nb}{range}\PY{p}{(}\PY{l+m+mi}{3}\PY{p}{)}\PY{p}{,}\PY{n}{randn}\PY{p}{(}\PY{l+m+mi}{3}\PY{p}{)}\PY{p}{)}\PY{p}{)}
\end{Verbatim}
%
\par%
\vspace{-1\smallerfontscale}}%
\end{addmargin}
\end{notebookcell}


    % Add contents below.

{\par%
\vspace{-1\baselineskip}%
\needspace{4\baselineskip}}%
\begin{notebookcell}[15]%
\begin{addmargin}[\cellleftmargin]{0em}% left, right
{\smaller%
\par%
%
\vspace{-1\smallerfontscale}%
\begin{Verbatim}[commandchars=\\\{\}]
\PY{k}{for} \PY{n}{item} \PY{o+ow}{in} \PY{n}{a}\PY{p}{:}
    \PY{k}{print} \PY{n}{item}
\end{Verbatim}
%
\par%
\vspace{-1\smallerfontscale}}%
\end{addmargin}
\end{notebookcell}

\par\vspace{1\smallerfontscale}%
    \needspace{4\baselineskip}%
    % Only render the prompt if the cell is pyout.  Note, the outputs prompt 
    % block isn't used since we need to check each indiviual output and only
    % add prompts to the pyout ones.
    %
    %
    \begin{addmargin}[\cellleftmargin]{0em}% left, right
    {\smaller%
    \vspace{-1\smallerfontscale}%
    
    \begin{Verbatim}[commandchars=\\\{\}]
0
1
2
    \end{Verbatim}
}%
    \end{addmargin}%
    % Add contents below.

{\par%
\vspace{-1\baselineskip}%
\needspace{4\baselineskip}}%
\begin{notebookcell}[18]%
\begin{addmargin}[\cellleftmargin]{0em}% left, right
{\smaller%
\par%
%
\vspace{-1\smallerfontscale}%
\begin{Verbatim}[commandchars=\\\{\}]
\PY{k}{for} \PY{n}{item} \PY{o+ow}{in} \PY{n}{a}\PY{o}{.}\PY{n}{iterkeys}\PY{p}{(}\PY{p}{)}\PY{p}{:}
    \PY{k}{print} \PY{n}{item}
\end{Verbatim}
%
\par%
\vspace{-1\smallerfontscale}}%
\end{addmargin}
\end{notebookcell}

\par\vspace{1\smallerfontscale}%
    \needspace{4\baselineskip}%
    % Only render the prompt if the cell is pyout.  Note, the outputs prompt 
    % block isn't used since we need to check each indiviual output and only
    % add prompts to the pyout ones.
    %
    %
    \begin{addmargin}[\cellleftmargin]{0em}% left, right
    {\smaller%
    \vspace{-1\smallerfontscale}%
    
    \begin{Verbatim}[commandchars=\\\{\}]
0
1
2
    \end{Verbatim}
}%
    \end{addmargin}%
    % Add contents below.

{\par%
\vspace{-1\baselineskip}%
\needspace{4\baselineskip}}%
\begin{notebookcell}[19]%
\begin{addmargin}[\cellleftmargin]{0em}% left, right
{\smaller%
\par%
%
\vspace{-1\smallerfontscale}%
\begin{Verbatim}[commandchars=\\\{\}]
\PY{k}{for} \PY{n}{item} \PY{o+ow}{in} \PY{n}{a}\PY{o}{.}\PY{n}{itervalues}\PY{p}{(}\PY{p}{)}\PY{p}{:}
    \PY{k}{print} \PY{n}{item}
\end{Verbatim}
%
\par%
\vspace{-1\smallerfontscale}}%
\end{addmargin}
\end{notebookcell}

\par\vspace{1\smallerfontscale}%
    \needspace{4\baselineskip}%
    % Only render the prompt if the cell is pyout.  Note, the outputs prompt 
    % block isn't used since we need to check each indiviual output and only
    % add prompts to the pyout ones.
    %
    %
    \begin{addmargin}[\cellleftmargin]{0em}% left, right
    {\smaller%
    \vspace{-1\smallerfontscale}%
    
    \begin{Verbatim}[commandchars=\\\{\}]
-0.445128793029
0.316719007573
2.06412261427
    \end{Verbatim}
}%
    \end{addmargin}%
    % Add contents below.

{\par%
\vspace{-1\baselineskip}%
\needspace{4\baselineskip}}%
\begin{notebookcell}[21]%
\begin{addmargin}[\cellleftmargin]{0em}% left, right
{\smaller%
\par%
%
\vspace{-1\smallerfontscale}%
\begin{Verbatim}[commandchars=\\\{\}]
\PY{k}{for} \PY{n}{item} \PY{o+ow}{in} \PY{n}{a}\PY{o}{.}\PY{n}{iteritems}\PY{p}{(}\PY{p}{)}\PY{p}{:}
    \PY{k}{print} \PY{n}{item}
\end{Verbatim}
%
\par%
\vspace{-1\smallerfontscale}}%
\end{addmargin}
\end{notebookcell}

\par\vspace{1\smallerfontscale}%
    \needspace{4\baselineskip}%
    % Only render the prompt if the cell is pyout.  Note, the outputs prompt 
    % block isn't used since we need to check each indiviual output and only
    % add prompts to the pyout ones.
    %
    %
    \begin{addmargin}[\cellleftmargin]{0em}% left, right
    {\smaller%
    \vspace{-1\smallerfontscale}%
    
    \begin{Verbatim}[commandchars=\\\{\}]
(0, -0.44512879302937886)
(1, 0.31671900757330107)
(2, 2.0641226142684475)
    \end{Verbatim}
}%
    \end{addmargin}%
    % Add contents below.

{\par%
\vspace{-1\baselineskip}%
\needspace{4\baselineskip}}%
\begin{notebookcell}[20]%
\begin{addmargin}[\cellleftmargin]{0em}% left, right
{\smaller%
\par%
%
\vspace{-1\smallerfontscale}%
\begin{Verbatim}[commandchars=\\\{\}]
\PY{k}{for} \PY{n}{k}\PY{p}{,}\PY{n}{v} \PY{o+ow}{in} \PY{n}{a}\PY{o}{.}\PY{n}{iteritems}\PY{p}{(}\PY{p}{)}\PY{p}{:}
    \PY{k}{print} \PY{n}{k}\PY{p}{,}\PY{n}{v}
\end{Verbatim}
%
\par%
\vspace{-1\smallerfontscale}}%
\end{addmargin}
\end{notebookcell}

\par\vspace{1\smallerfontscale}%
    \needspace{4\baselineskip}%
    % Only render the prompt if the cell is pyout.  Note, the outputs prompt 
    % block isn't used since we need to check each indiviual output and only
    % add prompts to the pyout ones.
    %
    %
    \begin{addmargin}[\cellleftmargin]{0em}% left, right
    {\smaller%
    \vspace{-1\smallerfontscale}%
    
    \begin{Verbatim}[commandchars=\\\{\}]
0 -0.445128793029
1 0.316719007573
2 2.06412261427
    \end{Verbatim}
}%
    \end{addmargin}%

    \subparagraph{Strings}


    % Add contents below.

{\par%
\vspace{-1\baselineskip}%
\needspace{4\baselineskip}}%
\begin{notebookcell}[24]%
\begin{addmargin}[\cellleftmargin]{0em}% left, right
{\smaller%
\par%
%
\vspace{-1\smallerfontscale}%
\begin{Verbatim}[commandchars=\\\{\}]
\PY{n}{a} \PY{o}{=} \PY{l+s}{\PYZsq{}}\PY{l+s}{Python is awesome!}\PY{l+s}{\PYZsq{}}
\end{Verbatim}
%
\par%
\vspace{-1\smallerfontscale}}%
\end{addmargin}
\end{notebookcell}


    % Add contents below.

{\par%
\vspace{-1\baselineskip}%
\needspace{4\baselineskip}}%
\begin{notebookcell}[25]%
\begin{addmargin}[\cellleftmargin]{0em}% left, right
{\smaller%
\par%
%
\vspace{-1\smallerfontscale}%
\begin{Verbatim}[commandchars=\\\{\}]
\PY{k}{for} \PY{n}{item} \PY{o+ow}{in} \PY{n}{a}\PY{p}{:}
    \PY{k}{print} \PY{n}{item}
\end{Verbatim}
%
\par%
\vspace{-1\smallerfontscale}}%
\end{addmargin}
\end{notebookcell}

\par\vspace{1\smallerfontscale}%
    \needspace{4\baselineskip}%
    % Only render the prompt if the cell is pyout.  Note, the outputs prompt 
    % block isn't used since we need to check each indiviual output and only
    % add prompts to the pyout ones.
    %
    %
    \begin{addmargin}[\cellleftmargin]{0em}% left, right
    {\smaller%
    \vspace{-1\smallerfontscale}%
    
    \begin{Verbatim}[commandchars=\\\{\}]
P
y
t
h
o
n
 
i
s
 
a
w
e
s
o
m
e
!
    \end{Verbatim}
}%
    \end{addmargin}%

    \subparagraph{Files}


    % Add contents below.

{\par%
\vspace{-1\baselineskip}%
\needspace{4\baselineskip}}%
\begin{notebookcell}[38]%
\begin{addmargin}[\cellleftmargin]{0em}% left, right
{\smaller%
\par%
%
\vspace{-1\smallerfontscale}%
\begin{Verbatim}[commandchars=\\\{\}]
\PY{n}{f} \PY{o}{=} \PY{n+nb}{open}\PY{p}{(}\PY{l+s}{\PYZsq{}}\PY{l+s}{iter\PYZus{}text.txt}\PY{l+s}{\PYZsq{}}\PY{p}{,}\PY{l+s}{\PYZsq{}}\PY{l+s}{rb}\PY{l+s}{\PYZsq{}}\PY{p}{)}
\end{Verbatim}
%
\par%
\vspace{-1\smallerfontscale}}%
\end{addmargin}
\end{notebookcell}


    % Add contents below.

{\par%
\vspace{-1\baselineskip}%
\needspace{4\baselineskip}}%
\begin{notebookcell}[39]%
\begin{addmargin}[\cellleftmargin]{0em}% left, right
{\smaller%
\par%
%
\vspace{-1\smallerfontscale}%
\begin{Verbatim}[commandchars=\\\{\}]
\PY{k}{for} \PY{n}{item} \PY{o+ow}{in} \PY{n}{f}\PY{p}{:}
    \PY{k}{print} \PY{n}{item}
\end{Verbatim}
%
\par%
\vspace{-1\smallerfontscale}}%
\end{addmargin}
\end{notebookcell}

\par\vspace{1\smallerfontscale}%
    \needspace{4\baselineskip}%
    % Only render the prompt if the cell is pyout.  Note, the outputs prompt 
    % block isn't used since we need to check each indiviual output and only
    % add prompts to the pyout ones.
    %
    %
    \begin{addmargin}[\cellleftmargin]{0em}% left, right
    {\smaller%
    \vspace{-1\smallerfontscale}%
    
    \begin{Verbatim}[commandchars=\\\{\}]
Line One

Line Two

Last line
    \end{Verbatim}
}%
    \end{addmargin}%
    % Add contents below.

{\par%
\vspace{-1\baselineskip}%
\needspace{4\baselineskip}}%
\begin{notebookcell}[44]%
\begin{addmargin}[\cellleftmargin]{0em}% left, right
{\smaller%
\par%
%
\vspace{-1\smallerfontscale}%
\begin{Verbatim}[commandchars=\\\{\}]
\PY{k}{for} \PY{n}{item} \PY{o+ow}{in} \PY{n}{f}\PY{o}{.}\PY{n}{readlines}\PY{p}{(}\PY{p}{)}\PY{p}{:}
    \PY{k}{print} \PY{n}{item}
\end{Verbatim}
%
\par%
\vspace{-1\smallerfontscale}}%
\end{addmargin}
\end{notebookcell}


    % Add contents below.

{\par%
\vspace{-1\baselineskip}%
\needspace{4\baselineskip}}%
\begin{notebookcell}[45]%
\begin{addmargin}[\cellleftmargin]{0em}% left, right
{\smaller%
\par%
%
\vspace{-1\smallerfontscale}%
\begin{Verbatim}[commandchars=\\\{\}]
\PY{n}{f} \PY{o}{=} \PY{n+nb}{open}\PY{p}{(}\PY{l+s}{\PYZsq{}}\PY{l+s}{iter\PYZus{}text.txt}\PY{l+s}{\PYZsq{}}\PY{p}{,}\PY{l+s}{\PYZsq{}}\PY{l+s}{rb}\PY{l+s}{\PYZsq{}}\PY{p}{)}
\PY{k}{for} \PY{n}{item} \PY{o+ow}{in} \PY{n}{f}\PY{o}{.}\PY{n}{readlines}\PY{p}{(}\PY{p}{)}\PY{p}{:}
    \PY{k}{print} \PY{n}{item}
\end{Verbatim}
%
\par%
\vspace{-1\smallerfontscale}}%
\end{addmargin}
\end{notebookcell}

\par\vspace{1\smallerfontscale}%
    \needspace{4\baselineskip}%
    % Only render the prompt if the cell is pyout.  Note, the outputs prompt 
    % block isn't used since we need to check each indiviual output and only
    % add prompts to the pyout ones.
    %
    %
    \begin{addmargin}[\cellleftmargin]{0em}% left, right
    {\smaller%
    \vspace{-1\smallerfontscale}%
    
    \begin{Verbatim}[commandchars=\\\{\}]
Line One

Line Two

Last line
    \end{Verbatim}
}%
    \end{addmargin}%
    We have two \n for each line, one from the text and one from the print
statement. A quick trick surpresses the one from print

    % Add contents below.

{\par%
\vspace{-1\baselineskip}%
\needspace{4\baselineskip}}%
\begin{notebookcell}[46]%
\begin{addmargin}[\cellleftmargin]{0em}% left, right
{\smaller%
\par%
%
\vspace{-1\smallerfontscale}%
\begin{Verbatim}[commandchars=\\\{\}]
\PY{n}{f} \PY{o}{=} \PY{n+nb}{open}\PY{p}{(}\PY{l+s}{\PYZsq{}}\PY{l+s}{iter\PYZus{}text.txt}\PY{l+s}{\PYZsq{}}\PY{p}{,}\PY{l+s}{\PYZsq{}}\PY{l+s}{rb}\PY{l+s}{\PYZsq{}}\PY{p}{)}
\PY{k}{for} \PY{n}{item} \PY{o+ow}{in} \PY{n}{f}\PY{p}{:}
    \PY{k}{print} \PY{n}{item}\PY{p}{,}
\end{Verbatim}
%
\par%
\vspace{-1\smallerfontscale}}%
\end{addmargin}
\end{notebookcell}

\par\vspace{1\smallerfontscale}%
    \needspace{4\baselineskip}%
    % Only render the prompt if the cell is pyout.  Note, the outputs prompt 
    % block isn't used since we need to check each indiviual output and only
    % add prompts to the pyout ones.
    %
    %
    \begin{addmargin}[\cellleftmargin]{0em}% left, right
    {\smaller%
    \vspace{-1\smallerfontscale}%
    
    \begin{Verbatim}[commandchars=\\\{\}]
Line One
Line Two
Last line
    \end{Verbatim}
}%
    \end{addmargin}%
    \textbf{Note}: As long as possible, use \emph{with} (controlled
execution) which closes the file for you. So instead of

\begin{verbatim}
f = open(filename,'r')
do something with f
f.close()
\end{verbatim}

you can write

\begin{verbatim}
with open(filename,'r') as f:
    do something with f
    
\end{verbatim}

where it automatically takes care of the clean-up for you


    \section{Generators}


    Generators are iterators which can only be iterated once and generate
the values on the fly

    % Add contents below.

{\par%
\vspace{-1\baselineskip}%
\needspace{4\baselineskip}}%
\begin{notebookcell}[52]%
\begin{addmargin}[\cellleftmargin]{0em}% left, right
{\smaller%
\par%
%
\vspace{-1\smallerfontscale}%
\begin{Verbatim}[commandchars=\\\{\}]
\PY{n}{my\PYZus{}generator} \PY{o}{=} \PY{p}{(}\PY{n}{j} \PY{k}{for} \PY{n}{j} \PY{o+ow}{in} \PY{n+nb}{range}\PY{p}{(}\PY{l+m+mi}{5}\PY{p}{)}\PY{p}{)}
\PY{k}{for} \PY{n}{i} \PY{o+ow}{in} \PY{n}{my\PYZus{}generator}\PY{p}{:}
    \PY{k}{print} \PY{n}{i}
\end{Verbatim}
%
\par%
\vspace{-1\smallerfontscale}}%
\end{addmargin}
\end{notebookcell}

\par\vspace{1\smallerfontscale}%
    \needspace{4\baselineskip}%
    % Only render the prompt if the cell is pyout.  Note, the outputs prompt 
    % block isn't used since we need to check each indiviual output and only
    % add prompts to the pyout ones.
    %
    %
    \begin{addmargin}[\cellleftmargin]{0em}% left, right
    {\smaller%
    \vspace{-1\smallerfontscale}%
    
    \begin{Verbatim}[commandchars=\\\{\}]
0
1
2
3
4
    \end{Verbatim}
}%
    \end{addmargin}%
    % Add contents below.

{\par%
\vspace{-1\baselineskip}%
\needspace{4\baselineskip}}%
\begin{notebookcell}[55]%
\begin{addmargin}[\cellleftmargin]{0em}% left, right
{\smaller%
\par%
%
\vspace{-1\smallerfontscale}%
\begin{Verbatim}[commandchars=\\\{\}]
\PY{n}{my\PYZus{}generator}\PY{o}{.}\PY{n}{next}\PY{p}{(}\PY{p}{)}
\end{Verbatim}
%
\par%
\vspace{-1\smallerfontscale}}%
\end{addmargin}
\end{notebookcell}

\par\vspace{1\smallerfontscale}%
    \needspace{4\baselineskip}%
    % Only render the prompt if the cell is pyout.  Note, the outputs prompt 
    % block isn't used since we need to check each indiviual output and only
    % add prompts to the pyout ones.
    %
    %
    \begin{addmargin}[\cellleftmargin]{0em}% left, right
    {\smaller%
    \vspace{-1\smallerfontscale}%
    
    \begin{Verbatim}[commandchars=\\\{\}]

        ---------------------------------------------------------------------------
    StopIteration                             Traceback (most recent call last)

        <ipython-input-55-125f388bb61b> in <module>()
    ----> 1 my\_generator.next()
    

        StopIteration: 

    \end{Verbatim}
}%
    \end{addmargin}%

    \subsection{Yield}


    \emph{yield} is the \emph{return} of generators. It outputs one value as
the \emph{next()} method of the generator is called

    % Add contents below.

{\par%
\vspace{-1\baselineskip}%
\needspace{4\baselineskip}}%
\begin{notebookcell}[6]%
\begin{addmargin}[\cellleftmargin]{0em}% left, right
{\smaller%
\par%
%
\vspace{-1\smallerfontscale}%
\begin{Verbatim}[commandchars=\\\{\}]
\PY{k}{def} \PY{n+nf}{createGenerator}\PY{p}{(}\PY{p}{)}\PY{p}{:}
    \PY{n}{mylist} \PY{o}{=} \PY{n+nb}{range}\PY{p}{(}\PY{l+m+mi}{5}\PY{p}{)}
    \PY{k}{for} \PY{n}{i} \PY{o+ow}{in} \PY{n}{mylist}\PY{p}{:}
        \PY{k}{yield} \PY{n}{i}
\end{Verbatim}
%
\par%
\vspace{-1\smallerfontscale}}%
\end{addmargin}
\end{notebookcell}


    % Add contents below.

{\par%
\vspace{-1\baselineskip}%
\needspace{4\baselineskip}}%
\begin{notebookcell}[7]%
\begin{addmargin}[\cellleftmargin]{0em}% left, right
{\smaller%
\par%
%
\vspace{-1\smallerfontscale}%
\begin{Verbatim}[commandchars=\\\{\}]
\PY{n}{mygenerator} \PY{o}{=} \PY{n}{createGenerator}\PY{p}{(}\PY{p}{)}
\end{Verbatim}
%
\par%
\vspace{-1\smallerfontscale}}%
\end{addmargin}
\end{notebookcell}


    % Add contents below.

{\par%
\vspace{-1\baselineskip}%
\needspace{4\baselineskip}}%
\begin{notebookcell}[8]%
\begin{addmargin}[\cellleftmargin]{0em}% left, right
{\smaller%
\par%
%
\vspace{-1\smallerfontscale}%
\begin{Verbatim}[commandchars=\\\{\}]
\PY{k}{print}\PY{p}{(}\PY{n}{mygenerator}\PY{p}{)}
\end{Verbatim}
%
\par%
\vspace{-1\smallerfontscale}}%
\end{addmargin}
\end{notebookcell}

\par\vspace{1\smallerfontscale}%
    \needspace{4\baselineskip}%
    % Only render the prompt if the cell is pyout.  Note, the outputs prompt 
    % block isn't used since we need to check each indiviual output and only
    % add prompts to the pyout ones.
    %
    %
    \begin{addmargin}[\cellleftmargin]{0em}% left, right
    {\smaller%
    \vspace{-1\smallerfontscale}%
    
    \begin{Verbatim}[commandchars=\\\{\}]
<generator object createGenerator at 0x7f2a5474fe60>
    \end{Verbatim}
}%
    \end{addmargin}%
    % Add contents below.

{\par%
\vspace{-1\baselineskip}%
\needspace{4\baselineskip}}%
\begin{notebookcell}[9]%
\begin{addmargin}[\cellleftmargin]{0em}% left, right
{\smaller%
\par%
%
\vspace{-1\smallerfontscale}%
\begin{Verbatim}[commandchars=\\\{\}]
\PY{k}{for} \PY{n}{i} \PY{o+ow}{in} \PY{n}{mygenerator}\PY{p}{:}
    \PY{k}{print} \PY{n}{i}
\end{Verbatim}
%
\par%
\vspace{-1\smallerfontscale}}%
\end{addmargin}
\end{notebookcell}

\par\vspace{1\smallerfontscale}%
    \needspace{4\baselineskip}%
    % Only render the prompt if the cell is pyout.  Note, the outputs prompt 
    % block isn't used since we need to check each indiviual output and only
    % add prompts to the pyout ones.
    %
    %
    \begin{addmargin}[\cellleftmargin]{0em}% left, right
    {\smaller%
    \vspace{-1\smallerfontscale}%
    
    \begin{Verbatim}[commandchars=\\\{\}]
0
1
2
3
4
    \end{Verbatim}
}%
    \end{addmargin}%
    % Add contents below.

{\par%
\vspace{-1\baselineskip}%
\needspace{4\baselineskip}}%
\begin{notebookcell}[10]%
\begin{addmargin}[\cellleftmargin]{0em}% left, right
{\smaller%
\par%
%
\vspace{-1\smallerfontscale}%
\begin{Verbatim}[commandchars=\\\{\}]
\PY{k}{class} \PY{n+nc}{Bank}\PY{p}{(}\PY{p}{)}\PY{p}{:}
    \PY{n}{crisis} \PY{o}{=} \PY{n+nb+bp}{False}
    \PY{k}{def} \PY{n+nf}{create\PYZus{}atm}\PY{p}{(}\PY{n+nb+bp}{self}\PY{p}{)}\PY{p}{:}
        \PY{k}{while} \PY{o+ow}{not} \PY{n+nb+bp}{self}\PY{o}{.}\PY{n}{crisis}\PY{p}{:}
            \PY{k}{yield} \PY{l+s}{\PYZsq{}}\PY{l+s}{100€}\PY{l+s}{\PYZsq{}}
\end{Verbatim}
%
\par%
\vspace{-1\smallerfontscale}}%
\end{addmargin}
\end{notebookcell}


    % Add contents below.

{\par%
\vspace{-1\baselineskip}%
\needspace{4\baselineskip}}%
\begin{notebookcell}[11]%
\begin{addmargin}[\cellleftmargin]{0em}% left, right
{\smaller%
\par%
%
\vspace{-1\smallerfontscale}%
\begin{Verbatim}[commandchars=\\\{\}]
\PY{n}{BES} \PY{o}{=} \PY{n}{Bank}\PY{p}{(}\PY{p}{)}
\end{Verbatim}
%
\par%
\vspace{-1\smallerfontscale}}%
\end{addmargin}
\end{notebookcell}


    % Add contents below.

{\par%
\vspace{-1\baselineskip}%
\needspace{4\baselineskip}}%
\begin{notebookcell}[12]%
\begin{addmargin}[\cellleftmargin]{0em}% left, right
{\smaller%
\par%
%
\vspace{-1\smallerfontscale}%
\begin{Verbatim}[commandchars=\\\{\}]
\PY{n}{Atm\PYZus{}Rossio} \PY{o}{=} \PY{n}{BES}\PY{o}{.}\PY{n}{create\PYZus{}atm}\PY{p}{(}\PY{p}{)}
\end{Verbatim}
%
\par%
\vspace{-1\smallerfontscale}}%
\end{addmargin}
\end{notebookcell}


    % Add contents below.

{\par%
\vspace{-1\baselineskip}%
\needspace{4\baselineskip}}%
\begin{notebookcell}[14]%
\begin{addmargin}[\cellleftmargin]{0em}% left, right
{\smaller%
\par%
%
\vspace{-1\smallerfontscale}%
\begin{Verbatim}[commandchars=\\\{\}]
\PY{k}{print}\PY{p}{(}\PY{n}{Atm\PYZus{}Rossio}\PY{o}{.}\PY{n}{next}\PY{p}{(}\PY{p}{)}\PY{p}{)}
\end{Verbatim}
%
\par%
\vspace{-1\smallerfontscale}}%
\end{addmargin}
\end{notebookcell}

\par\vspace{1\smallerfontscale}%
    \needspace{4\baselineskip}%
    % Only render the prompt if the cell is pyout.  Note, the outputs prompt 
    % block isn't used since we need to check each indiviual output and only
    % add prompts to the pyout ones.
    %
    %
    \begin{addmargin}[\cellleftmargin]{0em}% left, right
    {\smaller%
    \vspace{-1\smallerfontscale}%
    
    \begin{Verbatim}[commandchars=\\\{\}]
100€
    \end{Verbatim}
}%
    \end{addmargin}%
    % Add contents below.

{\par%
\vspace{-1\baselineskip}%
\needspace{4\baselineskip}}%
\begin{notebookcell}[15]%
\begin{addmargin}[\cellleftmargin]{0em}% left, right
{\smaller%
\par%
%
\vspace{-1\smallerfontscale}%
\begin{Verbatim}[commandchars=\\\{\}]
\PY{k}{print}\PY{p}{(}\PY{n}{Atm\PYZus{}Rossio}\PY{o}{.}\PY{n}{next}\PY{p}{(}\PY{p}{)}\PY{p}{)}
\end{Verbatim}
%
\par%
\vspace{-1\smallerfontscale}}%
\end{addmargin}
\end{notebookcell}

\par\vspace{1\smallerfontscale}%
    \needspace{4\baselineskip}%
    % Only render the prompt if the cell is pyout.  Note, the outputs prompt 
    % block isn't used since we need to check each indiviual output and only
    % add prompts to the pyout ones.
    %
    %
    \begin{addmargin}[\cellleftmargin]{0em}% left, right
    {\smaller%
    \vspace{-1\smallerfontscale}%
    
    \begin{Verbatim}[commandchars=\\\{\}]
100€
    \end{Verbatim}
}%
    \end{addmargin}%
    % Add contents below.

{\par%
\vspace{-1\baselineskip}%
\needspace{4\baselineskip}}%
\begin{notebookcell}[16]%
\begin{addmargin}[\cellleftmargin]{0em}% left, right
{\smaller%
\par%
%
\vspace{-1\smallerfontscale}%
\begin{Verbatim}[commandchars=\\\{\}]
\PY{n}{BES}\PY{o}{.}\PY{n}{crisis} \PY{o}{=} \PY{n+nb+bp}{True}
\end{Verbatim}
%
\par%
\vspace{-1\smallerfontscale}}%
\end{addmargin}
\end{notebookcell}


    % Add contents below.

{\par%
\vspace{-1\baselineskip}%
\needspace{4\baselineskip}}%
\begin{notebookcell}[18]%
\begin{addmargin}[\cellleftmargin]{0em}% left, right
{\smaller%
\par%
%
\vspace{-1\smallerfontscale}%
\begin{Verbatim}[commandchars=\\\{\}]
\PY{n}{Atm\PYZus{}Rossio}\PY{o}{.}\PY{n}{next}\PY{p}{(}\PY{p}{)}
\end{Verbatim}
%
\par%
\vspace{-1\smallerfontscale}}%
\end{addmargin}
\end{notebookcell}

\par\vspace{1\smallerfontscale}%
    \needspace{4\baselineskip}%
    % Only render the prompt if the cell is pyout.  Note, the outputs prompt 
    % block isn't used since we need to check each indiviual output and only
    % add prompts to the pyout ones.
    %
    %
    \begin{addmargin}[\cellleftmargin]{0em}% left, right
    {\smaller%
    \vspace{-1\smallerfontscale}%
    
    \begin{Verbatim}[commandchars=\\\{\}]

        ---------------------------------------------------------------------------
    StopIteration                             Traceback (most recent call last)

        <ipython-input-18-658b8cbaec0d> in <module>()
    ----> 1 Atm\_Rossio.next()
    

        StopIteration: 

    \end{Verbatim}
}%
    \end{addmargin}%
    % Add contents below.

{\par%
\vspace{-1\baselineskip}%
\needspace{4\baselineskip}}%
\begin{notebookcell}[19]%
\begin{addmargin}[\cellleftmargin]{0em}% left, right
{\smaller%
\par%
%
\vspace{-1\smallerfontscale}%
\begin{Verbatim}[commandchars=\\\{\}]
\PY{n}{BES}\PY{o}{.}\PY{n}{crisis} \PY{o}{=} \PY{n+nb+bp}{False}
\end{Verbatim}
%
\par%
\vspace{-1\smallerfontscale}}%
\end{addmargin}
\end{notebookcell}


    % Add contents below.

{\par%
\vspace{-1\baselineskip}%
\needspace{4\baselineskip}}%
\begin{notebookcell}[20]%
\begin{addmargin}[\cellleftmargin]{0em}% left, right
{\smaller%
\par%
%
\vspace{-1\smallerfontscale}%
\begin{Verbatim}[commandchars=\\\{\}]
\PY{k}{print}\PY{p}{(}\PY{n}{Atm\PYZus{}Rossio}\PY{o}{.}\PY{n}{next}\PY{p}{(}\PY{p}{)}\PY{p}{)}
\end{Verbatim}
%
\par%
\vspace{-1\smallerfontscale}}%
\end{addmargin}
\end{notebookcell}

\par\vspace{1\smallerfontscale}%
    \needspace{4\baselineskip}%
    % Only render the prompt if the cell is pyout.  Note, the outputs prompt 
    % block isn't used since we need to check each indiviual output and only
    % add prompts to the pyout ones.
    %
    %
    \begin{addmargin}[\cellleftmargin]{0em}% left, right
    {\smaller%
    \vspace{-1\smallerfontscale}%
    
    \begin{Verbatim}[commandchars=\\\{\}]

        ---------------------------------------------------------------------------
    StopIteration                             Traceback (most recent call last)

        <ipython-input-20-6b8e2bf0f31f> in <module>()
    ----> 1 print(Atm\_Rossio.next())
    

        StopIteration: 

    \end{Verbatim}
}%
    \end{addmargin}%
    % Add contents below.

{\par%
\vspace{-1\baselineskip}%
\needspace{4\baselineskip}}%
\begin{notebookcell}[21]%
\begin{addmargin}[\cellleftmargin]{0em}% left, right
{\smaller%
\par%
%
\vspace{-1\smallerfontscale}%
\begin{Verbatim}[commandchars=\\\{\}]
\PY{n}{Atm\PYZus{}NovoBanco} \PY{o}{=} \PY{n}{BES}\PY{o}{.}\PY{n}{create\PYZus{}atm}\PY{p}{(}\PY{p}{)}
\PY{k}{print}\PY{p}{(}\PY{n}{Atm\PYZus{}NovoBanco}\PY{o}{.}\PY{n}{next}\PY{p}{(}\PY{p}{)}\PY{p}{)}
\end{Verbatim}
%
\par%
\vspace{-1\smallerfontscale}}%
\end{addmargin}
\end{notebookcell}

\par\vspace{1\smallerfontscale}%
    \needspace{4\baselineskip}%
    % Only render the prompt if the cell is pyout.  Note, the outputs prompt 
    % block isn't used since we need to check each indiviual output and only
    % add prompts to the pyout ones.
    %
    %
    \begin{addmargin}[\cellleftmargin]{0em}% left, right
    {\smaller%
    \vspace{-1\smallerfontscale}%
    
    \begin{Verbatim}[commandchars=\\\{\}]
100€
    \end{Verbatim}
}%
    \end{addmargin}%
    % Add contents below.

{\par%
\vspace{-1\baselineskip}%
\needspace{4\baselineskip}}%
\begin{notebookcell}[3]%
\begin{addmargin}[\cellleftmargin]{0em}% left, right
{\smaller%
\par%
%
\vspace{-1\smallerfontscale}%
\begin{Verbatim}[commandchars=\\\{\}]
\PY{k+kn}{from} \PY{n+nn}{IPython.core.display} \PY{k+kn}{import} \PY{n}{HTML}
\PY{k}{def} \PY{n+nf}{css\PYZus{}styling}\PY{p}{(}\PY{p}{)}\PY{p}{:}
    \PY{n}{styles} \PY{o}{=} \PY{n+nb}{open}\PY{p}{(}\PY{l+s}{\PYZdq{}}\PY{l+s}{./styles/custom.css}\PY{l+s}{\PYZdq{}}\PY{p}{,} \PY{l+s}{\PYZdq{}}\PY{l+s}{r}\PY{l+s}{\PYZdq{}}\PY{p}{)}\PY{o}{.}\PY{n}{read}\PY{p}{(}\PY{p}{)}
    \PY{k}{return} \PY{n}{HTML}\PY{p}{(}\PY{n}{styles}\PY{p}{)}
\PY{n}{css\PYZus{}styling}\PY{p}{(}\PY{p}{)}
\end{Verbatim}
%
\par%
\vspace{-1\smallerfontscale}}%
\end{addmargin}
\end{notebookcell}

\par\vspace{1\smallerfontscale}%
    \needspace{4\baselineskip}%
    % Only render the prompt if the cell is pyout.  Note, the outputs prompt 
    % block isn't used since we need to check each indiviual output and only
    % add prompts to the pyout ones.
    
        {\par%
        \vspace{-1\smallerfontscale}%
        \noindent%
        \begin{minipage}{\cellleftmargin}%
    \hfill%
    {\smaller%
    \tt%
    \color{nbframe-out-prompt}%
    Out[3]:}%
    \hspace{\inputpadding}%
    \hspace{0em}%
    \hspace{3pt}%
    \end{minipage}%%
        }%
    %
    %
    \begin{addmargin}[\cellleftmargin]{0em}% left, right
    {\smaller%
    \vspace{-1\smallerfontscale}%
    
    
    
    \begin{verbatim}
<IPython.core.display.HTML at 0x7fa1d82598d0>
    \end{verbatim}

    
}%
    \end{addmargin}%
    % Add contents below.

{\par%
\vspace{-1\baselineskip}%
\needspace{4\baselineskip}}%
\begin{notebookcell}[]%
\begin{addmargin}[\cellleftmargin]{0em}% left, right
{\smaller%
\par%
%
\vspace{-1\smallerfontscale}%
\begin{Verbatim}[commandchars=\\\{\}]

\end{Verbatim}
%
\par%
\vspace{-1\smallerfontscale}}%
\end{addmargin}
\end{notebookcell}




%\newpage
%






    
    \section{II-\href{http://ipython.org/}{IPython}}\label{ii-ipython}

    \begin{itemize}
\itemsep1pt\parskip0pt\parsep0pt
\item
  \href{}{Tab Completion}
\item
  \href{}{Introspection}
\item
  \href{}{We need to go deeper}
\item
  \href{}{\%paste}
\item
  \href{}{Exception Handling}
\item
  \href{}{Magic Commands}
\item
  \href{}{Shell}
\item
  \href{}{\%pdb}
\item
  \href{}{\%time and \%timeit}
\item
  \href{}{Profiles}
\item
  \href{}{Miscellaneous}
\end{itemize}

    


    \subsection{Tab Completion}



    \subparagraph{Objects}


    % Add contents below.

{\par%
\vspace{-1\baselineskip}%
\needspace{4\baselineskip}}%
\begin{notebookcell}[17]%
\begin{addmargin}[\cellleftmargin]{0em}% left, right
{\smaller%
\par%
%
\vspace{-1\smallerfontscale}%
\begin{Verbatim}[commandchars=\\\{\}]
\PY{n}{stringA} \PY{o}{=} \PY{l+s}{\PYZsq{}}\PY{l+s}{Welcome to Python}\PY{l+s}{\PYZsq{}}
\end{Verbatim}
%
\par%
\vspace{-1\smallerfontscale}}%
\end{addmargin}
\end{notebookcell}


    % Add contents below.

{\par%
\vspace{-1\baselineskip}%
\needspace{4\baselineskip}}%
\begin{notebookcell}[21]%
\begin{addmargin}[\cellleftmargin]{0em}% left, right
{\smaller%
\par%
%
\vspace{-1\smallerfontscale}%
\begin{Verbatim}[commandchars=\\\{\}]
\PY{n}{stringA}\PY{o}{.}\PY{n}{capitalize}\PY{p}{(}\PY{p}{)}
\end{Verbatim}
%
\par%
\vspace{-1\smallerfontscale}}%
\end{addmargin}
\end{notebookcell}

\par\vspace{1\smallerfontscale}%
    \needspace{4\baselineskip}%
    % Only render the prompt if the cell is pyout.  Note, the outputs prompt 
    % block isn't used since we need to check each indiviual output and only
    % add prompts to the pyout ones.
    
        {\par%
        \vspace{-1\smallerfontscale}%
        \noindent%
        \begin{minipage}{\cellleftmargin}%
    \hfill%
    {\smaller%
    \tt%
    \color{nbframe-out-prompt}%
    Out[21]:}%
    \hspace{\inputpadding}%
    \hspace{0em}%
    \hspace{3pt}%
    \end{minipage}%%
        }%
    %
    %
    \begin{addmargin}[\cellleftmargin]{0em}% left, right
    {\smaller%
    \vspace{-1\smallerfontscale}%
    
    
    
    \begin{verbatim}
'Welcome to python'
    \end{verbatim}

    
}%
    \end{addmargin}%
    % Add contents below.

{\par%
\vspace{-1\baselineskip}%
\needspace{4\baselineskip}}%
\begin{notebookcell}[22]%
\begin{addmargin}[\cellleftmargin]{0em}% left, right
{\smaller%
\par%
%
\vspace{-1\smallerfontscale}%
\begin{Verbatim}[commandchars=\\\{\}]
\PY{n+nb}{dir}\PY{p}{(}\PY{n}{stringA}\PY{p}{)}
\end{Verbatim}
%
\par%
\vspace{-1\smallerfontscale}}%
\end{addmargin}
\end{notebookcell}

\par\vspace{1\smallerfontscale}%
    \needspace{4\baselineskip}%
    % Only render the prompt if the cell is pyout.  Note, the outputs prompt 
    % block isn't used since we need to check each indiviual output and only
    % add prompts to the pyout ones.
    
        {\par%
        \vspace{-1\smallerfontscale}%
        \noindent%
        \begin{minipage}{\cellleftmargin}%
    \hfill%
    {\smaller%
    \tt%
    \color{nbframe-out-prompt}%
    Out[22]:}%
    \hspace{\inputpadding}%
    \hspace{0em}%
    \hspace{3pt}%
    \end{minipage}%%
        }%
    %
    %
    \begin{addmargin}[\cellleftmargin]{0em}% left, right
    {\smaller%
    \vspace{-1\smallerfontscale}%
    
    
    
    \begin{verbatim}
['__add__',
 '__class__',
 '__contains__',
 '__delattr__',
 '__doc__',
 '__eq__',
 '__format__',
 '__ge__',
 '__getattribute__',
 '__getitem__',
 '__getnewargs__',
 '__getslice__',
 '__gt__',
 '__hash__',
 '__init__',
 '__le__',
 '__len__',
 '__lt__',
 '__mod__',
 '__mul__',
 '__ne__',
 '__new__',
 '__reduce__',
 '__reduce_ex__',
 '__repr__',
 '__rmod__',
 '__rmul__',
 '__setattr__',
 '__sizeof__',
 '__str__',
 '__subclasshook__',
 '_formatter_field_name_split',
 '_formatter_parser',
 'capitalize',
 'center',
 'count',
 'decode',
 'encode',
 'endswith',
 'expandtabs',
 'find',
 'format',
 'index',
 'isalnum',
 'isalpha',
 'isdigit',
 'islower',
 'isspace',
 'istitle',
 'isupper',
 'join',
 'ljust',
 'lower',
 'lstrip',
 'partition',
 'replace',
 'rfind',
 'rindex',
 'rjust',
 'rpartition',
 'rsplit',
 'rstrip',
 'split',
 'splitlines',
 'startswith',
 'strip',
 'swapcase',
 'title',
 'translate',
 'upper',
 'zfill']
    \end{verbatim}

    
}%
    \end{addmargin}%

    \subparagraph{Modules}


    % Add contents below.

{\par%
\vspace{-1\baselineskip}%
\needspace{4\baselineskip}}%
\begin{notebookcell}[12]%
\begin{addmargin}[\cellleftmargin]{0em}% left, right
{\smaller%
\par%
%
\vspace{-1\smallerfontscale}%
\begin{Verbatim}[commandchars=\\\{\}]
\PY{k+kn}{import} \PY{n+nn}{math}
\end{Verbatim}
%
\par%
\vspace{-1\smallerfontscale}}%
\end{addmargin}
\end{notebookcell}


    % Add contents below.

{\par%
\vspace{-1\baselineskip}%
\needspace{4\baselineskip}}%
\begin{notebookcell}[14]%
\begin{addmargin}[\cellleftmargin]{0em}% left, right
{\smaller%
\par%
%
\vspace{-1\smallerfontscale}%
\begin{Verbatim}[commandchars=\\\{\}]
\PY{n}{math}\PY{o}{.}\PY{n}{acos}
\end{Verbatim}
%
\par%
\vspace{-1\smallerfontscale}}%
\end{addmargin}
\end{notebookcell}

\par\vspace{1\smallerfontscale}%
    \needspace{4\baselineskip}%
    % Only render the prompt if the cell is pyout.  Note, the outputs prompt 
    % block isn't used since we need to check each indiviual output and only
    % add prompts to the pyout ones.
    
        {\par%
        \vspace{-1\smallerfontscale}%
        \noindent%
        \begin{minipage}{\cellleftmargin}%
    \hfill%
    {\smaller%
    \tt%
    \color{nbframe-out-prompt}%
    Out[14]:}%
    \hspace{\inputpadding}%
    \hspace{0em}%
    \hspace{3pt}%
    \end{minipage}%%
        }%
    %
    %
    \begin{addmargin}[\cellleftmargin]{0em}% left, right
    {\smaller%
    \vspace{-1\smallerfontscale}%
    
    
    
    \begin{verbatim}
<function math.acos>
    \end{verbatim}

    
}%
    \end{addmargin}%
    % Add contents below.

{\par%
\vspace{-1\baselineskip}%
\needspace{4\baselineskip}}%
\begin{notebookcell}[15]%
\begin{addmargin}[\cellleftmargin]{0em}% left, right
{\smaller%
\par%
%
\vspace{-1\smallerfontscale}%
\begin{Verbatim}[commandchars=\\\{\}]
\PY{n+nb}{dir}\PY{p}{(}\PY{n}{math}\PY{p}{)}
\end{Verbatim}
%
\par%
\vspace{-1\smallerfontscale}}%
\end{addmargin}
\end{notebookcell}

\par\vspace{1\smallerfontscale}%
    \needspace{4\baselineskip}%
    % Only render the prompt if the cell is pyout.  Note, the outputs prompt 
    % block isn't used since we need to check each indiviual output and only
    % add prompts to the pyout ones.
    
        {\par%
        \vspace{-1\smallerfontscale}%
        \noindent%
        \begin{minipage}{\cellleftmargin}%
    \hfill%
    {\smaller%
    \tt%
    \color{nbframe-out-prompt}%
    Out[15]:}%
    \hspace{\inputpadding}%
    \hspace{0em}%
    \hspace{3pt}%
    \end{minipage}%%
        }%
    %
    %
    \begin{addmargin}[\cellleftmargin]{0em}% left, right
    {\smaller%
    \vspace{-1\smallerfontscale}%
    
    
    
    \begin{verbatim}
['__doc__',
 '__file__',
 '__name__',
 '__package__',
 'acos',
 'acosh',
 'asin',
 'asinh',
 'atan',
 'atan2',
 'atanh',
 'ceil',
 'copysign',
 'cos',
 'cosh',
 'degrees',
 'e',
 'erf',
 'erfc',
 'exp',
 'expm1',
 'fabs',
 'factorial',
 'floor',
 'fmod',
 'frexp',
 'fsum',
 'gamma',
 'hypot',
 'isinf',
 'isnan',
 'ldexp',
 'lgamma',
 'log',
 'log10',
 'log1p',
 'modf',
 'pi',
 'pow',
 'radians',
 'sin',
 'sinh',
 'sqrt',
 'tan',
 'tanh',
 'trunc']
    \end{verbatim}

    
}%
    \end{addmargin}%

    \subparagraph{Command line}


    % Add contents below.

{\par%
\vspace{-1\baselineskip}%
\needspace{4\baselineskip}}%
\begin{notebookcell}[24]%
\begin{addmargin}[\cellleftmargin]{0em}% left, right
{\smaller%
\par%
%
\vspace{-1\smallerfontscale}%
\begin{Verbatim}[commandchars=\\\{\}]
\PY{o}{!}mkdir \PY{l+s+s1}{\PYZsq{}mkdir\PYZsq{}}
\end{Verbatim}
%
\par%
\vspace{-1\smallerfontscale}}%
\end{addmargin}
\end{notebookcell}


    % Add contents below.

{\par%
\vspace{-1\baselineskip}%
\needspace{4\baselineskip}}%
\begin{notebookcell}[25]%
\begin{addmargin}[\cellleftmargin]{0em}% left, right
{\smaller%
\par%
%
\vspace{-1\smallerfontscale}%
\begin{Verbatim}[commandchars=\\\{\}]
\PY{o}{!}rm \PYZhy{}r \PY{l+s+s1}{\PYZsq{}./mkdir/\PYZsq{}}
\end{Verbatim}
%
\par%
\vspace{-1\smallerfontscale}}%
\end{addmargin}
\end{notebookcell}


    % Add contents below.

{\par%
\vspace{-1\baselineskip}%
\needspace{4\baselineskip}}%
\begin{notebookcell}[31]%
\begin{addmargin}[\cellleftmargin]{0em}% left, right
{\smaller%
\par%
%
\vspace{-1\smallerfontscale}%
\begin{Verbatim}[commandchars=\\\{\}]
\PY{o}{!}ls
\end{Verbatim}
%
\par%
\vspace{-1\smallerfontscale}}%
\end{addmargin}
\end{notebookcell}

\par\vspace{1\smallerfontscale}%
    \needspace{4\baselineskip}%
    % Only render the prompt if the cell is pyout.  Note, the outputs prompt 
    % block isn't used since we need to check each indiviual output and only
    % add prompts to the pyout ones.
    %
    %
    \begin{addmargin}[\cellleftmargin]{0em}% left, right
    {\smaller%
    \vspace{-1\smallerfontscale}%
    
    \begin{Verbatim}[commandchars=\\\{\}]
Day1					    IV-SciPy.ipynb	tweet\_dumper.py		VII-Tips
Day2					    IX-Debugging.ipynb	tweet\_dumper.py\textasciitilde{}	VII-TipsandTricks.ipynb
Docs					    pairwise\_fort.f	tweet\_dumper.pyc	VI-Pandas.ipynb
Exercises.ipynb				    pairwise\_fort.so	version\_information	V-Matplotlib.ipynb
III-Numpy.ipynb				    README.md		VII-			X-Parallelization.ipynb
II-IPython.ipynb			    TOC2.ipynb		VIII-Benchmarks.ipynb
I-Introduction to Python Programming.ipynb  TOC.ipynb		VIII-Benchmarks.ipynb\textasciitilde{}
    \end{Verbatim}
}%
    \end{addmargin}%
    % Add contents below.

{\par%
\vspace{-1\baselineskip}%
\needspace{4\baselineskip}}%
\begin{notebookcell}[29]%
\begin{addmargin}[\cellleftmargin]{0em}% left, right
{\smaller%
\par%
%
\vspace{-1\smallerfontscale}%
\begin{Verbatim}[commandchars=\\\{\}]
\PY{o}{!}cat /proc/cpuinfo
\end{Verbatim}
%
\par%
\vspace{-1\smallerfontscale}}%
\end{addmargin}
\end{notebookcell}

\par\vspace{1\smallerfontscale}%
    \needspace{4\baselineskip}%
    % Only render the prompt if the cell is pyout.  Note, the outputs prompt 
    % block isn't used since we need to check each indiviual output and only
    % add prompts to the pyout ones.
    %
    %
    \begin{addmargin}[\cellleftmargin]{0em}% left, right
    {\smaller%
    \vspace{-1\smallerfontscale}%
    
    \begin{Verbatim}[commandchars=\\\{\}]
processor	: 0
vendor\_id	: GenuineIntel
cpu family	: 6
model		: 69
model name	: Intel(R) Core(TM) i5-4300U CPU @ 1.90GHz
stepping	: 1
microcode	: 0x14
cpu MHz		: 1900.000
cache size	: 3072 KB
physical id	: 0
siblings	: 4
core id		: 0
cpu cores	: 2
apicid		: 0
initial apicid	: 0
fpu		: yes
fpu\_exception	: yes
cpuid level	: 13
wp		: yes
flags		: fpu vme de pse tsc msr pae mce cx8 apic sep mtrr pge mca cmov pat pse36 clflush dts acpi mmx fxsr sse sse2 ss ht tm pbe syscall nx pdpe1gb rdtscp lm constant\_tsc arch\_perfmon pebs bts rep\_good nopl xtopology nonstop\_tsc aperfmperf eagerfpu pni pclmulqdq dtes64 monitor ds\_cpl vmx smx est tm2 ssse3 fma cx16 xtpr pdcm pcid sse4\_1 sse4\_2 x2apic movbe popcnt tsc\_deadline\_timer aes xsave avx f16c rdrand lahf\_lm abm ida arat epb xsaveopt pln pts dtherm tpr\_shadow vnmi flexpriority ept vpid fsgsbase tsc\_adjust bmi1 hle avx2 smep bmi2 erms invpcid rtm
bogomips	: 4988.54
clflush size	: 64
cache\_alignment	: 64
address sizes	: 39 bits physical, 48 bits virtual
power management:

processor	: 1
vendor\_id	: GenuineIntel
cpu family	: 6
model		: 69
model name	: Intel(R) Core(TM) i5-4300U CPU @ 1.90GHz
stepping	: 1
microcode	: 0x14
cpu MHz		: 1900.000
cache size	: 3072 KB
physical id	: 0
siblings	: 4
core id		: 1
cpu cores	: 2
apicid		: 2
initial apicid	: 2
fpu		: yes
fpu\_exception	: yes
cpuid level	: 13
wp		: yes
flags		: fpu vme de pse tsc msr pae mce cx8 apic sep mtrr pge mca cmov pat pse36 clflush dts acpi mmx fxsr sse sse2 ss ht tm pbe syscall nx pdpe1gb rdtscp lm constant\_tsc arch\_perfmon pebs bts rep\_good nopl xtopology nonstop\_tsc aperfmperf eagerfpu pni pclmulqdq dtes64 monitor ds\_cpl vmx smx est tm2 ssse3 fma cx16 xtpr pdcm pcid sse4\_1 sse4\_2 x2apic movbe popcnt tsc\_deadline\_timer aes xsave avx f16c rdrand lahf\_lm abm ida arat epb xsaveopt pln pts dtherm tpr\_shadow vnmi flexpriority ept vpid fsgsbase tsc\_adjust bmi1 hle avx2 smep bmi2 erms invpcid rtm
bogomips	: 4988.54
clflush size	: 64
cache\_alignment	: 64
address sizes	: 39 bits physical, 48 bits virtual
power management:

processor	: 2
vendor\_id	: GenuineIntel
cpu family	: 6
model		: 69
model name	: Intel(R) Core(TM) i5-4300U CPU @ 1.90GHz
stepping	: 1
microcode	: 0x14
cpu MHz		: 1500.000
cache size	: 3072 KB
physical id	: 0
siblings	: 4
core id		: 0
cpu cores	: 2
apicid		: 1
initial apicid	: 1
fpu		: yes
fpu\_exception	: yes
cpuid level	: 13
wp		: yes
flags		: fpu vme de pse tsc msr pae mce cx8 apic sep mtrr pge mca cmov pat pse36 clflush dts acpi mmx fxsr sse sse2 ss ht tm pbe syscall nx pdpe1gb rdtscp lm constant\_tsc arch\_perfmon pebs bts rep\_good nopl xtopology nonstop\_tsc aperfmperf eagerfpu pni pclmulqdq dtes64 monitor ds\_cpl vmx smx est tm2 ssse3 fma cx16 xtpr pdcm pcid sse4\_1 sse4\_2 x2apic movbe popcnt tsc\_deadline\_timer aes xsave avx f16c rdrand lahf\_lm abm ida arat epb xsaveopt pln pts dtherm tpr\_shadow vnmi flexpriority ept vpid fsgsbase tsc\_adjust bmi1 hle avx2 smep bmi2 erms invpcid rtm
bogomips	: 4988.54
clflush size	: 64
cache\_alignment	: 64
address sizes	: 39 bits physical, 48 bits virtual
power management:

processor	: 3
vendor\_id	: GenuineIntel
cpu family	: 6
model		: 69
model name	: Intel(R) Core(TM) i5-4300U CPU @ 1.90GHz
stepping	: 1
microcode	: 0x14
cpu MHz		: 1900.000
cache size	: 3072 KB
physical id	: 0
siblings	: 4
core id		: 1
cpu cores	: 2
apicid		: 3
initial apicid	: 3
fpu		: yes
fpu\_exception	: yes
cpuid level	: 13
wp		: yes
flags		: fpu vme de pse tsc msr pae mce cx8 apic sep mtrr pge mca cmov pat pse36 clflush dts acpi mmx fxsr sse sse2 ss ht tm pbe syscall nx pdpe1gb rdtscp lm constant\_tsc arch\_perfmon pebs bts rep\_good nopl xtopology nonstop\_tsc aperfmperf eagerfpu pni pclmulqdq dtes64 monitor ds\_cpl vmx smx est tm2 ssse3 fma cx16 xtpr pdcm pcid sse4\_1 sse4\_2 x2apic movbe popcnt tsc\_deadline\_timer aes xsave avx f16c rdrand lahf\_lm abm ida arat epb xsaveopt pln pts dtherm tpr\_shadow vnmi flexpriority ept vpid fsgsbase tsc\_adjust bmi1 hle avx2 smep bmi2 erms invpcid rtm
bogomips	: 4988.54
clflush size	: 64
cache\_alignment	: 64
address sizes	: 39 bits physical, 48 bits virtual
power management:
    \end{Verbatim}
}%
    \end{addmargin}%

    \subsubsection{What is the difference between atan and atan2?}


    % Add contents below.

{\par%
\vspace{-1\baselineskip}%
\needspace{4\baselineskip}}%
\begin{notebookcell}[32]%
\begin{addmargin}[\cellleftmargin]{0em}% left, right
{\smaller%
\par%
%
\vspace{-1\smallerfontscale}%
\begin{Verbatim}[commandchars=\\\{\}]
\PY{n+nb}{reload}\PY{p}{(}\PY{n}{math}\PY{p}{)}
\end{Verbatim}
%
\par%
\vspace{-1\smallerfontscale}}%
\end{addmargin}
\end{notebookcell}

\par\vspace{1\smallerfontscale}%
    \needspace{4\baselineskip}%
    % Only render the prompt if the cell is pyout.  Note, the outputs prompt 
    % block isn't used since we need to check each indiviual output and only
    % add prompts to the pyout ones.
    
        {\par%
        \vspace{-1\smallerfontscale}%
        \noindent%
        \begin{minipage}{\cellleftmargin}%
    \hfill%
    {\smaller%
    \tt%
    \color{nbframe-out-prompt}%
    Out[32]:}%
    \hspace{\inputpadding}%
    \hspace{0em}%
    \hspace{3pt}%
    \end{minipage}%%
        }%
    %
    %
    \begin{addmargin}[\cellleftmargin]{0em}% left, right
    {\smaller%
    \vspace{-1\smallerfontscale}%
    
    
    
    \begin{verbatim}
<module 'math' from '/home/jpsilva/anaconda/lib/python2.7/lib-dynload/math.so'>
    \end{verbatim}

    
}%
    \end{addmargin}%
    % Add contents below.

{\par%
\vspace{-1\baselineskip}%
\needspace{4\baselineskip}}%
\begin{notebookcell}[43]%
\begin{addmargin}[\cellleftmargin]{0em}% left, right
{\smaller%
\par%
%
\vspace{-1\smallerfontscale}%
\begin{Verbatim}[commandchars=\\\{\}]
\PY{k+kn}{import} \PY{n+nn}{numpy}
\end{Verbatim}
%
\par%
\vspace{-1\smallerfontscale}}%
\end{addmargin}
\end{notebookcell}


    % Add contents below.

{\par%
\vspace{-1\baselineskip}%
\needspace{4\baselineskip}}%
\begin{notebookcell}[50]%
\begin{addmargin}[\cellleftmargin]{0em}% left, right
{\smaller%
\par%
%
\vspace{-1\smallerfontscale}%
\begin{Verbatim}[commandchars=\\\{\}]
\PY{err}{?}\PY{n}{numpy}\PY{o}{.}\PY{n}{arctan}
\end{Verbatim}
%
\par%
\vspace{-1\smallerfontscale}}%
\end{addmargin}
\end{notebookcell}


    % Add contents below.

{\par%
\vspace{-1\baselineskip}%
\needspace{4\baselineskip}}%
\begin{notebookcell}[57]%
\begin{addmargin}[\cellleftmargin]{0em}% left, right
{\smaller%
\par%
%
\vspace{-1\smallerfontscale}%
\begin{Verbatim}[commandchars=\\\{\}]
\PY{n}{numpy}\PY{o}{.}\PY{n}{arctan}\PY{p}{(}\PY{n}{numpy}\PY{o}{.}\PY{n}{pi}\PY{o}{/}\PY{l+m+mi}{4}\PY{p}{)} \PY{o}{==} \PY{n}{math}\PY{o}{.}\PY{n}{atan}\PY{p}{(}\PY{n}{numpy}\PY{o}{.}\PY{n}{pi}\PY{o}{/}\PY{l+m+mi}{4}\PY{p}{)}
\PY{n}{numpy}\PY{o}{.}\PY{n}{sin}\PY{p}{(}\PY{n}{numpy}\PY{o}{.}\PY{n}{pi}\PY{o}{/}\PY{l+m+mi}{2}\PY{p}{)}
\end{Verbatim}
%
\par%
\vspace{-1\smallerfontscale}}%
\end{addmargin}
\end{notebookcell}

\par\vspace{1\smallerfontscale}%
    \needspace{4\baselineskip}%
    % Only render the prompt if the cell is pyout.  Note, the outputs prompt 
    % block isn't used since we need to check each indiviual output and only
    % add prompts to the pyout ones.
    
        {\par%
        \vspace{-1\smallerfontscale}%
        \noindent%
        \begin{minipage}{\cellleftmargin}%
    \hfill%
    {\smaller%
    \tt%
    \color{nbframe-out-prompt}%
    Out[57]:}%
    \hspace{\inputpadding}%
    \hspace{0em}%
    \hspace{3pt}%
    \end{minipage}%%
        }%
    %
    %
    \begin{addmargin}[\cellleftmargin]{0em}% left, right
    {\smaller%
    \vspace{-1\smallerfontscale}%
    
    
    
    \begin{verbatim}
1.0
    \end{verbatim}

    
}%
    \end{addmargin}%

    \subsection{Exception Handling}



    \subsection{Magic}


    % Add contents below.

{\par%
\vspace{-1\baselineskip}%
\needspace{4\baselineskip}}%
\begin{notebookcell}[58]%
\begin{addmargin}[\cellleftmargin]{0em}% left, right
{\smaller%
\par%
%
\vspace{-1\smallerfontscale}%
\begin{Verbatim}[commandchars=\\\{\}]
\PY{o}{\PYZpc{}}\PY{k}{quickref}
\end{Verbatim}
%
\par%
\vspace{-1\smallerfontscale}}%
\end{addmargin}
\end{notebookcell}


    % Add contents below.

{\par%
\vspace{-1\baselineskip}%
\needspace{4\baselineskip}}%
\begin{notebookcell}[1]%
\begin{addmargin}[\cellleftmargin]{0em}% left, right
{\smaller%
\par%
%
\vspace{-1\smallerfontscale}%
\begin{Verbatim}[commandchars=\\\{\}]
\PY{k+kn}{from} \PY{n+nn}{IPython.core.display} \PY{k+kn}{import} \PY{n}{HTML}
\PY{k}{def} \PY{n+nf}{css\PYZus{}styling}\PY{p}{(}\PY{p}{)}\PY{p}{:}
    \PY{n}{styles} \PY{o}{=} \PY{n+nb}{open}\PY{p}{(}\PY{l+s}{\PYZdq{}}\PY{l+s}{./styles/custom.css}\PY{l+s}{\PYZdq{}}\PY{p}{,} \PY{l+s}{\PYZdq{}}\PY{l+s}{r}\PY{l+s}{\PYZdq{}}\PY{p}{)}\PY{o}{.}\PY{n}{read}\PY{p}{(}\PY{p}{)}
    \PY{k}{return} \PY{n}{HTML}\PY{p}{(}\PY{n}{styles}\PY{p}{)}
\PY{n}{css\PYZus{}styling}\PY{p}{(}\PY{p}{)}
\end{Verbatim}
%
\par%
\vspace{-1\smallerfontscale}}%
\end{addmargin}
\end{notebookcell}

\par\vspace{1\smallerfontscale}%
    \needspace{4\baselineskip}%
    % Only render the prompt if the cell is pyout.  Note, the outputs prompt 
    % block isn't used since we need to check each indiviual output and only
    % add prompts to the pyout ones.
    
        {\par%
        \vspace{-1\smallerfontscale}%
        \noindent%
        \begin{minipage}{\cellleftmargin}%
    \hfill%
    {\smaller%
    \tt%
    \color{nbframe-out-prompt}%
    Out[1]:}%
    \hspace{\inputpadding}%
    \hspace{0em}%
    \hspace{3pt}%
    \end{minipage}%%
        }%
    %
    %
    \begin{addmargin}[\cellleftmargin]{0em}% left, right
    {\smaller%
    \vspace{-1\smallerfontscale}%
    
    
    
    \begin{verbatim}
<IPython.core.display.HTML at 0x7f8c3992f850>
    \end{verbatim}

    
}%
    \end{addmargin}%
    % Add contents below.

{\par%
\vspace{-1\baselineskip}%
\needspace{4\baselineskip}}%
\begin{notebookcell}[]%
\begin{addmargin}[\cellleftmargin]{0em}% left, right
{\smaller%
\par%
%
\vspace{-1\smallerfontscale}%
\begin{Verbatim}[commandchars=\\\{\}]

\end{Verbatim}
%
\par%
\vspace{-1\smallerfontscale}}%
\end{addmargin}
\end{notebookcell}




%\newpage
%






    
    \section{III-\href{http://www.numpy.org/}{NumPy} - Numerical
Python}\label{iii-numpy---numerical-python}

    NumPy provides the backbone to all scientific computing in Python. It
provides all the high-dimensional structures (arrays), operations and
interfaces to C and Fortran

    Usually it is imported as np

    % Add contents below.

{\par%
\vspace{-1\baselineskip}%
\needspace{4\baselineskip}}%
\begin{notebookcell}[1]%
\begin{addmargin}[\cellleftmargin]{0em}% left, right
{\smaller%
\par%
%
\vspace{-1\smallerfontscale}%
\begin{Verbatim}[commandchars=\\\{\}]
\PY{k+kn}{import} \PY{n+nn}{numpy} \PY{k+kn}{as} \PY{n+nn}{np}
\end{Verbatim}
%
\par%
\vspace{-1\smallerfontscale}}%
\end{addmargin}
\end{notebookcell}


    For ease of writing in these tutorials we will populate the entire
namespace with numpy

    % Add contents below.

{\par%
\vspace{-1\baselineskip}%
\needspace{4\baselineskip}}%
\begin{notebookcell}[2]%
\begin{addmargin}[\cellleftmargin]{0em}% left, right
{\smaller%
\par%
%
\vspace{-1\smallerfontscale}%
\begin{Verbatim}[commandchars=\\\{\}]
\PY{k+kn}{from} \PY{n+nn}{numpy} \PY{k+kn}{import} \PY{o}{*}
\end{Verbatim}
%
\par%
\vspace{-1\smallerfontscale}}%
\end{addmargin}
\end{notebookcell}


    ``NumPy's main object is the homogeneous multidimensional array. It is a
table of elements (usually numbers), all of the same type, indexed by a
tuple of positive integers. In Numpy dimensions are called axes. The
number of axes is rank.''

Let's create an array then


    \subparagraph{Create array from list}


    % Add contents below.

{\par%
\vspace{-1\baselineskip}%
\needspace{4\baselineskip}}%
\begin{notebookcell}[4]%
\begin{addmargin}[\cellleftmargin]{0em}% left, right
{\smaller%
\par%
%
\vspace{-1\smallerfontscale}%
\begin{Verbatim}[commandchars=\\\{\}]
\PY{n}{v} \PY{o}{=} \PY{p}{[}\PY{l+m+mi}{1}\PY{p}{,}\PY{l+m+mi}{2}\PY{p}{,}\PY{l+m+mi}{3}\PY{p}{,}\PY{l+m+mi}{4}\PY{p}{]}
\PY{n}{av} \PY{o}{=} \PY{n}{np}\PY{o}{.}\PY{n}{array}\PY{p}{(}\PY{n}{v}\PY{p}{)}
\PY{n}{av}
\end{Verbatim}
%
\par%
\vspace{-1\smallerfontscale}}%
\end{addmargin}
\end{notebookcell}

\par\vspace{1\smallerfontscale}%
    \needspace{4\baselineskip}%
    % Only render the prompt if the cell is pyout.  Note, the outputs prompt 
    % block isn't used since we need to check each indiviual output and only
    % add prompts to the pyout ones.
    
        {\par%
        \vspace{-1\smallerfontscale}%
        \noindent%
        \begin{minipage}{\cellleftmargin}%
    \hfill%
    {\smaller%
    \tt%
    \color{nbframe-out-prompt}%
    Out[4]:}%
    \hspace{\inputpadding}%
    \hspace{0em}%
    \hspace{3pt}%
    \end{minipage}%%
        }%
    %
    %
    \begin{addmargin}[\cellleftmargin]{0em}% left, right
    {\smaller%
    \vspace{-1\smallerfontscale}%
    
    
    
    \begin{verbatim}
array([1, 2, 3, 4])
    \end{verbatim}

    
}%
    \end{addmargin}%

    \subparagraph{Create matrix from list (of lists)}


    % Add contents below.

{\par%
\vspace{-1\baselineskip}%
\needspace{4\baselineskip}}%
\begin{notebookcell}[5]%
\begin{addmargin}[\cellleftmargin]{0em}% left, right
{\smaller%
\par%
%
\vspace{-1\smallerfontscale}%
\begin{Verbatim}[commandchars=\\\{\}]
\PY{n}{m} \PY{o}{=} \PY{p}{[}\PY{p}{[}\PY{l+m+mi}{1}\PY{p}{,}\PY{l+m+mi}{2}\PY{p}{]}\PY{p}{,}\PY{p}{[}\PY{l+m+mi}{3}\PY{p}{,}\PY{l+m+mi}{4}\PY{p}{]}\PY{p}{]}
\PY{n}{am} \PY{o}{=} \PY{n}{array}\PY{p}{(}\PY{n}{m}\PY{p}{)}
\PY{n}{am}
\end{Verbatim}
%
\par%
\vspace{-1\smallerfontscale}}%
\end{addmargin}
\end{notebookcell}

\par\vspace{1\smallerfontscale}%
    \needspace{4\baselineskip}%
    % Only render the prompt if the cell is pyout.  Note, the outputs prompt 
    % block isn't used since we need to check each indiviual output and only
    % add prompts to the pyout ones.
    
        {\par%
        \vspace{-1\smallerfontscale}%
        \noindent%
        \begin{minipage}{\cellleftmargin}%
    \hfill%
    {\smaller%
    \tt%
    \color{nbframe-out-prompt}%
    Out[5]:}%
    \hspace{\inputpadding}%
    \hspace{0em}%
    \hspace{3pt}%
    \end{minipage}%%
        }%
    %
    %
    \begin{addmargin}[\cellleftmargin]{0em}% left, right
    {\smaller%
    \vspace{-1\smallerfontscale}%
    
    
    
    \begin{verbatim}
array([[1, 2],
       [3, 4]])
    \end{verbatim}

    
}%
    \end{addmargin}%

    \subparagraph{Arrays have different methods and attributes}


    % Add contents below.

{\par%
\vspace{-1\baselineskip}%
\needspace{4\baselineskip}}%
\begin{notebookcell}[8]%
\begin{addmargin}[\cellleftmargin]{0em}% left, right
{\smaller%
\par%
%
\vspace{-1\smallerfontscale}%
\begin{Verbatim}[commandchars=\\\{\}]
\PY{n}{am}\PY{o}{.}\PY{n}{size} \PY{c}{\PYZsh{}total number of elements}
\end{Verbatim}
%
\par%
\vspace{-1\smallerfontscale}}%
\end{addmargin}
\end{notebookcell}

\par\vspace{1\smallerfontscale}%
    \needspace{4\baselineskip}%
    % Only render the prompt if the cell is pyout.  Note, the outputs prompt 
    % block isn't used since we need to check each indiviual output and only
    % add prompts to the pyout ones.
    
        {\par%
        \vspace{-1\smallerfontscale}%
        \noindent%
        \begin{minipage}{\cellleftmargin}%
    \hfill%
    {\smaller%
    \tt%
    \color{nbframe-out-prompt}%
    Out[8]:}%
    \hspace{\inputpadding}%
    \hspace{0em}%
    \hspace{3pt}%
    \end{minipage}%%
        }%
    %
    %
    \begin{addmargin}[\cellleftmargin]{0em}% left, right
    {\smaller%
    \vspace{-1\smallerfontscale}%
    
    
    
    \begin{verbatim}
4
    \end{verbatim}

    
}%
    \end{addmargin}%
    % Add contents below.

{\par%
\vspace{-1\baselineskip}%
\needspace{4\baselineskip}}%
\begin{notebookcell}[9]%
\begin{addmargin}[\cellleftmargin]{0em}% left, right
{\smaller%
\par%
%
\vspace{-1\smallerfontscale}%
\begin{Verbatim}[commandchars=\\\{\}]
\PY{n}{am}\PY{o}{.}\PY{n}{shape}   \PY{c}{\PYZsh{}array shape (equivalent to Matlab\PYZsq{}s size)}
\end{Verbatim}
%
\par%
\vspace{-1\smallerfontscale}}%
\end{addmargin}
\end{notebookcell}

\par\vspace{1\smallerfontscale}%
    \needspace{4\baselineskip}%
    % Only render the prompt if the cell is pyout.  Note, the outputs prompt 
    % block isn't used since we need to check each indiviual output and only
    % add prompts to the pyout ones.
    
        {\par%
        \vspace{-1\smallerfontscale}%
        \noindent%
        \begin{minipage}{\cellleftmargin}%
    \hfill%
    {\smaller%
    \tt%
    \color{nbframe-out-prompt}%
    Out[9]:}%
    \hspace{\inputpadding}%
    \hspace{0em}%
    \hspace{3pt}%
    \end{minipage}%%
        }%
    %
    %
    \begin{addmargin}[\cellleftmargin]{0em}% left, right
    {\smaller%
    \vspace{-1\smallerfontscale}%
    
    
    
    \begin{verbatim}
(2, 2)
    \end{verbatim}

    
}%
    \end{addmargin}%
    % Add contents below.

{\par%
\vspace{-1\baselineskip}%
\needspace{4\baselineskip}}%
\begin{notebookcell}[10]%
\begin{addmargin}[\cellleftmargin]{0em}% left, right
{\smaller%
\par%
%
\vspace{-1\smallerfontscale}%
\begin{Verbatim}[commandchars=\\\{\}]
\PY{n}{am}\PY{o}{.}\PY{n}{size}
\end{Verbatim}
%
\par%
\vspace{-1\smallerfontscale}}%
\end{addmargin}
\end{notebookcell}

\par\vspace{1\smallerfontscale}%
    \needspace{4\baselineskip}%
    % Only render the prompt if the cell is pyout.  Note, the outputs prompt 
    % block isn't used since we need to check each indiviual output and only
    % add prompts to the pyout ones.
    
        {\par%
        \vspace{-1\smallerfontscale}%
        \noindent%
        \begin{minipage}{\cellleftmargin}%
    \hfill%
    {\smaller%
    \tt%
    \color{nbframe-out-prompt}%
    Out[10]:}%
    \hspace{\inputpadding}%
    \hspace{0em}%
    \hspace{3pt}%
    \end{minipage}%%
        }%
    %
    %
    \begin{addmargin}[\cellleftmargin]{0em}% left, right
    {\smaller%
    \vspace{-1\smallerfontscale}%
    
    
    
    \begin{verbatim}
4
    \end{verbatim}

    
}%
    \end{addmargin}%
    % Add contents below.

{\par%
\vspace{-1\baselineskip}%
\needspace{4\baselineskip}}%
\begin{notebookcell}[11]%
\begin{addmargin}[\cellleftmargin]{0em}% left, right
{\smaller%
\par%
%
\vspace{-1\smallerfontscale}%
\begin{Verbatim}[commandchars=\\\{\}]
\PY{n}{am}\PY{o}{.}\PY{n}{shape}
\end{Verbatim}
%
\par%
\vspace{-1\smallerfontscale}}%
\end{addmargin}
\end{notebookcell}

\par\vspace{1\smallerfontscale}%
    \needspace{4\baselineskip}%
    % Only render the prompt if the cell is pyout.  Note, the outputs prompt 
    % block isn't used since we need to check each indiviual output and only
    % add prompts to the pyout ones.
    
        {\par%
        \vspace{-1\smallerfontscale}%
        \noindent%
        \begin{minipage}{\cellleftmargin}%
    \hfill%
    {\smaller%
    \tt%
    \color{nbframe-out-prompt}%
    Out[11]:}%
    \hspace{\inputpadding}%
    \hspace{0em}%
    \hspace{3pt}%
    \end{minipage}%%
        }%
    %
    %
    \begin{addmargin}[\cellleftmargin]{0em}% left, right
    {\smaller%
    \vspace{-1\smallerfontscale}%
    
    
    
    \begin{verbatim}
(2, 2)
    \end{verbatim}

    
}%
    \end{addmargin}%
    % Add contents below.

{\par%
\vspace{-1\baselineskip}%
\needspace{4\baselineskip}}%
\begin{notebookcell}[12]%
\begin{addmargin}[\cellleftmargin]{0em}% left, right
{\smaller%
\par%
%
\vspace{-1\smallerfontscale}%
\begin{Verbatim}[commandchars=\\\{\}]
\PY{n}{am}\PY{o}{.}\PY{n}{ndim}
\end{Verbatim}
%
\par%
\vspace{-1\smallerfontscale}}%
\end{addmargin}
\end{notebookcell}

\par\vspace{1\smallerfontscale}%
    \needspace{4\baselineskip}%
    % Only render the prompt if the cell is pyout.  Note, the outputs prompt 
    % block isn't used since we need to check each indiviual output and only
    % add prompts to the pyout ones.
    
        {\par%
        \vspace{-1\smallerfontscale}%
        \noindent%
        \begin{minipage}{\cellleftmargin}%
    \hfill%
    {\smaller%
    \tt%
    \color{nbframe-out-prompt}%
    Out[12]:}%
    \hspace{\inputpadding}%
    \hspace{0em}%
    \hspace{3pt}%
    \end{minipage}%%
        }%
    %
    %
    \begin{addmargin}[\cellleftmargin]{0em}% left, right
    {\smaller%
    \vspace{-1\smallerfontscale}%
    
    
    
    \begin{verbatim}
2
    \end{verbatim}

    
}%
    \end{addmargin}%
    In total, a lot of attributes and methods are available by default

    % Add contents below.

{\par%
\vspace{-1\baselineskip}%
\needspace{4\baselineskip}}%
\begin{notebookcell}[13]%
\begin{addmargin}[\cellleftmargin]{0em}% left, right
{\smaller%
\par%
%
\vspace{-1\smallerfontscale}%
\begin{Verbatim}[commandchars=\\\{\}]
\PY{n+nb}{len}\PY{p}{(}\PY{n+nb}{dir}\PY{p}{(}\PY{n}{am}\PY{p}{)}\PY{p}{)}
\end{Verbatim}
%
\par%
\vspace{-1\smallerfontscale}}%
\end{addmargin}
\end{notebookcell}

\par\vspace{1\smallerfontscale}%
    \needspace{4\baselineskip}%
    % Only render the prompt if the cell is pyout.  Note, the outputs prompt 
    % block isn't used since we need to check each indiviual output and only
    % add prompts to the pyout ones.
    
        {\par%
        \vspace{-1\smallerfontscale}%
        \noindent%
        \begin{minipage}{\cellleftmargin}%
    \hfill%
    {\smaller%
    \tt%
    \color{nbframe-out-prompt}%
    Out[13]:}%
    \hspace{\inputpadding}%
    \hspace{0em}%
    \hspace{3pt}%
    \end{minipage}%%
        }%
    %
    %
    \begin{addmargin}[\cellleftmargin]{0em}% left, right
    {\smaller%
    \vspace{-1\smallerfontscale}%
    
    
    
    \begin{verbatim}
163
    \end{verbatim}

    
}%
    \end{addmargin}%
    One important attribute is dtype. NumPy arrays are statically typed and
contain always the same type of data. (For the next generation of array
containers, see
\href{http://blaze.pydata.org/docs/latest/index.html}{Blaze} )
Therefore, if not specified, it is infered from the data when the array
is created

    % Add contents below.

{\par%
\vspace{-1\baselineskip}%
\needspace{4\baselineskip}}%
\begin{notebookcell}[14]%
\begin{addmargin}[\cellleftmargin]{0em}% left, right
{\smaller%
\par%
%
\vspace{-1\smallerfontscale}%
\begin{Verbatim}[commandchars=\\\{\}]
\PY{n}{am}\PY{o}{.}\PY{n}{dtype}
\end{Verbatim}
%
\par%
\vspace{-1\smallerfontscale}}%
\end{addmargin}
\end{notebookcell}

\par\vspace{1\smallerfontscale}%
    \needspace{4\baselineskip}%
    % Only render the prompt if the cell is pyout.  Note, the outputs prompt 
    % block isn't used since we need to check each indiviual output and only
    % add prompts to the pyout ones.
    
        {\par%
        \vspace{-1\smallerfontscale}%
        \noindent%
        \begin{minipage}{\cellleftmargin}%
    \hfill%
    {\smaller%
    \tt%
    \color{nbframe-out-prompt}%
    Out[14]:}%
    \hspace{\inputpadding}%
    \hspace{0em}%
    \hspace{3pt}%
    \end{minipage}%%
        }%
    %
    %
    \begin{addmargin}[\cellleftmargin]{0em}% left, right
    {\smaller%
    \vspace{-1\smallerfontscale}%
    
    
    
    \begin{verbatim}
dtype('int64')
    \end{verbatim}

    
}%
    \end{addmargin}%
    % Add contents below.

{\par%
\vspace{-1\baselineskip}%
\needspace{4\baselineskip}}%
\begin{notebookcell}[15]%
\begin{addmargin}[\cellleftmargin]{0em}% left, right
{\smaller%
\par%
%
\vspace{-1\smallerfontscale}%
\begin{Verbatim}[commandchars=\\\{\}]
\PY{n}{array}\PY{p}{(}\PY{p}{[}\PY{l+m+mi}{1}\PY{p}{,}\PY{l+m+mi}{2}\PY{p}{,}\PY{l+m+mi}{3}\PY{p}{]}\PY{p}{)}\PY{o}{.}\PY{n}{dtype}
\end{Verbatim}
%
\par%
\vspace{-1\smallerfontscale}}%
\end{addmargin}
\end{notebookcell}

\par\vspace{1\smallerfontscale}%
    \needspace{4\baselineskip}%
    % Only render the prompt if the cell is pyout.  Note, the outputs prompt 
    % block isn't used since we need to check each indiviual output and only
    % add prompts to the pyout ones.
    
        {\par%
        \vspace{-1\smallerfontscale}%
        \noindent%
        \begin{minipage}{\cellleftmargin}%
    \hfill%
    {\smaller%
    \tt%
    \color{nbframe-out-prompt}%
    Out[15]:}%
    \hspace{\inputpadding}%
    \hspace{0em}%
    \hspace{3pt}%
    \end{minipage}%%
        }%
    %
    %
    \begin{addmargin}[\cellleftmargin]{0em}% left, right
    {\smaller%
    \vspace{-1\smallerfontscale}%
    
    
    
    \begin{verbatim}
dtype('int64')
    \end{verbatim}

    
}%
    \end{addmargin}%
    % Add contents below.

{\par%
\vspace{-1\baselineskip}%
\needspace{4\baselineskip}}%
\begin{notebookcell}[16]%
\begin{addmargin}[\cellleftmargin]{0em}% left, right
{\smaller%
\par%
%
\vspace{-1\smallerfontscale}%
\begin{Verbatim}[commandchars=\\\{\}]
\PY{n}{array}\PY{p}{(}\PY{p}{[}\PY{l+m+mf}{1.1}\PY{p}{,}\PY{l+m+mf}{2.2}\PY{p}{,}\PY{l+m+mf}{3.4}\PY{p}{]}\PY{p}{)}\PY{o}{.}\PY{n}{dtype}
\end{Verbatim}
%
\par%
\vspace{-1\smallerfontscale}}%
\end{addmargin}
\end{notebookcell}

\par\vspace{1\smallerfontscale}%
    \needspace{4\baselineskip}%
    % Only render the prompt if the cell is pyout.  Note, the outputs prompt 
    % block isn't used since we need to check each indiviual output and only
    % add prompts to the pyout ones.
    
        {\par%
        \vspace{-1\smallerfontscale}%
        \noindent%
        \begin{minipage}{\cellleftmargin}%
    \hfill%
    {\smaller%
    \tt%
    \color{nbframe-out-prompt}%
    Out[16]:}%
    \hspace{\inputpadding}%
    \hspace{0em}%
    \hspace{3pt}%
    \end{minipage}%%
        }%
    %
    %
    \begin{addmargin}[\cellleftmargin]{0em}% left, right
    {\smaller%
    \vspace{-1\smallerfontscale}%
    
    
    
    \begin{verbatim}
dtype('float64')
    \end{verbatim}

    
}%
    \end{addmargin}%
    % Add contents below.

{\par%
\vspace{-1\baselineskip}%
\needspace{4\baselineskip}}%
\begin{notebookcell}[17]%
\begin{addmargin}[\cellleftmargin]{0em}% left, right
{\smaller%
\par%
%
\vspace{-1\smallerfontscale}%
\begin{Verbatim}[commandchars=\\\{\}]
\PY{n}{array}\PY{p}{(}\PY{p}{[}\PY{l+m+mi}{1}\PY{o}{+}\PY{l+m+mi}{2j}\PY{p}{,}\PY{l+m+mi}{2}\PY{o}{+}\PY{l+m+mi}{4j}\PY{p}{,}\PY{l+m+mi}{3}\PY{p}{]}\PY{p}{)}\PY{o}{.}\PY{n}{dtype}
\end{Verbatim}
%
\par%
\vspace{-1\smallerfontscale}}%
\end{addmargin}
\end{notebookcell}

\par\vspace{1\smallerfontscale}%
    \needspace{4\baselineskip}%
    % Only render the prompt if the cell is pyout.  Note, the outputs prompt 
    % block isn't used since we need to check each indiviual output and only
    % add prompts to the pyout ones.
    
        {\par%
        \vspace{-1\smallerfontscale}%
        \noindent%
        \begin{minipage}{\cellleftmargin}%
    \hfill%
    {\smaller%
    \tt%
    \color{nbframe-out-prompt}%
    Out[17]:}%
    \hspace{\inputpadding}%
    \hspace{0em}%
    \hspace{3pt}%
    \end{minipage}%%
        }%
    %
    %
    \begin{addmargin}[\cellleftmargin]{0em}% left, right
    {\smaller%
    \vspace{-1\smallerfontscale}%
    
    
    
    \begin{verbatim}
dtype('complex128')
    \end{verbatim}

    
}%
    \end{addmargin}%
    % Add contents below.

{\par%
\vspace{-1\baselineskip}%
\needspace{4\baselineskip}}%
\begin{notebookcell}[18]%
\begin{addmargin}[\cellleftmargin]{0em}% left, right
{\smaller%
\par%
%
\vspace{-1\smallerfontscale}%
\begin{Verbatim}[commandchars=\\\{\}]
\PY{n}{array}\PY{p}{(}\PY{p}{[}\PY{n+nb+bp}{True}\PY{p}{,}\PY{n+nb+bp}{False}\PY{p}{]}\PY{p}{)}\PY{o}{.}\PY{n}{dtype}
\end{Verbatim}
%
\par%
\vspace{-1\smallerfontscale}}%
\end{addmargin}
\end{notebookcell}

\par\vspace{1\smallerfontscale}%
    \needspace{4\baselineskip}%
    % Only render the prompt if the cell is pyout.  Note, the outputs prompt 
    % block isn't used since we need to check each indiviual output and only
    % add prompts to the pyout ones.
    
        {\par%
        \vspace{-1\smallerfontscale}%
        \noindent%
        \begin{minipage}{\cellleftmargin}%
    \hfill%
    {\smaller%
    \tt%
    \color{nbframe-out-prompt}%
    Out[18]:}%
    \hspace{\inputpadding}%
    \hspace{0em}%
    \hspace{3pt}%
    \end{minipage}%%
        }%
    %
    %
    \begin{addmargin}[\cellleftmargin]{0em}% left, right
    {\smaller%
    \vspace{-1\smallerfontscale}%
    
    
    
    \begin{verbatim}
dtype('bool')
    \end{verbatim}

    
}%
    \end{addmargin}%
    There is an extensive list of available dtypes

    % Add contents below.

{\par%
\vspace{-1\baselineskip}%
\needspace{4\baselineskip}}%
\begin{notebookcell}[19]%
\begin{addmargin}[\cellleftmargin]{0em}% left, right
{\smaller%
\par%
%
\vspace{-1\smallerfontscale}%
\begin{Verbatim}[commandchars=\\\{\}]
\PY{n}{sctypes}
\end{Verbatim}
%
\par%
\vspace{-1\smallerfontscale}}%
\end{addmargin}
\end{notebookcell}

\par\vspace{1\smallerfontscale}%
    \needspace{4\baselineskip}%
    % Only render the prompt if the cell is pyout.  Note, the outputs prompt 
    % block isn't used since we need to check each indiviual output and only
    % add prompts to the pyout ones.
    
        {\par%
        \vspace{-1\smallerfontscale}%
        \noindent%
        \begin{minipage}{\cellleftmargin}%
    \hfill%
    {\smaller%
    \tt%
    \color{nbframe-out-prompt}%
    Out[19]:}%
    \hspace{\inputpadding}%
    \hspace{0em}%
    \hspace{3pt}%
    \end{minipage}%%
        }%
    %
    %
    \begin{addmargin}[\cellleftmargin]{0em}% left, right
    {\smaller%
    \vspace{-1\smallerfontscale}%
    
    
    
    \begin{verbatim}
{'complex': [numpy.complex64, numpy.complex128, numpy.complex256],
 'float': [numpy.float16, numpy.float32, numpy.float64, numpy.float128],
 'int': [numpy.int8, numpy.int16, numpy.int32, numpy.int64],
 'others': [bool, object, str, unicode, numpy.void],
 'uint': [numpy.uint8, numpy.uint16, numpy.uint32, numpy.uint64]}
    \end{verbatim}

    
}%
    \end{addmargin}%

    \subparagraph{Array generating funtions}


    % Add contents below.

{\par%
\vspace{-1\baselineskip}%
\needspace{4\baselineskip}}%
\begin{notebookcell}[20]%
\begin{addmargin}[\cellleftmargin]{0em}% left, right
{\smaller%
\par%
%
\vspace{-1\smallerfontscale}%
\begin{Verbatim}[commandchars=\\\{\}]
\PY{n}{arange}\PY{p}{(}\PY{l+m+mi}{0}\PY{p}{,}\PY{l+m+mi}{10}\PY{p}{,}\PY{l+m+mi}{2}\PY{p}{)}
\end{Verbatim}
%
\par%
\vspace{-1\smallerfontscale}}%
\end{addmargin}
\end{notebookcell}

\par\vspace{1\smallerfontscale}%
    \needspace{4\baselineskip}%
    % Only render the prompt if the cell is pyout.  Note, the outputs prompt 
    % block isn't used since we need to check each indiviual output and only
    % add prompts to the pyout ones.
    
        {\par%
        \vspace{-1\smallerfontscale}%
        \noindent%
        \begin{minipage}{\cellleftmargin}%
    \hfill%
    {\smaller%
    \tt%
    \color{nbframe-out-prompt}%
    Out[20]:}%
    \hspace{\inputpadding}%
    \hspace{0em}%
    \hspace{3pt}%
    \end{minipage}%%
        }%
    %
    %
    \begin{addmargin}[\cellleftmargin]{0em}% left, right
    {\smaller%
    \vspace{-1\smallerfontscale}%
    
    
    
    \begin{verbatim}
array([0, 2, 4, 6, 8])
    \end{verbatim}

    
}%
    \end{addmargin}%
    % Add contents below.

{\par%
\vspace{-1\baselineskip}%
\needspace{4\baselineskip}}%
\begin{notebookcell}[21]%
\begin{addmargin}[\cellleftmargin]{0em}% left, right
{\smaller%
\par%
%
\vspace{-1\smallerfontscale}%
\begin{Verbatim}[commandchars=\\\{\}]
\PY{n}{arange}\PY{p}{(}\PY{o}{\PYZhy{}}\PY{l+m+mi}{1}\PY{p}{,}\PY{l+m+mi}{1}\PY{p}{,}\PY{l+m+mf}{0.1}\PY{p}{)}
\end{Verbatim}
%
\par%
\vspace{-1\smallerfontscale}}%
\end{addmargin}
\end{notebookcell}

\par\vspace{1\smallerfontscale}%
    \needspace{4\baselineskip}%
    % Only render the prompt if the cell is pyout.  Note, the outputs prompt 
    % block isn't used since we need to check each indiviual output and only
    % add prompts to the pyout ones.
    
        {\par%
        \vspace{-1\smallerfontscale}%
        \noindent%
        \begin{minipage}{\cellleftmargin}%
    \hfill%
    {\smaller%
    \tt%
    \color{nbframe-out-prompt}%
    Out[21]:}%
    \hspace{\inputpadding}%
    \hspace{0em}%
    \hspace{3pt}%
    \end{minipage}%%
        }%
    %
    %
    \begin{addmargin}[\cellleftmargin]{0em}% left, right
    {\smaller%
    \vspace{-1\smallerfontscale}%
    
    
    
    \begin{verbatim}
array([ -1.00000000e+00,  -9.00000000e-01,  -8.00000000e-01,
        -7.00000000e-01,  -6.00000000e-01,  -5.00000000e-01,
        -4.00000000e-01,  -3.00000000e-01,  -2.00000000e-01,
        -1.00000000e-01,  -2.22044605e-16,   1.00000000e-01,
         2.00000000e-01,   3.00000000e-01,   4.00000000e-01,
         5.00000000e-01,   6.00000000e-01,   7.00000000e-01,
         8.00000000e-01,   9.00000000e-01])
    \end{verbatim}

    
}%
    \end{addmargin}%
    % Add contents below.

{\par%
\vspace{-1\baselineskip}%
\needspace{4\baselineskip}}%
\begin{notebookcell}[22]%
\begin{addmargin}[\cellleftmargin]{0em}% left, right
{\smaller%
\par%
%
\vspace{-1\smallerfontscale}%
\begin{Verbatim}[commandchars=\\\{\}]
\PY{n}{linspace}\PY{p}{(}\PY{l+m+mi}{0}\PY{p}{,}\PY{l+m+mi}{10}\PY{p}{,}\PY{l+m+mi}{6}\PY{p}{)}
\end{Verbatim}
%
\par%
\vspace{-1\smallerfontscale}}%
\end{addmargin}
\end{notebookcell}

\par\vspace{1\smallerfontscale}%
    \needspace{4\baselineskip}%
    % Only render the prompt if the cell is pyout.  Note, the outputs prompt 
    % block isn't used since we need to check each indiviual output and only
    % add prompts to the pyout ones.
    
        {\par%
        \vspace{-1\smallerfontscale}%
        \noindent%
        \begin{minipage}{\cellleftmargin}%
    \hfill%
    {\smaller%
    \tt%
    \color{nbframe-out-prompt}%
    Out[22]:}%
    \hspace{\inputpadding}%
    \hspace{0em}%
    \hspace{3pt}%
    \end{minipage}%%
        }%
    %
    %
    \begin{addmargin}[\cellleftmargin]{0em}% left, right
    {\smaller%
    \vspace{-1\smallerfontscale}%
    
    
    
    \begin{verbatim}
array([  0.,   2.,   4.,   6.,   8.,  10.])
    \end{verbatim}

    
}%
    \end{addmargin}%
    % Add contents below.

{\par%
\vspace{-1\baselineskip}%
\needspace{4\baselineskip}}%
\begin{notebookcell}[23]%
\begin{addmargin}[\cellleftmargin]{0em}% left, right
{\smaller%
\par%
%
\vspace{-1\smallerfontscale}%
\begin{Verbatim}[commandchars=\\\{\}]
\PY{n}{\PYZus{}}\PY{o}{.}\PY{n}{dtype}
\end{Verbatim}
%
\par%
\vspace{-1\smallerfontscale}}%
\end{addmargin}
\end{notebookcell}

\par\vspace{1\smallerfontscale}%
    \needspace{4\baselineskip}%
    % Only render the prompt if the cell is pyout.  Note, the outputs prompt 
    % block isn't used since we need to check each indiviual output and only
    % add prompts to the pyout ones.
    
        {\par%
        \vspace{-1\smallerfontscale}%
        \noindent%
        \begin{minipage}{\cellleftmargin}%
    \hfill%
    {\smaller%
    \tt%
    \color{nbframe-out-prompt}%
    Out[23]:}%
    \hspace{\inputpadding}%
    \hspace{0em}%
    \hspace{3pt}%
    \end{minipage}%%
        }%
    %
    %
    \begin{addmargin}[\cellleftmargin]{0em}% left, right
    {\smaller%
    \vspace{-1\smallerfontscale}%
    
    
    
    \begin{verbatim}
dtype('float64')
    \end{verbatim}

    
}%
    \end{addmargin}%
    % Add contents below.

{\par%
\vspace{-1\baselineskip}%
\needspace{4\baselineskip}}%
\begin{notebookcell}[24]%
\begin{addmargin}[\cellleftmargin]{0em}% left, right
{\smaller%
\par%
%
\vspace{-1\smallerfontscale}%
\begin{Verbatim}[commandchars=\\\{\}]
\PY{n}{linspace}\PY{p}{(}\PY{l+m+mi}{0}\PY{p}{,}\PY{l+m+mi}{10}\PY{p}{,}\PY{l+m+mi}{6}\PY{p}{,}\PY{n}{dtype}\PY{o}{=}\PY{l+s}{\PYZsq{}}\PY{l+s}{uint32}\PY{l+s}{\PYZsq{}}\PY{p}{)}
\end{Verbatim}
%
\par%
\vspace{-1\smallerfontscale}}%
\end{addmargin}
\end{notebookcell}

\par\vspace{1\smallerfontscale}%
    \needspace{4\baselineskip}%
    % Only render the prompt if the cell is pyout.  Note, the outputs prompt 
    % block isn't used since we need to check each indiviual output and only
    % add prompts to the pyout ones.
    
        {\par%
        \vspace{-1\smallerfontscale}%
        \noindent%
        \begin{minipage}{\cellleftmargin}%
    \hfill%
    {\smaller%
    \tt%
    \color{nbframe-out-prompt}%
    Out[24]:}%
    \hspace{\inputpadding}%
    \hspace{0em}%
    \hspace{3pt}%
    \end{minipage}%%
        }%
    %
    %
    \begin{addmargin}[\cellleftmargin]{0em}% left, right
    {\smaller%
    \vspace{-1\smallerfontscale}%
    
    
    
    \begin{verbatim}
array([ 0,  2,  4,  6,  8, 10], dtype=uint32)
    \end{verbatim}

    
}%
    \end{addmargin}%
    % Add contents below.

{\par%
\vspace{-1\baselineskip}%
\needspace{4\baselineskip}}%
\begin{notebookcell}[25]%
\begin{addmargin}[\cellleftmargin]{0em}% left, right
{\smaller%
\par%
%
\vspace{-1\smallerfontscale}%
\begin{Verbatim}[commandchars=\\\{\}]
\PY{n}{\PYZus{}}\PY{o}{.}\PY{n}{dtype}
\end{Verbatim}
%
\par%
\vspace{-1\smallerfontscale}}%
\end{addmargin}
\end{notebookcell}

\par\vspace{1\smallerfontscale}%
    \needspace{4\baselineskip}%
    % Only render the prompt if the cell is pyout.  Note, the outputs prompt 
    % block isn't used since we need to check each indiviual output and only
    % add prompts to the pyout ones.
    
        {\par%
        \vspace{-1\smallerfontscale}%
        \noindent%
        \begin{minipage}{\cellleftmargin}%
    \hfill%
    {\smaller%
    \tt%
    \color{nbframe-out-prompt}%
    Out[25]:}%
    \hspace{\inputpadding}%
    \hspace{0em}%
    \hspace{3pt}%
    \end{minipage}%%
        }%
    %
    %
    \begin{addmargin}[\cellleftmargin]{0em}% left, right
    {\smaller%
    \vspace{-1\smallerfontscale}%
    
    
    
    \begin{verbatim}
dtype('uint32')
    \end{verbatim}

    
}%
    \end{addmargin}%
    % Add contents below.

{\par%
\vspace{-1\baselineskip}%
\needspace{4\baselineskip}}%
\begin{notebookcell}[26]%
\begin{addmargin}[\cellleftmargin]{0em}% left, right
{\smaller%
\par%
%
\vspace{-1\smallerfontscale}%
\begin{Verbatim}[commandchars=\\\{\}]
\PY{n}{linspace}\PY{p}{(}\PY{o}{\PYZhy{}}\PY{l+m+mi}{1}\PY{p}{,}\PY{l+m+mi}{10}\PY{p}{,}\PY{l+m+mi}{6}\PY{p}{,}\PY{n}{dtype}\PY{o}{=}\PY{l+s}{\PYZsq{}}\PY{l+s}{uint32}\PY{l+s}{\PYZsq{}}\PY{p}{)}
\end{Verbatim}
%
\par%
\vspace{-1\smallerfontscale}}%
\end{addmargin}
\end{notebookcell}

\par\vspace{1\smallerfontscale}%
    \needspace{4\baselineskip}%
    % Only render the prompt if the cell is pyout.  Note, the outputs prompt 
    % block isn't used since we need to check each indiviual output and only
    % add prompts to the pyout ones.
    
        {\par%
        \vspace{-1\smallerfontscale}%
        \noindent%
        \begin{minipage}{\cellleftmargin}%
    \hfill%
    {\smaller%
    \tt%
    \color{nbframe-out-prompt}%
    Out[26]:}%
    \hspace{\inputpadding}%
    \hspace{0em}%
    \hspace{3pt}%
    \end{minipage}%%
        }%
    %
    %
    \begin{addmargin}[\cellleftmargin]{0em}% left, right
    {\smaller%
    \vspace{-1\smallerfontscale}%
    
    
    
    \begin{verbatim}
array([4294967295,          1,          3,          5,          7,
               10], dtype=uint32)
    \end{verbatim}

    
}%
    \end{addmargin}%
    % Add contents below.

{\par%
\vspace{-1\baselineskip}%
\needspace{4\baselineskip}}%
\begin{notebookcell}[27]%
\begin{addmargin}[\cellleftmargin]{0em}% left, right
{\smaller%
\par%
%
\vspace{-1\smallerfontscale}%
\begin{Verbatim}[commandchars=\\\{\}]
\PY{n}{finfo}\PY{p}{(}\PY{l+s}{\PYZsq{}}\PY{l+s}{float64}\PY{l+s}{\PYZsq{}}\PY{p}{)}
\end{Verbatim}
%
\par%
\vspace{-1\smallerfontscale}}%
\end{addmargin}
\end{notebookcell}

\par\vspace{1\smallerfontscale}%
    \needspace{4\baselineskip}%
    % Only render the prompt if the cell is pyout.  Note, the outputs prompt 
    % block isn't used since we need to check each indiviual output and only
    % add prompts to the pyout ones.
    
        {\par%
        \vspace{-1\smallerfontscale}%
        \noindent%
        \begin{minipage}{\cellleftmargin}%
    \hfill%
    {\smaller%
    \tt%
    \color{nbframe-out-prompt}%
    Out[27]:}%
    \hspace{\inputpadding}%
    \hspace{0em}%
    \hspace{3pt}%
    \end{minipage}%%
        }%
    %
    %
    \begin{addmargin}[\cellleftmargin]{0em}% left, right
    {\smaller%
    \vspace{-1\smallerfontscale}%
    
    
    
    \begin{verbatim}
finfo(resolution=1e-15, min=-1.7976931348623157e+308, max=1.7976931348623157e+308, dtype=float64)
    \end{verbatim}

    
}%
    \end{addmargin}%
    % Add contents below.

{\par%
\vspace{-1\baselineskip}%
\needspace{4\baselineskip}}%
\begin{notebookcell}[28]%
\begin{addmargin}[\cellleftmargin]{0em}% left, right
{\smaller%
\par%
%
\vspace{-1\smallerfontscale}%
\begin{Verbatim}[commandchars=\\\{\}]
\PY{n}{iinfo}\PY{p}{(}\PY{n}{np}\PY{o}{.}\PY{n}{int8}\PY{p}{)}\PY{p}{,} \PY{n}{iinfo}\PY{p}{(}\PY{l+s}{\PYZsq{}}\PY{l+s}{uint8}\PY{l+s}{\PYZsq{}}\PY{p}{)}
\end{Verbatim}
%
\par%
\vspace{-1\smallerfontscale}}%
\end{addmargin}
\end{notebookcell}

\par\vspace{1\smallerfontscale}%
    \needspace{4\baselineskip}%
    % Only render the prompt if the cell is pyout.  Note, the outputs prompt 
    % block isn't used since we need to check each indiviual output and only
    % add prompts to the pyout ones.
    
        {\par%
        \vspace{-1\smallerfontscale}%
        \noindent%
        \begin{minipage}{\cellleftmargin}%
    \hfill%
    {\smaller%
    \tt%
    \color{nbframe-out-prompt}%
    Out[28]:}%
    \hspace{\inputpadding}%
    \hspace{0em}%
    \hspace{3pt}%
    \end{minipage}%%
        }%
    %
    %
    \begin{addmargin}[\cellleftmargin]{0em}% left, right
    {\smaller%
    \vspace{-1\smallerfontscale}%
    
    
    
    \begin{verbatim}
(iinfo(min=-128, max=127, dtype=int8), iinfo(min=0, max=255, dtype=uint8))
    \end{verbatim}

    
}%
    \end{addmargin}%
    % Add contents below.

{\par%
\vspace{-1\baselineskip}%
\needspace{4\baselineskip}}%
\begin{notebookcell}[29]%
\begin{addmargin}[\cellleftmargin]{0em}% left, right
{\smaller%
\par%
%
\vspace{-1\smallerfontscale}%
\begin{Verbatim}[commandchars=\\\{\}]
\PY{n}{logspace}\PY{p}{(}\PY{l+m+mi}{0}\PY{p}{,}\PY{l+m+mi}{1}\PY{p}{,}\PY{l+m+mi}{10}\PY{p}{)}
\end{Verbatim}
%
\par%
\vspace{-1\smallerfontscale}}%
\end{addmargin}
\end{notebookcell}

\par\vspace{1\smallerfontscale}%
    \needspace{4\baselineskip}%
    % Only render the prompt if the cell is pyout.  Note, the outputs prompt 
    % block isn't used since we need to check each indiviual output and only
    % add prompts to the pyout ones.
    
        {\par%
        \vspace{-1\smallerfontscale}%
        \noindent%
        \begin{minipage}{\cellleftmargin}%
    \hfill%
    {\smaller%
    \tt%
    \color{nbframe-out-prompt}%
    Out[29]:}%
    \hspace{\inputpadding}%
    \hspace{0em}%
    \hspace{3pt}%
    \end{minipage}%%
        }%
    %
    %
    \begin{addmargin}[\cellleftmargin]{0em}% left, right
    {\smaller%
    \vspace{-1\smallerfontscale}%
    
    
    
    \begin{verbatim}
array([  1.        ,   1.29154967,   1.66810054,   2.15443469,
         2.7825594 ,   3.59381366,   4.64158883,   5.9948425 ,
         7.74263683,  10.        ])
    \end{verbatim}

    
}%
    \end{addmargin}%
    % Add contents below.

{\par%
\vspace{-1\baselineskip}%
\needspace{4\baselineskip}}%
\begin{notebookcell}[30]%
\begin{addmargin}[\cellleftmargin]{0em}% left, right
{\smaller%
\par%
%
\vspace{-1\smallerfontscale}%
\begin{Verbatim}[commandchars=\\\{\}]
\PY{n}{x}\PY{p}{,}\PY{n}{y} \PY{o}{=} \PY{n}{mgrid}\PY{p}{[}\PY{l+m+mi}{0}\PY{p}{:}\PY{l+m+mi}{5}\PY{p}{,}\PY{l+m+mi}{0}\PY{p}{:}\PY{l+m+mi}{5}\PY{p}{]}
\PY{n}{x}\PY{p}{,} \PY{n}{y}
\end{Verbatim}
%
\par%
\vspace{-1\smallerfontscale}}%
\end{addmargin}
\end{notebookcell}

\par\vspace{1\smallerfontscale}%
    \needspace{4\baselineskip}%
    % Only render the prompt if the cell is pyout.  Note, the outputs prompt 
    % block isn't used since we need to check each indiviual output and only
    % add prompts to the pyout ones.
    
        {\par%
        \vspace{-1\smallerfontscale}%
        \noindent%
        \begin{minipage}{\cellleftmargin}%
    \hfill%
    {\smaller%
    \tt%
    \color{nbframe-out-prompt}%
    Out[30]:}%
    \hspace{\inputpadding}%
    \hspace{0em}%
    \hspace{3pt}%
    \end{minipage}%%
        }%
    %
    %
    \begin{addmargin}[\cellleftmargin]{0em}% left, right
    {\smaller%
    \vspace{-1\smallerfontscale}%
    
    
    
    \begin{verbatim}
(array([[0, 0, 0, 0, 0],
        [1, 1, 1, 1, 1],
        [2, 2, 2, 2, 2],
        [3, 3, 3, 3, 3],
        [4, 4, 4, 4, 4]]), array([[0, 1, 2, 3, 4],
        [0, 1, 2, 3, 4],
        [0, 1, 2, 3, 4],
        [0, 1, 2, 3, 4],
        [0, 1, 2, 3, 4]]))
    \end{verbatim}

    
}%
    \end{addmargin}%
    When we already have each of the axis we call meshgrid

    % Add contents below.

{\par%
\vspace{-1\baselineskip}%
\needspace{4\baselineskip}}%
\begin{notebookcell}[31]%
\begin{addmargin}[\cellleftmargin]{0em}% left, right
{\smaller%
\par%
%
\vspace{-1\smallerfontscale}%
\begin{Verbatim}[commandchars=\\\{\}]
\PY{n}{x} \PY{o}{=} \PY{n}{linspace}\PY{p}{(}\PY{l+m+mi}{0}\PY{p}{,}\PY{l+m+mi}{1}\PY{p}{,}\PY{l+m+mi}{10}\PY{p}{)}
\PY{n}{y} \PY{o}{=} \PY{n}{linspace}\PY{p}{(}\PY{o}{\PYZhy{}}\PY{l+m+mi}{1}\PY{p}{,}\PY{l+m+mi}{1}\PY{p}{,}\PY{l+m+mi}{5}\PY{p}{)}
\PY{n}{meshgrid}\PY{p}{(}\PY{n}{x}\PY{p}{,}\PY{n}{y}\PY{p}{)}
\end{Verbatim}
%
\par%
\vspace{-1\smallerfontscale}}%
\end{addmargin}
\end{notebookcell}

\par\vspace{1\smallerfontscale}%
    \needspace{4\baselineskip}%
    % Only render the prompt if the cell is pyout.  Note, the outputs prompt 
    % block isn't used since we need to check each indiviual output and only
    % add prompts to the pyout ones.
    
        {\par%
        \vspace{-1\smallerfontscale}%
        \noindent%
        \begin{minipage}{\cellleftmargin}%
    \hfill%
    {\smaller%
    \tt%
    \color{nbframe-out-prompt}%
    Out[31]:}%
    \hspace{\inputpadding}%
    \hspace{0em}%
    \hspace{3pt}%
    \end{minipage}%%
        }%
    %
    %
    \begin{addmargin}[\cellleftmargin]{0em}% left, right
    {\smaller%
    \vspace{-1\smallerfontscale}%
    
    
    
    \begin{verbatim}
[array([[ 0.        ,  0.11111111,  0.22222222,  0.33333333,  0.44444444,
          0.55555556,  0.66666667,  0.77777778,  0.88888889,  1.        ],
        [ 0.        ,  0.11111111,  0.22222222,  0.33333333,  0.44444444,
          0.55555556,  0.66666667,  0.77777778,  0.88888889,  1.        ],
        [ 0.        ,  0.11111111,  0.22222222,  0.33333333,  0.44444444,
          0.55555556,  0.66666667,  0.77777778,  0.88888889,  1.        ],
        [ 0.        ,  0.11111111,  0.22222222,  0.33333333,  0.44444444,
          0.55555556,  0.66666667,  0.77777778,  0.88888889,  1.        ],
        [ 0.        ,  0.11111111,  0.22222222,  0.33333333,  0.44444444,
          0.55555556,  0.66666667,  0.77777778,  0.88888889,  1.        ]]),
 array([[-1. , -1. , -1. , -1. , -1. , -1. , -1. , -1. , -1. , -1. ],
        [-0.5, -0.5, -0.5, -0.5, -0.5, -0.5, -0.5, -0.5, -0.5, -0.5],
        [ 0. ,  0. ,  0. ,  0. ,  0. ,  0. ,  0. ,  0. ,  0. ,  0. ],
        [ 0.5,  0.5,  0.5,  0.5,  0.5,  0.5,  0.5,  0.5,  0.5,  0.5],
        [ 1. ,  1. ,  1. ,  1. ,  1. ,  1. ,  1. ,  1. ,  1. ,  1. ]])]
    \end{verbatim}

    
}%
    \end{addmargin}%

    \subparagraph{A lot of times we will want to generate random data to test our
implementations}


    % Add contents below.

{\par%
\vspace{-1\baselineskip}%
\needspace{4\baselineskip}}%
\begin{notebookcell}[32]%
\begin{addmargin}[\cellleftmargin]{0em}% left, right
{\smaller%
\par%
%
\vspace{-1\smallerfontscale}%
\begin{Verbatim}[commandchars=\\\{\}]
\PY{k+kn}{from} \PY{n+nn}{numpy} \PY{k+kn}{import} \PY{n}{random}
\end{Verbatim}
%
\par%
\vspace{-1\smallerfontscale}}%
\end{addmargin}
\end{notebookcell}


    % Add contents below.

{\par%
\vspace{-1\baselineskip}%
\needspace{4\baselineskip}}%
\begin{notebookcell}[33]%
\begin{addmargin}[\cellleftmargin]{0em}% left, right
{\smaller%
\par%
%
\vspace{-1\smallerfontscale}%
\begin{Verbatim}[commandchars=\\\{\}]
\PY{n}{random}\PY{o}{.}\PY{n}{normal}\PY{p}{(}\PY{l+m+mi}{0}\PY{p}{,}\PY{l+m+mi}{1}\PY{p}{)}
\end{Verbatim}
%
\par%
\vspace{-1\smallerfontscale}}%
\end{addmargin}
\end{notebookcell}

\par\vspace{1\smallerfontscale}%
    \needspace{4\baselineskip}%
    % Only render the prompt if the cell is pyout.  Note, the outputs prompt 
    % block isn't used since we need to check each indiviual output and only
    % add prompts to the pyout ones.
    
        {\par%
        \vspace{-1\smallerfontscale}%
        \noindent%
        \begin{minipage}{\cellleftmargin}%
    \hfill%
    {\smaller%
    \tt%
    \color{nbframe-out-prompt}%
    Out[33]:}%
    \hspace{\inputpadding}%
    \hspace{0em}%
    \hspace{3pt}%
    \end{minipage}%%
        }%
    %
    %
    \begin{addmargin}[\cellleftmargin]{0em}% left, right
    {\smaller%
    \vspace{-1\smallerfontscale}%
    
    
    
    \begin{verbatim}
0.6633650258161853
    \end{verbatim}

    
}%
    \end{addmargin}%
    % Add contents below.

{\par%
\vspace{-1\baselineskip}%
\needspace{4\baselineskip}}%
\begin{notebookcell}[34]%
\begin{addmargin}[\cellleftmargin]{0em}% left, right
{\smaller%
\par%
%
\vspace{-1\smallerfontscale}%
\begin{Verbatim}[commandchars=\\\{\}]
\PY{n}{random}\PY{o}{.}\PY{n}{normal}\PY{p}{(}\PY{o}{\PYZhy{}}\PY{l+m+mi}{1}\PY{p}{,}\PY{l+m+mf}{0.001}\PY{p}{)}
\end{Verbatim}
%
\par%
\vspace{-1\smallerfontscale}}%
\end{addmargin}
\end{notebookcell}

\par\vspace{1\smallerfontscale}%
    \needspace{4\baselineskip}%
    % Only render the prompt if the cell is pyout.  Note, the outputs prompt 
    % block isn't used since we need to check each indiviual output and only
    % add prompts to the pyout ones.
    
        {\par%
        \vspace{-1\smallerfontscale}%
        \noindent%
        \begin{minipage}{\cellleftmargin}%
    \hfill%
    {\smaller%
    \tt%
    \color{nbframe-out-prompt}%
    Out[34]:}%
    \hspace{\inputpadding}%
    \hspace{0em}%
    \hspace{3pt}%
    \end{minipage}%%
        }%
    %
    %
    \begin{addmargin}[\cellleftmargin]{0em}% left, right
    {\smaller%
    \vspace{-1\smallerfontscale}%
    
    
    
    \begin{verbatim}
-1.0001122377014229
    \end{verbatim}

    
}%
    \end{addmargin}%
    % Add contents below.

{\par%
\vspace{-1\baselineskip}%
\needspace{4\baselineskip}}%
\begin{notebookcell}[35]%
\begin{addmargin}[\cellleftmargin]{0em}% left, right
{\smaller%
\par%
%
\vspace{-1\smallerfontscale}%
\begin{Verbatim}[commandchars=\\\{\}]
\PY{n}{random}\PY{o}{.}\PY{n}{normal}\PY{p}{(}\PY{l+m+mi}{0}\PY{p}{,}\PY{l+m+mi}{1}\PY{p}{,}\PY{p}{[}\PY{l+m+mi}{2}\PY{p}{,}\PY{l+m+mi}{3}\PY{p}{]}\PY{p}{)}
\end{Verbatim}
%
\par%
\vspace{-1\smallerfontscale}}%
\end{addmargin}
\end{notebookcell}

\par\vspace{1\smallerfontscale}%
    \needspace{4\baselineskip}%
    % Only render the prompt if the cell is pyout.  Note, the outputs prompt 
    % block isn't used since we need to check each indiviual output and only
    % add prompts to the pyout ones.
    
        {\par%
        \vspace{-1\smallerfontscale}%
        \noindent%
        \begin{minipage}{\cellleftmargin}%
    \hfill%
    {\smaller%
    \tt%
    \color{nbframe-out-prompt}%
    Out[35]:}%
    \hspace{\inputpadding}%
    \hspace{0em}%
    \hspace{3pt}%
    \end{minipage}%%
        }%
    %
    %
    \begin{addmargin}[\cellleftmargin]{0em}% left, right
    {\smaller%
    \vspace{-1\smallerfontscale}%
    
    
    
    \begin{verbatim}
array([[-0.0965079 ,  0.61404973, -0.64773592],
       [ 0.71630596,  0.76143585, -0.46887797]])
    \end{verbatim}

    
}%
    \end{addmargin}%
    % Add contents below.

{\par%
\vspace{-1\baselineskip}%
\needspace{4\baselineskip}}%
\begin{notebookcell}[36]%
\begin{addmargin}[\cellleftmargin]{0em}% left, right
{\smaller%
\par%
%
\vspace{-1\smallerfontscale}%
\begin{Verbatim}[commandchars=\\\{\}]
\PY{n}{random}\PY{o}{.}\PY{n}{poisson}\PY{p}{(}\PY{p}{)}
\end{Verbatim}
%
\par%
\vspace{-1\smallerfontscale}}%
\end{addmargin}
\end{notebookcell}

\par\vspace{1\smallerfontscale}%
    \needspace{4\baselineskip}%
    % Only render the prompt if the cell is pyout.  Note, the outputs prompt 
    % block isn't used since we need to check each indiviual output and only
    % add prompts to the pyout ones.
    
        {\par%
        \vspace{-1\smallerfontscale}%
        \noindent%
        \begin{minipage}{\cellleftmargin}%
    \hfill%
    {\smaller%
    \tt%
    \color{nbframe-out-prompt}%
    Out[36]:}%
    \hspace{\inputpadding}%
    \hspace{0em}%
    \hspace{3pt}%
    \end{minipage}%%
        }%
    %
    %
    \begin{addmargin}[\cellleftmargin]{0em}% left, right
    {\smaller%
    \vspace{-1\smallerfontscale}%
    
    
    
    \begin{verbatim}
0
    \end{verbatim}

    
}%
    \end{addmargin}%
    % Add contents below.

{\par%
\vspace{-1\baselineskip}%
\needspace{4\baselineskip}}%
\begin{notebookcell}[37]%
\begin{addmargin}[\cellleftmargin]{0em}% left, right
{\smaller%
\par%
%
\vspace{-1\smallerfontscale}%
\begin{Verbatim}[commandchars=\\\{\}]
\PY{n}{random}\PY{o}{.}\PY{n}{rand}\PY{p}{(}\PY{l+m+mi}{5}\PY{p}{,}\PY{l+m+mi}{5}\PY{p}{)}
\end{Verbatim}
%
\par%
\vspace{-1\smallerfontscale}}%
\end{addmargin}
\end{notebookcell}

\par\vspace{1\smallerfontscale}%
    \needspace{4\baselineskip}%
    % Only render the prompt if the cell is pyout.  Note, the outputs prompt 
    % block isn't used since we need to check each indiviual output and only
    % add prompts to the pyout ones.
    
        {\par%
        \vspace{-1\smallerfontscale}%
        \noindent%
        \begin{minipage}{\cellleftmargin}%
    \hfill%
    {\smaller%
    \tt%
    \color{nbframe-out-prompt}%
    Out[37]:}%
    \hspace{\inputpadding}%
    \hspace{0em}%
    \hspace{3pt}%
    \end{minipage}%%
        }%
    %
    %
    \begin{addmargin}[\cellleftmargin]{0em}% left, right
    {\smaller%
    \vspace{-1\smallerfontscale}%
    
    
    
    \begin{verbatim}
array([[ 0.87390908,  0.24764657,  0.99397698,  0.38955504,  0.48710422],
       [ 0.73542102,  0.87108142,  0.27943569,  0.04596862,  0.93223845],
       [ 0.12292627,  0.03837154,  0.94424234,  0.54406483,  0.25399445],
       [ 0.04583676,  0.09379588,  0.91877364,  0.51265172,  0.36080634],
       [ 0.56637395,  0.39093857,  0.52816819,  0.65556558,  0.94067937]])
    \end{verbatim}

    
}%
    \end{addmargin}%

    \subparagraph{Indexing}


    Unlike Matlab, where parenthesis is used for indexing, in NumPy it is
done using square brackets

\textbf{Note:}Check with your colleagues if you both have the same array
\emph{a} in the next step. It is very important to have the same values
at this point

    % Add contents below.

{\par%
\vspace{-1\baselineskip}%
\needspace{4\baselineskip}}%
\begin{notebookcell}[46]%
\begin{addmargin}[\cellleftmargin]{0em}% left, right
{\smaller%
\par%
%
\vspace{-1\smallerfontscale}%
\begin{Verbatim}[commandchars=\\\{\}]
\PY{n}{a} \PY{o}{=} \PY{n}{random}\PY{o}{.}\PY{n}{rand}\PY{p}{(}\PY{l+m+mi}{5}\PY{p}{,}\PY{l+m+mi}{5}\PY{p}{)}
\PY{n}{a}
\end{Verbatim}
%
\par%
\vspace{-1\smallerfontscale}}%
\end{addmargin}
\end{notebookcell}

\par\vspace{1\smallerfontscale}%
    \needspace{4\baselineskip}%
    % Only render the prompt if the cell is pyout.  Note, the outputs prompt 
    % block isn't used since we need to check each indiviual output and only
    % add prompts to the pyout ones.
    
        {\par%
        \vspace{-1\smallerfontscale}%
        \noindent%
        \begin{minipage}{\cellleftmargin}%
    \hfill%
    {\smaller%
    \tt%
    \color{nbframe-out-prompt}%
    Out[46]:}%
    \hspace{\inputpadding}%
    \hspace{0em}%
    \hspace{3pt}%
    \end{minipage}%%
        }%
    %
    %
    \begin{addmargin}[\cellleftmargin]{0em}% left, right
    {\smaller%
    \vspace{-1\smallerfontscale}%
    
    
    
    \begin{verbatim}
array([[ 0.75916444,  0.94064948,  0.35674098,  0.76391983,  0.02329327],
       [ 0.26212745,  0.55539317,  0.71104502,  0.31895431,  0.56786133],
       [ 0.43040634,  0.2584529 ,  0.34181508,  0.24693744,  0.71610839],
       [ 0.78713362,  0.21469213,  0.00365139,  0.21454292,  0.94248957],
       [ 0.53490537,  0.61891915,  0.73147886,  0.72756939,  0.26000907]])
    \end{verbatim}

    
}%
    \end{addmargin}%
    % Add contents below.

{\par%
\vspace{-1\baselineskip}%
\needspace{4\baselineskip}}%
\begin{notebookcell}[47]%
\begin{addmargin}[\cellleftmargin]{0em}% left, right
{\smaller%
\par%
%
\vspace{-1\smallerfontscale}%
\begin{Verbatim}[commandchars=\\\{\}]
\PY{n}{a}\PY{p}{[}\PY{l+m+mi}{0}\PY{p}{,}\PY{l+m+mi}{0}\PY{p}{]}
\end{Verbatim}
%
\par%
\vspace{-1\smallerfontscale}}%
\end{addmargin}
\end{notebookcell}

\par\vspace{1\smallerfontscale}%
    \needspace{4\baselineskip}%
    % Only render the prompt if the cell is pyout.  Note, the outputs prompt 
    % block isn't used since we need to check each indiviual output and only
    % add prompts to the pyout ones.
    
        {\par%
        \vspace{-1\smallerfontscale}%
        \noindent%
        \begin{minipage}{\cellleftmargin}%
    \hfill%
    {\smaller%
    \tt%
    \color{nbframe-out-prompt}%
    Out[47]:}%
    \hspace{\inputpadding}%
    \hspace{0em}%
    \hspace{3pt}%
    \end{minipage}%%
        }%
    %
    %
    \begin{addmargin}[\cellleftmargin]{0em}% left, right
    {\smaller%
    \vspace{-1\smallerfontscale}%
    
    
    
    \begin{verbatim}
0.75916443575824311
    \end{verbatim}

    
}%
    \end{addmargin}%
    % Add contents below.

{\par%
\vspace{-1\baselineskip}%
\needspace{4\baselineskip}}%
\begin{notebookcell}[48]%
\begin{addmargin}[\cellleftmargin]{0em}% left, right
{\smaller%
\par%
%
\vspace{-1\smallerfontscale}%
\begin{Verbatim}[commandchars=\\\{\}]
\PY{n}{a}\PY{p}{[}\PY{l+m+mi}{0}\PY{p}{]}
\end{Verbatim}
%
\par%
\vspace{-1\smallerfontscale}}%
\end{addmargin}
\end{notebookcell}

\par\vspace{1\smallerfontscale}%
    \needspace{4\baselineskip}%
    % Only render the prompt if the cell is pyout.  Note, the outputs prompt 
    % block isn't used since we need to check each indiviual output and only
    % add prompts to the pyout ones.
    
        {\par%
        \vspace{-1\smallerfontscale}%
        \noindent%
        \begin{minipage}{\cellleftmargin}%
    \hfill%
    {\smaller%
    \tt%
    \color{nbframe-out-prompt}%
    Out[48]:}%
    \hspace{\inputpadding}%
    \hspace{0em}%
    \hspace{3pt}%
    \end{minipage}%%
        }%
    %
    %
    \begin{addmargin}[\cellleftmargin]{0em}% left, right
    {\smaller%
    \vspace{-1\smallerfontscale}%
    
    
    
    \begin{verbatim}
array([ 0.75916444,  0.94064948,  0.35674098,  0.76391983,  0.02329327])
    \end{verbatim}

    
}%
    \end{addmargin}%
    % Add contents below.

{\par%
\vspace{-1\baselineskip}%
\needspace{4\baselineskip}}%
\begin{notebookcell}[49]%
\begin{addmargin}[\cellleftmargin]{0em}% left, right
{\smaller%
\par%
%
\vspace{-1\smallerfontscale}%
\begin{Verbatim}[commandchars=\\\{\}]
\PY{n}{a}\PY{p}{[}\PY{p}{:}\PY{p}{,}\PY{l+m+mi}{0}\PY{p}{]}
\end{Verbatim}
%
\par%
\vspace{-1\smallerfontscale}}%
\end{addmargin}
\end{notebookcell}

\par\vspace{1\smallerfontscale}%
    \needspace{4\baselineskip}%
    % Only render the prompt if the cell is pyout.  Note, the outputs prompt 
    % block isn't used since we need to check each indiviual output and only
    % add prompts to the pyout ones.
    
        {\par%
        \vspace{-1\smallerfontscale}%
        \noindent%
        \begin{minipage}{\cellleftmargin}%
    \hfill%
    {\smaller%
    \tt%
    \color{nbframe-out-prompt}%
    Out[49]:}%
    \hspace{\inputpadding}%
    \hspace{0em}%
    \hspace{3pt}%
    \end{minipage}%%
        }%
    %
    %
    \begin{addmargin}[\cellleftmargin]{0em}% left, right
    {\smaller%
    \vspace{-1\smallerfontscale}%
    
    
    
    \begin{verbatim}
array([ 0.75916444,  0.26212745,  0.43040634,  0.78713362,  0.53490537])
    \end{verbatim}

    
}%
    \end{addmargin}%
    Very important: Slicing and indexing return views of the original array.
Can you see any difference?

    % Add contents below.

{\par%
\vspace{-1\baselineskip}%
\needspace{4\baselineskip}}%
\begin{notebookcell}[50]%
\begin{addmargin}[\cellleftmargin]{0em}% left, right
{\smaller%
\par%
%
\vspace{-1\smallerfontscale}%
\begin{Verbatim}[commandchars=\\\{\}]
\PY{n}{aview1} \PY{o}{=} \PY{n}{a}\PY{p}{[}\PY{p}{[}\PY{l+m+mi}{0}\PY{p}{]}\PY{p}{,}\PY{p}{:}\PY{p}{]}
\PY{n}{aview1}
\end{Verbatim}
%
\par%
\vspace{-1\smallerfontscale}}%
\end{addmargin}
\end{notebookcell}

\par\vspace{1\smallerfontscale}%
    \needspace{4\baselineskip}%
    % Only render the prompt if the cell is pyout.  Note, the outputs prompt 
    % block isn't used since we need to check each indiviual output and only
    % add prompts to the pyout ones.
    
        {\par%
        \vspace{-1\smallerfontscale}%
        \noindent%
        \begin{minipage}{\cellleftmargin}%
    \hfill%
    {\smaller%
    \tt%
    \color{nbframe-out-prompt}%
    Out[50]:}%
    \hspace{\inputpadding}%
    \hspace{0em}%
    \hspace{3pt}%
    \end{minipage}%%
        }%
    %
    %
    \begin{addmargin}[\cellleftmargin]{0em}% left, right
    {\smaller%
    \vspace{-1\smallerfontscale}%
    
    
    
    \begin{verbatim}
array([[ 0.75916444,  0.94064948,  0.35674098,  0.76391983,  0.02329327]])
    \end{verbatim}

    
}%
    \end{addmargin}%
    % Add contents below.

{\par%
\vspace{-1\baselineskip}%
\needspace{4\baselineskip}}%
\begin{notebookcell}[51]%
\begin{addmargin}[\cellleftmargin]{0em}% left, right
{\smaller%
\par%
%
\vspace{-1\smallerfontscale}%
\begin{Verbatim}[commandchars=\\\{\}]
\PY{n}{aview2} \PY{o}{=} \PY{n}{a}\PY{p}{[}\PY{l+m+mi}{0}\PY{p}{]}
\PY{n}{aview2}
\end{Verbatim}
%
\par%
\vspace{-1\smallerfontscale}}%
\end{addmargin}
\end{notebookcell}

\par\vspace{1\smallerfontscale}%
    \needspace{4\baselineskip}%
    % Only render the prompt if the cell is pyout.  Note, the outputs prompt 
    % block isn't used since we need to check each indiviual output and only
    % add prompts to the pyout ones.
    
        {\par%
        \vspace{-1\smallerfontscale}%
        \noindent%
        \begin{minipage}{\cellleftmargin}%
    \hfill%
    {\smaller%
    \tt%
    \color{nbframe-out-prompt}%
    Out[51]:}%
    \hspace{\inputpadding}%
    \hspace{0em}%
    \hspace{3pt}%
    \end{minipage}%%
        }%
    %
    %
    \begin{addmargin}[\cellleftmargin]{0em}% left, right
    {\smaller%
    \vspace{-1\smallerfontscale}%
    
    
    
    \begin{verbatim}
array([ 0.75916444,  0.94064948,  0.35674098,  0.76391983,  0.02329327])
    \end{verbatim}

    
}%
    \end{addmargin}%
    % Add contents below.

{\par%
\vspace{-1\baselineskip}%
\needspace{4\baselineskip}}%
\begin{notebookcell}[52]%
\begin{addmargin}[\cellleftmargin]{0em}% left, right
{\smaller%
\par%
%
\vspace{-1\smallerfontscale}%
\begin{Verbatim}[commandchars=\\\{\}]
\PY{n}{aview3} \PY{o}{=} \PY{n}{a}\PY{p}{[}\PY{p}{[}\PY{l+m+mi}{0}\PY{p}{]}\PY{p}{]}
\PY{n}{aview3}
\end{Verbatim}
%
\par%
\vspace{-1\smallerfontscale}}%
\end{addmargin}
\end{notebookcell}

\par\vspace{1\smallerfontscale}%
    \needspace{4\baselineskip}%
    % Only render the prompt if the cell is pyout.  Note, the outputs prompt 
    % block isn't used since we need to check each indiviual output and only
    % add prompts to the pyout ones.
    
        {\par%
        \vspace{-1\smallerfontscale}%
        \noindent%
        \begin{minipage}{\cellleftmargin}%
    \hfill%
    {\smaller%
    \tt%
    \color{nbframe-out-prompt}%
    Out[52]:}%
    \hspace{\inputpadding}%
    \hspace{0em}%
    \hspace{3pt}%
    \end{minipage}%%
        }%
    %
    %
    \begin{addmargin}[\cellleftmargin]{0em}% left, right
    {\smaller%
    \vspace{-1\smallerfontscale}%
    
    
    
    \begin{verbatim}
array([[ 0.75916444,  0.94064948,  0.35674098,  0.76391983,  0.02329327]])
    \end{verbatim}

    
}%
    \end{addmargin}%
    % Add contents below.

{\par%
\vspace{-1\baselineskip}%
\needspace{4\baselineskip}}%
\begin{notebookcell}[53]%
\begin{addmargin}[\cellleftmargin]{0em}% left, right
{\smaller%
\par%
%
\vspace{-1\smallerfontscale}%
\begin{Verbatim}[commandchars=\\\{\}]
\PY{k}{print} \PY{n}{aview1}\PY{o}{.}\PY{n}{shape}\PY{p}{,} \PY{n}{aview2}\PY{o}{.}\PY{n}{shape}\PY{p}{,} \PY{n}{aview3}\PY{o}{.}\PY{n}{shape}
\end{Verbatim}
%
\par%
\vspace{-1\smallerfontscale}}%
\end{addmargin}
\end{notebookcell}

\par\vspace{1\smallerfontscale}%
    \needspace{4\baselineskip}%
    % Only render the prompt if the cell is pyout.  Note, the outputs prompt 
    % block isn't used since we need to check each indiviual output and only
    % add prompts to the pyout ones.
    %
    %
    \begin{addmargin}[\cellleftmargin]{0em}% left, right
    {\smaller%
    \vspace{-1\smallerfontscale}%
    
    \begin{Verbatim}[commandchars=\\\{\}]
(1, 5) (5,) (1, 5)
    \end{Verbatim}
}%
    \end{addmargin}%

    \subparagraph{Slicing}


    % Add contents below.

{\par%
\vspace{-1\baselineskip}%
\needspace{4\baselineskip}}%
\begin{notebookcell}[56]%
\begin{addmargin}[\cellleftmargin]{0em}% left, right
{\smaller%
\par%
%
\vspace{-1\smallerfontscale}%
\begin{Verbatim}[commandchars=\\\{\}]
\PY{k+kn}{from} \PY{n+nn}{IPython.display} \PY{k+kn}{import} \PY{n}{YouTubeVideo}
\PY{n}{YouTubeVideo}\PY{p}{(}\PY{l+s}{\PYZsq{}}\PY{l+s}{q\PYZus{}2TBbfMLzs}\PY{l+s}{\PYZsq{}}\PY{p}{,}\PY{n}{start}\PY{o}{=}\PY{l+m+mi}{15}\PY{p}{)}
\end{Verbatim}
%
\par%
\vspace{-1\smallerfontscale}}%
\end{addmargin}
\end{notebookcell}

\par\vspace{1\smallerfontscale}%
    \needspace{4\baselineskip}%
    % Only render the prompt if the cell is pyout.  Note, the outputs prompt 
    % block isn't used since we need to check each indiviual output and only
    % add prompts to the pyout ones.
    
        {\par%
        \vspace{-1\smallerfontscale}%
        \noindent%
        \begin{minipage}{\cellleftmargin}%
    \hfill%
    {\smaller%
    \tt%
    \color{nbframe-out-prompt}%
    Out[56]:}%
    \hspace{\inputpadding}%
    \hspace{0em}%
    \hspace{3pt}%
    \end{minipage}%%
        }%
    %
    %
    \begin{addmargin}[\cellleftmargin]{0em}% left, right
    {\smaller%
    \vspace{-1\smallerfontscale}%
    
    
    
    \begin{verbatim}
<IPython.lib.display.YouTubeVideo at 0x7f21ecc46490>
    \end{verbatim}

    
}%
    \end{addmargin}%
    % Add contents below.

{\par%
\vspace{-1\baselineskip}%
\needspace{4\baselineskip}}%
\begin{notebookcell}[57]%
\begin{addmargin}[\cellleftmargin]{0em}% left, right
{\smaller%
\par%
%
\vspace{-1\smallerfontscale}%
\begin{Verbatim}[commandchars=\\\{\}]
\PY{n}{a}
\end{Verbatim}
%
\par%
\vspace{-1\smallerfontscale}}%
\end{addmargin}
\end{notebookcell}

\par\vspace{1\smallerfontscale}%
    \needspace{4\baselineskip}%
    % Only render the prompt if the cell is pyout.  Note, the outputs prompt 
    % block isn't used since we need to check each indiviual output and only
    % add prompts to the pyout ones.
    
        {\par%
        \vspace{-1\smallerfontscale}%
        \noindent%
        \begin{minipage}{\cellleftmargin}%
    \hfill%
    {\smaller%
    \tt%
    \color{nbframe-out-prompt}%
    Out[57]:}%
    \hspace{\inputpadding}%
    \hspace{0em}%
    \hspace{3pt}%
    \end{minipage}%%
        }%
    %
    %
    \begin{addmargin}[\cellleftmargin]{0em}% left, right
    {\smaller%
    \vspace{-1\smallerfontscale}%
    
    
    
    \begin{verbatim}
array([[ 0.75916444,  0.94064948,  0.35674098,  0.76391983,  0.02329327],
       [ 0.26212745,  0.55539317,  0.71104502,  0.31895431,  0.56786133],
       [ 0.43040634,  0.2584529 ,  0.34181508,  0.24693744,  0.71610839],
       [ 0.78713362,  0.21469213,  0.00365139,  0.21454292,  0.94248957],
       [ 0.53490537,  0.61891915,  0.73147886,  0.72756939,  0.26000907]])
    \end{verbatim}

    
}%
    \end{addmargin}%
    % Add contents below.

{\par%
\vspace{-1\baselineskip}%
\needspace{4\baselineskip}}%
\begin{notebookcell}[58]%
\begin{addmargin}[\cellleftmargin]{0em}% left, right
{\smaller%
\par%
%
\vspace{-1\smallerfontscale}%
\begin{Verbatim}[commandchars=\\\{\}]
\PY{n}{a}\PY{p}{[}\PY{p}{:}\PY{p}{,}\PY{p}{:}\PY{p}{]}
\end{Verbatim}
%
\par%
\vspace{-1\smallerfontscale}}%
\end{addmargin}
\end{notebookcell}

\par\vspace{1\smallerfontscale}%
    \needspace{4\baselineskip}%
    % Only render the prompt if the cell is pyout.  Note, the outputs prompt 
    % block isn't used since we need to check each indiviual output and only
    % add prompts to the pyout ones.
    
        {\par%
        \vspace{-1\smallerfontscale}%
        \noindent%
        \begin{minipage}{\cellleftmargin}%
    \hfill%
    {\smaller%
    \tt%
    \color{nbframe-out-prompt}%
    Out[58]:}%
    \hspace{\inputpadding}%
    \hspace{0em}%
    \hspace{3pt}%
    \end{minipage}%%
        }%
    %
    %
    \begin{addmargin}[\cellleftmargin]{0em}% left, right
    {\smaller%
    \vspace{-1\smallerfontscale}%
    
    
    
    \begin{verbatim}
array([[ 0.75916444,  0.94064948,  0.35674098,  0.76391983,  0.02329327],
       [ 0.26212745,  0.55539317,  0.71104502,  0.31895431,  0.56786133],
       [ 0.43040634,  0.2584529 ,  0.34181508,  0.24693744,  0.71610839],
       [ 0.78713362,  0.21469213,  0.00365139,  0.21454292,  0.94248957],
       [ 0.53490537,  0.61891915,  0.73147886,  0.72756939,  0.26000907]])
    \end{verbatim}

    
}%
    \end{addmargin}%
    % Add contents below.

{\par%
\vspace{-1\baselineskip}%
\needspace{4\baselineskip}}%
\begin{notebookcell}[59]%
\begin{addmargin}[\cellleftmargin]{0em}% left, right
{\smaller%
\par%
%
\vspace{-1\smallerfontscale}%
\begin{Verbatim}[commandchars=\\\{\}]
\PY{n}{a}\PY{p}{[}\PY{l+m+mi}{0}\PY{p}{:}\PY{l+m+mi}{2}\PY{p}{,}\PY{l+m+mi}{1}\PY{p}{:}\PY{l+m+mi}{3}\PY{p}{]}
\end{Verbatim}
%
\par%
\vspace{-1\smallerfontscale}}%
\end{addmargin}
\end{notebookcell}

\par\vspace{1\smallerfontscale}%
    \needspace{4\baselineskip}%
    % Only render the prompt if the cell is pyout.  Note, the outputs prompt 
    % block isn't used since we need to check each indiviual output and only
    % add prompts to the pyout ones.
    
        {\par%
        \vspace{-1\smallerfontscale}%
        \noindent%
        \begin{minipage}{\cellleftmargin}%
    \hfill%
    {\smaller%
    \tt%
    \color{nbframe-out-prompt}%
    Out[59]:}%
    \hspace{\inputpadding}%
    \hspace{0em}%
    \hspace{3pt}%
    \end{minipage}%%
        }%
    %
    %
    \begin{addmargin}[\cellleftmargin]{0em}% left, right
    {\smaller%
    \vspace{-1\smallerfontscale}%
    
    
    
    \begin{verbatim}
array([[ 0.94064948,  0.35674098],
       [ 0.55539317,  0.71104502]])
    \end{verbatim}

    
}%
    \end{addmargin}%
    % Add contents below.

{\par%
\vspace{-1\baselineskip}%
\needspace{4\baselineskip}}%
\begin{notebookcell}[60]%
\begin{addmargin}[\cellleftmargin]{0em}% left, right
{\smaller%
\par%
%
\vspace{-1\smallerfontscale}%
\begin{Verbatim}[commandchars=\\\{\}]
\PY{n}{a} \PY{o}{=} \PY{n}{arange}\PY{p}{(}\PY{l+m+mi}{10}\PY{p}{)}
\end{Verbatim}
%
\par%
\vspace{-1\smallerfontscale}}%
\end{addmargin}
\end{notebookcell}


    % Add contents below.

{\par%
\vspace{-1\baselineskip}%
\needspace{4\baselineskip}}%
\begin{notebookcell}[61]%
\begin{addmargin}[\cellleftmargin]{0em}% left, right
{\smaller%
\par%
%
\vspace{-1\smallerfontscale}%
\begin{Verbatim}[commandchars=\\\{\}]
\PY{n}{a}\PY{p}{[}\PY{l+m+mi}{1}\PY{p}{:}\PY{l+m+mi}{6}\PY{p}{:}\PY{l+m+mi}{3}\PY{p}{]}
\end{Verbatim}
%
\par%
\vspace{-1\smallerfontscale}}%
\end{addmargin}
\end{notebookcell}

\par\vspace{1\smallerfontscale}%
    \needspace{4\baselineskip}%
    % Only render the prompt if the cell is pyout.  Note, the outputs prompt 
    % block isn't used since we need to check each indiviual output and only
    % add prompts to the pyout ones.
    
        {\par%
        \vspace{-1\smallerfontscale}%
        \noindent%
        \begin{minipage}{\cellleftmargin}%
    \hfill%
    {\smaller%
    \tt%
    \color{nbframe-out-prompt}%
    Out[61]:}%
    \hspace{\inputpadding}%
    \hspace{0em}%
    \hspace{3pt}%
    \end{minipage}%%
        }%
    %
    %
    \begin{addmargin}[\cellleftmargin]{0em}% left, right
    {\smaller%
    \vspace{-1\smallerfontscale}%
    
    
    
    \begin{verbatim}
array([1, 4])
    \end{verbatim}

    
}%
    \end{addmargin}%
    % Add contents below.

{\par%
\vspace{-1\baselineskip}%
\needspace{4\baselineskip}}%
\begin{notebookcell}[62]%
\begin{addmargin}[\cellleftmargin]{0em}% left, right
{\smaller%
\par%
%
\vspace{-1\smallerfontscale}%
\begin{Verbatim}[commandchars=\\\{\}]
\PY{n}{a}\PY{p}{[}\PY{p}{:}\PY{p}{:}\PY{l+m+mi}{2}\PY{p}{]}
\end{Verbatim}
%
\par%
\vspace{-1\smallerfontscale}}%
\end{addmargin}
\end{notebookcell}

\par\vspace{1\smallerfontscale}%
    \needspace{4\baselineskip}%
    % Only render the prompt if the cell is pyout.  Note, the outputs prompt 
    % block isn't used since we need to check each indiviual output and only
    % add prompts to the pyout ones.
    
        {\par%
        \vspace{-1\smallerfontscale}%
        \noindent%
        \begin{minipage}{\cellleftmargin}%
    \hfill%
    {\smaller%
    \tt%
    \color{nbframe-out-prompt}%
    Out[62]:}%
    \hspace{\inputpadding}%
    \hspace{0em}%
    \hspace{3pt}%
    \end{minipage}%%
        }%
    %
    %
    \begin{addmargin}[\cellleftmargin]{0em}% left, right
    {\smaller%
    \vspace{-1\smallerfontscale}%
    
    
    
    \begin{verbatim}
array([0, 2, 4, 6, 8])
    \end{verbatim}

    
}%
    \end{addmargin}%
    % Add contents below.

{\par%
\vspace{-1\baselineskip}%
\needspace{4\baselineskip}}%
\begin{notebookcell}[63]%
\begin{addmargin}[\cellleftmargin]{0em}% left, right
{\smaller%
\par%
%
\vspace{-1\smallerfontscale}%
\begin{Verbatim}[commandchars=\\\{\}]
\PY{n}{a}\PY{p}{[}\PY{p}{:}\PY{p}{:}\PY{o}{\PYZhy{}}\PY{l+m+mi}{1}\PY{p}{]}
\end{Verbatim}
%
\par%
\vspace{-1\smallerfontscale}}%
\end{addmargin}
\end{notebookcell}

\par\vspace{1\smallerfontscale}%
    \needspace{4\baselineskip}%
    % Only render the prompt if the cell is pyout.  Note, the outputs prompt 
    % block isn't used since we need to check each indiviual output and only
    % add prompts to the pyout ones.
    
        {\par%
        \vspace{-1\smallerfontscale}%
        \noindent%
        \begin{minipage}{\cellleftmargin}%
    \hfill%
    {\smaller%
    \tt%
    \color{nbframe-out-prompt}%
    Out[63]:}%
    \hspace{\inputpadding}%
    \hspace{0em}%
    \hspace{3pt}%
    \end{minipage}%%
        }%
    %
    %
    \begin{addmargin}[\cellleftmargin]{0em}% left, right
    {\smaller%
    \vspace{-1\smallerfontscale}%
    
    
    
    \begin{verbatim}
array([9, 8, 7, 6, 5, 4, 3, 2, 1, 0])
    \end{verbatim}

    
}%
    \end{addmargin}%
    % Add contents below.

{\par%
\vspace{-1\baselineskip}%
\needspace{4\baselineskip}}%
\begin{notebookcell}[64]%
\begin{addmargin}[\cellleftmargin]{0em}% left, right
{\smaller%
\par%
%
\vspace{-1\smallerfontscale}%
\begin{Verbatim}[commandchars=\\\{\}]
\PY{o}{\PYZpc{}\PYZpc{}}\PY{k}{timeit}
\PY{n}{a}\PY{p}{[}\PY{p}{:}\PY{p}{:}\PY{o}{\PYZhy{}}\PY{l+m+mi}{1}\PY{p}{]}
\end{Verbatim}
%
\par%
\vspace{-1\smallerfontscale}}%
\end{addmargin}
\end{notebookcell}

\par\vspace{1\smallerfontscale}%
    \needspace{4\baselineskip}%
    % Only render the prompt if the cell is pyout.  Note, the outputs prompt 
    % block isn't used since we need to check each indiviual output and only
    % add prompts to the pyout ones.
    %
    %
    \begin{addmargin}[\cellleftmargin]{0em}% left, right
    {\smaller%
    \vspace{-1\smallerfontscale}%
    
    \begin{Verbatim}[commandchars=\\\{\}]
1000000 loops, best of 3: 222 ns per loop
    \end{Verbatim}
}%
    \end{addmargin}%
    % Add contents below.

{\par%
\vspace{-1\baselineskip}%
\needspace{4\baselineskip}}%
\begin{notebookcell}[65]%
\begin{addmargin}[\cellleftmargin]{0em}% left, right
{\smaller%
\par%
%
\vspace{-1\smallerfontscale}%
\begin{Verbatim}[commandchars=\\\{\}]
\PY{o}{\PYZpc{}\PYZpc{}}\PY{k}{timeit}
\PY{n+nb}{reversed}\PY{p}{(}\PY{n}{a}\PY{p}{)}
\end{Verbatim}
%
\par%
\vspace{-1\smallerfontscale}}%
\end{addmargin}
\end{notebookcell}

\par\vspace{1\smallerfontscale}%
    \needspace{4\baselineskip}%
    % Only render the prompt if the cell is pyout.  Note, the outputs prompt 
    % block isn't used since we need to check each indiviual output and only
    % add prompts to the pyout ones.
    %
    %
    \begin{addmargin}[\cellleftmargin]{0em}% left, right
    {\smaller%
    \vspace{-1\smallerfontscale}%
    
    \begin{Verbatim}[commandchars=\\\{\}]
1000000 loops, best of 3: 182 ns per loop
    \end{Verbatim}
}%
    \end{addmargin}%
    % Add contents below.

{\par%
\vspace{-1\baselineskip}%
\needspace{4\baselineskip}}%
\begin{notebookcell}[66]%
\begin{addmargin}[\cellleftmargin]{0em}% left, right
{\smaller%
\par%
%
\vspace{-1\smallerfontscale}%
\begin{Verbatim}[commandchars=\\\{\}]
\PY{k}{print} \PY{n}{a}\PY{p}{[}\PY{l+m+mi}{5}\PY{p}{:}\PY{p}{]}\PY{p}{,} \PY{n}{a}\PY{p}{[}\PY{p}{:}\PY{l+m+mi}{4}\PY{p}{]}\PY{p}{,} \PY{n}{a}\PY{p}{[}\PY{o}{\PYZhy{}}\PY{l+m+mi}{2}\PY{p}{:}\PY{p}{]}\PY{p}{,} \PY{n}{a}\PY{p}{[}\PY{p}{:}\PY{o}{\PYZhy{}}\PY{l+m+mi}{4}\PY{p}{]}
\end{Verbatim}
%
\par%
\vspace{-1\smallerfontscale}}%
\end{addmargin}
\end{notebookcell}

\par\vspace{1\smallerfontscale}%
    \needspace{4\baselineskip}%
    % Only render the prompt if the cell is pyout.  Note, the outputs prompt 
    % block isn't used since we need to check each indiviual output and only
    % add prompts to the pyout ones.
    %
    %
    \begin{addmargin}[\cellleftmargin]{0em}% left, right
    {\smaller%
    \vspace{-1\smallerfontscale}%
    
    \begin{Verbatim}[commandchars=\\\{\}]
[5 6 7 8 9] [0 1 2 3] [8 9] [0 1 2 3 4 5]
    \end{Verbatim}
}%
    \end{addmargin}%
    % Add contents below.

{\par%
\vspace{-1\baselineskip}%
\needspace{4\baselineskip}}%
\begin{notebookcell}[67]%
\begin{addmargin}[\cellleftmargin]{0em}% left, right
{\smaller%
\par%
%
\vspace{-1\smallerfontscale}%
\begin{Verbatim}[commandchars=\\\{\}]
\PY{n}{a} \PY{o}{=} \PY{n}{random}\PY{o}{.}\PY{n}{rand}\PY{p}{(}\PY{l+m+mi}{5}\PY{p}{,}\PY{l+m+mi}{5}\PY{p}{)}
\PY{n}{a}
\end{Verbatim}
%
\par%
\vspace{-1\smallerfontscale}}%
\end{addmargin}
\end{notebookcell}

\par\vspace{1\smallerfontscale}%
    \needspace{4\baselineskip}%
    % Only render the prompt if the cell is pyout.  Note, the outputs prompt 
    % block isn't used since we need to check each indiviual output and only
    % add prompts to the pyout ones.
    
        {\par%
        \vspace{-1\smallerfontscale}%
        \noindent%
        \begin{minipage}{\cellleftmargin}%
    \hfill%
    {\smaller%
    \tt%
    \color{nbframe-out-prompt}%
    Out[67]:}%
    \hspace{\inputpadding}%
    \hspace{0em}%
    \hspace{3pt}%
    \end{minipage}%%
        }%
    %
    %
    \begin{addmargin}[\cellleftmargin]{0em}% left, right
    {\smaller%
    \vspace{-1\smallerfontscale}%
    
    
    
    \begin{verbatim}
array([[ 0.63611189,  0.45387041,  0.84324789,  0.41564385,  0.8736876 ],
       [ 0.15186863,  0.75865389,  0.11447856,  0.03593258,  0.35814001],
       [ 0.76592388,  0.7361502 ,  0.36358767,  0.25983878,  0.94569104],
       [ 0.43403234,  0.23080921,  0.64767593,  0.50096602,  0.27094573],
       [ 0.79041938,  0.41486588,  0.4061916 ,  0.33896987,  0.68529799]])
    \end{verbatim}

    
}%
    \end{addmargin}%
    % Add contents below.

{\par%
\vspace{-1\baselineskip}%
\needspace{4\baselineskip}}%
\begin{notebookcell}[68]%
\begin{addmargin}[\cellleftmargin]{0em}% left, right
{\smaller%
\par%
%
\vspace{-1\smallerfontscale}%
\begin{Verbatim}[commandchars=\\\{\}]
\PY{n}{a}\PY{p}{[}\PY{l+m+mi}{0}\PY{p}{]}
\end{Verbatim}
%
\par%
\vspace{-1\smallerfontscale}}%
\end{addmargin}
\end{notebookcell}

\par\vspace{1\smallerfontscale}%
    \needspace{4\baselineskip}%
    % Only render the prompt if the cell is pyout.  Note, the outputs prompt 
    % block isn't used since we need to check each indiviual output and only
    % add prompts to the pyout ones.
    
        {\par%
        \vspace{-1\smallerfontscale}%
        \noindent%
        \begin{minipage}{\cellleftmargin}%
    \hfill%
    {\smaller%
    \tt%
    \color{nbframe-out-prompt}%
    Out[68]:}%
    \hspace{\inputpadding}%
    \hspace{0em}%
    \hspace{3pt}%
    \end{minipage}%%
        }%
    %
    %
    \begin{addmargin}[\cellleftmargin]{0em}% left, right
    {\smaller%
    \vspace{-1\smallerfontscale}%
    
    
    
    \begin{verbatim}
array([ 0.63611189,  0.45387041,  0.84324789,  0.41564385,  0.8736876 ])
    \end{verbatim}

    
}%
    \end{addmargin}%
    % Add contents below.

{\par%
\vspace{-1\baselineskip}%
\needspace{4\baselineskip}}%
\begin{notebookcell}[69]%
\begin{addmargin}[\cellleftmargin]{0em}% left, right
{\smaller%
\par%
%
\vspace{-1\smallerfontscale}%
\begin{Verbatim}[commandchars=\\\{\}]
\PY{n}{rows} \PY{o}{=} \PY{p}{[}\PY{l+m+mi}{1}\PY{p}{,}\PY{l+m+mi}{3}\PY{p}{]}
\PY{n}{cols} \PY{o}{=} \PY{p}{[}\PY{l+m+mi}{0}\PY{p}{,}\PY{l+m+mi}{2}\PY{p}{]}
\PY{n}{a}\PY{p}{[}\PY{n}{rows}\PY{p}{]}\PY{p}{,} \PY{n}{a}\PY{p}{[}\PY{p}{:}\PY{p}{,}\PY{n}{cols}\PY{p}{]}\PY{p}{,} \PY{n}{a}\PY{p}{[}\PY{n}{rows}\PY{p}{,}\PY{n}{cols}\PY{p}{]}
\end{Verbatim}
%
\par%
\vspace{-1\smallerfontscale}}%
\end{addmargin}
\end{notebookcell}

\par\vspace{1\smallerfontscale}%
    \needspace{4\baselineskip}%
    % Only render the prompt if the cell is pyout.  Note, the outputs prompt 
    % block isn't used since we need to check each indiviual output and only
    % add prompts to the pyout ones.
    
        {\par%
        \vspace{-1\smallerfontscale}%
        \noindent%
        \begin{minipage}{\cellleftmargin}%
    \hfill%
    {\smaller%
    \tt%
    \color{nbframe-out-prompt}%
    Out[69]:}%
    \hspace{\inputpadding}%
    \hspace{0em}%
    \hspace{3pt}%
    \end{minipage}%%
        }%
    %
    %
    \begin{addmargin}[\cellleftmargin]{0em}% left, right
    {\smaller%
    \vspace{-1\smallerfontscale}%
    
    
    
    \begin{verbatim}
(array([[ 0.15186863,  0.75865389,  0.11447856,  0.03593258,  0.35814001],
        [ 0.43403234,  0.23080921,  0.64767593,  0.50096602,  0.27094573]]),
 array([[ 0.63611189,  0.84324789],
        [ 0.15186863,  0.11447856],
        [ 0.76592388,  0.36358767],
        [ 0.43403234,  0.64767593],
        [ 0.79041938,  0.4061916 ]]),
 array([ 0.15186863,  0.64767593]))
    \end{verbatim}

    
}%
    \end{addmargin}%
    Vorsicht! Round vs np.round

    % Add contents below.

{\par%
\vspace{-1\baselineskip}%
\needspace{4\baselineskip}}%
\begin{notebookcell}[70]%
\begin{addmargin}[\cellleftmargin]{0em}% left, right
{\smaller%
\par%
%
\vspace{-1\smallerfontscale}%
\begin{Verbatim}[commandchars=\\\{\}]
\PY{n}{mask} \PY{o}{=} \PY{n}{np}\PY{o}{.}\PY{n}{around}\PY{p}{(}\PY{n}{a}\PY{p}{)}
\PY{n}{mask} \PY{o}{=} \PY{n}{mask}\PY{o}{.}\PY{n}{astype}\PY{p}{(}\PY{l+s}{\PYZsq{}}\PY{l+s}{bool}\PY{l+s}{\PYZsq{}}\PY{p}{)}
\end{Verbatim}
%
\par%
\vspace{-1\smallerfontscale}}%
\end{addmargin}
\end{notebookcell}


    % Add contents below.

{\par%
\vspace{-1\baselineskip}%
\needspace{4\baselineskip}}%
\begin{notebookcell}[71]%
\begin{addmargin}[\cellleftmargin]{0em}% left, right
{\smaller%
\par%
%
\vspace{-1\smallerfontscale}%
\begin{Verbatim}[commandchars=\\\{\}]
\PY{n}{a}\PY{p}{[}\PY{n}{mask}\PY{p}{]}
\end{Verbatim}
%
\par%
\vspace{-1\smallerfontscale}}%
\end{addmargin}
\end{notebookcell}

\par\vspace{1\smallerfontscale}%
    \needspace{4\baselineskip}%
    % Only render the prompt if the cell is pyout.  Note, the outputs prompt 
    % block isn't used since we need to check each indiviual output and only
    % add prompts to the pyout ones.
    
        {\par%
        \vspace{-1\smallerfontscale}%
        \noindent%
        \begin{minipage}{\cellleftmargin}%
    \hfill%
    {\smaller%
    \tt%
    \color{nbframe-out-prompt}%
    Out[71]:}%
    \hspace{\inputpadding}%
    \hspace{0em}%
    \hspace{3pt}%
    \end{minipage}%%
        }%
    %
    %
    \begin{addmargin}[\cellleftmargin]{0em}% left, right
    {\smaller%
    \vspace{-1\smallerfontscale}%
    
    
    
    \begin{verbatim}
array([ 0.63611189,  0.84324789,  0.8736876 ,  0.75865389,  0.76592388,
        0.7361502 ,  0.94569104,  0.64767593,  0.50096602,  0.79041938,
        0.68529799])
    \end{verbatim}

    
}%
    \end{addmargin}%
    % Add contents below.

{\par%
\vspace{-1\baselineskip}%
\needspace{4\baselineskip}}%
\begin{notebookcell}[72]%
\begin{addmargin}[\cellleftmargin]{0em}% left, right
{\smaller%
\par%
%
\vspace{-1\smallerfontscale}%
\begin{Verbatim}[commandchars=\\\{\}]
\PY{n}{mask} \PY{o}{=} \PY{n}{a} \PY{o}{\PYZgt{}} \PY{l+m+mf}{1.0}
\PY{n}{a}\PY{p}{[}\PY{n}{mask}\PY{p}{]}
\end{Verbatim}
%
\par%
\vspace{-1\smallerfontscale}}%
\end{addmargin}
\end{notebookcell}

\par\vspace{1\smallerfontscale}%
    \needspace{4\baselineskip}%
    % Only render the prompt if the cell is pyout.  Note, the outputs prompt 
    % block isn't used since we need to check each indiviual output and only
    % add prompts to the pyout ones.
    
        {\par%
        \vspace{-1\smallerfontscale}%
        \noindent%
        \begin{minipage}{\cellleftmargin}%
    \hfill%
    {\smaller%
    \tt%
    \color{nbframe-out-prompt}%
    Out[72]:}%
    \hspace{\inputpadding}%
    \hspace{0em}%
    \hspace{3pt}%
    \end{minipage}%%
        }%
    %
    %
    \begin{addmargin}[\cellleftmargin]{0em}% left, right
    {\smaller%
    \vspace{-1\smallerfontscale}%
    
    
    
    \begin{verbatim}
array([], dtype=float64)
    \end{verbatim}

    
}%
    \end{addmargin}%
    % Add contents below.

{\par%
\vspace{-1\baselineskip}%
\needspace{4\baselineskip}}%
\begin{notebookcell}[73]%
\begin{addmargin}[\cellleftmargin]{0em}% left, right
{\smaller%
\par%
%
\vspace{-1\smallerfontscale}%
\begin{Verbatim}[commandchars=\\\{\}]
\PY{n}{mask} \PY{o}{=} \PY{n}{a} \PY{o}{\PYZlt{}} \PY{l+m+mf}{1.0}
\PY{n}{a}\PY{p}{[}\PY{n}{mask}\PY{p}{]}
\end{Verbatim}
%
\par%
\vspace{-1\smallerfontscale}}%
\end{addmargin}
\end{notebookcell}

\par\vspace{1\smallerfontscale}%
    \needspace{4\baselineskip}%
    % Only render the prompt if the cell is pyout.  Note, the outputs prompt 
    % block isn't used since we need to check each indiviual output and only
    % add prompts to the pyout ones.
    
        {\par%
        \vspace{-1\smallerfontscale}%
        \noindent%
        \begin{minipage}{\cellleftmargin}%
    \hfill%
    {\smaller%
    \tt%
    \color{nbframe-out-prompt}%
    Out[73]:}%
    \hspace{\inputpadding}%
    \hspace{0em}%
    \hspace{3pt}%
    \end{minipage}%%
        }%
    %
    %
    \begin{addmargin}[\cellleftmargin]{0em}% left, right
    {\smaller%
    \vspace{-1\smallerfontscale}%
    
    
    
    \begin{verbatim}
array([ 0.63611189,  0.45387041,  0.84324789,  0.41564385,  0.8736876 ,
        0.15186863,  0.75865389,  0.11447856,  0.03593258,  0.35814001,
        0.76592388,  0.7361502 ,  0.36358767,  0.25983878,  0.94569104,
        0.43403234,  0.23080921,  0.64767593,  0.50096602,  0.27094573,
        0.79041938,  0.41486588,  0.4061916 ,  0.33896987,  0.68529799])
    \end{verbatim}

    
}%
    \end{addmargin}%
    % Add contents below.

{\par%
\vspace{-1\baselineskip}%
\needspace{4\baselineskip}}%
\begin{notebookcell}[74]%
\begin{addmargin}[\cellleftmargin]{0em}% left, right
{\smaller%
\par%
%
\vspace{-1\smallerfontscale}%
\begin{Verbatim}[commandchars=\\\{\}]
\PY{n}{mask} \PY{o}{=} \PY{n}{a} \PY{o}{\PYZgt{}} \PY{l+m+mf}{0.7}
\PY{n}{a}\PY{p}{[}\PY{n}{mask}\PY{p}{]}
\end{Verbatim}
%
\par%
\vspace{-1\smallerfontscale}}%
\end{addmargin}
\end{notebookcell}

\par\vspace{1\smallerfontscale}%
    \needspace{4\baselineskip}%
    % Only render the prompt if the cell is pyout.  Note, the outputs prompt 
    % block isn't used since we need to check each indiviual output and only
    % add prompts to the pyout ones.
    
        {\par%
        \vspace{-1\smallerfontscale}%
        \noindent%
        \begin{minipage}{\cellleftmargin}%
    \hfill%
    {\smaller%
    \tt%
    \color{nbframe-out-prompt}%
    Out[74]:}%
    \hspace{\inputpadding}%
    \hspace{0em}%
    \hspace{3pt}%
    \end{minipage}%%
        }%
    %
    %
    \begin{addmargin}[\cellleftmargin]{0em}% left, right
    {\smaller%
    \vspace{-1\smallerfontscale}%
    
    
    
    \begin{verbatim}
array([ 0.84324789,  0.8736876 ,  0.75865389,  0.76592388,  0.7361502 ,
        0.94569104,  0.79041938])
    \end{verbatim}

    
}%
    \end{addmargin}%
    % Add contents below.

{\par%
\vspace{-1\baselineskip}%
\needspace{4\baselineskip}}%
\begin{notebookcell}[75]%
\begin{addmargin}[\cellleftmargin]{0em}% left, right
{\smaller%
\par%
%
\vspace{-1\smallerfontscale}%
\begin{Verbatim}[commandchars=\\\{\}]
\PY{n}{where}\PY{p}{(}\PY{n}{mask}\PY{p}{)}
\end{Verbatim}
%
\par%
\vspace{-1\smallerfontscale}}%
\end{addmargin}
\end{notebookcell}

\par\vspace{1\smallerfontscale}%
    \needspace{4\baselineskip}%
    % Only render the prompt if the cell is pyout.  Note, the outputs prompt 
    % block isn't used since we need to check each indiviual output and only
    % add prompts to the pyout ones.
    
        {\par%
        \vspace{-1\smallerfontscale}%
        \noindent%
        \begin{minipage}{\cellleftmargin}%
    \hfill%
    {\smaller%
    \tt%
    \color{nbframe-out-prompt}%
    Out[75]:}%
    \hspace{\inputpadding}%
    \hspace{0em}%
    \hspace{3pt}%
    \end{minipage}%%
        }%
    %
    %
    \begin{addmargin}[\cellleftmargin]{0em}% left, right
    {\smaller%
    \vspace{-1\smallerfontscale}%
    
    
    
    \begin{verbatim}
(array([0, 0, 1, 2, 2, 2, 4]), array([2, 4, 1, 0, 1, 4, 0]))
    \end{verbatim}

    
}%
    \end{addmargin}%

    \subparagraph{Linear Algebra}


    % Add contents below.

{\par%
\vspace{-1\baselineskip}%
\needspace{4\baselineskip}}%
\begin{notebookcell}[76]%
\begin{addmargin}[\cellleftmargin]{0em}% left, right
{\smaller%
\par%
%
\vspace{-1\smallerfontscale}%
\begin{Verbatim}[commandchars=\\\{\}]
\PY{k+kn}{from} \PY{n+nn}{numpy.linalg} \PY{k+kn}{import} \PY{o}{*}
\end{Verbatim}
%
\par%
\vspace{-1\smallerfontscale}}%
\end{addmargin}
\end{notebookcell}


    % Add contents below.

{\par%
\vspace{-1\baselineskip}%
\needspace{4\baselineskip}}%
\begin{notebookcell}[77]%
\begin{addmargin}[\cellleftmargin]{0em}% left, right
{\smaller%
\par%
%
\vspace{-1\smallerfontscale}%
\begin{Verbatim}[commandchars=\\\{\}]
\PY{n}{x} \PY{o}{=} \PY{n}{arange}\PY{p}{(}\PY{l+m+mi}{10}\PY{p}{)}
\PY{n}{b} \PY{o}{=} \PY{n}{random}\PY{o}{.}\PY{n}{rand}\PY{p}{(}\PY{l+m+mi}{10}\PY{p}{,}\PY{l+m+mi}{1}\PY{p}{)}
\PY{n}{A} \PY{o}{=} \PY{n}{random}\PY{o}{.}\PY{n}{randn}\PY{p}{(}\PY{l+m+mi}{10}\PY{p}{,}\PY{l+m+mi}{10}\PY{p}{)}
\PY{k}{print} \PY{n}{x}
\end{Verbatim}
%
\par%
\vspace{-1\smallerfontscale}}%
\end{addmargin}
\end{notebookcell}

\par\vspace{1\smallerfontscale}%
    \needspace{4\baselineskip}%
    % Only render the prompt if the cell is pyout.  Note, the outputs prompt 
    % block isn't used since we need to check each indiviual output and only
    % add prompts to the pyout ones.
    %
    %
    \begin{addmargin}[\cellleftmargin]{0em}% left, right
    {\smaller%
    \vspace{-1\smallerfontscale}%
    
    \begin{Verbatim}[commandchars=\\\{\}]
[0 1 2 3 4 5 6 7 8 9]
    \end{Verbatim}
}%
    \end{addmargin}%
    % Add contents below.

{\par%
\vspace{-1\baselineskip}%
\needspace{4\baselineskip}}%
\begin{notebookcell}[78]%
\begin{addmargin}[\cellleftmargin]{0em}% left, right
{\smaller%
\par%
%
\vspace{-1\smallerfontscale}%
\begin{Verbatim}[commandchars=\\\{\}]
\PY{n}{x}\PY{o}{*}\PY{l+m+mi}{2}
\end{Verbatim}
%
\par%
\vspace{-1\smallerfontscale}}%
\end{addmargin}
\end{notebookcell}

\par\vspace{1\smallerfontscale}%
    \needspace{4\baselineskip}%
    % Only render the prompt if the cell is pyout.  Note, the outputs prompt 
    % block isn't used since we need to check each indiviual output and only
    % add prompts to the pyout ones.
    
        {\par%
        \vspace{-1\smallerfontscale}%
        \noindent%
        \begin{minipage}{\cellleftmargin}%
    \hfill%
    {\smaller%
    \tt%
    \color{nbframe-out-prompt}%
    Out[78]:}%
    \hspace{\inputpadding}%
    \hspace{0em}%
    \hspace{3pt}%
    \end{minipage}%%
        }%
    %
    %
    \begin{addmargin}[\cellleftmargin]{0em}% left, right
    {\smaller%
    \vspace{-1\smallerfontscale}%
    
    
    
    \begin{verbatim}
array([ 0,  2,  4,  6,  8, 10, 12, 14, 16, 18])
    \end{verbatim}

    
}%
    \end{addmargin}%
    % Add contents below.

{\par%
\vspace{-1\baselineskip}%
\needspace{4\baselineskip}}%
\begin{notebookcell}[79]%
\begin{addmargin}[\cellleftmargin]{0em}% left, right
{\smaller%
\par%
%
\vspace{-1\smallerfontscale}%
\begin{Verbatim}[commandchars=\\\{\}]
\PY{n}{x}\PY{o}{+}\PY{l+m+mi}{5}
\end{Verbatim}
%
\par%
\vspace{-1\smallerfontscale}}%
\end{addmargin}
\end{notebookcell}

\par\vspace{1\smallerfontscale}%
    \needspace{4\baselineskip}%
    % Only render the prompt if the cell is pyout.  Note, the outputs prompt 
    % block isn't used since we need to check each indiviual output and only
    % add prompts to the pyout ones.
    
        {\par%
        \vspace{-1\smallerfontscale}%
        \noindent%
        \begin{minipage}{\cellleftmargin}%
    \hfill%
    {\smaller%
    \tt%
    \color{nbframe-out-prompt}%
    Out[79]:}%
    \hspace{\inputpadding}%
    \hspace{0em}%
    \hspace{3pt}%
    \end{minipage}%%
        }%
    %
    %
    \begin{addmargin}[\cellleftmargin]{0em}% left, right
    {\smaller%
    \vspace{-1\smallerfontscale}%
    
    
    
    \begin{verbatim}
array([ 5,  6,  7,  8,  9, 10, 11, 12, 13, 14])
    \end{verbatim}

    
}%
    \end{addmargin}%
    % Add contents below.

{\par%
\vspace{-1\baselineskip}%
\needspace{4\baselineskip}}%
\begin{notebookcell}[80]%
\begin{addmargin}[\cellleftmargin]{0em}% left, right
{\smaller%
\par%
%
\vspace{-1\smallerfontscale}%
\begin{Verbatim}[commandchars=\\\{\}]
\PY{n}{x}\PY{o}{*}\PY{o}{*}\PY{l+m+mi}{2}
\end{Verbatim}
%
\par%
\vspace{-1\smallerfontscale}}%
\end{addmargin}
\end{notebookcell}

\par\vspace{1\smallerfontscale}%
    \needspace{4\baselineskip}%
    % Only render the prompt if the cell is pyout.  Note, the outputs prompt 
    % block isn't used since we need to check each indiviual output and only
    % add prompts to the pyout ones.
    
        {\par%
        \vspace{-1\smallerfontscale}%
        \noindent%
        \begin{minipage}{\cellleftmargin}%
    \hfill%
    {\smaller%
    \tt%
    \color{nbframe-out-prompt}%
    Out[80]:}%
    \hspace{\inputpadding}%
    \hspace{0em}%
    \hspace{3pt}%
    \end{minipage}%%
        }%
    %
    %
    \begin{addmargin}[\cellleftmargin]{0em}% left, right
    {\smaller%
    \vspace{-1\smallerfontscale}%
    
    
    
    \begin{verbatim}
array([ 0,  1,  4,  9, 16, 25, 36, 49, 64, 81])
    \end{verbatim}

    
}%
    \end{addmargin}%
    % Add contents below.

{\par%
\vspace{-1\baselineskip}%
\needspace{4\baselineskip}}%
\begin{notebookcell}[81]%
\begin{addmargin}[\cellleftmargin]{0em}% left, right
{\smaller%
\par%
%
\vspace{-1\smallerfontscale}%
\begin{Verbatim}[commandchars=\\\{\}]
\PY{n}{x}\PY{o}{*}\PY{n}{x}
\end{Verbatim}
%
\par%
\vspace{-1\smallerfontscale}}%
\end{addmargin}
\end{notebookcell}

\par\vspace{1\smallerfontscale}%
    \needspace{4\baselineskip}%
    % Only render the prompt if the cell is pyout.  Note, the outputs prompt 
    % block isn't used since we need to check each indiviual output and only
    % add prompts to the pyout ones.
    
        {\par%
        \vspace{-1\smallerfontscale}%
        \noindent%
        \begin{minipage}{\cellleftmargin}%
    \hfill%
    {\smaller%
    \tt%
    \color{nbframe-out-prompt}%
    Out[81]:}%
    \hspace{\inputpadding}%
    \hspace{0em}%
    \hspace{3pt}%
    \end{minipage}%%
        }%
    %
    %
    \begin{addmargin}[\cellleftmargin]{0em}% left, right
    {\smaller%
    \vspace{-1\smallerfontscale}%
    
    
    
    \begin{verbatim}
array([ 0,  1,  4,  9, 16, 25, 36, 49, 64, 81])
    \end{verbatim}

    
}%
    \end{addmargin}%
    And one of the best features: broadcasting

    % Add contents below.

{\par%
\vspace{-1\baselineskip}%
\needspace{4\baselineskip}}%
\begin{notebookcell}[82]%
\begin{addmargin}[\cellleftmargin]{0em}% left, right
{\smaller%
\par%
%
\vspace{-1\smallerfontscale}%
\begin{Verbatim}[commandchars=\\\{\}]
\PY{k+kn}{from} \PY{n+nn}{IPython.display} \PY{k+kn}{import} \PY{n}{Image}
\PY{n}{Image}\PY{p}{(}\PY{l+s}{\PYZsq{}}\PY{l+s}{https://scipy\PYZhy{}lectures.github.io/\PYZus{}images/numpy\PYZus{}broadcasting.png}\PY{l+s}{\PYZsq{}}\PY{p}{,}\PY{n}{width}\PY{o}{=}\PY{l+m+mi}{800}\PY{p}{)}
\end{Verbatim}
%
\par%
\vspace{-1\smallerfontscale}}%
\end{addmargin}
\end{notebookcell}

\par\vspace{1\smallerfontscale}%
    \needspace{4\baselineskip}%
    % Only render the prompt if the cell is pyout.  Note, the outputs prompt 
    % block isn't used since we need to check each indiviual output and only
    % add prompts to the pyout ones.
    
        {\par%
        \vspace{-1\smallerfontscale}%
        \noindent%
        \begin{minipage}{\cellleftmargin}%
    \hfill%
    {\smaller%
    \tt%
    \color{nbframe-out-prompt}%
    Out[82]:}%
    \hspace{\inputpadding}%
    \hspace{0em}%
    \hspace{3pt}%
    \end{minipage}%%
        }%
    %
    %
    \begin{addmargin}[\cellleftmargin]{0em}% left, right
    {\smaller%
    \vspace{-1\smallerfontscale}%
    
    
    \begin{center}
    \adjustimage{max size={0.9\linewidth}{0.9\paperheight}}{III-Numpy_files/III-Numpy_89_0.png}
    \end{center}
    { \hspace*{\fill} \\}
    
}%
    \end{addmargin}%
    % Add contents below.

{\par%
\vspace{-1\baselineskip}%
\needspace{4\baselineskip}}%
\begin{notebookcell}[83]%
\begin{addmargin}[\cellleftmargin]{0em}% left, right
{\smaller%
\par%
%
\vspace{-1\smallerfontscale}%
\begin{Verbatim}[commandchars=\\\{\}]
\PY{n}{x}\PY{o}{.}\PY{n}{shape}\PY{p}{,} \PY{n}{b}\PY{o}{.}\PY{n}{shape}
\end{Verbatim}
%
\par%
\vspace{-1\smallerfontscale}}%
\end{addmargin}
\end{notebookcell}

\par\vspace{1\smallerfontscale}%
    \needspace{4\baselineskip}%
    % Only render the prompt if the cell is pyout.  Note, the outputs prompt 
    % block isn't used since we need to check each indiviual output and only
    % add prompts to the pyout ones.
    
        {\par%
        \vspace{-1\smallerfontscale}%
        \noindent%
        \begin{minipage}{\cellleftmargin}%
    \hfill%
    {\smaller%
    \tt%
    \color{nbframe-out-prompt}%
    Out[83]:}%
    \hspace{\inputpadding}%
    \hspace{0em}%
    \hspace{3pt}%
    \end{minipage}%%
        }%
    %
    %
    \begin{addmargin}[\cellleftmargin]{0em}% left, right
    {\smaller%
    \vspace{-1\smallerfontscale}%
    
    
    
    \begin{verbatim}
((10,), (10, 1))
    \end{verbatim}

    
}%
    \end{addmargin}%
    % Add contents below.

{\par%
\vspace{-1\baselineskip}%
\needspace{4\baselineskip}}%
\begin{notebookcell}[84]%
\begin{addmargin}[\cellleftmargin]{0em}% left, right
{\smaller%
\par%
%
\vspace{-1\smallerfontscale}%
\begin{Verbatim}[commandchars=\\\{\}]
\PY{n}{x}\PY{o}{+}\PY{n}{b}
\end{Verbatim}
%
\par%
\vspace{-1\smallerfontscale}}%
\end{addmargin}
\end{notebookcell}

\par\vspace{1\smallerfontscale}%
    \needspace{4\baselineskip}%
    % Only render the prompt if the cell is pyout.  Note, the outputs prompt 
    % block isn't used since we need to check each indiviual output and only
    % add prompts to the pyout ones.
    
        {\par%
        \vspace{-1\smallerfontscale}%
        \noindent%
        \begin{minipage}{\cellleftmargin}%
    \hfill%
    {\smaller%
    \tt%
    \color{nbframe-out-prompt}%
    Out[84]:}%
    \hspace{\inputpadding}%
    \hspace{0em}%
    \hspace{3pt}%
    \end{minipage}%%
        }%
    %
    %
    \begin{addmargin}[\cellleftmargin]{0em}% left, right
    {\smaller%
    \vspace{-1\smallerfontscale}%
    
    
    
    \begin{verbatim}
array([[ 0.62627322,  1.62627322,  2.62627322,  3.62627322,  4.62627322,
         5.62627322,  6.62627322,  7.62627322,  8.62627322,  9.62627322],
       [ 0.01516076,  1.01516076,  2.01516076,  3.01516076,  4.01516076,
         5.01516076,  6.01516076,  7.01516076,  8.01516076,  9.01516076],
       [ 0.28310379,  1.28310379,  2.28310379,  3.28310379,  4.28310379,
         5.28310379,  6.28310379,  7.28310379,  8.28310379,  9.28310379],
       [ 0.79613997,  1.79613997,  2.79613997,  3.79613997,  4.79613997,
         5.79613997,  6.79613997,  7.79613997,  8.79613997,  9.79613997],
       [ 0.6313566 ,  1.6313566 ,  2.6313566 ,  3.6313566 ,  4.6313566 ,
         5.6313566 ,  6.6313566 ,  7.6313566 ,  8.6313566 ,  9.6313566 ],
       [ 0.29106039,  1.29106039,  2.29106039,  3.29106039,  4.29106039,
         5.29106039,  6.29106039,  7.29106039,  8.29106039,  9.29106039],
       [ 0.6707564 ,  1.6707564 ,  2.6707564 ,  3.6707564 ,  4.6707564 ,
         5.6707564 ,  6.6707564 ,  7.6707564 ,  8.6707564 ,  9.6707564 ],
       [ 0.67791273,  1.67791273,  2.67791273,  3.67791273,  4.67791273,
         5.67791273,  6.67791273,  7.67791273,  8.67791273,  9.67791273],
       [ 0.25119872,  1.25119872,  2.25119872,  3.25119872,  4.25119872,
         5.25119872,  6.25119872,  7.25119872,  8.25119872,  9.25119872],
       [ 0.39701644,  1.39701644,  2.39701644,  3.39701644,  4.39701644,
         5.39701644,  6.39701644,  7.39701644,  8.39701644,  9.39701644]])
    \end{verbatim}

    
}%
    \end{addmargin}%
    % Add contents below.

{\par%
\vspace{-1\baselineskip}%
\needspace{4\baselineskip}}%
\begin{notebookcell}[85]%
\begin{addmargin}[\cellleftmargin]{0em}% left, right
{\smaller%
\par%
%
\vspace{-1\smallerfontscale}%
\begin{Verbatim}[commandchars=\\\{\}]
\PY{n}{x2} \PY{o}{=} \PY{n}{solve}\PY{p}{(}\PY{n}{A}\PY{p}{,}\PY{n}{b}\PY{p}{)}
\end{Verbatim}
%
\par%
\vspace{-1\smallerfontscale}}%
\end{addmargin}
\end{notebookcell}


    % Add contents below.

{\par%
\vspace{-1\baselineskip}%
\needspace{4\baselineskip}}%
\begin{notebookcell}[86]%
\begin{addmargin}[\cellleftmargin]{0em}% left, right
{\smaller%
\par%
%
\vspace{-1\smallerfontscale}%
\begin{Verbatim}[commandchars=\\\{\}]
\PY{n}{A}\PY{o}{*}\PY{n}{x2}\PY{o}{\PYZhy{}}\PY{n}{b}
\end{Verbatim}
%
\par%
\vspace{-1\smallerfontscale}}%
\end{addmargin}
\end{notebookcell}

\par\vspace{1\smallerfontscale}%
    \needspace{4\baselineskip}%
    % Only render the prompt if the cell is pyout.  Note, the outputs prompt 
    % block isn't used since we need to check each indiviual output and only
    % add prompts to the pyout ones.
    
        {\par%
        \vspace{-1\smallerfontscale}%
        \noindent%
        \begin{minipage}{\cellleftmargin}%
    \hfill%
    {\smaller%
    \tt%
    \color{nbframe-out-prompt}%
    Out[86]:}%
    \hspace{\inputpadding}%
    \hspace{0em}%
    \hspace{3pt}%
    \end{minipage}%%
        }%
    %
    %
    \begin{addmargin}[\cellleftmargin]{0em}% left, right
    {\smaller%
    \vspace{-1\smallerfontscale}%
    
    
    
    \begin{verbatim}
array([[ -3.95409374e-01,   4.13685767e-01,   2.14409468e-01,
         -6.11842916e-01,   9.99325848e-02,  -1.80705335e-01,
          4.60119301e-01,   1.85241297e-02,   1.83146509e-01,
         -5.73630120e-01],
       [ -6.88983026e-01,  -9.55154870e-01,   2.54459630e-01,
          2.43030937e-01,  -2.06568041e-01,  -4.28551990e-01,
          6.72499735e-01,  -4.51001736e-01,   1.46594431e-02,
         -4.69708206e-01],
       [ -3.46408297e-01,  -1.19438477e-01,  -3.27836245e-01,
         -1.09431832e-01,  -3.93504767e-01,  -1.93695090e-01,
         -2.53422338e-01,  -4.14724995e-01,  -2.97374203e-01,
         -3.93128055e-01],
       [ -7.78827743e-01,  -9.17404612e-01,  -7.45494711e-01,
         -9.22583512e-01,  -9.41219026e-01,  -7.53176435e-01,
         -8.45786322e-01,  -6.66898316e-01,  -7.99559214e-01,
         -8.22526185e-01],
       [ -1.03200940e+00,  -6.94543850e-01,  -1.15150462e+00,
         -2.35762676e-01,  -5.29343164e-01,  -4.93288471e-01,
         -5.16766367e-01,  -1.07569529e+00,  -8.58115139e-02,
         -1.35220025e+00],
       [ -7.80664600e-01,   2.68440722e-01,  -2.41399829e-01,
         -3.06844995e-01,   3.31572377e-01,   9.34426675e-01,
         -5.17301177e-01,   5.37634677e-01,   2.54035523e-04,
          1.01212903e-01],
       [ -1.72659782e+00,  -1.61006484e+00,  -1.73012235e-01,
         -1.22882031e+00,  -1.87884725e+00,  -1.00480080e+00,
         -2.72855063e-01,   7.06177709e-01,  -1.02047558e+00,
         -1.53621766e+00],
       [ -4.25488959e-01,  -9.28754504e-01,  -1.09543169e+00,
         -8.60858296e-01,  -4.05409203e-01,  -5.91867877e-01,
         -8.29765544e-01,  -6.12709584e-01,  -4.15395974e-01,
         -6.09868912e-01],
       [ -2.50242819e-01,  -2.87483231e-01,  -2.42666143e-01,
         -2.39819874e-01,  -2.31899871e-01,  -2.65062491e-01,
         -2.44599600e-01,  -2.49597843e-01,  -2.76084286e-01,
         -2.74808172e-01],
       [ -2.44695529e-01,  -2.42136197e-01,  -2.85091549e-01,
         -2.06551965e-01,  -2.18391036e-01,  -2.75058884e-01,
         -4.99375982e-01,  -3.36798199e-01,  -2.14202261e-01,
         -2.67609670e-01]])
    \end{verbatim}

    
}%
    \end{addmargin}%
    % Add contents below.

{\par%
\vspace{-1\baselineskip}%
\needspace{4\baselineskip}}%
\begin{notebookcell}[87]%
\begin{addmargin}[\cellleftmargin]{0em}% left, right
{\smaller%
\par%
%
\vspace{-1\smallerfontscale}%
\begin{Verbatim}[commandchars=\\\{\}]
\PY{n}{A}\PY{o}{.}\PY{n}{dot}\PY{p}{(}\PY{n}{x2}\PY{p}{)}\PY{o}{\PYZhy{}}\PY{n}{b}
\end{Verbatim}
%
\par%
\vspace{-1\smallerfontscale}}%
\end{addmargin}
\end{notebookcell}

\par\vspace{1\smallerfontscale}%
    \needspace{4\baselineskip}%
    % Only render the prompt if the cell is pyout.  Note, the outputs prompt 
    % block isn't used since we need to check each indiviual output and only
    % add prompts to the pyout ones.
    
        {\par%
        \vspace{-1\smallerfontscale}%
        \noindent%
        \begin{minipage}{\cellleftmargin}%
    \hfill%
    {\smaller%
    \tt%
    \color{nbframe-out-prompt}%
    Out[87]:}%
    \hspace{\inputpadding}%
    \hspace{0em}%
    \hspace{3pt}%
    \end{minipage}%%
        }%
    %
    %
    \begin{addmargin}[\cellleftmargin]{0em}% left, right
    {\smaller%
    \vspace{-1\smallerfontscale}%
    
    
    
    \begin{verbatim}
array([[  0.00000000e+00],
       [ -4.16333634e-17],
       [  2.22044605e-16],
       [ -1.11022302e-16],
       [  0.00000000e+00],
       [ -6.66133815e-16],
       [ -1.11022302e-16],
       [  4.44089210e-16],
       [ -5.55111512e-17],
       [  0.00000000e+00]])
    \end{verbatim}

    
}%
    \end{addmargin}%
    % Add contents below.

{\par%
\vspace{-1\baselineskip}%
\needspace{4\baselineskip}}%
\begin{notebookcell}[88]%
\begin{addmargin}[\cellleftmargin]{0em}% left, right
{\smaller%
\par%
%
\vspace{-1\smallerfontscale}%
\begin{Verbatim}[commandchars=\\\{\}]
\PY{n}{np}\PY{o}{.}\PY{n}{allclose}\PY{p}{(}\PY{n}{A}\PY{o}{.}\PY{n}{dot}\PY{p}{(}\PY{n}{x2}\PY{p}{)}\PY{p}{,}\PY{n}{b}\PY{p}{)}
\end{Verbatim}
%
\par%
\vspace{-1\smallerfontscale}}%
\end{addmargin}
\end{notebookcell}

\par\vspace{1\smallerfontscale}%
    \needspace{4\baselineskip}%
    % Only render the prompt if the cell is pyout.  Note, the outputs prompt 
    % block isn't used since we need to check each indiviual output and only
    % add prompts to the pyout ones.
    
        {\par%
        \vspace{-1\smallerfontscale}%
        \noindent%
        \begin{minipage}{\cellleftmargin}%
    \hfill%
    {\smaller%
    \tt%
    \color{nbframe-out-prompt}%
    Out[88]:}%
    \hspace{\inputpadding}%
    \hspace{0em}%
    \hspace{3pt}%
    \end{minipage}%%
        }%
    %
    %
    \begin{addmargin}[\cellleftmargin]{0em}% left, right
    {\smaller%
    \vspace{-1\smallerfontscale}%
    
    
    
    \begin{verbatim}
True
    \end{verbatim}

    
}%
    \end{addmargin}%
    % Add contents below.

{\par%
\vspace{-1\baselineskip}%
\needspace{4\baselineskip}}%
\begin{notebookcell}[89]%
\begin{addmargin}[\cellleftmargin]{0em}% left, right
{\smaller%
\par%
%
\vspace{-1\smallerfontscale}%
\begin{Verbatim}[commandchars=\\\{\}]
\PY{n}{invA} \PY{o}{=} \PY{n}{inv}\PY{p}{(}\PY{n}{A}\PY{p}{)}
\PY{n}{np}\PY{o}{.}\PY{n}{allclose}\PY{p}{(}\PY{n}{invA}\PY{o}{.}\PY{n}{dot}\PY{p}{(}\PY{n}{A}\PY{p}{)}\PY{p}{,}\PY{n}{identity}\PY{p}{(}\PY{n}{shape}\PY{p}{(}\PY{n}{A}\PY{p}{)}\PY{p}{[}\PY{l+m+mi}{0}\PY{p}{]}\PY{p}{)}\PY{p}{)}
\end{Verbatim}
%
\par%
\vspace{-1\smallerfontscale}}%
\end{addmargin}
\end{notebookcell}

\par\vspace{1\smallerfontscale}%
    \needspace{4\baselineskip}%
    % Only render the prompt if the cell is pyout.  Note, the outputs prompt 
    % block isn't used since we need to check each indiviual output and only
    % add prompts to the pyout ones.
    
        {\par%
        \vspace{-1\smallerfontscale}%
        \noindent%
        \begin{minipage}{\cellleftmargin}%
    \hfill%
    {\smaller%
    \tt%
    \color{nbframe-out-prompt}%
    Out[89]:}%
    \hspace{\inputpadding}%
    \hspace{0em}%
    \hspace{3pt}%
    \end{minipage}%%
        }%
    %
    %
    \begin{addmargin}[\cellleftmargin]{0em}% left, right
    {\smaller%
    \vspace{-1\smallerfontscale}%
    
    
    
    \begin{verbatim}
True
    \end{verbatim}

    
}%
    \end{addmargin}%
    % Add contents below.

{\par%
\vspace{-1\baselineskip}%
\needspace{4\baselineskip}}%
\begin{notebookcell}[90]%
\begin{addmargin}[\cellleftmargin]{0em}% left, right
{\smaller%
\par%
%
\vspace{-1\smallerfontscale}%
\begin{Verbatim}[commandchars=\\\{\}]
\PY{n}{A}\PY{o}{.}\PY{n}{transpose}\PY{p}{(}\PY{p}{)}
\end{Verbatim}
%
\par%
\vspace{-1\smallerfontscale}}%
\end{addmargin}
\end{notebookcell}

\par\vspace{1\smallerfontscale}%
    \needspace{4\baselineskip}%
    % Only render the prompt if the cell is pyout.  Note, the outputs prompt 
    % block isn't used since we need to check each indiviual output and only
    % add prompts to the pyout ones.
    
        {\par%
        \vspace{-1\smallerfontscale}%
        \noindent%
        \begin{minipage}{\cellleftmargin}%
    \hfill%
    {\smaller%
    \tt%
    \color{nbframe-out-prompt}%
    Out[90]:}%
    \hspace{\inputpadding}%
    \hspace{0em}%
    \hspace{3pt}%
    \end{minipage}%%
        }%
    %
    %
    \begin{addmargin}[\cellleftmargin]{0em}% left, right
    {\smaller%
    \vspace{-1\smallerfontscale}%
    
    
    
    \begin{verbatim}
array([[ 0.37271708,  1.46971809,  0.57517022,  0.11021651,  1.27367445,
        -0.80129291, -2.09517358,  0.80999239,  0.05558992,  1.3773463 ],
       [ 1.67895704,  2.05028301, -1.48702545, -0.77201885,  0.20087213,
         0.91568713, -1.86392975, -0.80491598, -2.11011398,  1.40048879],
       [ 1.35723633, -0.58808677,  0.40642882,  0.32242783,  1.65354952,
         0.08127515,  0.98770556, -1.33975964,  0.49620902,  1.0120694 ],
       [ 0.02329694, -0.56315889, -1.57794353, -0.80498979, -1.2575923 ,
        -0.0258333 , -1.10740186, -0.58704659,  0.66173285,  1.72225559],
       [ 1.17241964,  0.41749102,  1.00307793, -0.92363086, -0.32430052,
         1.01900926, -2.39729186,  0.8744255 ,  1.12231799,  1.615202  ],
       [ 0.71934504,  0.90167482, -0.81234691,  0.27352293, -0.43891831,
         2.00564881, -0.66286563,  0.27610583, -0.80624321,  1.10278876],
       [ 1.75392144, -1.49990155, -0.26967886, -0.31606838, -0.36428213,
        -0.37026875,  0.78958106, -0.48727433,  0.38376945, -0.92557577],
       [ 1.04099014,  0.95064144,  1.19588005,  0.82280369,  1.4125518 ,
         1.35625362,  2.73233834,  0.20922773,  0.09309828,  0.54451729],
       [ 1.30676399, -0.0650428 ,  0.1296577 , -0.02176826, -1.73428674,
         0.47676915, -0.6939701 ,  0.84237936, -1.44721209,  1.65307857],
       [ 0.08498941,  0.99144335,  0.99965521, -0.16798512,  2.2915605 ,
         0.64199981, -1.71739006,  0.21834304, -1.37300009,  1.17014751]])
    \end{verbatim}

    
}%
    \end{addmargin}%
    % Add contents below.

{\par%
\vspace{-1\baselineskip}%
\needspace{4\baselineskip}}%
\begin{notebookcell}[91]%
\begin{addmargin}[\cellleftmargin]{0em}% left, right
{\smaller%
\par%
%
\vspace{-1\smallerfontscale}%
\begin{Verbatim}[commandchars=\\\{\}]
\PY{n}{A}\PY{o}{.}\PY{n}{sum}\PY{p}{(}\PY{p}{)}
\end{Verbatim}
%
\par%
\vspace{-1\smallerfontscale}}%
\end{addmargin}
\end{notebookcell}

\par\vspace{1\smallerfontscale}%
    \needspace{4\baselineskip}%
    % Only render the prompt if the cell is pyout.  Note, the outputs prompt 
    % block isn't used since we need to check each indiviual output and only
    % add prompts to the pyout ones.
    
        {\par%
        \vspace{-1\smallerfontscale}%
        \noindent%
        \begin{minipage}{\cellleftmargin}%
    \hfill%
    {\smaller%
    \tt%
    \color{nbframe-out-prompt}%
    Out[91]:}%
    \hspace{\inputpadding}%
    \hspace{0em}%
    \hspace{3pt}%
    \end{minipage}%%
        }%
    %
    %
    \begin{addmargin}[\cellleftmargin]{0em}% left, right
    {\smaller%
    \vspace{-1\smallerfontscale}%
    
    
    
    \begin{verbatim}
22.004705986670082
    \end{verbatim}

    
}%
    \end{addmargin}%
    % Add contents below.

{\par%
\vspace{-1\baselineskip}%
\needspace{4\baselineskip}}%
\begin{notebookcell}[92]%
\begin{addmargin}[\cellleftmargin]{0em}% left, right
{\smaller%
\par%
%
\vspace{-1\smallerfontscale}%
\begin{Verbatim}[commandchars=\\\{\}]
\PY{n}{np}\PY{o}{.}\PY{n}{sum}\PY{p}{(}\PY{n}{A}\PY{p}{)}
\end{Verbatim}
%
\par%
\vspace{-1\smallerfontscale}}%
\end{addmargin}
\end{notebookcell}

\par\vspace{1\smallerfontscale}%
    \needspace{4\baselineskip}%
    % Only render the prompt if the cell is pyout.  Note, the outputs prompt 
    % block isn't used since we need to check each indiviual output and only
    % add prompts to the pyout ones.
    
        {\par%
        \vspace{-1\smallerfontscale}%
        \noindent%
        \begin{minipage}{\cellleftmargin}%
    \hfill%
    {\smaller%
    \tt%
    \color{nbframe-out-prompt}%
    Out[92]:}%
    \hspace{\inputpadding}%
    \hspace{0em}%
    \hspace{3pt}%
    \end{minipage}%%
        }%
    %
    %
    \begin{addmargin}[\cellleftmargin]{0em}% left, right
    {\smaller%
    \vspace{-1\smallerfontscale}%
    
    
    
    \begin{verbatim}
22.004705986670082
    \end{verbatim}

    
}%
    \end{addmargin}%
    % Add contents below.

{\par%
\vspace{-1\baselineskip}%
\needspace{4\baselineskip}}%
\begin{notebookcell}[93]%
\begin{addmargin}[\cellleftmargin]{0em}% left, right
{\smaller%
\par%
%
\vspace{-1\smallerfontscale}%
\begin{Verbatim}[commandchars=\\\{\}]
\PY{n+nb}{sum}\PY{p}{(}\PY{n}{A}\PY{p}{)}
\end{Verbatim}
%
\par%
\vspace{-1\smallerfontscale}}%
\end{addmargin}
\end{notebookcell}

\par\vspace{1\smallerfontscale}%
    \needspace{4\baselineskip}%
    % Only render the prompt if the cell is pyout.  Note, the outputs prompt 
    % block isn't used since we need to check each indiviual output and only
    % add prompts to the pyout ones.
    
        {\par%
        \vspace{-1\smallerfontscale}%
        \noindent%
        \begin{minipage}{\cellleftmargin}%
    \hfill%
    {\smaller%
    \tt%
    \color{nbframe-out-prompt}%
    Out[93]:}%
    \hspace{\inputpadding}%
    \hspace{0em}%
    \hspace{3pt}%
    \end{minipage}%%
        }%
    %
    %
    \begin{addmargin}[\cellleftmargin]{0em}% left, right
    {\smaller%
    \vspace{-1\smallerfontscale}%
    
    
    
    \begin{verbatim}
22.004705986670082
    \end{verbatim}

    
}%
    \end{addmargin}%
    % Add contents below.

{\par%
\vspace{-1\baselineskip}%
\needspace{4\baselineskip}}%
\begin{notebookcell}[94]%
\begin{addmargin}[\cellleftmargin]{0em}% left, right
{\smaller%
\par%
%
\vspace{-1\smallerfontscale}%
\begin{Verbatim}[commandchars=\\\{\}]
\PY{n+nb}{sum} \PY{o+ow}{is} \PY{n}{np}\PY{o}{.}\PY{n}{sum}
\end{Verbatim}
%
\par%
\vspace{-1\smallerfontscale}}%
\end{addmargin}
\end{notebookcell}

\par\vspace{1\smallerfontscale}%
    \needspace{4\baselineskip}%
    % Only render the prompt if the cell is pyout.  Note, the outputs prompt 
    % block isn't used since we need to check each indiviual output and only
    % add prompts to the pyout ones.
    
        {\par%
        \vspace{-1\smallerfontscale}%
        \noindent%
        \begin{minipage}{\cellleftmargin}%
    \hfill%
    {\smaller%
    \tt%
    \color{nbframe-out-prompt}%
    Out[94]:}%
    \hspace{\inputpadding}%
    \hspace{0em}%
    \hspace{3pt}%
    \end{minipage}%%
        }%
    %
    %
    \begin{addmargin}[\cellleftmargin]{0em}% left, right
    {\smaller%
    \vspace{-1\smallerfontscale}%
    
    
    
    \begin{verbatim}
True
    \end{verbatim}

    
}%
    \end{addmargin}%
    % Add contents below.

{\par%
\vspace{-1\baselineskip}%
\needspace{4\baselineskip}}%
\begin{notebookcell}[]%
\begin{addmargin}[\cellleftmargin]{0em}% left, right
{\smaller%
\par%
%
\vspace{-1\smallerfontscale}%
\begin{Verbatim}[commandchars=\\\{\}]

\end{Verbatim}
%
\par%
\vspace{-1\smallerfontscale}}%
\end{addmargin}
\end{notebookcell}


    % Add contents below.

{\par%
\vspace{-1\baselineskip}%
\needspace{4\baselineskip}}%
\begin{notebookcell}[]%
\begin{addmargin}[\cellleftmargin]{0em}% left, right
{\smaller%
\par%
%
\vspace{-1\smallerfontscale}%
\begin{Verbatim}[commandchars=\\\{\}]

\end{Verbatim}
%
\par%
\vspace{-1\smallerfontscale}}%
\end{addmargin}
\end{notebookcell}


    % Add contents below.

{\par%
\vspace{-1\baselineskip}%
\needspace{4\baselineskip}}%
\begin{notebookcell}[]%
\begin{addmargin}[\cellleftmargin]{0em}% left, right
{\smaller%
\par%
%
\vspace{-1\smallerfontscale}%
\begin{Verbatim}[commandchars=\\\{\}]

\end{Verbatim}
%
\par%
\vspace{-1\smallerfontscale}}%
\end{addmargin}
\end{notebookcell}


    % Add contents below.

{\par%
\vspace{-1\baselineskip}%
\needspace{4\baselineskip}}%
\begin{notebookcell}[]%
\begin{addmargin}[\cellleftmargin]{0em}% left, right
{\smaller%
\par%
%
\vspace{-1\smallerfontscale}%
\begin{Verbatim}[commandchars=\\\{\}]

\end{Verbatim}
%
\par%
\vspace{-1\smallerfontscale}}%
\end{addmargin}
\end{notebookcell}


    % Add contents below.

{\par%
\vspace{-1\baselineskip}%
\needspace{4\baselineskip}}%
\begin{notebookcell}[]%
\begin{addmargin}[\cellleftmargin]{0em}% left, right
{\smaller%
\par%
%
\vspace{-1\smallerfontscale}%
\begin{Verbatim}[commandchars=\\\{\}]

\end{Verbatim}
%
\par%
\vspace{-1\smallerfontscale}}%
\end{addmargin}
\end{notebookcell}


    % Add contents below.

{\par%
\vspace{-1\baselineskip}%
\needspace{4\baselineskip}}%
\begin{notebookcell}[]%
\begin{addmargin}[\cellleftmargin]{0em}% left, right
{\smaller%
\par%
%
\vspace{-1\smallerfontscale}%
\begin{Verbatim}[commandchars=\\\{\}]

\end{Verbatim}
%
\par%
\vspace{-1\smallerfontscale}}%
\end{addmargin}
\end{notebookcell}


    \href{http://wiki.scipy.org/NumPy_for_Matlab_Users}{Numpy for Matlab
Users}

\href{http://mathesaurus.sourceforge.net/}{Other syntax conversions
between languages}

    % Add contents below.

{\par%
\vspace{-1\baselineskip}%
\needspace{4\baselineskip}}%
\begin{notebookcell}[2]%
\begin{addmargin}[\cellleftmargin]{0em}% left, right
{\smaller%
\par%
%
\vspace{-1\smallerfontscale}%
\begin{Verbatim}[commandchars=\\\{\}]
\PY{k+kn}{from} \PY{n+nn}{IPython.core.display} \PY{k+kn}{import} \PY{n}{HTML}
\PY{k}{def} \PY{n+nf}{css\PYZus{}styling}\PY{p}{(}\PY{p}{)}\PY{p}{:}
    \PY{n}{styles} \PY{o}{=} \PY{n+nb}{open}\PY{p}{(}\PY{l+s}{\PYZdq{}}\PY{l+s}{./styles/custom.css}\PY{l+s}{\PYZdq{}}\PY{p}{,} \PY{l+s}{\PYZdq{}}\PY{l+s}{r}\PY{l+s}{\PYZdq{}}\PY{p}{)}\PY{o}{.}\PY{n}{read}\PY{p}{(}\PY{p}{)}
    \PY{k}{return} \PY{n}{HTML}\PY{p}{(}\PY{n}{styles}\PY{p}{)}
\PY{n}{css\PYZus{}styling}\PY{p}{(}\PY{p}{)}
\end{Verbatim}
%
\par%
\vspace{-1\smallerfontscale}}%
\end{addmargin}
\end{notebookcell}

\par\vspace{1\smallerfontscale}%
    \needspace{4\baselineskip}%
    % Only render the prompt if the cell is pyout.  Note, the outputs prompt 
    % block isn't used since we need to check each indiviual output and only
    % add prompts to the pyout ones.
    
        {\par%
        \vspace{-1\smallerfontscale}%
        \noindent%
        \begin{minipage}{\cellleftmargin}%
    \hfill%
    {\smaller%
    \tt%
    \color{nbframe-out-prompt}%
    Out[2]:}%
    \hspace{\inputpadding}%
    \hspace{0em}%
    \hspace{3pt}%
    \end{minipage}%%
        }%
    %
    %
    \begin{addmargin}[\cellleftmargin]{0em}% left, right
    {\smaller%
    \vspace{-1\smallerfontscale}%
    
    
    
    \begin{verbatim}
<IPython.core.display.HTML at 0x7f5e10322510>
    \end{verbatim}

    
}%
    \end{addmargin}%
    % Add contents below.

{\par%
\vspace{-1\baselineskip}%
\needspace{4\baselineskip}}%
\begin{notebookcell}[]%
\begin{addmargin}[\cellleftmargin]{0em}% left, right
{\smaller%
\par%
%
\vspace{-1\smallerfontscale}%
\begin{Verbatim}[commandchars=\\\{\}]

\end{Verbatim}
%
\par%
\vspace{-1\smallerfontscale}}%
\end{addmargin}
\end{notebookcell}




%\newpage
%
% Default to the notebook output style

    


% Inherit from the specified cell style.




    
\documentclass{article}

    
    
    \usepackage{graphicx} % Used to insert images
    \usepackage{adjustbox} % Used to constrain images to a maximum size 
    \usepackage{color} % Allow colors to be defined
    \usepackage{enumerate} % Needed for markdown enumerations to work
    \usepackage{geometry} % Used to adjust the document margins
    \usepackage{amsmath} % Equations
    \usepackage{amssymb} % Equations
    \usepackage[mathletters]{ucs} % Extended unicode (utf-8) support
    \usepackage[utf8x]{inputenc} % Allow utf-8 characters in the tex document
    \usepackage{fancyvrb} % verbatim replacement that allows latex
    \usepackage{grffile} % extends the file name processing of package graphics 
                         % to support a larger range 
    % The hyperref package gives us a pdf with properly built
    % internal navigation ('pdf bookmarks' for the table of contents,
    % internal cross-reference links, web links for URLs, etc.)
    \usepackage{hyperref}
    \usepackage{longtable} % longtable support required by pandoc >1.10
    \usepackage{booktabs}  % table support for pandoc > 1.12.2
    

    
    
    \definecolor{orange}{cmyk}{0,0.4,0.8,0.2}
    \definecolor{darkorange}{rgb}{.71,0.21,0.01}
    \definecolor{darkgreen}{rgb}{.12,.54,.11}
    \definecolor{myteal}{rgb}{.26, .44, .56}
    \definecolor{gray}{gray}{0.45}
    \definecolor{lightgray}{gray}{.95}
    \definecolor{mediumgray}{gray}{.8}
    \definecolor{inputbackground}{rgb}{.95, .95, .85}
    \definecolor{outputbackground}{rgb}{.95, .95, .95}
    \definecolor{traceback}{rgb}{1, .95, .95}
    % ansi colors
    \definecolor{red}{rgb}{.6,0,0}
    \definecolor{green}{rgb}{0,.65,0}
    \definecolor{brown}{rgb}{0.6,0.6,0}
    \definecolor{blue}{rgb}{0,.145,.698}
    \definecolor{purple}{rgb}{.698,.145,.698}
    \definecolor{cyan}{rgb}{0,.698,.698}
    \definecolor{lightgray}{gray}{0.5}
    
    % bright ansi colors
    \definecolor{darkgray}{gray}{0.25}
    \definecolor{lightred}{rgb}{1.0,0.39,0.28}
    \definecolor{lightgreen}{rgb}{0.48,0.99,0.0}
    \definecolor{lightblue}{rgb}{0.53,0.81,0.92}
    \definecolor{lightpurple}{rgb}{0.87,0.63,0.87}
    \definecolor{lightcyan}{rgb}{0.5,1.0,0.83}
    
    % commands and environments needed by pandoc snippets
    % extracted from the output of `pandoc -s`
    \DefineVerbatimEnvironment{Highlighting}{Verbatim}{commandchars=\\\{\}}
    % Add ',fontsize=\small' for more characters per line
    \newenvironment{Shaded}{}{}
    \newcommand{\KeywordTok}[1]{\textcolor[rgb]{0.00,0.44,0.13}{\textbf{{#1}}}}
    \newcommand{\DataTypeTok}[1]{\textcolor[rgb]{0.56,0.13,0.00}{{#1}}}
    \newcommand{\DecValTok}[1]{\textcolor[rgb]{0.25,0.63,0.44}{{#1}}}
    \newcommand{\BaseNTok}[1]{\textcolor[rgb]{0.25,0.63,0.44}{{#1}}}
    \newcommand{\FloatTok}[1]{\textcolor[rgb]{0.25,0.63,0.44}{{#1}}}
    \newcommand{\CharTok}[1]{\textcolor[rgb]{0.25,0.44,0.63}{{#1}}}
    \newcommand{\StringTok}[1]{\textcolor[rgb]{0.25,0.44,0.63}{{#1}}}
    \newcommand{\CommentTok}[1]{\textcolor[rgb]{0.38,0.63,0.69}{\textit{{#1}}}}
    \newcommand{\OtherTok}[1]{\textcolor[rgb]{0.00,0.44,0.13}{{#1}}}
    \newcommand{\AlertTok}[1]{\textcolor[rgb]{1.00,0.00,0.00}{\textbf{{#1}}}}
    \newcommand{\FunctionTok}[1]{\textcolor[rgb]{0.02,0.16,0.49}{{#1}}}
    \newcommand{\RegionMarkerTok}[1]{{#1}}
    \newcommand{\ErrorTok}[1]{\textcolor[rgb]{1.00,0.00,0.00}{\textbf{{#1}}}}
    \newcommand{\NormalTok}[1]{{#1}}
    
    % Define a nice break command that doesn't care if a line doesn't already
    % exist.
    \def\br{\hspace*{\fill} \\* }
    % Math Jax compatability definitions
    \def\gt{>}
    \def\lt{<}
    % Document parameters
    \title{IV-SciPy}
    
    
    

    % Pygments definitions
    
\makeatletter
\def\PY@reset{\let\PY@it=\relax \let\PY@bf=\relax%
    \let\PY@ul=\relax \let\PY@tc=\relax%
    \let\PY@bc=\relax \let\PY@ff=\relax}
\def\PY@tok#1{\csname PY@tok@#1\endcsname}
\def\PY@toks#1+{\ifx\relax#1\empty\else%
    \PY@tok{#1}\expandafter\PY@toks\fi}
\def\PY@do#1{\PY@bc{\PY@tc{\PY@ul{%
    \PY@it{\PY@bf{\PY@ff{#1}}}}}}}
\def\PY#1#2{\PY@reset\PY@toks#1+\relax+\PY@do{#2}}

\expandafter\def\csname PY@tok@gd\endcsname{\def\PY@tc##1{\textcolor[rgb]{0.63,0.00,0.00}{##1}}}
\expandafter\def\csname PY@tok@gu\endcsname{\let\PY@bf=\textbf\def\PY@tc##1{\textcolor[rgb]{0.50,0.00,0.50}{##1}}}
\expandafter\def\csname PY@tok@gt\endcsname{\def\PY@tc##1{\textcolor[rgb]{0.00,0.27,0.87}{##1}}}
\expandafter\def\csname PY@tok@gs\endcsname{\let\PY@bf=\textbf}
\expandafter\def\csname PY@tok@gr\endcsname{\def\PY@tc##1{\textcolor[rgb]{1.00,0.00,0.00}{##1}}}
\expandafter\def\csname PY@tok@cm\endcsname{\let\PY@it=\textit\def\PY@tc##1{\textcolor[rgb]{0.25,0.50,0.50}{##1}}}
\expandafter\def\csname PY@tok@vg\endcsname{\def\PY@tc##1{\textcolor[rgb]{0.10,0.09,0.49}{##1}}}
\expandafter\def\csname PY@tok@m\endcsname{\def\PY@tc##1{\textcolor[rgb]{0.40,0.40,0.40}{##1}}}
\expandafter\def\csname PY@tok@mh\endcsname{\def\PY@tc##1{\textcolor[rgb]{0.40,0.40,0.40}{##1}}}
\expandafter\def\csname PY@tok@go\endcsname{\def\PY@tc##1{\textcolor[rgb]{0.53,0.53,0.53}{##1}}}
\expandafter\def\csname PY@tok@ge\endcsname{\let\PY@it=\textit}
\expandafter\def\csname PY@tok@vc\endcsname{\def\PY@tc##1{\textcolor[rgb]{0.10,0.09,0.49}{##1}}}
\expandafter\def\csname PY@tok@il\endcsname{\def\PY@tc##1{\textcolor[rgb]{0.40,0.40,0.40}{##1}}}
\expandafter\def\csname PY@tok@cs\endcsname{\let\PY@it=\textit\def\PY@tc##1{\textcolor[rgb]{0.25,0.50,0.50}{##1}}}
\expandafter\def\csname PY@tok@cp\endcsname{\def\PY@tc##1{\textcolor[rgb]{0.74,0.48,0.00}{##1}}}
\expandafter\def\csname PY@tok@gi\endcsname{\def\PY@tc##1{\textcolor[rgb]{0.00,0.63,0.00}{##1}}}
\expandafter\def\csname PY@tok@gh\endcsname{\let\PY@bf=\textbf\def\PY@tc##1{\textcolor[rgb]{0.00,0.00,0.50}{##1}}}
\expandafter\def\csname PY@tok@ni\endcsname{\let\PY@bf=\textbf\def\PY@tc##1{\textcolor[rgb]{0.60,0.60,0.60}{##1}}}
\expandafter\def\csname PY@tok@nl\endcsname{\def\PY@tc##1{\textcolor[rgb]{0.63,0.63,0.00}{##1}}}
\expandafter\def\csname PY@tok@nn\endcsname{\let\PY@bf=\textbf\def\PY@tc##1{\textcolor[rgb]{0.00,0.00,1.00}{##1}}}
\expandafter\def\csname PY@tok@no\endcsname{\def\PY@tc##1{\textcolor[rgb]{0.53,0.00,0.00}{##1}}}
\expandafter\def\csname PY@tok@na\endcsname{\def\PY@tc##1{\textcolor[rgb]{0.49,0.56,0.16}{##1}}}
\expandafter\def\csname PY@tok@nb\endcsname{\def\PY@tc##1{\textcolor[rgb]{0.00,0.50,0.00}{##1}}}
\expandafter\def\csname PY@tok@nc\endcsname{\let\PY@bf=\textbf\def\PY@tc##1{\textcolor[rgb]{0.00,0.00,1.00}{##1}}}
\expandafter\def\csname PY@tok@nd\endcsname{\def\PY@tc##1{\textcolor[rgb]{0.67,0.13,1.00}{##1}}}
\expandafter\def\csname PY@tok@ne\endcsname{\let\PY@bf=\textbf\def\PY@tc##1{\textcolor[rgb]{0.82,0.25,0.23}{##1}}}
\expandafter\def\csname PY@tok@nf\endcsname{\def\PY@tc##1{\textcolor[rgb]{0.00,0.00,1.00}{##1}}}
\expandafter\def\csname PY@tok@si\endcsname{\let\PY@bf=\textbf\def\PY@tc##1{\textcolor[rgb]{0.73,0.40,0.53}{##1}}}
\expandafter\def\csname PY@tok@s2\endcsname{\def\PY@tc##1{\textcolor[rgb]{0.73,0.13,0.13}{##1}}}
\expandafter\def\csname PY@tok@vi\endcsname{\def\PY@tc##1{\textcolor[rgb]{0.10,0.09,0.49}{##1}}}
\expandafter\def\csname PY@tok@nt\endcsname{\let\PY@bf=\textbf\def\PY@tc##1{\textcolor[rgb]{0.00,0.50,0.00}{##1}}}
\expandafter\def\csname PY@tok@nv\endcsname{\def\PY@tc##1{\textcolor[rgb]{0.10,0.09,0.49}{##1}}}
\expandafter\def\csname PY@tok@s1\endcsname{\def\PY@tc##1{\textcolor[rgb]{0.73,0.13,0.13}{##1}}}
\expandafter\def\csname PY@tok@kd\endcsname{\let\PY@bf=\textbf\def\PY@tc##1{\textcolor[rgb]{0.00,0.50,0.00}{##1}}}
\expandafter\def\csname PY@tok@sh\endcsname{\def\PY@tc##1{\textcolor[rgb]{0.73,0.13,0.13}{##1}}}
\expandafter\def\csname PY@tok@sc\endcsname{\def\PY@tc##1{\textcolor[rgb]{0.73,0.13,0.13}{##1}}}
\expandafter\def\csname PY@tok@sx\endcsname{\def\PY@tc##1{\textcolor[rgb]{0.00,0.50,0.00}{##1}}}
\expandafter\def\csname PY@tok@bp\endcsname{\def\PY@tc##1{\textcolor[rgb]{0.00,0.50,0.00}{##1}}}
\expandafter\def\csname PY@tok@c1\endcsname{\let\PY@it=\textit\def\PY@tc##1{\textcolor[rgb]{0.25,0.50,0.50}{##1}}}
\expandafter\def\csname PY@tok@kc\endcsname{\let\PY@bf=\textbf\def\PY@tc##1{\textcolor[rgb]{0.00,0.50,0.00}{##1}}}
\expandafter\def\csname PY@tok@c\endcsname{\let\PY@it=\textit\def\PY@tc##1{\textcolor[rgb]{0.25,0.50,0.50}{##1}}}
\expandafter\def\csname PY@tok@mf\endcsname{\def\PY@tc##1{\textcolor[rgb]{0.40,0.40,0.40}{##1}}}
\expandafter\def\csname PY@tok@err\endcsname{\def\PY@bc##1{\setlength{\fboxsep}{0pt}\fcolorbox[rgb]{1.00,0.00,0.00}{1,1,1}{\strut ##1}}}
\expandafter\def\csname PY@tok@mb\endcsname{\def\PY@tc##1{\textcolor[rgb]{0.40,0.40,0.40}{##1}}}
\expandafter\def\csname PY@tok@ss\endcsname{\def\PY@tc##1{\textcolor[rgb]{0.10,0.09,0.49}{##1}}}
\expandafter\def\csname PY@tok@sr\endcsname{\def\PY@tc##1{\textcolor[rgb]{0.73,0.40,0.53}{##1}}}
\expandafter\def\csname PY@tok@mo\endcsname{\def\PY@tc##1{\textcolor[rgb]{0.40,0.40,0.40}{##1}}}
\expandafter\def\csname PY@tok@kn\endcsname{\let\PY@bf=\textbf\def\PY@tc##1{\textcolor[rgb]{0.00,0.50,0.00}{##1}}}
\expandafter\def\csname PY@tok@mi\endcsname{\def\PY@tc##1{\textcolor[rgb]{0.40,0.40,0.40}{##1}}}
\expandafter\def\csname PY@tok@gp\endcsname{\let\PY@bf=\textbf\def\PY@tc##1{\textcolor[rgb]{0.00,0.00,0.50}{##1}}}
\expandafter\def\csname PY@tok@o\endcsname{\def\PY@tc##1{\textcolor[rgb]{0.40,0.40,0.40}{##1}}}
\expandafter\def\csname PY@tok@kr\endcsname{\let\PY@bf=\textbf\def\PY@tc##1{\textcolor[rgb]{0.00,0.50,0.00}{##1}}}
\expandafter\def\csname PY@tok@s\endcsname{\def\PY@tc##1{\textcolor[rgb]{0.73,0.13,0.13}{##1}}}
\expandafter\def\csname PY@tok@kp\endcsname{\def\PY@tc##1{\textcolor[rgb]{0.00,0.50,0.00}{##1}}}
\expandafter\def\csname PY@tok@w\endcsname{\def\PY@tc##1{\textcolor[rgb]{0.73,0.73,0.73}{##1}}}
\expandafter\def\csname PY@tok@kt\endcsname{\def\PY@tc##1{\textcolor[rgb]{0.69,0.00,0.25}{##1}}}
\expandafter\def\csname PY@tok@ow\endcsname{\let\PY@bf=\textbf\def\PY@tc##1{\textcolor[rgb]{0.67,0.13,1.00}{##1}}}
\expandafter\def\csname PY@tok@sb\endcsname{\def\PY@tc##1{\textcolor[rgb]{0.73,0.13,0.13}{##1}}}
\expandafter\def\csname PY@tok@k\endcsname{\let\PY@bf=\textbf\def\PY@tc##1{\textcolor[rgb]{0.00,0.50,0.00}{##1}}}
\expandafter\def\csname PY@tok@se\endcsname{\let\PY@bf=\textbf\def\PY@tc##1{\textcolor[rgb]{0.73,0.40,0.13}{##1}}}
\expandafter\def\csname PY@tok@sd\endcsname{\let\PY@it=\textit\def\PY@tc##1{\textcolor[rgb]{0.73,0.13,0.13}{##1}}}

\def\PYZbs{\char`\\}
\def\PYZus{\char`\_}
\def\PYZob{\char`\{}
\def\PYZcb{\char`\}}
\def\PYZca{\char`\^}
\def\PYZam{\char`\&}
\def\PYZlt{\char`\<}
\def\PYZgt{\char`\>}
\def\PYZsh{\char`\#}
\def\PYZpc{\char`\%}
\def\PYZdl{\char`\$}
\def\PYZhy{\char`\-}
\def\PYZsq{\char`\'}
\def\PYZdq{\char`\"}
\def\PYZti{\char`\~}
% for compatibility with earlier versions
\def\PYZat{@}
\def\PYZlb{[}
\def\PYZrb{]}
\makeatother


    % Exact colors from NB
    \definecolor{incolor}{rgb}{0.0, 0.0, 0.5}
    \definecolor{outcolor}{rgb}{0.545, 0.0, 0.0}



    
    % Prevent overflowing lines due to hard-to-break entities
    \sloppy 
    % Setup hyperref package
    \hypersetup{
      breaklinks=true,  % so long urls are correctly broken across lines
      colorlinks=true,
      urlcolor=blue,
      linkcolor=darkorange,
      citecolor=darkgreen,
      }
    % Slightly bigger margins than the latex defaults
    
    \geometry{verbose,tmargin=1in,bmargin=1in,lmargin=1in,rmargin=1in}
    
    

    \begin{document}
    
    
    \maketitle
    
    

    
    \section{IV-\href{http://www.scipy.org/}{SciPy} - Scientific
Python}\label{iv-scipy---scientific-python}

    \subsubsection{Index}\label{index}

\begin{itemize}
\itemsep1pt\parskip0pt\parsep0pt
\item
  \hyperref[quadrature]{Quadrature}
\item
  \hyperref[ode]{ODE Integrate}

  \begin{itemize}
  \itemsep1pt\parskip0pt\parsep0pt
  \item
    \hyperref[]{Way 1}
  \item
    \hyperref[]{Way 2}
  \end{itemize}
\item
  \hyperref[linearalgebra]{Linear Algebra}
\item
  \hyperref[optimize]{Optimize}
\item
  \hyperref[]{}
\item
  \hyperref[]{}
\item
  \hyperref[]{}
\end{itemize}

    

    \begin{Verbatim}[commandchars=\\\{\}]
{\color{incolor}In [{\color{incolor}1}]:} \PY{k+kn}{from} \PY{n+nn}{IPython.display} \PY{k+kn}{import} \PY{n}{Image}\PY{p}{,} \PY{n}{YouTubeVideo}
\end{Verbatim}

    \begin{Verbatim}[commandchars=\\\{\}]
{\color{incolor}In [{\color{incolor}2}]:} \PY{k+kn}{import} \PY{n+nn}{scipy} \PY{k+kn}{as} \PY{n+nn}{scp}
        \PY{o}{\PYZpc{}}\PY{k}{pylab}
\end{Verbatim}

    \begin{Verbatim}[commandchars=\\\{\}]
Using matplotlib backend: Qt4Agg
Populating the interactive namespace from numpy and matplotlib
    \end{Verbatim}

    \begin{Verbatim}[commandchars=\\\{\}]
{\color{incolor}In [{\color{incolor}13}]:} \PY{c}{\PYZsh{} help(scp)}
\end{Verbatim}


    \subsubsection{Integrate}


    Although not completely unrelated, scipy.integrate contains the
functions for quadrature and for ode integration.


    \subparagraph{Quadrature}


    \begin{Verbatim}[commandchars=\\\{\}]
{\color{incolor}In [{\color{incolor}14}]:} \PY{k+kn}{from}  \PY{n+nn}{scipy} \PY{k+kn}{import} \PY{n}{integrate}
\end{Verbatim}

    \begin{Verbatim}[commandchars=\\\{\}]
{\color{incolor}In [{\color{incolor}15}]:} \PY{n+nb}{dir}\PY{p}{(}\PY{n}{integrate}\PY{p}{)}
\end{Verbatim}

            \begin{Verbatim}[commandchars=\\\{\}]
{\color{outcolor}Out[{\color{outcolor}15}]:} ['IntegrationWarning',
          'Tester',
          '\_\_all\_\_',
          '\_\_builtins\_\_',
          '\_\_doc\_\_',
          '\_\_file\_\_',
          '\_\_name\_\_',
          '\_\_package\_\_',
          '\_\_path\_\_',
          '\_dop',
          '\_ode',
          '\_odepack',
          '\_quadpack',
          'absolute\_import',
          'complex\_ode',
          'cumtrapz',
          'dblquad',
          'division',
          'fixed\_quad',
          'lsoda',
          'newton\_cotes',
          'nquad',
          'ode',
          'odeint',
          'odepack',
          'print\_function',
          'quad',
          'quad\_explain',
          'quadpack',
          'quadrature',
          'romb',
          'romberg',
          's',
          'simps',
          'test',
          'tplquad',
          'trapz',
          'vode']
\end{Verbatim}
        
    \begin{Verbatim}[commandchars=\\\{\}]
{\color{incolor}In [{\color{incolor}16}]:} \PY{k+kn}{from} \PY{n+nn}{scipy.integrate} \PY{k+kn}{import} \PY{o}{*}
\end{Verbatim}

    Let's calculate an integral of the following type

    \[\int ^a_b f(x)dx \]

    We start by a simple example, \textbf{$f(x)=x^2$}

    \begin{Verbatim}[commandchars=\\\{\}]
{\color{incolor}In [{\color{incolor}17}]:} \PY{k}{def} \PY{n+nf}{f}\PY{p}{(}\PY{n}{x}\PY{p}{)}\PY{p}{:}
             \PY{k}{return} \PY{n}{x}\PY{o}{*}\PY{o}{*}\PY{l+m+mi}{2}
\end{Verbatim}

    We define $a$ and $b$

    \begin{Verbatim}[commandchars=\\\{\}]
{\color{incolor}In [{\color{incolor}18}]:} \PY{n}{a}\PY{p}{,} \PY{n}{b} \PY{o}{=} \PY{o}{\PYZhy{}}\PY{l+m+mi}{1}\PY{p}{,} \PY{l+m+mi}{1}
\end{Verbatim}

    \begin{Verbatim}[commandchars=\\\{\}]
{\color{incolor}In [{\color{incolor}19}]:} \PY{n}{quad}\PY{p}{(}\PY{n}{f}\PY{p}{,}\PY{n}{a}\PY{p}{,}\PY{n}{b}\PY{p}{)}
\end{Verbatim}

            \begin{Verbatim}[commandchars=\\\{\}]
{\color{outcolor}Out[{\color{outcolor}19}]:} (0.6666666666666666, 7.401486830834376e-15)
\end{Verbatim}
        
    Trapezoidal Rule

    \begin{Verbatim}[commandchars=\\\{\}]
{\color{incolor}In [{\color{incolor}20}]:} \PY{n}{Image}\PY{p}{(}\PY{l+s}{\PYZsq{}}\PY{l+s}{http://upload.wikimedia.org/wikipedia/commons/4/42/Composite\PYZus{}trapezoidal\PYZus{}rule\PYZus{}illustration.png}\PY{l+s}{\PYZsq{}}\PY{p}{)}
\end{Verbatim}
\texttt{\color{outcolor}Out[{\color{outcolor}20}]:}
    
    \begin{center}
    \adjustimage{max size={0.9\linewidth}{0.9\paperheight}}{IV-SciPy_files/IV-SciPy_20_0.png}
    \end{center}
    { \hspace*{\fill} \\}
    

    \begin{Verbatim}[commandchars=\\\{\}]
{\color{incolor}In [{\color{incolor}21}]:} \PY{n}{x} \PY{o}{=} \PY{n}{linspace}\PY{p}{(}\PY{o}{\PYZhy{}}\PY{l+m+mi}{1}\PY{p}{,}\PY{l+m+mi}{1}\PY{p}{,}\PY{l+m+mi}{100}\PY{p}{)}
         \PY{n}{y} \PY{o}{=} \PY{n}{f}\PY{p}{(}\PY{n}{x}\PY{p}{)}
         \PY{n}{trapz}\PY{p}{(}\PY{n}{y}\PY{p}{,}\PY{n}{x}\PY{p}{)}
\end{Verbatim}

            \begin{Verbatim}[commandchars=\\\{\}]
{\color{outcolor}Out[{\color{outcolor}21}]:} 0.66680270720674772
\end{Verbatim}
        
    We can check convergence of Trapezoidal Rule easily

    \begin{Verbatim}[commandchars=\\\{\}]
{\color{incolor}In [{\color{incolor}22}]:} \PY{n}{N} \PY{o}{=} \PY{l+m+mi}{10}\PY{o}{*}\PY{o}{*}\PY{n}{arange}\PY{p}{(}\PY{l+m+mi}{2}\PY{p}{,}\PY{l+m+mi}{6}\PY{p}{)}
         
         \PY{n}{trap\PYZus{}res} \PY{o}{=} \PY{p}{[}\PY{p}{]}
         \PY{k}{for} \PY{n}{n} \PY{o+ow}{in} \PY{n}{N}\PY{p}{:}
             \PY{n}{x} \PY{o}{=} \PY{n}{linspace}\PY{p}{(}\PY{o}{\PYZhy{}}\PY{l+m+mi}{1}\PY{p}{,}\PY{l+m+mi}{1}\PY{p}{,}\PY{n}{n}\PY{p}{)}
             \PY{n}{y} \PY{o}{=} \PY{n}{f}\PY{p}{(}\PY{n}{x}\PY{p}{)}
             \PY{n}{trap\PYZus{}res}\PY{o}{.}\PY{n}{append}\PY{p}{(}\PY{n}{trapz}\PY{p}{(}\PY{n}{y}\PY{p}{,}\PY{n}{x}\PY{p}{)}\PY{p}{)}
         \PY{n}{err\PYZus{}trap} \PY{o}{=} \PY{n+nb}{abs}\PY{p}{(}\PY{l+m+mf}{2.}\PY{o}{/}\PY{l+m+mi}{3}\PY{o}{\PYZhy{}}\PY{n}{asarray}\PY{p}{(}\PY{n}{trap\PYZus{}res}\PY{p}{)}\PY{p}{)}
         \PY{n}{plot}\PY{p}{(}\PY{n}{N}\PY{p}{,}\PY{n}{err\PYZus{}trap}\PY{p}{)}
         \PY{n}{semilogy}\PY{p}{(}\PY{p}{)}
         \PY{n}{semilogx}\PY{p}{(}\PY{p}{)}
\end{Verbatim}

            \begin{Verbatim}[commandchars=\\\{\}]
{\color{outcolor}Out[{\color{outcolor}22}]:} []
\end{Verbatim}
        
    Simpson's Rule

    \begin{Verbatim}[commandchars=\\\{\}]
{\color{incolor}In [{\color{incolor}23}]:} \PY{n}{N} \PY{o}{=} \PY{l+m+mi}{10}\PY{o}{*}\PY{o}{*}\PY{n}{arange}\PY{p}{(}\PY{l+m+mi}{2}\PY{p}{,}\PY{l+m+mi}{6}\PY{p}{)}
         
         \PY{n}{simps\PYZus{}res} \PY{o}{=} \PY{p}{[}\PY{p}{]}
         \PY{k}{for} \PY{n}{n} \PY{o+ow}{in} \PY{n}{N}\PY{p}{:}
             \PY{n}{x} \PY{o}{=} \PY{n}{linspace}\PY{p}{(}\PY{o}{\PYZhy{}}\PY{l+m+mi}{1}\PY{p}{,}\PY{l+m+mi}{1}\PY{p}{,}\PY{n}{n}\PY{p}{)}
             \PY{n}{y} \PY{o}{=} \PY{n}{f}\PY{p}{(}\PY{n}{x}\PY{p}{)}
             \PY{n}{simps\PYZus{}res}\PY{o}{.}\PY{n}{append}\PY{p}{(}\PY{n}{simps}\PY{p}{(}\PY{n}{y}\PY{p}{,}\PY{n}{x}\PY{p}{)}\PY{p}{)}
         \PY{n}{err\PYZus{}simps} \PY{o}{=} \PY{n+nb}{abs}\PY{p}{(}\PY{l+m+mf}{2.}\PY{o}{/}\PY{l+m+mi}{3}\PY{o}{\PYZhy{}}\PY{n}{asarray}\PY{p}{(}\PY{n}{simps\PYZus{}res}\PY{p}{)}\PY{p}{)}
         \PY{n}{plot}\PY{p}{(}\PY{n}{N}\PY{p}{,}\PY{n}{err\PYZus{}simps}\PY{p}{)}
         \PY{n}{semilogy}\PY{p}{(}\PY{p}{)}
         \PY{n}{semilogx}\PY{p}{(}\PY{p}{)}
\end{Verbatim}

            \begin{Verbatim}[commandchars=\\\{\}]
{\color{outcolor}Out[{\color{outcolor}23}]:} []
\end{Verbatim}
        
    Romb Integration

    \begin{Verbatim}[commandchars=\\\{\}]
{\color{incolor}In [{\color{incolor}24}]:} \PY{n}{romb\PYZus{}res} \PY{o}{=} \PY{p}{[}\PY{p}{]}
         \PY{n}{Nromb} \PY{o}{=} \PY{l+m+mi}{2}\PY{o}{*}\PY{o}{*}\PY{n}{arange}\PY{p}{(}\PY{l+m+mi}{2}\PY{p}{,}\PY{l+m+mi}{10}\PY{p}{)}\PY{o}{+}\PY{l+m+mi}{1}
         \PY{k}{for} \PY{n}{n} \PY{o+ow}{in} \PY{n}{Nromb}\PY{p}{:}
             \PY{n}{x}\PY{p}{,} \PY{n}{dx} \PY{o}{=} \PY{n}{linspace}\PY{p}{(}\PY{o}{\PYZhy{}}\PY{l+m+mi}{1}\PY{p}{,}\PY{l+m+mi}{1}\PY{p}{,}\PY{n}{n}\PY{p}{,}\PY{n}{retstep}\PY{o}{=}\PY{n+nb+bp}{True}\PY{p}{)}
             \PY{n}{y} \PY{o}{=} \PY{n}{f}\PY{p}{(}\PY{n}{x}\PY{p}{)}
             \PY{n}{romb\PYZus{}res}\PY{o}{.}\PY{n}{append}\PY{p}{(}\PY{n}{romb}\PY{p}{(}\PY{n}{y}\PY{p}{,}\PY{n}{dx}\PY{p}{)}\PY{p}{)}
         \PY{n}{err\PYZus{}romb} \PY{o}{=} \PY{n+nb}{abs}\PY{p}{(}\PY{l+m+mf}{2.}\PY{o}{/}\PY{l+m+mi}{3}\PY{o}{\PYZhy{}}\PY{n}{array}\PY{p}{(}\PY{n}{romb\PYZus{}res}\PY{p}{)}\PY{p}{)}
         \PY{n}{plot}\PY{p}{(}\PY{n}{Nromb}\PY{p}{,}\PY{n}{err\PYZus{}romb}\PY{p}{)}
         \PY{n}{semilogy}\PY{p}{(}\PY{p}{)}
         \PY{n}{semilogx}\PY{p}{(}\PY{p}{)}
\end{Verbatim}

            \begin{Verbatim}[commandchars=\\\{\}]
{\color{outcolor}Out[{\color{outcolor}24}]:} []
\end{Verbatim}
        
    \begin{Verbatim}[commandchars=\\\{\}]
{\color{incolor}In [{\color{incolor}25}]:} \PY{n}{plot}\PY{p}{(}\PY{n}{N}\PY{p}{,}\PY{n}{err\PYZus{}trap}\PY{p}{,}\PY{l+s}{\PYZsq{}}\PY{l+s}{r}\PY{l+s}{\PYZsq{}}\PY{p}{,}\PY{n}{N}\PY{p}{,}\PY{n}{err\PYZus{}simps}\PY{p}{,}\PY{l+s}{\PYZsq{}}\PY{l+s}{b}\PY{l+s}{\PYZsq{}}\PY{p}{,}\PY{n}{Nromb}\PY{p}{,}\PY{n}{err\PYZus{}romb}\PY{p}{,}\PY{l+s}{\PYZsq{}}\PY{l+s}{g}\PY{l+s}{\PYZsq{}}\PY{p}{)}
         \PY{n}{semilogy}\PY{p}{(}\PY{p}{)}
         \PY{n}{semilogx}\PY{p}{(}\PY{p}{)}
\end{Verbatim}

            \begin{Verbatim}[commandchars=\\\{\}]
{\color{outcolor}Out[{\color{outcolor}25}]:} []
\end{Verbatim}
        
    Indefinite intervals are also possible

    \begin{Verbatim}[commandchars=\\\{\}]
{\color{incolor}In [{\color{incolor}26}]:} \PY{n}{quad}\PY{p}{(}\PY{k}{lambda} \PY{n}{x}\PY{p}{:} \PY{n}{exp}\PY{p}{(}\PY{o}{\PYZhy{}}\PY{n}{x} \PY{o}{*}\PY{o}{*} \PY{l+m+mi}{2}\PY{p}{)}\PY{p}{,} \PY{o}{\PYZhy{}}\PY{n}{Inf}\PY{p}{,} \PY{n}{Inf}\PY{p}{)}
\end{Verbatim}

            \begin{Verbatim}[commandchars=\\\{\}]
{\color{outcolor}Out[{\color{outcolor}26}]:} (1.7724538509055159, 1.4202636780944923e-08)
\end{Verbatim}
        
    Double integral

    \begin{Verbatim}[commandchars=\\\{\}]
{\color{incolor}In [{\color{incolor}27}]:} \PY{k}{def} \PY{n+nf}{integrand}\PY{p}{(}\PY{n}{x}\PY{p}{,} \PY{n}{y}\PY{p}{)}\PY{p}{:}
             \PY{k}{return} \PY{n}{exp}\PY{p}{(}\PY{o}{\PYZhy{}}\PY{n}{x}\PY{o}{*}\PY{o}{*}\PY{l+m+mi}{2}\PY{o}{\PYZhy{}}\PY{n}{y}\PY{o}{*}\PY{o}{*}\PY{l+m+mi}{2}\PY{p}{)}
         
         \PY{n}{x\PYZus{}lower} \PY{o}{=} \PY{l+m+mi}{0}  
         \PY{n}{x\PYZus{}upper} \PY{o}{=} \PY{l+m+mi}{10}
         \PY{n}{y\PYZus{}lower} \PY{o}{=} \PY{l+m+mi}{0}
         \PY{n}{y\PYZus{}upper} \PY{o}{=} \PY{l+m+mi}{10}
         
         \PY{n}{dblquad}\PY{p}{(}\PY{n}{integrand}\PY{p}{,} \PY{n}{x\PYZus{}lower}\PY{p}{,} \PY{n}{x\PYZus{}upper}\PY{p}{,} \PY{k}{lambda} \PY{n}{x} \PY{p}{:} \PY{n}{y\PYZus{}lower}\PY{p}{,} \PY{k}{lambda} \PY{n}{x}\PY{p}{:} \PY{n}{y\PYZus{}upper}\PY{p}{)}
\end{Verbatim}

            \begin{Verbatim}[commandchars=\\\{\}]
{\color{outcolor}Out[{\color{outcolor}27}]:} (0.7853981633974476, 1.638229942140971e-13)
\end{Verbatim}
        
    Triple Integral

    \begin{Verbatim}[commandchars=\\\{\}]
{\color{incolor}In [{\color{incolor}28}]:} \PY{k}{def} \PY{n+nf}{integrand}\PY{p}{(}\PY{n}{x}\PY{p}{,} \PY{n}{y}\PY{p}{,}\PY{n}{z}\PY{p}{)}\PY{p}{:}
             \PY{k}{return} \PY{n}{exp}\PY{p}{(}\PY{o}{\PYZhy{}}\PY{n}{x}\PY{o}{*}\PY{o}{*}\PY{l+m+mi}{2}\PY{o}{\PYZhy{}}\PY{n}{y}\PY{o}{*}\PY{o}{*}\PY{l+m+mi}{2}\PY{o}{\PYZhy{}}\PY{n}{z}\PY{o}{*}\PY{o}{*}\PY{l+m+mi}{2}\PY{p}{)}
         
         \PY{n}{x\PYZus{}lower} \PY{o}{=} \PY{l+m+mi}{0}  
         \PY{n}{x\PYZus{}upper} \PY{o}{=} \PY{l+m+mi}{10}
         \PY{n}{y\PYZus{}lower} \PY{o}{=} \PY{l+m+mi}{0}
         \PY{n}{y\PYZus{}upper} \PY{o}{=} \PY{l+m+mi}{10}
         \PY{n}{z\PYZus{}lower} \PY{o}{=} \PY{l+m+mi}{0}
         \PY{n}{z\PYZus{}upper} \PY{o}{=} \PY{l+m+mi}{10}
         
         \PY{n}{tplquad}\PY{p}{(}\PY{n}{integrand}\PY{p}{,} \PY{n}{x\PYZus{}lower}\PY{p}{,} \PY{n}{x\PYZus{}upper}\PY{p}{,} \PY{k}{lambda} \PY{n}{x} \PY{p}{:} \PY{n}{y\PYZus{}lower}\PY{p}{,} \PY{k}{lambda} \PY{n}{x}\PY{p}{:} \PY{n}{y\PYZus{}upper}\PY{p}{,}\PYZbs{}
                 \PY{k}{lambda} \PY{n}{x}\PY{p}{,}\PY{n}{y}\PY{p}{:} \PY{n}{z\PYZus{}lower}\PY{p}{,} \PY{k}{lambda} \PY{n}{x}\PY{p}{,}\PY{n}{y}\PY{p}{:} \PY{n}{z\PYZus{}upper}\PY{p}{)}
\end{Verbatim}

            \begin{Verbatim}[commandchars=\\\{\}]
{\color{outcolor}Out[{\color{outcolor}28}]:} (0.6960409996039545, 1.4506309421028255e-13)
\end{Verbatim}
        
    Exercise

    \subsection{Numerical Integration}\label{numerical-integration}

    \begin{Verbatim}[commandchars=\\\{\}]
{\color{incolor}In [{\color{incolor}3}]:} \PY{k+kn}{import} \PY{n+nn}{scipy} \PY{k+kn}{as} \PY{n+nn}{scp}
        \PY{k+kn}{from} \PY{n+nn}{pylab} \PY{k+kn}{import} \PY{o}{*}
        \PY{o}{\PYZpc{}}\PY{k}{matplotlib}
        \PY{k+kn}{from} \PY{n+nn}{scipy.integrate} \PY{k+kn}{import} \PY{n}{odeint}\PY{p}{,} \PY{n}{ode}
\end{Verbatim}

    \begin{Verbatim}[commandchars=\\\{\}]
Using matplotlib backend: Qt4Agg
    \end{Verbatim}

    


    \paragraph{Way 1}


    The first way is to use \emph{odeint}. scipy.integrate.odeint is a
wrapper to LSODA (Fortran Library). It chooses which solver to use
depending on the stiffness of the problem. Using the docstring of
odeint, we obtain the following information:

    \begin{verbatim}
Parameters
----------
func : callable(y, t0, ...)
    Computes the derivative of y at t0.
y0 : array
    Initial condition on y (can be a vector).
t : array
    A sequence of time points for which to solve for y.  The initial
    value point should be the first element of this sequence.
args : tuple, optional
    Extra arguments to pass to function.
Dfun : callable(y, t0, ...)
    Gradient (Jacobian) of `func`.
col_deriv : bool, optional
    True if `Dfun` defines derivatives down columns (faster),
    otherwise `Dfun` should define derivatives across rows.
full_output : bool, optional
    True if to return a dictionary of optional outputs as the second output
printmessg : bool, optional
    Whether to print the convergence message

Returns
-------
y : array, shape (len(t), len(y0))
    Array containing the value of y for each desired time in t,
    with the initial value `y0` in the first row.
infodict : dict, only returned if full_output == True
\end{verbatim}

    Let us try and solve the equations of motion of a simple pendulum

    \begin{Verbatim}[commandchars=\\\{\}]
{\color{incolor}In [{\color{incolor}35}]:} \PY{n}{g} \PY{o}{=} \PY{l+m+mf}{9.8}
         \PY{n}{L} \PY{o}{=} \PY{l+m+mf}{0.5}
         \PY{n}{m} \PY{o}{=} \PY{l+m+mf}{0.1}
\end{Verbatim}

    \begin{Verbatim}[commandchars=\\\{\}]
{\color{incolor}In [{\color{incolor}32}]:} \PY{k}{def} \PY{n+nf}{f}\PY{p}{(}\PY{n}{x}\PY{p}{,} \PY{n}{t}\PY{p}{,} \PY{n}{g}\PY{p}{,} \PY{n}{l}\PY{p}{)}\PY{p}{:}
             \PY{l+s+sd}{\PYZsq{}\PYZsq{}\PYZsq{}RHS of pendulum\PYZsq{}\PYZsq{}\PYZsq{}}
             \PY{n}{x1}\PY{p}{,} \PY{n}{x2} \PY{o}{=} \PY{n}{x}
             \PY{n}{dx1} \PY{o}{=} \PY{n}{x2}
             \PY{n}{dx2} \PY{o}{=} \PY{o}{\PYZhy{}} \PY{n}{g}\PY{o}{/}\PY{n}{l}\PY{o}{*}\PY{n}{sin}\PY{p}{(}\PY{n}{x1}\PY{p}{)}
             
             \PY{k}{return} \PY{p}{[}\PY{n}{dx1}\PY{p}{,} \PY{n}{dx2}\PY{p}{]}
\end{Verbatim}

    \begin{Verbatim}[commandchars=\\\{\}]
{\color{incolor}In [{\color{incolor}33}]:} \PY{n}{x0} \PY{o}{=} \PY{p}{[}\PY{n}{pi}\PY{o}{/}\PY{l+m+mi}{4}\PY{p}{,} \PY{l+m+mi}{0}\PY{p}{]}
         \PY{n}{t} \PY{o}{=} \PY{n}{linspace}\PY{p}{(}\PY{l+m+mi}{0}\PY{p}{,} \PY{l+m+mi}{10}\PY{p}{,} \PY{l+m+mi}{250}\PY{p}{)}
\end{Verbatim}

    \begin{Verbatim}[commandchars=\\\{\}]
{\color{incolor}In [{\color{incolor}37}]:} \PY{n}{x} \PY{o}{=} \PY{n}{odeint}\PY{p}{(}\PY{n}{f}\PY{p}{,} \PY{n}{x0}\PY{p}{,} \PY{n}{t}\PY{p}{,} \PY{p}{(}\PY{n}{g}\PY{p}{,}\PY{n}{L}\PY{p}{,}\PY{p}{)}\PY{p}{)}
\end{Verbatim}

    \begin{Verbatim}[commandchars=\\\{\}]
{\color{incolor}In [{\color{incolor}45}]:} \PY{n}{fig}\PY{p}{,} \PY{n}{ax} \PY{o}{=} \PY{n}{subplots}\PY{p}{(}\PY{l+m+mi}{1}\PY{p}{,}\PY{l+m+mi}{2}\PY{p}{)}
         \PY{n}{ax}\PY{p}{[}\PY{l+m+mi}{0}\PY{p}{]}\PY{o}{.}\PY{n}{plot}\PY{p}{(}\PY{n}{t}\PY{p}{,} \PY{n}{x}\PY{p}{[}\PY{p}{:}\PY{p}{,} \PY{l+m+mi}{0}\PY{p}{]}\PY{p}{,} \PY{l+s}{\PYZsq{}}\PY{l+s}{r}\PY{l+s}{\PYZsq{}}\PY{p}{,} \PY{n}{label}\PY{o}{=}\PY{l+s}{\PYZdq{}}\PY{l+s}{theta1}\PY{l+s}{\PYZdq{}}\PY{p}{)}
         \PY{n}{x1} \PY{o}{=} \PY{o}{+} \PY{n}{L}\PY{o}{*}\PY{n}{sin}\PY{p}{(}\PY{n}{x}\PY{p}{[}\PY{p}{:}\PY{p}{,} \PY{l+m+mi}{0}\PY{p}{]}\PY{p}{)}
         \PY{n}{y1} \PY{o}{=} \PY{o}{\PYZhy{}} \PY{n}{L}\PY{o}{*}\PY{n}{cos}\PY{p}{(}\PY{n}{x}\PY{p}{[}\PY{p}{:}\PY{p}{,} \PY{l+m+mi}{0}\PY{p}{]}\PY{p}{)}
         \PY{n}{ax}\PY{p}{[}\PY{l+m+mi}{1}\PY{p}{]}\PY{o}{.}\PY{n}{plot}\PY{p}{(}\PY{n}{x1}\PY{p}{,}\PY{n}{y1}\PY{p}{,}\PY{l+s}{\PYZsq{}}\PY{l+s}{b}\PY{l+s}{\PYZsq{}}\PY{p}{)}
\end{Verbatim}

            \begin{Verbatim}[commandchars=\\\{\}]
{\color{outcolor}Out[{\color{outcolor}45}]:} [<matplotlib.lines.Line2D at 0x7fcf5ecd3d10>]
\end{Verbatim}
        
    \begin{Verbatim}[commandchars=\\\{\}]
{\color{incolor}In [{\color{incolor}46}]:} \PY{k+kn}{from} \PY{n+nn}{matplotlib.widgets} \PY{k+kn}{import} \PY{n}{Slider}
\end{Verbatim}

    \begin{Verbatim}[commandchars=\\\{\}]
{\color{incolor}In [{\color{incolor}73}]:} \PY{n}{axanim} \PY{o}{=} \PY{n}{axes}\PY{p}{(}\PY{p}{[}\PY{l+m+mf}{0.1}\PY{p}{,} \PY{l+m+mf}{0.25}\PY{p}{,} \PY{l+m+mf}{0.8}\PY{p}{,} \PY{l+m+mf}{0.6}\PY{p}{]}\PY{p}{)}
         \PY{n}{pl}\PY{p}{,} \PY{o}{=} \PY{n}{axanim}\PY{o}{.}\PY{n}{plot}\PY{p}{(}\PY{p}{[}\PY{l+m+mi}{0}\PY{p}{,}\PY{n}{x1}\PY{p}{[}\PY{l+m+mi}{0}\PY{p}{]}\PY{p}{]}\PY{p}{,}\PY{p}{[}\PY{l+m+mi}{0}\PY{p}{,}\PY{n}{y1}\PY{p}{[}\PY{l+m+mi}{0}\PY{p}{]}\PY{p}{]}\PY{p}{,}\PY{l+s}{\PYZsq{}}\PY{l+s}{k}\PY{l+s}{\PYZsq{}}\PY{p}{)}
         \PY{n}{pl2}\PY{p}{,} \PY{o}{=} \PY{n}{plot}\PY{p}{(}\PY{n}{x1}\PY{p}{[}\PY{l+m+mi}{0}\PY{p}{]}\PY{p}{,}\PY{n}{y1}\PY{p}{[}\PY{l+m+mi}{0}\PY{p}{]}\PY{p}{,}\PY{l+s}{\PYZsq{}}\PY{l+s}{bo}\PY{l+s}{\PYZsq{}}\PY{p}{)}
         \PY{n}{pl3}\PY{p}{,} \PY{o}{=} \PY{n}{plot}\PY{p}{(}\PY{n}{x1}\PY{p}{[}\PY{l+m+mi}{0}\PY{p}{]}\PY{p}{,}\PY{n}{y1}\PY{p}{[}\PY{l+m+mi}{0}\PY{p}{]}\PY{p}{,}\PY{l+s}{\PYZsq{}}\PY{l+s}{r}\PY{l+s}{\PYZsq{}}\PY{p}{,}\PY{n}{alpha}\PY{o}{=}\PY{l+m+mf}{0.25}\PY{p}{)}
         \PY{n}{xlim}\PY{p}{(}\PY{p}{[}\PY{o}{\PYZhy{}}\PY{l+m+mf}{1.5}\PY{o}{*}\PY{n}{L}\PY{p}{,} \PY{l+m+mf}{1.5}\PY{o}{*}\PY{n}{L}\PY{p}{]}\PY{p}{)}
         \PY{n}{ylim}\PY{p}{(}\PY{p}{[}\PY{o}{\PYZhy{}}\PY{l+m+mf}{1.5}\PY{o}{*}\PY{n}{L}\PY{p}{,} \PY{l+m+mi}{0}\PY{p}{]}\PY{p}{)}
         
         \PY{n}{axsl} \PY{o}{=} \PY{n}{axes}\PY{p}{(}\PY{p}{[}\PY{l+m+mf}{0.1}\PY{p}{,} \PY{l+m+mf}{0.1}\PY{p}{,} \PY{l+m+mf}{0.8}\PY{p}{,} \PY{l+m+mf}{0.1}\PY{p}{]}\PY{p}{)}
         \PY{n}{sl} \PY{o}{=} \PY{n}{Slider}\PY{p}{(}\PY{n}{axsl}\PY{p}{,}\PY{l+s}{\PYZsq{}}\PY{l+s}{Time}\PY{l+s}{\PYZsq{}}\PY{p}{,}\PY{n}{t}\PY{o}{.}\PY{n}{min}\PY{p}{(}\PY{p}{)}\PY{p}{,} \PY{n}{t}\PY{o}{.}\PY{n}{max}\PY{p}{(}\PY{p}{)}\PY{p}{,}\PY{n}{valinit}\PY{o}{=}\PY{l+m+mi}{0}\PY{p}{,} \PY{n}{valfmt}\PY{o}{=}\PY{l+s}{\PYZsq{}}\PY{l+s+si}{\PYZpc{}.2f}\PY{l+s}{ s}\PY{l+s}{\PYZsq{}}\PY{p}{)}
         
         \PY{k}{def} \PY{n+nf}{update}\PY{p}{(}\PY{n}{data}\PY{p}{)}\PY{p}{:}
             \PY{n}{it} \PY{o}{=} \PY{n+nb}{abs}\PY{p}{(}\PY{n}{t}\PY{o}{\PYZhy{}}\PY{n}{data}\PY{p}{)}\PY{o}{.}\PY{n}{argmin}\PY{p}{(}\PY{p}{)}
             \PY{n}{pl}\PY{o}{.}\PY{n}{set\PYZus{}data}\PY{p}{(}\PY{p}{[}\PY{l+m+mi}{0}\PY{p}{,}\PY{n}{x1}\PY{p}{[}\PY{n}{it}\PY{p}{]}\PY{p}{]}\PY{p}{,}\PY{p}{[}\PY{l+m+mi}{0}\PY{p}{,}\PY{n}{y1}\PY{p}{[}\PY{n}{it}\PY{p}{]}\PY{p}{]}\PY{p}{)}
             \PY{n}{pl2}\PY{o}{.}\PY{n}{set\PYZus{}data}\PY{p}{(}\PY{n}{x1}\PY{p}{[}\PY{n}{it}\PY{p}{]}\PY{p}{,}\PY{n}{y1}\PY{p}{[}\PY{n}{it}\PY{p}{]}\PY{p}{)}
             \PY{n}{pl3}\PY{o}{.}\PY{n}{set\PYZus{}data}\PY{p}{(}\PY{n}{x1}\PY{p}{[}\PY{p}{:}\PY{n}{it}\PY{o}{+}\PY{l+m+mi}{1}\PY{p}{]}\PY{p}{,}\PY{n}{y1}\PY{p}{[}\PY{p}{:}\PY{n}{it}\PY{o}{+}\PY{l+m+mi}{1}\PY{p}{]}\PY{p}{)}
             \PY{n}{draw}\PY{p}{(}\PY{p}{)}
         
         \PY{n}{sl}\PY{o}{.}\PY{n}{on\PYZus{}changed}\PY{p}{(}\PY{n}{update}\PY{p}{)}
\end{Verbatim}

            \begin{Verbatim}[commandchars=\\\{\}]
{\color{outcolor}Out[{\color{outcolor}73}]:} 0
\end{Verbatim}
        
    


    \paragraph{Way 2}


    Another possibility is to use the ode package. They differ in the
control they give the user. Let's run the simple pendulum example.

    \begin{Verbatim}[commandchars=\\\{\}]
{\color{incolor}In [{\color{incolor}25}]:} \PY{n}{g} \PY{o}{=} \PY{l+m+mf}{9.8}
         \PY{n}{L} \PY{o}{=} \PY{l+m+mf}{0.5}
         \PY{n}{m} \PY{o}{=} \PY{l+m+mf}{0.1}
\end{Verbatim}

    \begin{Verbatim}[commandchars=\\\{\}]
{\color{incolor}In [{\color{incolor}26}]:} \PY{k}{def} \PY{n+nf}{f}\PY{p}{(}\PY{n}{x}\PY{p}{,} \PY{n}{t}\PY{p}{,} \PY{n}{g}\PY{p}{,} \PY{n}{l}\PY{p}{)}\PY{p}{:}
             \PY{l+s+sd}{\PYZsq{}\PYZsq{}\PYZsq{}RHS of pendulum\PYZsq{}\PYZsq{}\PYZsq{}}
             \PY{n}{x1}\PY{p}{,} \PY{n}{x2} \PY{o}{=} \PY{n}{x}
             \PY{n}{dx1} \PY{o}{=} \PY{n}{x2}
             \PY{n}{dx2} \PY{o}{=} \PY{o}{\PYZhy{}} \PY{n}{g}\PY{o}{/}\PY{n}{l}\PY{o}{*}\PY{n}{sin}\PY{p}{(}\PY{n}{x1}\PY{p}{)}
             
             \PY{k}{return} \PY{p}{[}\PY{n}{dx1}\PY{p}{,} \PY{n}{dx2}\PY{p}{]}
         
         \PY{n}{x0} \PY{o}{=} \PY{p}{[}\PY{n}{pi}\PY{o}{/}\PY{l+m+mi}{4}\PY{p}{,} \PY{l+m+mi}{0}\PY{p}{]}
         \PY{n}{t} \PY{o}{=} \PY{n}{linspace}\PY{p}{(}\PY{l+m+mi}{0}\PY{p}{,} \PY{l+m+mi}{10}\PY{p}{,} \PY{l+m+mi}{250}\PY{p}{)}
\end{Verbatim}

    \begin{Verbatim}[commandchars=\\\{\}]
{\color{incolor}In [{\color{incolor}27}]:} \PY{n}{solver} \PY{o}{=} \PY{n}{ode}\PY{p}{(}\PY{n}{f}\PY{p}{)}
         \PY{n}{solver}\PY{o}{.}\PY{n}{set\PYZus{}f\PYZus{}params}\PY{p}{(}\PY{n}{g}\PY{p}{,}\PY{n}{L}\PY{p}{)}
         \PY{n}{solver}\PY{o}{.}\PY{n}{set\PYZus{}initial\PYZus{}value}\PY{p}{(}\PY{n}{x0}\PY{p}{,}\PY{n}{t}\PY{p}{[}\PY{l+m+mi}{0}\PY{p}{]}\PY{p}{)}
\end{Verbatim}

            \begin{Verbatim}[commandchars=\\\{\}]
{\color{outcolor}Out[{\color{outcolor}27}]:} <scipy.integrate.\_ode.ode at 0x7f30221c6990>
\end{Verbatim}
        
    \begin{Verbatim}[commandchars=\\\{\}]
{\color{incolor}In [{\color{incolor}28}]:} \PY{n}{dt} \PY{o}{=} \PY{n}{diff}\PY{p}{(}\PY{n}{t}\PY{p}{)}\PY{p}{[}\PY{l+m+mi}{0}\PY{p}{]}
         \PY{k}{while} \PY{n}{solver}\PY{o}{.}\PY{n}{successful}\PY{p}{(}\PY{p}{)} \PY{o+ow}{and} \PY{n}{solver}\PY{o}{.}\PY{n}{t} \PY{o}{\PYZlt{}} \PY{n}{t}\PY{p}{[}\PY{o}{\PYZhy{}}\PY{l+m+mi}{1}\PY{p}{]}\PY{p}{:}
             \PY{n}{solver}\PY{o}{.}\PY{n}{integrate}\PY{p}{(}\PY{n}{solver}\PY{o}{.}\PY{n}{t}\PY{o}{+}\PY{n}{dt}\PY{p}{)}
             \PY{k}{print} \PY{p}{(}\PY{n}{solver}\PY{o}{.}\PY{n}{t}\PY{p}{,} \PY{n}{solver}\PY{o}{.}\PY{n}{y}\PY{p}{)}
\end{Verbatim}

    \begin{Verbatim}[commandchars=\\\{\}]

        ---------------------------------------------------------------------------
    TypeError                                 Traceback (most recent call last)

        <ipython-input-28-5598cddd42a8> in <module>()
          1 dt = diff(t)[0]
          2 while solver.successful() and solver.t < t[-1]:
    ----> 3     solver.integrate(solver.t+dt)
          4     print (solver.t, solver.y)


        /home/jpsilva/anaconda/lib/python2.7/site-packages/scipy/integrate/\_ode.pyc in integrate(self, t, step, relax)
        386             self.\_y, self.t = mth(self.f, self.jac or (lambda: None),
        387                                 self.\_y, self.t, t,
    --> 388                                 self.f\_params, self.jac\_params)
        389         except SystemError:
        390             \# f2py issue with tuple returns, see ticket 1187.


        /home/jpsilva/anaconda/lib/python2.7/site-packages/scipy/integrate/\_ode.pyc in run(self, *args)
        735             self.acquire\_new\_handle()
        736         y1, t, istate = self.runner(*(args[:5] + tuple(self.call\_args) +
    --> 737                                       args[5:]))
        738         if istate < 0:
        739             warnings.warn('vode: ' +


        <ipython-input-26-e141f1e3624a> in f(x, t, g, l)
          1 def f(x, t, g, l):
          2     '''RHS of pendulum'''
    ----> 3     x1, x2 = x
          4     dx1 = x2
          5     dx2 = - g/l*sin(x1)


        TypeError: 'float' object is not iterable

    \end{Verbatim}

    The reason for this error is the arguments' order convention. Unline
odeint which asks for a callable f(x,t\ldots{}), ode calls for
f(t,x,\ldots{}). We can either define a new function or use a lambda
function to solve this

    \begin{Verbatim}[commandchars=\\\{\}]
{\color{incolor}In [{\color{incolor}29}]:} \PY{n}{solver} \PY{o}{=} \PY{n}{ode}\PY{p}{(}\PY{k}{lambda} \PY{n}{t}\PY{p}{,} \PY{n}{x}\PY{p}{:} \PY{n}{f}\PY{p}{(}\PY{n}{x}\PY{p}{,}\PY{n}{t}\PY{p}{,}\PY{n}{g}\PY{p}{,}\PY{n}{L}\PY{p}{)}\PY{p}{)}\PY{o}{.}\PY{n}{set\PYZus{}initial\PYZus{}value}\PY{p}{(}\PY{n}{x0}\PY{p}{,}\PY{n}{t}\PY{p}{[}\PY{l+m+mi}{0}\PY{p}{]}\PY{p}{)} 
         \PY{c}{\PYZsh{} Note that we eliminated the need for extra arguments}
\end{Verbatim}

    \begin{Verbatim}[commandchars=\\\{\}]
{\color{incolor}In [{\color{incolor}31}]:} \PY{n}{dt} \PY{o}{=} \PY{n}{diff}\PY{p}{(}\PY{n}{t}\PY{p}{)}\PY{p}{[}\PY{l+m+mi}{0}\PY{p}{]}
         \PY{k}{while} \PY{n}{solver}\PY{o}{.}\PY{n}{successful}\PY{p}{(}\PY{p}{)} \PY{o+ow}{and} \PY{n}{solver}\PY{o}{.}\PY{n}{t} \PY{o}{\PYZlt{}} \PY{n}{t}\PY{p}{[}\PY{o}{\PYZhy{}}\PY{l+m+mi}{1}\PY{p}{]}\PY{p}{:}
             \PY{n}{solver}\PY{o}{.}\PY{n}{integrate}\PY{p}{(}\PY{n}{solver}\PY{o}{.}\PY{n}{t}\PY{o}{+}\PY{n}{dt}\PY{p}{)}
\end{Verbatim}

    \begin{Verbatim}[commandchars=\\\{\}]
{\color{incolor}In [{\color{incolor}32}]:} \PY{n}{solver}\PY{o}{.}\PY{n}{y}
\end{Verbatim}

            \begin{Verbatim}[commandchars=\\\{\}]
{\color{outcolor}Out[{\color{outcolor}32}]:} array([ 0.25731756,  3.19230143])
\end{Verbatim}
        
    If we want to store the solution for all time steps, we have to
explicitly store it

    \begin{Verbatim}[commandchars=\\\{\}]
{\color{incolor}In [{\color{incolor}}]:} 
\end{Verbatim}

    \begin{Verbatim}[commandchars=\\\{\}]
{\color{incolor}In [{\color{incolor}}]:} 
\end{Verbatim}

    \begin{Verbatim}[commandchars=\\\{\}]
{\color{incolor}In [{\color{incolor}}]:} 
\end{Verbatim}

    


    \\*\textit{Version Information}


    \begin{Verbatim}[commandchars=\\\{\}]
{\color{incolor}In [{\color{incolor}11}]:} \PY{o}{\PYZpc{}}\PY{k}{load\PYZus{}ext} \PY{n}{version\PYZus{}information}
         \PY{o}{\PYZpc{}}\PY{k}{version\PYZus{}information} \PY{n}{scipy}\PY{p}{,} \PY{n}{matplotlib}
\end{Verbatim}
\texttt{\color{outcolor}Out[{\color{outcolor}11}]:}
    
    \begin{tabular}{|l|l|}\hline
{\bf Software} & {\bf Version} \\ \hline\hline
Python & 2.7.8 |Anaconda 2.1.0 (64-bit)| (default, Aug 21 2014, 18:22:21) [GCC 4.4.7 20120313 (Red Hat 4.4.7-1)] \\ \hline
IPython & 2.3.1 \\ \hline
OS & posix [linux2] \\ \hline
scipy & 0.14.0 \\ \hline
matplotlib & 1.4.2 \\ \hline
\hline \multicolumn{2}{|l|}{Fri Dec 05 10:14:13 2014 CET} \\ \hline
\end{tabular}


    

    \begin{Verbatim}[commandchars=\\\{\}]
{\color{incolor}In [{\color{incolor}1}]:} \PY{k+kn}{from} \PY{n+nn}{IPython.core.display} \PY{k+kn}{import} \PY{n}{HTML}
        \PY{k}{def} \PY{n+nf}{css\PYZus{}styling}\PY{p}{(}\PY{p}{)}\PY{p}{:}
            \PY{n}{styles} \PY{o}{=} \PY{n+nb}{open}\PY{p}{(}\PY{l+s}{\PYZdq{}}\PY{l+s}{./styles/custom.css}\PY{l+s}{\PYZdq{}}\PY{p}{,} \PY{l+s}{\PYZdq{}}\PY{l+s}{r}\PY{l+s}{\PYZdq{}}\PY{p}{)}\PY{o}{.}\PY{n}{read}\PY{p}{(}\PY{p}{)}
            \PY{k}{return} \PY{n}{HTML}\PY{p}{(}\PY{n}{styles}\PY{p}{)}
        \PY{n}{css\PYZus{}styling}\PY{p}{(}\PY{p}{)}
\end{Verbatim}

            \begin{Verbatim}[commandchars=\\\{\}]
{\color{outcolor}Out[{\color{outcolor}1}]:} <IPython.core.display.HTML at 0x7f345c671850>
\end{Verbatim}
        
    \begin{Verbatim}[commandchars=\\\{\}]
{\color{incolor}In [{\color{incolor}}]:} 
\end{Verbatim}


    % Add a bibliography block to the postdoc
    
    
    
    \end{document}


%\newpage
%
% Default to the notebook output style

    


% Inherit from the specified cell style.




    
\documentclass{article}

    
    
    \usepackage{graphicx} % Used to insert images
    \usepackage{adjustbox} % Used to constrain images to a maximum size 
    \usepackage{color} % Allow colors to be defined
    \usepackage{enumerate} % Needed for markdown enumerations to work
    \usepackage{geometry} % Used to adjust the document margins
    \usepackage{amsmath} % Equations
    \usepackage{amssymb} % Equations
    \usepackage[mathletters]{ucs} % Extended unicode (utf-8) support
    \usepackage[utf8x]{inputenc} % Allow utf-8 characters in the tex document
    \usepackage{fancyvrb} % verbatim replacement that allows latex
    \usepackage{grffile} % extends the file name processing of package graphics 
                         % to support a larger range 
    % The hyperref package gives us a pdf with properly built
    % internal navigation ('pdf bookmarks' for the table of contents,
    % internal cross-reference links, web links for URLs, etc.)
    \usepackage{hyperref}
    \usepackage{longtable} % longtable support required by pandoc >1.10
    \usepackage{booktabs}  % table support for pandoc > 1.12.2
    

    
    
    \definecolor{orange}{cmyk}{0,0.4,0.8,0.2}
    \definecolor{darkorange}{rgb}{.71,0.21,0.01}
    \definecolor{darkgreen}{rgb}{.12,.54,.11}
    \definecolor{myteal}{rgb}{.26, .44, .56}
    \definecolor{gray}{gray}{0.45}
    \definecolor{lightgray}{gray}{.95}
    \definecolor{mediumgray}{gray}{.8}
    \definecolor{inputbackground}{rgb}{.95, .95, .85}
    \definecolor{outputbackground}{rgb}{.95, .95, .95}
    \definecolor{traceback}{rgb}{1, .95, .95}
    % ansi colors
    \definecolor{red}{rgb}{.6,0,0}
    \definecolor{green}{rgb}{0,.65,0}
    \definecolor{brown}{rgb}{0.6,0.6,0}
    \definecolor{blue}{rgb}{0,.145,.698}
    \definecolor{purple}{rgb}{.698,.145,.698}
    \definecolor{cyan}{rgb}{0,.698,.698}
    \definecolor{lightgray}{gray}{0.5}
    
    % bright ansi colors
    \definecolor{darkgray}{gray}{0.25}
    \definecolor{lightred}{rgb}{1.0,0.39,0.28}
    \definecolor{lightgreen}{rgb}{0.48,0.99,0.0}
    \definecolor{lightblue}{rgb}{0.53,0.81,0.92}
    \definecolor{lightpurple}{rgb}{0.87,0.63,0.87}
    \definecolor{lightcyan}{rgb}{0.5,1.0,0.83}
    
    % commands and environments needed by pandoc snippets
    % extracted from the output of `pandoc -s`
    \DefineVerbatimEnvironment{Highlighting}{Verbatim}{commandchars=\\\{\}}
    % Add ',fontsize=\small' for more characters per line
    \newenvironment{Shaded}{}{}
    \newcommand{\KeywordTok}[1]{\textcolor[rgb]{0.00,0.44,0.13}{\textbf{{#1}}}}
    \newcommand{\DataTypeTok}[1]{\textcolor[rgb]{0.56,0.13,0.00}{{#1}}}
    \newcommand{\DecValTok}[1]{\textcolor[rgb]{0.25,0.63,0.44}{{#1}}}
    \newcommand{\BaseNTok}[1]{\textcolor[rgb]{0.25,0.63,0.44}{{#1}}}
    \newcommand{\FloatTok}[1]{\textcolor[rgb]{0.25,0.63,0.44}{{#1}}}
    \newcommand{\CharTok}[1]{\textcolor[rgb]{0.25,0.44,0.63}{{#1}}}
    \newcommand{\StringTok}[1]{\textcolor[rgb]{0.25,0.44,0.63}{{#1}}}
    \newcommand{\CommentTok}[1]{\textcolor[rgb]{0.38,0.63,0.69}{\textit{{#1}}}}
    \newcommand{\OtherTok}[1]{\textcolor[rgb]{0.00,0.44,0.13}{{#1}}}
    \newcommand{\AlertTok}[1]{\textcolor[rgb]{1.00,0.00,0.00}{\textbf{{#1}}}}
    \newcommand{\FunctionTok}[1]{\textcolor[rgb]{0.02,0.16,0.49}{{#1}}}
    \newcommand{\RegionMarkerTok}[1]{{#1}}
    \newcommand{\ErrorTok}[1]{\textcolor[rgb]{1.00,0.00,0.00}{\textbf{{#1}}}}
    \newcommand{\NormalTok}[1]{{#1}}
    
    % Define a nice break command that doesn't care if a line doesn't already
    % exist.
    \def\br{\hspace*{\fill} \\* }
    % Math Jax compatability definitions
    \def\gt{>}
    \def\lt{<}
    % Document parameters
    \title{V-Matplotlib}
    
    
    

    % Pygments definitions
    
\makeatletter
\def\PY@reset{\let\PY@it=\relax \let\PY@bf=\relax%
    \let\PY@ul=\relax \let\PY@tc=\relax%
    \let\PY@bc=\relax \let\PY@ff=\relax}
\def\PY@tok#1{\csname PY@tok@#1\endcsname}
\def\PY@toks#1+{\ifx\relax#1\empty\else%
    \PY@tok{#1}\expandafter\PY@toks\fi}
\def\PY@do#1{\PY@bc{\PY@tc{\PY@ul{%
    \PY@it{\PY@bf{\PY@ff{#1}}}}}}}
\def\PY#1#2{\PY@reset\PY@toks#1+\relax+\PY@do{#2}}

\expandafter\def\csname PY@tok@gd\endcsname{\def\PY@tc##1{\textcolor[rgb]{0.63,0.00,0.00}{##1}}}
\expandafter\def\csname PY@tok@gu\endcsname{\let\PY@bf=\textbf\def\PY@tc##1{\textcolor[rgb]{0.50,0.00,0.50}{##1}}}
\expandafter\def\csname PY@tok@gt\endcsname{\def\PY@tc##1{\textcolor[rgb]{0.00,0.27,0.87}{##1}}}
\expandafter\def\csname PY@tok@gs\endcsname{\let\PY@bf=\textbf}
\expandafter\def\csname PY@tok@gr\endcsname{\def\PY@tc##1{\textcolor[rgb]{1.00,0.00,0.00}{##1}}}
\expandafter\def\csname PY@tok@cm\endcsname{\let\PY@it=\textit\def\PY@tc##1{\textcolor[rgb]{0.25,0.50,0.50}{##1}}}
\expandafter\def\csname PY@tok@vg\endcsname{\def\PY@tc##1{\textcolor[rgb]{0.10,0.09,0.49}{##1}}}
\expandafter\def\csname PY@tok@m\endcsname{\def\PY@tc##1{\textcolor[rgb]{0.40,0.40,0.40}{##1}}}
\expandafter\def\csname PY@tok@mh\endcsname{\def\PY@tc##1{\textcolor[rgb]{0.40,0.40,0.40}{##1}}}
\expandafter\def\csname PY@tok@go\endcsname{\def\PY@tc##1{\textcolor[rgb]{0.53,0.53,0.53}{##1}}}
\expandafter\def\csname PY@tok@ge\endcsname{\let\PY@it=\textit}
\expandafter\def\csname PY@tok@vc\endcsname{\def\PY@tc##1{\textcolor[rgb]{0.10,0.09,0.49}{##1}}}
\expandafter\def\csname PY@tok@il\endcsname{\def\PY@tc##1{\textcolor[rgb]{0.40,0.40,0.40}{##1}}}
\expandafter\def\csname PY@tok@cs\endcsname{\let\PY@it=\textit\def\PY@tc##1{\textcolor[rgb]{0.25,0.50,0.50}{##1}}}
\expandafter\def\csname PY@tok@cp\endcsname{\def\PY@tc##1{\textcolor[rgb]{0.74,0.48,0.00}{##1}}}
\expandafter\def\csname PY@tok@gi\endcsname{\def\PY@tc##1{\textcolor[rgb]{0.00,0.63,0.00}{##1}}}
\expandafter\def\csname PY@tok@gh\endcsname{\let\PY@bf=\textbf\def\PY@tc##1{\textcolor[rgb]{0.00,0.00,0.50}{##1}}}
\expandafter\def\csname PY@tok@ni\endcsname{\let\PY@bf=\textbf\def\PY@tc##1{\textcolor[rgb]{0.60,0.60,0.60}{##1}}}
\expandafter\def\csname PY@tok@nl\endcsname{\def\PY@tc##1{\textcolor[rgb]{0.63,0.63,0.00}{##1}}}
\expandafter\def\csname PY@tok@nn\endcsname{\let\PY@bf=\textbf\def\PY@tc##1{\textcolor[rgb]{0.00,0.00,1.00}{##1}}}
\expandafter\def\csname PY@tok@no\endcsname{\def\PY@tc##1{\textcolor[rgb]{0.53,0.00,0.00}{##1}}}
\expandafter\def\csname PY@tok@na\endcsname{\def\PY@tc##1{\textcolor[rgb]{0.49,0.56,0.16}{##1}}}
\expandafter\def\csname PY@tok@nb\endcsname{\def\PY@tc##1{\textcolor[rgb]{0.00,0.50,0.00}{##1}}}
\expandafter\def\csname PY@tok@nc\endcsname{\let\PY@bf=\textbf\def\PY@tc##1{\textcolor[rgb]{0.00,0.00,1.00}{##1}}}
\expandafter\def\csname PY@tok@nd\endcsname{\def\PY@tc##1{\textcolor[rgb]{0.67,0.13,1.00}{##1}}}
\expandafter\def\csname PY@tok@ne\endcsname{\let\PY@bf=\textbf\def\PY@tc##1{\textcolor[rgb]{0.82,0.25,0.23}{##1}}}
\expandafter\def\csname PY@tok@nf\endcsname{\def\PY@tc##1{\textcolor[rgb]{0.00,0.00,1.00}{##1}}}
\expandafter\def\csname PY@tok@si\endcsname{\let\PY@bf=\textbf\def\PY@tc##1{\textcolor[rgb]{0.73,0.40,0.53}{##1}}}
\expandafter\def\csname PY@tok@s2\endcsname{\def\PY@tc##1{\textcolor[rgb]{0.73,0.13,0.13}{##1}}}
\expandafter\def\csname PY@tok@vi\endcsname{\def\PY@tc##1{\textcolor[rgb]{0.10,0.09,0.49}{##1}}}
\expandafter\def\csname PY@tok@nt\endcsname{\let\PY@bf=\textbf\def\PY@tc##1{\textcolor[rgb]{0.00,0.50,0.00}{##1}}}
\expandafter\def\csname PY@tok@nv\endcsname{\def\PY@tc##1{\textcolor[rgb]{0.10,0.09,0.49}{##1}}}
\expandafter\def\csname PY@tok@s1\endcsname{\def\PY@tc##1{\textcolor[rgb]{0.73,0.13,0.13}{##1}}}
\expandafter\def\csname PY@tok@kd\endcsname{\let\PY@bf=\textbf\def\PY@tc##1{\textcolor[rgb]{0.00,0.50,0.00}{##1}}}
\expandafter\def\csname PY@tok@sh\endcsname{\def\PY@tc##1{\textcolor[rgb]{0.73,0.13,0.13}{##1}}}
\expandafter\def\csname PY@tok@sc\endcsname{\def\PY@tc##1{\textcolor[rgb]{0.73,0.13,0.13}{##1}}}
\expandafter\def\csname PY@tok@sx\endcsname{\def\PY@tc##1{\textcolor[rgb]{0.00,0.50,0.00}{##1}}}
\expandafter\def\csname PY@tok@bp\endcsname{\def\PY@tc##1{\textcolor[rgb]{0.00,0.50,0.00}{##1}}}
\expandafter\def\csname PY@tok@c1\endcsname{\let\PY@it=\textit\def\PY@tc##1{\textcolor[rgb]{0.25,0.50,0.50}{##1}}}
\expandafter\def\csname PY@tok@kc\endcsname{\let\PY@bf=\textbf\def\PY@tc##1{\textcolor[rgb]{0.00,0.50,0.00}{##1}}}
\expandafter\def\csname PY@tok@c\endcsname{\let\PY@it=\textit\def\PY@tc##1{\textcolor[rgb]{0.25,0.50,0.50}{##1}}}
\expandafter\def\csname PY@tok@mf\endcsname{\def\PY@tc##1{\textcolor[rgb]{0.40,0.40,0.40}{##1}}}
\expandafter\def\csname PY@tok@err\endcsname{\def\PY@bc##1{\setlength{\fboxsep}{0pt}\fcolorbox[rgb]{1.00,0.00,0.00}{1,1,1}{\strut ##1}}}
\expandafter\def\csname PY@tok@mb\endcsname{\def\PY@tc##1{\textcolor[rgb]{0.40,0.40,0.40}{##1}}}
\expandafter\def\csname PY@tok@ss\endcsname{\def\PY@tc##1{\textcolor[rgb]{0.10,0.09,0.49}{##1}}}
\expandafter\def\csname PY@tok@sr\endcsname{\def\PY@tc##1{\textcolor[rgb]{0.73,0.40,0.53}{##1}}}
\expandafter\def\csname PY@tok@mo\endcsname{\def\PY@tc##1{\textcolor[rgb]{0.40,0.40,0.40}{##1}}}
\expandafter\def\csname PY@tok@kn\endcsname{\let\PY@bf=\textbf\def\PY@tc##1{\textcolor[rgb]{0.00,0.50,0.00}{##1}}}
\expandafter\def\csname PY@tok@mi\endcsname{\def\PY@tc##1{\textcolor[rgb]{0.40,0.40,0.40}{##1}}}
\expandafter\def\csname PY@tok@gp\endcsname{\let\PY@bf=\textbf\def\PY@tc##1{\textcolor[rgb]{0.00,0.00,0.50}{##1}}}
\expandafter\def\csname PY@tok@o\endcsname{\def\PY@tc##1{\textcolor[rgb]{0.40,0.40,0.40}{##1}}}
\expandafter\def\csname PY@tok@kr\endcsname{\let\PY@bf=\textbf\def\PY@tc##1{\textcolor[rgb]{0.00,0.50,0.00}{##1}}}
\expandafter\def\csname PY@tok@s\endcsname{\def\PY@tc##1{\textcolor[rgb]{0.73,0.13,0.13}{##1}}}
\expandafter\def\csname PY@tok@kp\endcsname{\def\PY@tc##1{\textcolor[rgb]{0.00,0.50,0.00}{##1}}}
\expandafter\def\csname PY@tok@w\endcsname{\def\PY@tc##1{\textcolor[rgb]{0.73,0.73,0.73}{##1}}}
\expandafter\def\csname PY@tok@kt\endcsname{\def\PY@tc##1{\textcolor[rgb]{0.69,0.00,0.25}{##1}}}
\expandafter\def\csname PY@tok@ow\endcsname{\let\PY@bf=\textbf\def\PY@tc##1{\textcolor[rgb]{0.67,0.13,1.00}{##1}}}
\expandafter\def\csname PY@tok@sb\endcsname{\def\PY@tc##1{\textcolor[rgb]{0.73,0.13,0.13}{##1}}}
\expandafter\def\csname PY@tok@k\endcsname{\let\PY@bf=\textbf\def\PY@tc##1{\textcolor[rgb]{0.00,0.50,0.00}{##1}}}
\expandafter\def\csname PY@tok@se\endcsname{\let\PY@bf=\textbf\def\PY@tc##1{\textcolor[rgb]{0.73,0.40,0.13}{##1}}}
\expandafter\def\csname PY@tok@sd\endcsname{\let\PY@it=\textit\def\PY@tc##1{\textcolor[rgb]{0.73,0.13,0.13}{##1}}}

\def\PYZbs{\char`\\}
\def\PYZus{\char`\_}
\def\PYZob{\char`\{}
\def\PYZcb{\char`\}}
\def\PYZca{\char`\^}
\def\PYZam{\char`\&}
\def\PYZlt{\char`\<}
\def\PYZgt{\char`\>}
\def\PYZsh{\char`\#}
\def\PYZpc{\char`\%}
\def\PYZdl{\char`\$}
\def\PYZhy{\char`\-}
\def\PYZsq{\char`\'}
\def\PYZdq{\char`\"}
\def\PYZti{\char`\~}
% for compatibility with earlier versions
\def\PYZat{@}
\def\PYZlb{[}
\def\PYZrb{]}
\makeatother


    % Exact colors from NB
    \definecolor{incolor}{rgb}{0.0, 0.0, 0.5}
    \definecolor{outcolor}{rgb}{0.545, 0.0, 0.0}



    
    % Prevent overflowing lines due to hard-to-break entities
    \sloppy 
    % Setup hyperref package
    \hypersetup{
      breaklinks=true,  % so long urls are correctly broken across lines
      colorlinks=true,
      urlcolor=blue,
      linkcolor=darkorange,
      citecolor=darkgreen,
      }
    % Slightly bigger margins than the latex defaults
    
    \geometry{verbose,tmargin=1in,bmargin=1in,lmargin=1in,rmargin=1in}
    
    

    \begin{document}
    
    
    \maketitle
    
    

    
    \section{V-\href{http://matplotlib.org/}{Matplotlib}}\label{v-matplotlib}

    

    \subsubsection{Index}\label{index}

\begin{itemize}
\itemsep1pt\parskip0pt\parsep0pt
\item
  \hyperref[simpleux5fplot]{Simple Plot}
\item
  \hyperref[colormap]{Colormap}
\item
  \hyperref[3d]{3D}
\item
  \hyperref[histogram]{Histogram}
\item
  \hyperref[animation]{Animation}
\end{itemize}

    \begin{Verbatim}[commandchars=\\\{\}]
{\color{incolor}In [{\color{incolor}1}]:} \PY{k+kn}{import} \PY{n+nn}{matplotlib} \PY{k+kn}{as} \PY{n+nn}{mpl}
        \PY{o}{\PYZpc{}}\PY{k}{pylab} \PY{n}{inline}
\end{Verbatim}

    \begin{Verbatim}[commandchars=\\\{\}]
Populating the interactive namespace from numpy and matplotlib
    \end{Verbatim}

    


    \subparagraph{Simple Plot}


    \begin{Verbatim}[commandchars=\\\{\}]
{\color{incolor}In [{\color{incolor}2}]:} \PY{n}{x} \PY{o}{=} \PY{n}{linspace}\PY{p}{(}\PY{l+m+mi}{0}\PY{p}{,}\PY{l+m+mi}{1}\PY{p}{,}\PY{l+m+mi}{100}\PY{p}{)}
        \PY{n}{y} \PY{o}{=} \PY{n}{x}\PY{o}{*}\PY{o}{*}\PY{l+m+mi}{2}
\end{Verbatim}

    \begin{Verbatim}[commandchars=\\\{\}]
{\color{incolor}In [{\color{incolor}3}]:} \PY{n}{plot}\PY{p}{(}\PY{n}{y}\PY{p}{)}
\end{Verbatim}

            \begin{Verbatim}[commandchars=\\\{\}]
{\color{outcolor}Out[{\color{outcolor}3}]:} [<matplotlib.lines.Line2D at 0x7fce14bd80d0>]
\end{Verbatim}
        
    \begin{center}
    \adjustimage{max size={0.9\linewidth}{0.9\paperheight}}{V-Matplotlib_files/V-Matplotlib_7_1.pdf}
    \end{center}
    { \hspace*{\fill} \\}
    
    \begin{Verbatim}[commandchars=\\\{\}]
{\color{incolor}In [{\color{incolor}4}]:} \PY{n}{figure}\PY{p}{(}\PY{p}{)}
        \PY{n}{plot}\PY{p}{(}\PY{n}{x}\PY{p}{,}\PY{n}{y}\PY{p}{)}
        \PY{n}{xlabel}\PY{p}{(}\PY{l+s}{\PYZsq{}}\PY{l+s}{x}\PY{l+s}{\PYZsq{}}\PY{p}{)}
        \PY{n}{ylabel}\PY{p}{(}\PY{l+s}{\PYZsq{}}\PY{l+s}{y}\PY{l+s}{\PYZsq{}}\PY{p}{)}
        \PY{n}{title}\PY{p}{(}\PY{l+s}{\PYZsq{}}\PY{l+s}{Quadratic function}\PY{l+s}{\PYZsq{}}\PY{p}{)}
        \PY{n}{show}\PY{p}{(}\PY{p}{)}
\end{Verbatim}

    \begin{center}
    \adjustimage{max size={0.9\linewidth}{0.9\paperheight}}{V-Matplotlib_files/V-Matplotlib_8_0.pdf}
    \end{center}
    { \hspace*{\fill} \\}
    
    \begin{Verbatim}[commandchars=\\\{\}]
{\color{incolor}In [{\color{incolor}5}]:} \PY{n}{figure}\PY{p}{(}\PY{p}{)}
        \PY{n}{plot}\PY{p}{(}\PY{n}{x}\PY{p}{,}\PY{n}{y}\PY{p}{)}
        \PY{n}{xlabel}\PY{p}{(}\PY{l+s}{\PYZsq{}}\PY{l+s}{\PYZdl{}x\PYZdl{}}\PY{l+s}{\PYZsq{}}\PY{p}{)}
        \PY{n}{ylabel}\PY{p}{(}\PY{l+s}{\PYZsq{}}\PY{l+s}{\PYZdl{}y\PYZdl{}}\PY{l+s}{\PYZsq{}}\PY{p}{)}
        \PY{n}{title}\PY{p}{(}\PY{l+s}{\PYZsq{}}\PY{l+s}{\PYZdl{}y=x\PYZca{}2\PYZdl{}}\PY{l+s}{\PYZsq{}}\PY{p}{)}
        \PY{n}{show}\PY{p}{(}\PY{p}{)}
\end{Verbatim}

    \begin{center}
    \adjustimage{max size={0.9\linewidth}{0.9\paperheight}}{V-Matplotlib_files/V-Matplotlib_9_0.pdf}
    \end{center}
    { \hspace*{\fill} \\}
    
    \begin{Verbatim}[commandchars=\\\{\}]
{\color{incolor}In [{\color{incolor}6}]:} \PY{n}{fig} \PY{o}{=} \PY{n}{plt}\PY{o}{.}\PY{n}{figure}\PY{p}{(}\PY{p}{)}
        
        \PY{n}{axes} \PY{o}{=} \PY{n}{fig}\PY{o}{.}\PY{n}{add\PYZus{}axes}\PY{p}{(}\PY{p}{[}\PY{l+m+mf}{0.1}\PY{p}{,} \PY{l+m+mf}{0.1}\PY{p}{,} \PY{l+m+mf}{0.8}\PY{p}{,} \PY{l+m+mf}{0.8}\PY{p}{]}\PY{p}{)} \PY{c}{\PYZsh{} left, bottom, width, height (range 0 to 1)}
        \PY{n}{axes}\PY{o}{.}\PY{n}{plot}\PY{p}{(}\PY{n}{x}\PY{p}{,} \PY{n}{y}\PY{p}{,} \PY{l+s}{\PYZsq{}}\PY{l+s}{r}\PY{l+s}{\PYZsq{}}\PY{p}{)}
        \PY{n}{axes}\PY{o}{.}\PY{n}{set\PYZus{}xlabel}\PY{p}{(}\PY{l+s}{\PYZsq{}}\PY{l+s}{x}\PY{l+s}{\PYZsq{}}\PY{p}{)}
        \PY{n}{axes}\PY{o}{.}\PY{n}{set\PYZus{}ylabel}\PY{p}{(}\PY{l+s}{\PYZsq{}}\PY{l+s}{y}\PY{l+s}{\PYZsq{}}\PY{p}{)}
        \PY{n}{axes}\PY{o}{.}\PY{n}{set\PYZus{}title}\PY{p}{(}\PY{l+s}{\PYZsq{}}\PY{l+s}{title}\PY{l+s}{\PYZsq{}}\PY{p}{)}\PY{p}{;}
\end{Verbatim}

    \begin{center}
    \adjustimage{max size={0.9\linewidth}{0.9\paperheight}}{V-Matplotlib_files/V-Matplotlib_10_0.pdf}
    \end{center}
    { \hspace*{\fill} \\}
    
    We can change the appearance, similar to MATLAB

    \begin{Verbatim}[commandchars=\\\{\}]
{\color{incolor}In [{\color{incolor}7}]:} \PY{n}{x} \PY{o}{=} \PY{n}{linspace}\PY{p}{(}\PY{l+m+mi}{0}\PY{p}{,}\PY{l+m+mi}{1}\PY{p}{,}\PY{l+m+mi}{10}\PY{p}{)}
        \PY{n}{y1} \PY{o}{=} \PY{n}{x}\PY{o}{*}\PY{o}{*}\PY{l+m+mi}{2}
        \PY{n}{y2} \PY{o}{=} \PY{n}{x}\PY{o}{*}\PY{o}{*}\PY{l+m+mi}{2} \PY{o}{+} \PY{l+m+mf}{0.1}
        \PY{n}{y3} \PY{o}{=} \PY{n}{x}\PY{o}{*}\PY{o}{*}\PY{l+m+mi}{2} \PY{o}{+} \PY{l+m+mf}{0.2}
\end{Verbatim}

    \begin{Verbatim}[commandchars=\\\{\}]
{\color{incolor}In [{\color{incolor}8}]:} \PY{n}{figure}\PY{p}{(}\PY{p}{)}
        \PY{n}{plot}\PY{p}{(}\PY{n}{x}\PY{p}{,} \PY{n}{y1}\PY{p}{,} \PY{n}{color}\PY{o}{=}\PY{l+s}{\PYZsq{}}\PY{l+s}{blue}\PY{l+s}{\PYZsq{}}\PY{p}{,} \PY{n}{linestyle}\PY{o}{=}\PY{l+s}{\PYZsq{}}\PY{l+s}{dashed}\PY{l+s}{\PYZsq{}}\PY{p}{)}
        \PY{n}{plot}\PY{p}{(}\PY{n}{x}\PY{p}{,} \PY{n}{y2}\PY{p}{,} \PY{n}{color}\PY{o}{=}\PY{l+s}{\PYZsq{}}\PY{l+s}{red}\PY{l+s}{\PYZsq{}}\PY{p}{)}
        \PY{n}{plot}\PY{p}{(}\PY{n}{x}\PY{p}{,} \PY{n}{y3}\PY{p}{,} \PY{n}{color}\PY{o}{=}\PY{l+s}{\PYZsq{}}\PY{l+s}{green}\PY{l+s}{\PYZsq{}}\PY{p}{,} \PY{n}{linestyle}\PY{o}{=}\PY{l+s}{\PYZsq{}}\PY{l+s}{.}\PY{l+s}{\PYZsq{}}\PY{p}{,}\PY{n}{marker}\PY{o}{=}\PY{l+s}{\PYZsq{}}\PY{l+s}{*}\PY{l+s}{\PYZsq{}}\PY{p}{)}
        \PY{n}{xlabel}\PY{p}{(}\PY{l+s}{\PYZsq{}}\PY{l+s}{\PYZdl{}x\PYZdl{}}\PY{l+s}{\PYZsq{}}\PY{p}{)}
        \PY{n}{ylabel}\PY{p}{(}\PY{l+s}{\PYZsq{}}\PY{l+s}{\PYZdl{}y\PYZdl{}}\PY{l+s}{\PYZsq{}}\PY{p}{)}
        \PY{n}{title}\PY{p}{(}\PY{l+s}{\PYZsq{}}\PY{l+s}{\PYZdl{}y=x\PYZca{}2\PYZdl{}}\PY{l+s}{\PYZsq{}}\PY{p}{)}
        \PY{n}{show}\PY{p}{(}\PY{p}{)}
\end{Verbatim}

    \begin{center}
    \adjustimage{max size={0.9\linewidth}{0.9\paperheight}}{V-Matplotlib_files/V-Matplotlib_13_0.pdf}
    \end{center}
    { \hspace*{\fill} \\}
    
    \begin{Verbatim}[commandchars=\\\{\}]
{\color{incolor}In [{\color{incolor}9}]:} \PY{n}{figure}\PY{p}{(}\PY{p}{)}
        \PY{n}{plot}\PY{p}{(}\PY{n}{x}\PY{p}{,}\PY{n}{y1}\PY{p}{,}\PY{l+s}{\PYZsq{}}\PY{l+s}{\PYZhy{}\PYZhy{}}\PY{l+s}{\PYZsq{}}\PY{p}{,}\PY{n}{x}\PY{p}{,}\PY{n}{y2}\PY{p}{,}\PY{l+s}{\PYZsq{}}\PY{l+s}{r}\PY{l+s}{\PYZsq{}}\PY{p}{,}\PY{n}{x}\PY{p}{,}\PY{n}{y3}\PY{p}{,}\PY{l+s}{\PYZsq{}}\PY{l+s}{*}\PY{l+s}{\PYZsq{}}\PY{p}{)}
        \PY{n}{xlabel}\PY{p}{(}\PY{l+s}{\PYZsq{}}\PY{l+s}{\PYZdl{}x\PYZdl{}}\PY{l+s}{\PYZsq{}}\PY{p}{)}
        \PY{n}{ylabel}\PY{p}{(}\PY{l+s}{\PYZsq{}}\PY{l+s}{\PYZdl{}y\PYZdl{}}\PY{l+s}{\PYZsq{}}\PY{p}{)}
        \PY{n}{title}\PY{p}{(}\PY{l+s}{\PYZsq{}}\PY{l+s}{\PYZdl{}y=x\PYZca{}2\PYZdl{}}\PY{l+s}{\PYZsq{}}\PY{p}{)}
        \PY{n}{show}\PY{p}{(}\PY{p}{)}
\end{Verbatim}

    \begin{center}
    \adjustimage{max size={0.9\linewidth}{0.9\paperheight}}{V-Matplotlib_files/V-Matplotlib_14_0.pdf}
    \end{center}
    { \hspace*{\fill} \\}
    
    or more definition based

    \begin{Verbatim}[commandchars=\\\{\}]
{\color{incolor}In [{\color{incolor}10}]:} \PY{n}{figure}\PY{p}{(}\PY{p}{)}
         \PY{n}{plot}\PY{p}{(}\PY{n}{x}\PY{p}{,} \PY{n}{y1}\PY{p}{,} \PY{n}{color}\PY{o}{=}\PY{l+s}{\PYZsq{}}\PY{l+s}{green}\PY{l+s}{\PYZsq{}}\PY{p}{,} \PY{n}{linestyle}\PY{o}{=}\PY{l+s}{\PYZsq{}}\PY{l+s}{dashed}\PY{l+s}{\PYZsq{}}\PY{p}{)}
         \PY{n}{plot}\PY{p}{(}\PY{n}{x}\PY{p}{,} \PY{n}{y2}\PY{p}{,} \PY{n}{color}\PY{o}{=}\PY{l+s}{\PYZsq{}}\PY{l+s}{blue}\PY{l+s}{\PYZsq{}}\PY{p}{,} \PY{n}{linestyle}\PY{o}{=}\PY{l+s}{\PYZsq{}}\PY{l+s}{.}\PY{l+s}{\PYZsq{}}\PY{p}{,} \PY{n}{marker}\PY{o}{=}\PY{l+s}{\PYZsq{}}\PY{l+s}{o}\PY{l+s}{\PYZsq{}}\PY{p}{,} \PY{n}{markerfacecolor}\PY{o}{=}\PY{l+s}{\PYZsq{}}\PY{l+s}{blue}\PY{l+s}{\PYZsq{}}\PY{p}{,} \PY{n}{markersize}\PY{o}{=}\PY{l+m+mi}{10}\PY{p}{,} \PY{n}{alpha} \PY{o}{=} \PY{l+m+mf}{0.3}\PY{p}{)}
         \PY{n}{xlabel}\PY{p}{(}\PY{l+s}{\PYZsq{}}\PY{l+s}{\PYZdl{}x\PYZdl{}}\PY{l+s}{\PYZsq{}}\PY{p}{)}
         \PY{n}{ylabel}\PY{p}{(}\PY{l+s}{\PYZsq{}}\PY{l+s}{\PYZdl{}y\PYZdl{}}\PY{l+s}{\PYZsq{}}\PY{p}{)}
         \PY{n}{title}\PY{p}{(}\PY{l+s}{\PYZsq{}}\PY{l+s}{\PYZdl{}y=x\PYZca{}2\PYZdl{}}\PY{l+s}{\PYZsq{}}\PY{p}{)}
         \PY{n}{show}\PY{p}{(}\PY{p}{)}
\end{Verbatim}

    \begin{center}
    \adjustimage{max size={0.9\linewidth}{0.9\paperheight}}{V-Matplotlib_files/V-Matplotlib_16_0.pdf}
    \end{center}
    { \hspace*{\fill} \\}
    
    \begin{Verbatim}[commandchars=\\\{\}]
{\color{incolor}In [{\color{incolor}11}]:} \PY{n}{fig}\PY{p}{,} \PY{n}{ax} \PY{o}{=} \PY{n}{plt}\PY{o}{.}\PY{n}{subplots}\PY{p}{(}\PY{l+m+mi}{2}\PY{p}{,}\PY{l+m+mi}{2}\PY{p}{,}\PY{n}{figsize}\PY{o}{=}\PY{p}{(}\PY{l+m+mi}{12}\PY{p}{,}\PY{l+m+mi}{6}\PY{p}{)}\PY{p}{)}
         
         \PY{n}{ax}\PY{p}{[}\PY{l+m+mi}{0}\PY{p}{,}\PY{l+m+mi}{0}\PY{p}{]}\PY{o}{.}\PY{n}{plot}\PY{p}{(}\PY{n}{x}\PY{p}{,} \PY{n}{x}\PY{o}{+}\PY{l+m+mi}{1}\PY{p}{,} \PY{n}{color}\PY{o}{=}\PY{l+s}{\PYZdq{}}\PY{l+s}{blue}\PY{l+s}{\PYZdq{}}\PY{p}{,} \PY{n}{linewidth}\PY{o}{=}\PY{l+m+mf}{0.25}\PY{p}{)}
         \PY{n}{ax}\PY{p}{[}\PY{l+m+mi}{0}\PY{p}{,}\PY{l+m+mi}{0}\PY{p}{]}\PY{o}{.}\PY{n}{plot}\PY{p}{(}\PY{n}{x}\PY{p}{,} \PY{n}{x}\PY{o}{+}\PY{l+m+mi}{2}\PY{p}{,} \PY{n}{color}\PY{o}{=}\PY{l+s}{\PYZdq{}}\PY{l+s}{blue}\PY{l+s}{\PYZdq{}}\PY{p}{,} \PY{n}{linewidth}\PY{o}{=}\PY{l+m+mf}{0.50}\PY{p}{)}
         \PY{n}{ax}\PY{p}{[}\PY{l+m+mi}{0}\PY{p}{,}\PY{l+m+mi}{0}\PY{p}{]}\PY{o}{.}\PY{n}{plot}\PY{p}{(}\PY{n}{x}\PY{p}{,} \PY{n}{x}\PY{o}{+}\PY{l+m+mi}{3}\PY{p}{,} \PY{n}{color}\PY{o}{=}\PY{l+s}{\PYZdq{}}\PY{l+s}{blue}\PY{l+s}{\PYZdq{}}\PY{p}{,} \PY{n}{linewidth}\PY{o}{=}\PY{l+m+mf}{1.00}\PY{p}{)}    \PY{c}{\PYZsh{} default}
         \PY{n}{ax}\PY{p}{[}\PY{l+m+mi}{0}\PY{p}{,}\PY{l+m+mi}{0}\PY{p}{]}\PY{o}{.}\PY{n}{plot}\PY{p}{(}\PY{n}{x}\PY{p}{,} \PY{n}{x}\PY{o}{+}\PY{l+m+mi}{4}\PY{p}{,} \PY{n}{color}\PY{o}{=}\PY{l+s}{\PYZdq{}}\PY{l+s}{blue}\PY{l+s}{\PYZdq{}}\PY{p}{,} \PY{n}{linewidth}\PY{o}{=}\PY{l+m+mf}{2.00}\PY{p}{)}
         
         \PY{c}{\PYZsh{} possible linestype options ‘\PYZhy{}‘, ‘–’, ‘\PYZhy{}.’, ‘:’, ‘.’, ‘steps’}
         \PY{n}{ax}\PY{p}{[}\PY{l+m+mi}{0}\PY{p}{,}\PY{l+m+mi}{0}\PY{p}{]}\PY{o}{.}\PY{n}{plot}\PY{p}{(}\PY{n}{x}\PY{p}{,} \PY{n}{x}\PY{o}{+}\PY{l+m+mi}{5}\PY{p}{,} \PY{n}{color}\PY{o}{=}\PY{l+s}{\PYZdq{}}\PY{l+s}{red}\PY{l+s}{\PYZdq{}}\PY{p}{,} \PY{n}{lw}\PY{o}{=}\PY{l+m+mi}{2}\PY{p}{,} \PY{n}{linestyle}\PY{o}{=}\PY{l+s}{\PYZsq{}}\PY{l+s}{\PYZhy{}}\PY{l+s}{\PYZsq{}}\PY{p}{)}
         \PY{n}{ax}\PY{p}{[}\PY{l+m+mi}{0}\PY{p}{,}\PY{l+m+mi}{0}\PY{p}{]}\PY{o}{.}\PY{n}{plot}\PY{p}{(}\PY{n}{x}\PY{p}{,} \PY{n}{x}\PY{o}{+}\PY{l+m+mi}{6}\PY{p}{,} \PY{n}{color}\PY{o}{=}\PY{l+s}{\PYZdq{}}\PY{l+s}{red}\PY{l+s}{\PYZdq{}}\PY{p}{,} \PY{n}{lw}\PY{o}{=}\PY{l+m+mi}{2}\PY{p}{,} \PY{n}{ls}\PY{o}{=}\PY{l+s}{\PYZsq{}}\PY{l+s}{\PYZhy{}\PYZhy{}}\PY{l+s}{\PYZsq{}}\PY{p}{)}
         \PY{n}{ax}\PY{p}{[}\PY{l+m+mi}{0}\PY{p}{,}\PY{l+m+mi}{0}\PY{p}{]}\PY{o}{.}\PY{n}{plot}\PY{p}{(}\PY{n}{x}\PY{p}{,} \PY{n}{x}\PY{o}{+}\PY{l+m+mi}{7}\PY{p}{,} \PY{n}{color}\PY{o}{=}\PY{l+s}{\PYZdq{}}\PY{l+s}{red}\PY{l+s}{\PYZdq{}}\PY{p}{,} \PY{n}{lw}\PY{o}{=}\PY{l+m+mi}{2}\PY{p}{,} \PY{n}{ls}\PY{o}{=}\PY{l+s}{\PYZsq{}}\PY{l+s}{\PYZhy{}.}\PY{l+s}{\PYZsq{}}\PY{p}{)}
         \PY{n}{ax}\PY{p}{[}\PY{l+m+mi}{0}\PY{p}{,}\PY{l+m+mi}{0}\PY{p}{]}\PY{o}{.}\PY{n}{plot}\PY{p}{(}\PY{n}{x}\PY{p}{,} \PY{n}{x}\PY{o}{+}\PY{l+m+mi}{8}\PY{p}{,} \PY{n}{color}\PY{o}{=}\PY{l+s}{\PYZdq{}}\PY{l+s}{red}\PY{l+s}{\PYZdq{}}\PY{p}{,} \PY{n}{lw}\PY{o}{=}\PY{l+m+mi}{2}\PY{p}{,} \PY{n}{ls}\PY{o}{=}\PY{l+s}{\PYZsq{}}\PY{l+s}{:}\PY{l+s}{\PYZsq{}}\PY{p}{)}
         \PY{n}{ax}\PY{p}{[}\PY{l+m+mi}{0}\PY{p}{,}\PY{l+m+mi}{0}\PY{p}{]}\PY{o}{.}\PY{n}{plot}\PY{p}{(}\PY{n}{x}\PY{p}{,} \PY{n}{x}\PY{o}{+}\PY{l+m+mi}{9}\PY{p}{,} \PY{n}{color}\PY{o}{=}\PY{l+s}{\PYZdq{}}\PY{l+s}{red}\PY{l+s}{\PYZdq{}}\PY{p}{,} \PY{n}{lw}\PY{o}{=}\PY{l+m+mi}{2}\PY{p}{,} \PY{n}{ls}\PY{o}{=}\PY{l+s}{\PYZsq{}}\PY{l+s}{.}\PY{l+s}{\PYZsq{}}\PY{p}{)}
         
         \PY{c}{\PYZsh{} custom}
         \PY{n}{line}\PY{p}{,} \PY{o}{=} \PY{n}{ax}\PY{p}{[}\PY{l+m+mi}{0}\PY{p}{,}\PY{l+m+mi}{0}\PY{p}{]}\PY{o}{.}\PY{n}{plot}\PY{p}{(}\PY{n}{x}\PY{p}{,} \PY{n}{x}\PY{o}{+}\PY{l+m+mi}{10}\PY{p}{,} \PY{n}{color}\PY{o}{=}\PY{l+s}{\PYZdq{}}\PY{l+s}{black}\PY{l+s}{\PYZdq{}}\PY{p}{,} \PY{n}{lw}\PY{o}{=}\PY{l+m+mf}{1.50}\PY{p}{)}
         \PY{n}{line}\PY{o}{.}\PY{n}{set\PYZus{}dashes}\PY{p}{(}\PY{p}{[}\PY{l+m+mi}{5}\PY{p}{,} \PY{l+m+mi}{10}\PY{p}{,} \PY{l+m+mi}{15}\PY{p}{,} \PY{l+m+mi}{10}\PY{p}{]}\PY{p}{)} \PY{c}{\PYZsh{} format: line length, space length, ...}
         
         \PY{c}{\PYZsh{} Marker Symbols}
         \PY{n}{ax}\PY{p}{[}\PY{l+m+mi}{1}\PY{p}{,}\PY{l+m+mi}{0}\PY{p}{]}\PY{o}{.}\PY{n}{plot}\PY{p}{(}\PY{n}{x}\PY{p}{,} \PY{n}{x}\PY{p}{,} \PY{n}{lw}\PY{o}{=}\PY{l+m+mi}{2}\PY{p}{,} \PY{n}{ls}\PY{o}{=}\PY{l+s}{\PYZsq{}}\PY{l+s}{*}\PY{l+s}{\PYZsq{}}\PY{p}{,} \PY{n}{marker}\PY{o}{=}\PY{l+s}{\PYZsq{}}\PY{l+s}{+}\PY{l+s}{\PYZsq{}}\PY{p}{)}
         \PY{n}{ax}\PY{p}{[}\PY{l+m+mi}{1}\PY{p}{,}\PY{l+m+mi}{0}\PY{p}{]}\PY{o}{.}\PY{n}{plot}\PY{p}{(}\PY{n}{x}\PY{p}{,} \PY{n}{x}\PY{o}{+}\PY{l+m+mi}{1}\PY{p}{,} \PY{n}{lw}\PY{o}{=}\PY{l+m+mi}{2}\PY{p}{,} \PY{n}{ls}\PY{o}{=}\PY{l+s}{\PYZsq{}}\PY{l+s}{*}\PY{l+s}{\PYZsq{}}\PY{p}{,} \PY{n}{marker}\PY{o}{=}\PY{l+s}{\PYZsq{}}\PY{l+s}{o}\PY{l+s}{\PYZsq{}}\PY{p}{)}
         \PY{n}{ax}\PY{p}{[}\PY{l+m+mi}{1}\PY{p}{,}\PY{l+m+mi}{0}\PY{p}{]}\PY{o}{.}\PY{n}{plot}\PY{p}{(}\PY{n}{x}\PY{p}{,} \PY{n}{x}\PY{o}{+}\PY{l+m+mi}{2}\PY{p}{,} \PY{n}{lw}\PY{o}{=}\PY{l+m+mi}{2}\PY{p}{,} \PY{n}{ls}\PY{o}{=}\PY{l+s}{\PYZsq{}}\PY{l+s}{*}\PY{l+s}{\PYZsq{}}\PY{p}{,} \PY{n}{marker}\PY{o}{=}\PY{l+s}{\PYZsq{}}\PY{l+s}{v}\PY{l+s}{\PYZsq{}}\PY{p}{)}
         \PY{n}{ax}\PY{p}{[}\PY{l+m+mi}{1}\PY{p}{,}\PY{l+m+mi}{0}\PY{p}{]}\PY{o}{.}\PY{n}{plot}\PY{p}{(}\PY{n}{x}\PY{p}{,} \PY{n}{x}\PY{o}{+}\PY{l+m+mi}{3}\PY{p}{,} \PY{n}{lw}\PY{o}{=}\PY{l+m+mi}{2}\PY{p}{,} \PY{n}{ls}\PY{o}{=}\PY{l+s}{\PYZsq{}}\PY{l+s}{*}\PY{l+s}{\PYZsq{}}\PY{p}{,} \PY{n}{marker}\PY{o}{=}\PY{l+s}{\PYZsq{}}\PY{l+s}{\PYZca{}}\PY{l+s}{\PYZsq{}}\PY{p}{)}
         \PY{n}{ax}\PY{p}{[}\PY{l+m+mi}{1}\PY{p}{,}\PY{l+m+mi}{0}\PY{p}{]}\PY{o}{.}\PY{n}{plot}\PY{p}{(}\PY{n}{x}\PY{p}{,} \PY{n}{x}\PY{o}{+}\PY{l+m+mi}{4}\PY{p}{,} \PY{n}{lw}\PY{o}{=}\PY{l+m+mi}{2}\PY{p}{,} \PY{n}{ls}\PY{o}{=}\PY{l+s}{\PYZsq{}}\PY{l+s}{*}\PY{l+s}{\PYZsq{}}\PY{p}{,} \PY{n}{marker}\PY{o}{=}\PY{l+s}{\PYZsq{}}\PY{l+s}{\PYZlt{}}\PY{l+s}{\PYZsq{}}\PY{p}{)}
         \PY{n}{ax}\PY{p}{[}\PY{l+m+mi}{1}\PY{p}{,}\PY{l+m+mi}{0}\PY{p}{]}\PY{o}{.}\PY{n}{plot}\PY{p}{(}\PY{n}{x}\PY{p}{,} \PY{n}{x}\PY{o}{+}\PY{l+m+mi}{5}\PY{p}{,} \PY{n}{lw}\PY{o}{=}\PY{l+m+mi}{2}\PY{p}{,} \PY{n}{ls}\PY{o}{=}\PY{l+s}{\PYZsq{}}\PY{l+s}{*}\PY{l+s}{\PYZsq{}}\PY{p}{,} \PY{n}{marker}\PY{o}{=}\PY{l+s}{\PYZsq{}}\PY{l+s}{\PYZgt{}}\PY{l+s}{\PYZsq{}}\PY{p}{)}
         \PY{n}{ax}\PY{p}{[}\PY{l+m+mi}{1}\PY{p}{,}\PY{l+m+mi}{0}\PY{p}{]}\PY{o}{.}\PY{n}{plot}\PY{p}{(}\PY{n}{x}\PY{p}{,} \PY{n}{x}\PY{o}{+}\PY{l+m+mi}{6}\PY{p}{,} \PY{n}{lw}\PY{o}{=}\PY{l+m+mi}{2}\PY{p}{,} \PY{n}{ls}\PY{o}{=}\PY{l+s}{\PYZsq{}}\PY{l+s}{*}\PY{l+s}{\PYZsq{}}\PY{p}{,} \PY{n}{marker}\PY{o}{=}\PY{l+s}{\PYZsq{}}\PY{l+s}{1}\PY{l+s}{\PYZsq{}}\PY{p}{)}
         \PY{n}{ax}\PY{p}{[}\PY{l+m+mi}{1}\PY{p}{,}\PY{l+m+mi}{0}\PY{p}{]}\PY{o}{.}\PY{n}{plot}\PY{p}{(}\PY{n}{x}\PY{p}{,} \PY{n}{x}\PY{o}{+}\PY{l+m+mi}{7}\PY{p}{,} \PY{n}{lw}\PY{o}{=}\PY{l+m+mi}{2}\PY{p}{,} \PY{n}{ls}\PY{o}{=}\PY{l+s}{\PYZsq{}}\PY{l+s}{*}\PY{l+s}{\PYZsq{}}\PY{p}{,} \PY{n}{marker}\PY{o}{=}\PY{l+s}{\PYZsq{}}\PY{l+s}{2}\PY{l+s}{\PYZsq{}}\PY{p}{)}
         \PY{n}{ax}\PY{p}{[}\PY{l+m+mi}{1}\PY{p}{,}\PY{l+m+mi}{0}\PY{p}{]}\PY{o}{.}\PY{n}{plot}\PY{p}{(}\PY{n}{x}\PY{p}{,} \PY{n}{x}\PY{o}{+}\PY{l+m+mi}{8}\PY{p}{,} \PY{n}{lw}\PY{o}{=}\PY{l+m+mi}{2}\PY{p}{,} \PY{n}{ls}\PY{o}{=}\PY{l+s}{\PYZsq{}}\PY{l+s}{*}\PY{l+s}{\PYZsq{}}\PY{p}{,} \PY{n}{marker}\PY{o}{=}\PY{l+s}{\PYZsq{}}\PY{l+s}{3}\PY{l+s}{\PYZsq{}}\PY{p}{)}
         \PY{n}{ax}\PY{p}{[}\PY{l+m+mi}{1}\PY{p}{,}\PY{l+m+mi}{0}\PY{p}{]}\PY{o}{.}\PY{n}{plot}\PY{p}{(}\PY{n}{x}\PY{p}{,} \PY{n}{x}\PY{o}{+}\PY{l+m+mi}{9}\PY{p}{,} \PY{n}{lw}\PY{o}{=}\PY{l+m+mi}{2}\PY{p}{,} \PY{n}{ls}\PY{o}{=}\PY{l+s}{\PYZsq{}}\PY{l+s}{*}\PY{l+s}{\PYZsq{}}\PY{p}{,} \PY{n}{marker}\PY{o}{=}\PY{l+s}{\PYZsq{}}\PY{l+s}{4}\PY{l+s}{\PYZsq{}}\PY{p}{)}
         \PY{n}{ax}\PY{p}{[}\PY{l+m+mi}{1}\PY{p}{,}\PY{l+m+mi}{0}\PY{p}{]}\PY{o}{.}\PY{n}{plot}\PY{p}{(}\PY{n}{x}\PY{p}{,} \PY{n}{x}\PY{o}{+}\PY{l+m+mi}{10}\PY{p}{,} \PY{n}{lw}\PY{o}{=}\PY{l+m+mi}{2}\PY{p}{,} \PY{n}{ls}\PY{o}{=}\PY{l+s}{\PYZsq{}}\PY{l+s}{*}\PY{l+s}{\PYZsq{}}\PY{p}{,} \PY{n}{marker}\PY{o}{=}\PY{l+s}{\PYZsq{}}\PY{l+s}{s}\PY{l+s}{\PYZsq{}}\PY{p}{)}
         \PY{n}{ax}\PY{p}{[}\PY{l+m+mi}{1}\PY{p}{,}\PY{l+m+mi}{0}\PY{p}{]}\PY{o}{.}\PY{n}{plot}\PY{p}{(}\PY{n}{x}\PY{p}{,} \PY{n}{x}\PY{o}{+}\PY{l+m+mi}{11}\PY{p}{,} \PY{n}{lw}\PY{o}{=}\PY{l+m+mi}{2}\PY{p}{,} \PY{n}{ls}\PY{o}{=}\PY{l+s}{\PYZsq{}}\PY{l+s}{*}\PY{l+s}{\PYZsq{}}\PY{p}{,} \PY{n}{marker}\PY{o}{=}\PY{l+s}{\PYZsq{}}\PY{l+s}{p}\PY{l+s}{\PYZsq{}}\PY{p}{)}
         \PY{n}{ax}\PY{p}{[}\PY{l+m+mi}{1}\PY{p}{,}\PY{l+m+mi}{0}\PY{p}{]}\PY{o}{.}\PY{n}{plot}\PY{p}{(}\PY{n}{x}\PY{p}{,} \PY{n}{x}\PY{o}{+}\PY{l+m+mi}{12}\PY{p}{,} \PY{n}{lw}\PY{o}{=}\PY{l+m+mi}{2}\PY{p}{,} \PY{n}{ls}\PY{o}{=}\PY{l+s}{\PYZsq{}}\PY{l+s}{*}\PY{l+s}{\PYZsq{}}\PY{p}{,} \PY{n}{marker}\PY{o}{=}\PY{l+s}{\PYZsq{}}\PY{l+s}{*}\PY{l+s}{\PYZsq{}}\PY{p}{)}
         \PY{n}{ax}\PY{p}{[}\PY{l+m+mi}{1}\PY{p}{,}\PY{l+m+mi}{0}\PY{p}{]}\PY{o}{.}\PY{n}{plot}\PY{p}{(}\PY{n}{x}\PY{p}{,} \PY{n}{x}\PY{o}{+}\PY{l+m+mi}{13}\PY{p}{,} \PY{n}{lw}\PY{o}{=}\PY{l+m+mi}{2}\PY{p}{,} \PY{n}{ls}\PY{o}{=}\PY{l+s}{\PYZsq{}}\PY{l+s}{*}\PY{l+s}{\PYZsq{}}\PY{p}{,} \PY{n}{marker}\PY{o}{=}\PY{l+s}{\PYZsq{}}\PY{l+s}{h}\PY{l+s}{\PYZsq{}}\PY{p}{)}
         \PY{n}{ax}\PY{p}{[}\PY{l+m+mi}{1}\PY{p}{,}\PY{l+m+mi}{0}\PY{p}{]}\PY{o}{.}\PY{n}{plot}\PY{p}{(}\PY{n}{x}\PY{p}{,} \PY{n}{x}\PY{o}{+}\PY{l+m+mi}{14}\PY{p}{,} \PY{n}{lw}\PY{o}{=}\PY{l+m+mi}{2}\PY{p}{,} \PY{n}{ls}\PY{o}{=}\PY{l+s}{\PYZsq{}}\PY{l+s}{*}\PY{l+s}{\PYZsq{}}\PY{p}{,} \PY{n}{marker}\PY{o}{=}\PY{l+s}{\PYZsq{}}\PY{l+s}{H}\PY{l+s}{\PYZsq{}}\PY{p}{)}
         \PY{n}{ax}\PY{p}{[}\PY{l+m+mi}{1}\PY{p}{,}\PY{l+m+mi}{0}\PY{p}{]}\PY{o}{.}\PY{n}{plot}\PY{p}{(}\PY{n}{x}\PY{p}{,} \PY{n}{x}\PY{o}{+}\PY{l+m+mi}{15}\PY{p}{,} \PY{n}{lw}\PY{o}{=}\PY{l+m+mi}{2}\PY{p}{,} \PY{n}{ls}\PY{o}{=}\PY{l+s}{\PYZsq{}}\PY{l+s}{*}\PY{l+s}{\PYZsq{}}\PY{p}{,} \PY{n}{marker}\PY{o}{=}\PY{l+s}{\PYZsq{}}\PY{l+s}{+}\PY{l+s}{\PYZsq{}}\PY{p}{)}
         \PY{n}{ax}\PY{p}{[}\PY{l+m+mi}{1}\PY{p}{,}\PY{l+m+mi}{0}\PY{p}{]}\PY{o}{.}\PY{n}{plot}\PY{p}{(}\PY{n}{x}\PY{p}{,} \PY{n}{x}\PY{o}{+}\PY{l+m+mi}{16}\PY{p}{,} \PY{n}{lw}\PY{o}{=}\PY{l+m+mi}{2}\PY{p}{,} \PY{n}{ls}\PY{o}{=}\PY{l+s}{\PYZsq{}}\PY{l+s}{*}\PY{l+s}{\PYZsq{}}\PY{p}{,} \PY{n}{marker}\PY{o}{=}\PY{l+s}{\PYZsq{}}\PY{l+s}{x}\PY{l+s}{\PYZsq{}}\PY{p}{)}
         \PY{n}{ax}\PY{p}{[}\PY{l+m+mi}{1}\PY{p}{,}\PY{l+m+mi}{0}\PY{p}{]}\PY{o}{.}\PY{n}{plot}\PY{p}{(}\PY{n}{x}\PY{p}{,} \PY{n}{x}\PY{o}{+}\PY{l+m+mi}{17}\PY{p}{,} \PY{n}{lw}\PY{o}{=}\PY{l+m+mi}{2}\PY{p}{,} \PY{n}{ls}\PY{o}{=}\PY{l+s}{\PYZsq{}}\PY{l+s}{*}\PY{l+s}{\PYZsq{}}\PY{p}{,} \PY{n}{marker}\PY{o}{=}\PY{l+s}{\PYZsq{}}\PY{l+s}{D}\PY{l+s}{\PYZsq{}}\PY{p}{)}
         \PY{n}{ax}\PY{p}{[}\PY{l+m+mi}{1}\PY{p}{,}\PY{l+m+mi}{0}\PY{p}{]}\PY{o}{.}\PY{n}{plot}\PY{p}{(}\PY{n}{x}\PY{p}{,} \PY{n}{x}\PY{o}{+}\PY{l+m+mi}{18}\PY{p}{,} \PY{n}{lw}\PY{o}{=}\PY{l+m+mi}{2}\PY{p}{,} \PY{n}{ls}\PY{o}{=}\PY{l+s}{\PYZsq{}}\PY{l+s}{*}\PY{l+s}{\PYZsq{}}\PY{p}{,} \PY{n}{marker}\PY{o}{=}\PY{l+s}{\PYZsq{}}\PY{l+s}{d}\PY{l+s}{\PYZsq{}}\PY{p}{)}
         
         
         \PY{c}{\PYZsh{} marker size and color}
         \PY{n}{ax}\PY{p}{[}\PY{l+m+mi}{0}\PY{p}{,}\PY{l+m+mi}{1}\PY{p}{]}\PY{o}{.}\PY{n}{plot}\PY{p}{(}\PY{n}{x}\PY{p}{,} \PY{n}{x}\PY{o}{+}\PY{l+m+mi}{4}\PY{p}{,} \PY{n}{color}\PY{o}{=}\PY{l+s}{\PYZdq{}}\PY{l+s}{purple}\PY{l+s}{\PYZdq{}}\PY{p}{,} \PY{n}{lw}\PY{o}{=}\PY{l+m+mi}{1}\PY{p}{,} \PY{n}{ls}\PY{o}{=}\PY{l+s}{\PYZsq{}}\PY{l+s}{\PYZhy{}}\PY{l+s}{\PYZsq{}}\PY{p}{,} \PY{n}{marker}\PY{o}{=}\PY{l+s}{\PYZsq{}}\PY{l+s}{o}\PY{l+s}{\PYZsq{}}\PY{p}{,} \PY{n}{markersize}\PY{o}{=}\PY{l+m+mi}{2}\PY{p}{)}
         \PY{n}{ax}\PY{p}{[}\PY{l+m+mi}{0}\PY{p}{,}\PY{l+m+mi}{1}\PY{p}{]}\PY{o}{.}\PY{n}{plot}\PY{p}{(}\PY{n}{x}\PY{p}{,} \PY{n}{x}\PY{o}{+}\PY{l+m+mi}{5}\PY{p}{,} \PY{n}{color}\PY{o}{=}\PY{l+s}{\PYZdq{}}\PY{l+s}{purple}\PY{l+s}{\PYZdq{}}\PY{p}{,} \PY{n}{lw}\PY{o}{=}\PY{l+m+mi}{1}\PY{p}{,} \PY{n}{ls}\PY{o}{=}\PY{l+s}{\PYZsq{}}\PY{l+s}{\PYZhy{}}\PY{l+s}{\PYZsq{}}\PY{p}{,} \PY{n}{marker}\PY{o}{=}\PY{l+s}{\PYZsq{}}\PY{l+s}{o}\PY{l+s}{\PYZsq{}}\PY{p}{,} \PY{n}{markersize}\PY{o}{=}\PY{l+m+mi}{4}\PY{p}{)}
         \PY{n}{ax}\PY{p}{[}\PY{l+m+mi}{0}\PY{p}{,}\PY{l+m+mi}{1}\PY{p}{]}\PY{o}{.}\PY{n}{plot}\PY{p}{(}\PY{n}{x}\PY{p}{,} \PY{n}{x}\PY{o}{+}\PY{l+m+mi}{6}\PY{p}{,} \PY{n}{color}\PY{o}{=}\PY{l+s}{\PYZdq{}}\PY{l+s}{green}\PY{l+s}{\PYZdq{}}\PY{p}{,} \PY{n}{lw}\PY{o}{=}\PY{l+m+mi}{1}\PY{p}{,} \PY{n}{ls}\PY{o}{=}\PY{l+s}{\PYZsq{}}\PY{l+s}{\PYZhy{}}\PY{l+s}{\PYZsq{}}\PY{p}{,} \PY{n}{marker}\PY{o}{=}\PY{l+s}{\PYZsq{}}\PY{l+s}{o}\PY{l+s}{\PYZsq{}}\PY{p}{,} \PY{n}{markersize}\PY{o}{=}\PY{l+m+mi}{8}\PY{p}{,} \PY{n}{markerfacecolor}\PY{o}{=}\PY{l+s}{\PYZdq{}}\PY{l+s}{red}\PY{l+s}{\PYZdq{}}\PY{p}{)}
         \PY{n}{ax}\PY{p}{[}\PY{l+m+mi}{0}\PY{p}{,}\PY{l+m+mi}{1}\PY{p}{]}\PY{o}{.}\PY{n}{plot}\PY{p}{(}\PY{n}{x}\PY{p}{,} \PY{n}{x}\PY{o}{+}\PY{l+m+mi}{7}\PY{p}{,} \PY{n}{color}\PY{o}{=}\PY{l+s}{\PYZdq{}}\PY{l+s}{blue}\PY{l+s}{\PYZdq{}}\PY{p}{,} \PY{n}{lw}\PY{o}{=}\PY{l+m+mi}{1}\PY{p}{,} \PY{n}{ls}\PY{o}{=}\PY{l+s}{\PYZsq{}}\PY{l+s}{\PYZhy{}}\PY{l+s}{\PYZsq{}}\PY{p}{,} \PY{n}{marker}\PY{o}{=}\PY{l+s}{\PYZsq{}}\PY{l+s}{s}\PY{l+s}{\PYZsq{}}\PY{p}{,} \PY{n}{markersize}\PY{o}{=}\PY{l+m+mi}{8}\PY{p}{,} 
                 \PY{n}{markerfacecolor}\PY{o}{=}\PY{l+s}{\PYZdq{}}\PY{l+s}{yellow}\PY{l+s}{\PYZdq{}}\PY{p}{,} \PY{n}{markeredgewidth}\PY{o}{=}\PY{l+m+mi}{2}\PY{p}{,} \PY{n}{markeredgecolor}\PY{o}{=}\PY{l+s}{\PYZdq{}}\PY{l+s}{green}\PY{l+s}{\PYZdq{}}\PY{p}{)}\PY{p}{;}
\end{Verbatim}

    \begin{center}
    \adjustimage{max size={0.9\linewidth}{0.9\paperheight}}{V-Matplotlib_files/V-Matplotlib_17_0.pdf}
    \end{center}
    { \hspace*{\fill} \\}
    
    Subplots

    \begin{Verbatim}[commandchars=\\\{\}]
{\color{incolor}In [{\color{incolor}12}]:} \PY{n}{fig}\PY{p}{,} \PY{n}{ax} \PY{o}{=} \PY{n}{plt}\PY{o}{.}\PY{n}{subplots}\PY{p}{(}\PY{l+m+mi}{2}\PY{p}{,} \PY{l+m+mi}{3}\PY{p}{)}
         \PY{n}{fig}\PY{o}{.}\PY{n}{tight\PYZus{}layout}\PY{p}{(}\PY{p}{)}
\end{Verbatim}

    \begin{center}
    \adjustimage{max size={0.9\linewidth}{0.9\paperheight}}{V-Matplotlib_files/V-Matplotlib_19_0.pdf}
    \end{center}
    { \hspace*{\fill} \\}
    
    


    \subsubsection{Colormap}


    cmap with \_r is a reversed one

    \begin{Verbatim}[commandchars=\\\{\}]
{\color{incolor}In [{\color{incolor}13}]:} \PY{n}{data} \PY{o}{=} \PY{n}{np}\PY{o}{.}\PY{n}{random}\PY{o}{.}\PY{n}{randn}\PY{p}{(}\PY{l+m+mi}{50}\PY{p}{,}\PY{l+m+mi}{50}\PY{p}{)}
         \PY{n}{fig}\PY{p}{,} \PY{n}{ax} \PY{o}{=} \PY{n}{subplots}\PY{p}{(}\PY{l+m+mi}{1}\PY{p}{,}\PY{l+m+mi}{3}\PY{p}{)}
         
         \PY{n}{plt1} \PY{o}{=} \PY{n}{ax}\PY{p}{[}\PY{l+m+mi}{0}\PY{p}{]}\PY{o}{.}\PY{n}{imshow}\PY{p}{(}\PY{n}{data}\PY{p}{,}\PY{n}{cmap}\PY{o}{=}\PY{l+s}{\PYZsq{}}\PY{l+s}{BuGn}\PY{l+s}{\PYZsq{}}\PY{p}{)}
         \PY{n}{fig}\PY{o}{.}\PY{n}{colorbar}\PY{p}{(}\PY{n}{plt1}\PY{p}{,}\PY{n}{ax}\PY{o}{=}\PY{n}{ax}\PY{p}{[}\PY{l+m+mi}{0}\PY{p}{]}\PY{p}{,}\PY{n}{fraction}\PY{o}{=}\PY{l+m+mf}{0.05}\PY{p}{)}
         \PY{n}{plt2} \PY{o}{=} \PY{n}{ax}\PY{p}{[}\PY{l+m+mi}{1}\PY{p}{]}\PY{o}{.}\PY{n}{imshow}\PY{p}{(}\PY{n}{data}\PY{p}{,}\PY{n}{cmap}\PY{o}{=}\PY{l+s}{\PYZsq{}}\PY{l+s}{Reds}\PY{l+s}{\PYZsq{}}\PY{p}{)}
         \PY{n}{fig}\PY{o}{.}\PY{n}{colorbar}\PY{p}{(}\PY{n}{plt2}\PY{p}{,}\PY{n}{ax}\PY{o}{=}\PY{n}{ax}\PY{p}{[}\PY{l+m+mi}{1}\PY{p}{]}\PY{p}{,}\PY{n}{fraction}\PY{o}{=}\PY{l+m+mf}{0.1}\PY{p}{)}
         \PY{n}{plt3} \PY{o}{=} \PY{n}{ax}\PY{p}{[}\PY{l+m+mi}{2}\PY{p}{]}\PY{o}{.}\PY{n}{imshow}\PY{p}{(}\PY{n}{data}\PY{p}{,}\PY{n}{cmap}\PY{o}{=}\PY{l+s}{\PYZsq{}}\PY{l+s}{Blues}\PY{l+s}{\PYZsq{}}\PY{p}{)}
         \PY{n}{fig}\PY{o}{.}\PY{n}{colorbar}\PY{p}{(}\PY{n}{plt3}\PY{p}{,}\PY{n}{ax}\PY{o}{=}\PY{n}{ax}\PY{p}{[}\PY{l+m+mi}{2}\PY{p}{]}\PY{p}{)}
         \PY{n}{fig}\PY{o}{.}\PY{n}{tight\PYZus{}layout}\PY{p}{(}\PY{p}{)}
\end{Verbatim}

    \begin{center}
    \adjustimage{max size={0.9\linewidth}{0.9\paperheight}}{V-Matplotlib_files/V-Matplotlib_23_0.pdf}
    \end{center}
    { \hspace*{\fill} \\}
    
    


    \subsubsection{3D}


    \begin{Verbatim}[commandchars=\\\{\}]
{\color{incolor}In [{\color{incolor}14}]:} \PY{k+kn}{from} \PY{n+nn}{mpl\PYZus{}toolkits.mplot3d} \PY{k+kn}{import} \PY{n}{Axes3D}
\end{Verbatim}

    \begin{Verbatim}[commandchars=\\\{\}]
{\color{incolor}In [{\color{incolor}15}]:} \PY{n}{x} \PY{o}{=} \PY{n}{linspace}\PY{p}{(}\PY{o}{\PYZhy{}}\PY{l+m+mi}{1}\PY{p}{,}\PY{l+m+mi}{1}\PY{p}{,}\PY{l+m+mi}{50}\PY{p}{)}
         \PY{n}{y} \PY{o}{=} \PY{n}{linspace}\PY{p}{(}\PY{o}{\PYZhy{}}\PY{l+m+mi}{1}\PY{p}{,}\PY{l+m+mi}{1}\PY{p}{,}\PY{l+m+mi}{50}\PY{p}{)}
         \PY{n}{X}\PY{p}{,}\PY{n}{Y} \PY{o}{=} \PY{n}{meshgrid}\PY{p}{(}\PY{n}{x}\PY{p}{,}\PY{n}{y}\PY{p}{)}
         \PY{n}{Z} \PY{o}{=} \PY{n}{exp}\PY{p}{(}\PY{o}{\PYZhy{}}\PY{n}{X}\PY{o}{*}\PY{o}{*}\PY{l+m+mi}{2}\PY{o}{\PYZhy{}}\PY{n}{Y}\PY{o}{*}\PY{o}{*}\PY{l+m+mi}{2}\PY{p}{)}
\end{Verbatim}

    \begin{Verbatim}[commandchars=\\\{\}]
{\color{incolor}In [{\color{incolor}16}]:} \PY{n}{fig}\PY{p}{,} \PY{n}{ax} \PY{o}{=} \PY{n}{subplots}\PY{p}{(}\PY{l+m+mi}{2}\PY{p}{,}\PY{l+m+mi}{2}\PY{p}{,}\PY{n}{subplot\PYZus{}kw}\PY{o}{=}\PY{n+nb}{dict}\PY{p}{(}\PY{n}{projection}\PY{o}{=}\PY{l+s}{\PYZsq{}}\PY{l+s}{3d}\PY{l+s}{\PYZsq{}}\PY{p}{)}\PY{p}{)}
         \PY{n}{ax}\PY{p}{[}\PY{l+m+mi}{0}\PY{p}{,}\PY{l+m+mi}{0}\PY{p}{]}\PY{o}{.}\PY{n}{plot\PYZus{}surface}\PY{p}{(}\PY{n}{X}\PY{p}{,}\PY{n}{Y}\PY{p}{,}\PY{n}{Z}\PY{p}{)}
         \PY{n}{ax}\PY{p}{[}\PY{l+m+mi}{0}\PY{p}{,}\PY{l+m+mi}{1}\PY{p}{]}\PY{o}{.}\PY{n}{plot\PYZus{}wireframe}\PY{p}{(}\PY{n}{X}\PY{p}{,}\PY{n}{Y}\PY{p}{,}\PY{n}{Z}\PY{p}{)}
         \PY{n}{ax}\PY{p}{[}\PY{l+m+mi}{1}\PY{p}{,}\PY{l+m+mi}{0}\PY{p}{]}\PY{o}{.}\PY{n}{plot\PYZus{}surface}\PY{p}{(}\PY{n}{X}\PY{p}{,}\PY{n}{Y}\PY{p}{,}\PY{n}{Z}\PY{p}{,}\PY{n}{cmap}\PY{o}{=}\PY{n}{cm}\PY{o}{.}\PY{n}{coolwarm}\PY{p}{,}\PY{n}{rstride}\PY{o}{=}\PY{l+m+mi}{5}\PY{p}{,}\PY{n}{cstride}\PY{o}{=}\PY{l+m+mi}{5}\PY{p}{)}
         \PY{n}{cset} \PY{o}{=} \PY{n}{ax}\PY{p}{[}\PY{l+m+mi}{1}\PY{p}{,}\PY{l+m+mi}{0}\PY{p}{]}\PY{o}{.}\PY{n}{contour}\PY{p}{(}\PY{n}{X}\PY{p}{,} \PY{n}{Y}\PY{p}{,} \PY{n}{Z}\PY{p}{,} \PY{n}{zdir}\PY{o}{=}\PY{l+s}{\PYZsq{}}\PY{l+s}{z}\PY{l+s}{\PYZsq{}}\PY{p}{,} \PY{n}{offset}\PY{o}{=}\PY{o}{\PYZhy{}}\PY{l+m+mf}{0.25}\PY{p}{,} \PY{n}{cmap}\PY{o}{=}\PY{n}{cm}\PY{o}{.}\PY{n}{coolwarm}\PY{p}{)}
         \PY{n}{cset} \PY{o}{=} \PY{n}{ax}\PY{p}{[}\PY{l+m+mi}{1}\PY{p}{,}\PY{l+m+mi}{0}\PY{p}{]}\PY{o}{.}\PY{n}{contour}\PY{p}{(}\PY{n}{X}\PY{p}{,} \PY{n}{Y}\PY{p}{,} \PY{n}{Z}\PY{p}{,} \PY{n}{zdir}\PY{o}{=}\PY{l+s}{\PYZsq{}}\PY{l+s}{x}\PY{l+s}{\PYZsq{}}\PY{p}{,} \PY{n}{offset}\PY{o}{=}\PY{o}{\PYZhy{}}\PY{l+m+mf}{1.5}\PY{p}{,} \PY{n}{cmap}\PY{o}{=}\PY{n}{cm}\PY{o}{.}\PY{n}{coolwarm}\PY{p}{)}
         \PY{n}{cset} \PY{o}{=} \PY{n}{ax}\PY{p}{[}\PY{l+m+mi}{1}\PY{p}{,}\PY{l+m+mi}{0}\PY{p}{]}\PY{o}{.}\PY{n}{contour}\PY{p}{(}\PY{n}{X}\PY{p}{,} \PY{n}{Y}\PY{p}{,} \PY{n}{Z}\PY{p}{,} \PY{n}{zdir}\PY{o}{=}\PY{l+s}{\PYZsq{}}\PY{l+s}{y}\PY{l+s}{\PYZsq{}}\PY{p}{,} \PY{n}{offset}\PY{o}{=}\PY{l+m+mf}{1.5}\PY{p}{,} \PY{n}{cmap}\PY{o}{=}\PY{n}{cm}\PY{o}{.}\PY{n}{coolwarm}\PY{p}{)}
         
         \PY{n}{ax}\PY{p}{[}\PY{l+m+mi}{1}\PY{p}{,}\PY{l+m+mi}{0}\PY{p}{]}\PY{o}{.}\PY{n}{set\PYZus{}xlim}\PY{p}{(}\PY{p}{[}\PY{o}{\PYZhy{}}\PY{l+m+mf}{1.5}\PY{p}{,} \PY{l+m+mi}{1}\PY{p}{]}\PY{p}{)}
         \PY{n}{ax}\PY{p}{[}\PY{l+m+mi}{1}\PY{p}{,}\PY{l+m+mi}{0}\PY{p}{]}\PY{o}{.}\PY{n}{set\PYZus{}ylim}\PY{p}{(}\PY{p}{[}\PY{o}{\PYZhy{}}\PY{l+m+mi}{1}\PY{p}{,} \PY{l+m+mf}{1.5}\PY{p}{]}\PY{p}{)}
         \PY{n}{ax}\PY{p}{[}\PY{l+m+mi}{1}\PY{p}{,}\PY{l+m+mi}{0}\PY{p}{]}\PY{o}{.}\PY{n}{set\PYZus{}zlim}\PY{p}{(}\PY{p}{[}\PY{o}{\PYZhy{}}\PY{l+m+mf}{0.25}\PY{p}{,} \PY{l+m+mf}{1.25}\PY{p}{]}\PY{p}{)}
         
         \PY{n}{ax}\PY{p}{[}\PY{l+m+mi}{1}\PY{p}{,}\PY{l+m+mi}{1}\PY{p}{]}\PY{o}{.}\PY{n}{plot\PYZus{}wireframe}\PY{p}{(}\PY{n}{X}\PY{p}{,}\PY{n}{Y}\PY{p}{,}\PY{n}{Z}\PY{p}{,}\PY{n}{rstride}\PY{o}{=}\PY{l+m+mi}{4}\PY{p}{,}\PY{n}{cstride}\PY{o}{=}\PY{l+m+mi}{4}\PY{p}{)}
\end{Verbatim}

            \begin{Verbatim}[commandchars=\\\{\}]
{\color{outcolor}Out[{\color{outcolor}16}]:} <mpl\_toolkits.mplot3d.art3d.Line3DCollection at 0x7fce12c1ee10>
\end{Verbatim}
        
    \begin{center}
    \adjustimage{max size={0.9\linewidth}{0.9\paperheight}}{V-Matplotlib_files/V-Matplotlib_28_1.pdf}
    \end{center}
    { \hspace*{\fill} \\}
    
    


    \subsubsection{Histogram}


    \begin{Verbatim}[commandchars=\\\{\}]
{\color{incolor}In [{\color{incolor}17}]:} \PY{k+kn}{import} \PY{n+nn}{numpy} \PY{k+kn}{as} \PY{n+nn}{np}
         \PY{k+kn}{import} \PY{n+nn}{matplotlib.pyplot} \PY{k+kn}{as} \PY{n+nn}{plt}
         \PY{k+kn}{from} \PY{n+nn}{matplotlib.ticker} \PY{k+kn}{import} \PY{n}{NullFormatter}
         
         \PY{c}{\PYZsh{} the random data}
         \PY{n}{x} \PY{o}{=} \PY{n}{np}\PY{o}{.}\PY{n}{random}\PY{o}{.}\PY{n}{randn}\PY{p}{(}\PY{l+m+mi}{1000}\PY{p}{)}
         \PY{n}{y} \PY{o}{=} \PY{n}{np}\PY{o}{.}\PY{n}{random}\PY{o}{.}\PY{n}{randn}\PY{p}{(}\PY{l+m+mi}{1000}\PY{p}{)}
         
         \PY{n}{nullfmt}   \PY{o}{=} \PY{n}{NullFormatter}\PY{p}{(}\PY{p}{)}         \PY{c}{\PYZsh{} no labels}
         
         \PY{c}{\PYZsh{} definitions for the axes}
         \PY{n}{left}\PY{p}{,} \PY{n}{width} \PY{o}{=} \PY{l+m+mf}{0.1}\PY{p}{,} \PY{l+m+mf}{0.65}
         \PY{n}{bottom}\PY{p}{,} \PY{n}{height} \PY{o}{=} \PY{l+m+mf}{0.1}\PY{p}{,} \PY{l+m+mf}{0.65}
         \PY{n}{bottom\PYZus{}h} \PY{o}{=} \PY{n}{left\PYZus{}h} \PY{o}{=} \PY{n}{left}\PY{o}{+}\PY{n}{width}\PY{o}{+}\PY{l+m+mf}{0.02}
         
         \PY{n}{rect\PYZus{}scatter} \PY{o}{=} \PY{p}{[}\PY{n}{left}\PY{p}{,} \PY{n}{bottom}\PY{p}{,} \PY{n}{width}\PY{p}{,} \PY{n}{height}\PY{p}{]}
         \PY{n}{rect\PYZus{}histx} \PY{o}{=} \PY{p}{[}\PY{n}{left}\PY{p}{,} \PY{n}{bottom\PYZus{}h}\PY{p}{,} \PY{n}{width}\PY{p}{,} \PY{l+m+mf}{0.2}\PY{p}{]}
         \PY{n}{rect\PYZus{}histy} \PY{o}{=} \PY{p}{[}\PY{n}{left\PYZus{}h}\PY{p}{,} \PY{n}{bottom}\PY{p}{,} \PY{l+m+mf}{0.2}\PY{p}{,} \PY{n}{height}\PY{p}{]}
         
         \PY{c}{\PYZsh{} start with a rectangular Figure}
         \PY{n}{plt}\PY{o}{.}\PY{n}{figure}\PY{p}{(}\PY{l+m+mi}{1}\PY{p}{,} \PY{n}{figsize}\PY{o}{=}\PY{p}{(}\PY{l+m+mi}{8}\PY{p}{,}\PY{l+m+mi}{8}\PY{p}{)}\PY{p}{)}
         
         \PY{n}{axScatter} \PY{o}{=} \PY{n}{plt}\PY{o}{.}\PY{n}{axes}\PY{p}{(}\PY{n}{rect\PYZus{}scatter}\PY{p}{)}
         \PY{n}{axHistx} \PY{o}{=} \PY{n}{plt}\PY{o}{.}\PY{n}{axes}\PY{p}{(}\PY{n}{rect\PYZus{}histx}\PY{p}{)}
         \PY{n}{axHisty} \PY{o}{=} \PY{n}{plt}\PY{o}{.}\PY{n}{axes}\PY{p}{(}\PY{n}{rect\PYZus{}histy}\PY{p}{)}
         
         \PY{c}{\PYZsh{} no labels}
         \PY{n}{axHistx}\PY{o}{.}\PY{n}{xaxis}\PY{o}{.}\PY{n}{set\PYZus{}major\PYZus{}formatter}\PY{p}{(}\PY{n}{nullfmt}\PY{p}{)}
         \PY{n}{axHisty}\PY{o}{.}\PY{n}{yaxis}\PY{o}{.}\PY{n}{set\PYZus{}major\PYZus{}formatter}\PY{p}{(}\PY{n}{nullfmt}\PY{p}{)}
         
         \PY{c}{\PYZsh{} the scatter plot:}
         \PY{n}{axScatter}\PY{o}{.}\PY{n}{scatter}\PY{p}{(}\PY{n}{x}\PY{p}{,} \PY{n}{y}\PY{p}{)}
         
         \PY{c}{\PYZsh{} now determine nice limits by hand:}
         \PY{n}{binwidth} \PY{o}{=} \PY{l+m+mf}{0.25}
         \PY{n}{xymax} \PY{o}{=} \PY{n}{np}\PY{o}{.}\PY{n}{max}\PY{p}{(} \PY{p}{[}\PY{n}{np}\PY{o}{.}\PY{n}{max}\PY{p}{(}\PY{n}{np}\PY{o}{.}\PY{n}{fabs}\PY{p}{(}\PY{n}{x}\PY{p}{)}\PY{p}{)}\PY{p}{,} \PY{n}{np}\PY{o}{.}\PY{n}{max}\PY{p}{(}\PY{n}{np}\PY{o}{.}\PY{n}{fabs}\PY{p}{(}\PY{n}{y}\PY{p}{)}\PY{p}{)}\PY{p}{]} \PY{p}{)}
         \PY{n}{lim} \PY{o}{=} \PY{p}{(} \PY{n+nb}{int}\PY{p}{(}\PY{n}{xymax}\PY{o}{/}\PY{n}{binwidth}\PY{p}{)} \PY{o}{+} \PY{l+m+mi}{1}\PY{p}{)} \PY{o}{*} \PY{n}{binwidth}
         
         \PY{n}{axScatter}\PY{o}{.}\PY{n}{set\PYZus{}xlim}\PY{p}{(} \PY{p}{(}\PY{o}{\PYZhy{}}\PY{n}{lim}\PY{p}{,} \PY{n}{lim}\PY{p}{)} \PY{p}{)}
         \PY{n}{axScatter}\PY{o}{.}\PY{n}{set\PYZus{}ylim}\PY{p}{(} \PY{p}{(}\PY{o}{\PYZhy{}}\PY{n}{lim}\PY{p}{,} \PY{n}{lim}\PY{p}{)} \PY{p}{)}
         
         \PY{n}{bins} \PY{o}{=} \PY{n}{np}\PY{o}{.}\PY{n}{arange}\PY{p}{(}\PY{o}{\PYZhy{}}\PY{n}{lim}\PY{p}{,} \PY{n}{lim} \PY{o}{+} \PY{n}{binwidth}\PY{p}{,} \PY{n}{binwidth}\PY{p}{)}
         \PY{n}{axHistx}\PY{o}{.}\PY{n}{hist}\PY{p}{(}\PY{n}{x}\PY{p}{,} \PY{n}{bins}\PY{o}{=}\PY{n}{bins}\PY{p}{)}
         \PY{n}{axHisty}\PY{o}{.}\PY{n}{hist}\PY{p}{(}\PY{n}{y}\PY{p}{,} \PY{n}{bins}\PY{o}{=}\PY{n}{bins}\PY{p}{,} \PY{n}{orientation}\PY{o}{=}\PY{l+s}{\PYZsq{}}\PY{l+s}{horizontal}\PY{l+s}{\PYZsq{}}\PY{p}{)}
         
         \PY{n}{axHistx}\PY{o}{.}\PY{n}{set\PYZus{}xlim}\PY{p}{(} \PY{n}{axScatter}\PY{o}{.}\PY{n}{get\PYZus{}xlim}\PY{p}{(}\PY{p}{)} \PY{p}{)}
         \PY{n}{axHisty}\PY{o}{.}\PY{n}{set\PYZus{}ylim}\PY{p}{(} \PY{n}{axScatter}\PY{o}{.}\PY{n}{get\PYZus{}ylim}\PY{p}{(}\PY{p}{)} \PY{p}{)}
         
         \PY{n}{plt}\PY{o}{.}\PY{n}{show}\PY{p}{(}\PY{p}{)}
\end{Verbatim}

    \begin{center}
    \adjustimage{max size={0.9\linewidth}{0.9\paperheight}}{V-Matplotlib_files/V-Matplotlib_31_0.pdf}
    \end{center}
    { \hspace*{\fill} \\}
    
    we can run any example from the online gallery
\href{http://matplotlib.org/gallery.html}{examples}

    \begin{Verbatim}[commandchars=\\\{\}]
{\color{incolor}In [{\color{incolor}18}]:} \PY{k+kn}{import} \PY{n+nn}{os}
         \PY{n}{os}\PY{o}{.}\PY{n}{system}\PY{p}{(}\PY{l+s}{\PYZsq{}}\PY{l+s}{wget http://matplotlib.org/mpl\PYZus{}examples/mplot3d/scatter3d\PYZus{}demo.py}\PY{l+s}{\PYZsq{}}\PY{p}{)}
         
         \PY{o}{\PYZpc{}}\PY{k}{run} \PY{l+s}{\PYZsq{}}\PY{l+s}{scatter3d\PYZus{}demo.py}\PY{l+s}{\PYZsq{}}
\end{Verbatim}

    \begin{center}
    \adjustimage{max size={0.9\linewidth}{0.9\paperheight}}{V-Matplotlib_files/V-Matplotlib_33_0.pdf}
    \end{center}
    { \hspace*{\fill} \\}
    
    


    \subsubsection{Animation}


    \begin{Verbatim}[commandchars=\\\{\}]
{\color{incolor}In [{\color{incolor}21}]:} \PY{k+kn}{from} \PY{n+nn}{scipy.integrate} \PY{k+kn}{import} \PY{n}{odeint}
         \PY{k+kn}{from} \PY{n+nn}{matplotlib} \PY{k+kn}{import} \PY{n}{animation}
         
         \PY{n}{g} \PY{o}{=} \PY{l+m+mf}{9.82}\PY{p}{;} \PY{n}{L} \PY{o}{=} \PY{l+m+mf}{0.5}\PY{p}{;} \PY{n}{m} \PY{o}{=} \PY{l+m+mf}{0.1}
         
         \PY{k}{def} \PY{n+nf}{dx}\PY{p}{(}\PY{n}{x}\PY{p}{,} \PY{n}{t}\PY{p}{)}\PY{p}{:}
             \PY{n}{x1}\PY{p}{,} \PY{n}{x2}\PY{p}{,} \PY{n}{x3}\PY{p}{,} \PY{n}{x4} \PY{o}{=} \PY{n}{x}\PY{p}{[}\PY{l+m+mi}{0}\PY{p}{]}\PY{p}{,} \PY{n}{x}\PY{p}{[}\PY{l+m+mi}{1}\PY{p}{]}\PY{p}{,} \PY{n}{x}\PY{p}{[}\PY{l+m+mi}{2}\PY{p}{]}\PY{p}{,} \PY{n}{x}\PY{p}{[}\PY{l+m+mi}{3}\PY{p}{]}
             
             \PY{n}{dx1} \PY{o}{=} \PY{l+m+mf}{6.0}\PY{o}{/}\PY{p}{(}\PY{n}{m}\PY{o}{*}\PY{n}{L}\PY{o}{*}\PY{o}{*}\PY{l+m+mi}{2}\PY{p}{)} \PY{o}{*} \PY{p}{(}\PY{l+m+mi}{2} \PY{o}{*} \PY{n}{x3} \PY{o}{\PYZhy{}} \PY{l+m+mi}{3} \PY{o}{*} \PY{n}{cos}\PY{p}{(}\PY{n}{x1}\PY{o}{\PYZhy{}}\PY{n}{x2}\PY{p}{)} \PY{o}{*} \PY{n}{x4}\PY{p}{)}\PY{o}{/}\PY{p}{(}\PY{l+m+mi}{16} \PY{o}{\PYZhy{}} \PY{l+m+mi}{9} \PY{o}{*} \PY{n}{cos}\PY{p}{(}\PY{n}{x1}\PY{o}{\PYZhy{}}\PY{n}{x2}\PY{p}{)}\PY{o}{*}\PY{o}{*}\PY{l+m+mi}{2}\PY{p}{)}
             \PY{n}{dx2} \PY{o}{=} \PY{l+m+mf}{6.0}\PY{o}{/}\PY{p}{(}\PY{n}{m}\PY{o}{*}\PY{n}{L}\PY{o}{*}\PY{o}{*}\PY{l+m+mi}{2}\PY{p}{)} \PY{o}{*} \PY{p}{(}\PY{l+m+mi}{8} \PY{o}{*} \PY{n}{x4} \PY{o}{\PYZhy{}} \PY{l+m+mi}{3} \PY{o}{*} \PY{n}{cos}\PY{p}{(}\PY{n}{x1}\PY{o}{\PYZhy{}}\PY{n}{x2}\PY{p}{)} \PY{o}{*} \PY{n}{x3}\PY{p}{)}\PY{o}{/}\PY{p}{(}\PY{l+m+mi}{16} \PY{o}{\PYZhy{}} \PY{l+m+mi}{9} \PY{o}{*} \PY{n}{cos}\PY{p}{(}\PY{n}{x1}\PY{o}{\PYZhy{}}\PY{n}{x2}\PY{p}{)}\PY{o}{*}\PY{o}{*}\PY{l+m+mi}{2}\PY{p}{)}
             \PY{n}{dx3} \PY{o}{=} \PY{o}{\PYZhy{}}\PY{l+m+mf}{0.5} \PY{o}{*} \PY{n}{m} \PY{o}{*} \PY{n}{L}\PY{o}{*}\PY{o}{*}\PY{l+m+mi}{2} \PY{o}{*} \PY{p}{(} \PY{n}{dx1} \PY{o}{*} \PY{n}{dx2} \PY{o}{*} \PY{n}{sin}\PY{p}{(}\PY{n}{x1}\PY{o}{\PYZhy{}}\PY{n}{x2}\PY{p}{)} \PY{o}{+} \PY{l+m+mi}{3} \PY{o}{*} \PY{p}{(}\PY{n}{g}\PY{o}{/}\PY{n}{L}\PY{p}{)} \PY{o}{*} \PY{n}{sin}\PY{p}{(}\PY{n}{x1}\PY{p}{)}\PY{p}{)}
             \PY{n}{dx4} \PY{o}{=} \PY{o}{\PYZhy{}}\PY{l+m+mf}{0.5} \PY{o}{*} \PY{n}{m} \PY{o}{*} \PY{n}{L}\PY{o}{*}\PY{o}{*}\PY{l+m+mi}{2} \PY{o}{*} \PY{p}{(}\PY{o}{\PYZhy{}}\PY{n}{dx1} \PY{o}{*} \PY{n}{dx2} \PY{o}{*} \PY{n}{sin}\PY{p}{(}\PY{n}{x1}\PY{o}{\PYZhy{}}\PY{n}{x2}\PY{p}{)} \PY{o}{+} \PY{p}{(}\PY{n}{g}\PY{o}{/}\PY{n}{L}\PY{p}{)} \PY{o}{*} \PY{n}{sin}\PY{p}{(}\PY{n}{x2}\PY{p}{)}\PY{p}{)}
             \PY{k}{return} \PY{p}{[}\PY{n}{dx1}\PY{p}{,} \PY{n}{dx2}\PY{p}{,} \PY{n}{dx3}\PY{p}{,} \PY{n}{dx4}\PY{p}{]}
         
         \PY{n}{x0} \PY{o}{=} \PY{p}{[}\PY{n}{pi}\PY{o}{/}\PY{l+m+mi}{2}\PY{p}{,} \PY{n}{pi}\PY{o}{/}\PY{l+m+mi}{2}\PY{p}{,} \PY{l+m+mi}{0}\PY{p}{,} \PY{l+m+mi}{0}\PY{p}{]}  \PY{c}{\PYZsh{} initial state}
         \PY{n}{t} \PY{o}{=} \PY{n}{linspace}\PY{p}{(}\PY{l+m+mi}{0}\PY{p}{,} \PY{l+m+mi}{10}\PY{p}{,} \PY{l+m+mi}{250}\PY{p}{)} \PY{c}{\PYZsh{} time coordinates}
         \PY{n}{x} \PY{o}{=} \PY{n}{odeint}\PY{p}{(}\PY{n}{dx}\PY{p}{,} \PY{n}{x0}\PY{p}{,} \PY{n}{t}\PY{p}{)}    \PY{c}{\PYZsh{} solve the ODE problem}
         
         \PY{n}{fig}\PY{p}{,} \PY{n}{ax} \PY{o}{=} \PY{n}{plt}\PY{o}{.}\PY{n}{subplots}\PY{p}{(}\PY{n}{figsize}\PY{o}{=}\PY{p}{(}\PY{l+m+mi}{5}\PY{p}{,}\PY{l+m+mi}{5}\PY{p}{)}\PY{p}{)}
         
         \PY{n}{ax}\PY{o}{.}\PY{n}{set\PYZus{}ylim}\PY{p}{(}\PY{p}{[}\PY{o}{\PYZhy{}}\PY{l+m+mf}{1.5}\PY{p}{,} \PY{l+m+mf}{0.5}\PY{p}{]}\PY{p}{)}
         \PY{n}{ax}\PY{o}{.}\PY{n}{set\PYZus{}xlim}\PY{p}{(}\PY{p}{[}\PY{l+m+mi}{1}\PY{p}{,} \PY{o}{\PYZhy{}}\PY{l+m+mi}{1}\PY{p}{]}\PY{p}{)}
         \PY{n}{t} \PY{o}{=} \PY{n}{linspace}\PY{p}{(}\PY{l+m+mi}{0}\PY{p}{,} \PY{l+m+mi}{10}\PY{p}{,} \PY{l+m+mi}{250}\PY{p}{)} \PY{c}{\PYZsh{} time coordinates}
         
         
         \PY{n}{pendulum1}\PY{p}{,} \PY{o}{=} \PY{n}{ax}\PY{o}{.}\PY{n}{plot}\PY{p}{(}\PY{p}{[}\PY{p}{]}\PY{p}{,} \PY{p}{[}\PY{p}{]}\PY{p}{,} \PY{n}{color}\PY{o}{=}\PY{l+s}{\PYZdq{}}\PY{l+s}{red}\PY{l+s}{\PYZdq{}}\PY{p}{,} \PY{n}{lw}\PY{o}{=}\PY{l+m+mi}{2}\PY{p}{)}
         \PY{n}{pendulum2}\PY{p}{,} \PY{o}{=} \PY{n}{ax}\PY{o}{.}\PY{n}{plot}\PY{p}{(}\PY{p}{[}\PY{p}{]}\PY{p}{,} \PY{p}{[}\PY{p}{]}\PY{p}{,} \PY{n}{color}\PY{o}{=}\PY{l+s}{\PYZdq{}}\PY{l+s}{blue}\PY{l+s}{\PYZdq{}}\PY{p}{,} \PY{n}{lw}\PY{o}{=}\PY{l+m+mi}{2}\PY{p}{)}
         
         \PY{k}{def} \PY{n+nf}{init}\PY{p}{(}\PY{p}{)}\PY{p}{:}
             \PY{n}{pendulum1}\PY{o}{.}\PY{n}{set\PYZus{}data}\PY{p}{(}\PY{p}{[}\PY{p}{]}\PY{p}{,} \PY{p}{[}\PY{p}{]}\PY{p}{)}
             \PY{n}{pendulum2}\PY{o}{.}\PY{n}{set\PYZus{}data}\PY{p}{(}\PY{p}{[}\PY{p}{]}\PY{p}{,} \PY{p}{[}\PY{p}{]}\PY{p}{)}
         
         \PY{k}{def} \PY{n+nf}{update}\PY{p}{(}\PY{n}{n}\PY{p}{)}\PY{p}{:} 
             \PY{c}{\PYZsh{} n = frame counter}
             \PY{c}{\PYZsh{} calculate the positions of the pendulums}
             \PY{n}{x1} \PY{o}{=} \PY{o}{+} \PY{n}{L} \PY{o}{*} \PY{n}{sin}\PY{p}{(}\PY{n}{x}\PY{p}{[}\PY{n}{n}\PY{p}{,} \PY{l+m+mi}{0}\PY{p}{]}\PY{p}{)}
             \PY{n}{y1} \PY{o}{=} \PY{o}{\PYZhy{}} \PY{n}{L} \PY{o}{*} \PY{n}{cos}\PY{p}{(}\PY{n}{x}\PY{p}{[}\PY{n}{n}\PY{p}{,} \PY{l+m+mi}{0}\PY{p}{]}\PY{p}{)}
             \PY{n}{x2} \PY{o}{=} \PY{n}{x1} \PY{o}{+} \PY{n}{L} \PY{o}{*} \PY{n}{sin}\PY{p}{(}\PY{n}{x}\PY{p}{[}\PY{n}{n}\PY{p}{,} \PY{l+m+mi}{1}\PY{p}{]}\PY{p}{)}
             \PY{n}{y2} \PY{o}{=} \PY{n}{y1} \PY{o}{\PYZhy{}} \PY{n}{L} \PY{o}{*} \PY{n}{cos}\PY{p}{(}\PY{n}{x}\PY{p}{[}\PY{n}{n}\PY{p}{,} \PY{l+m+mi}{1}\PY{p}{]}\PY{p}{)}
             
             \PY{c}{\PYZsh{} update the line data}
             \PY{n}{pendulum1}\PY{o}{.}\PY{n}{set\PYZus{}data}\PY{p}{(}\PY{p}{[}\PY{l+m+mi}{0} \PY{p}{,}\PY{n}{x1}\PY{p}{]}\PY{p}{,} \PY{p}{[}\PY{l+m+mi}{0} \PY{p}{,}\PY{n}{y1}\PY{p}{]}\PY{p}{)}
             \PY{n}{pendulum2}\PY{o}{.}\PY{n}{set\PYZus{}data}\PY{p}{(}\PY{p}{[}\PY{n}{x1}\PY{p}{,}\PY{n}{x2}\PY{p}{]}\PY{p}{,} \PY{p}{[}\PY{n}{y1}\PY{p}{,}\PY{n}{y2}\PY{p}{]}\PY{p}{)}
         
         \PY{n}{anim} \PY{o}{=} \PY{n}{animation}\PY{o}{.}\PY{n}{FuncAnimation}\PY{p}{(}\PY{n}{fig}\PY{p}{,} \PY{n}{update}\PY{p}{,} \PY{n}{init\PYZus{}func}\PY{o}{=}\PY{n}{init}\PY{p}{,} \PY{n}{frames}\PY{o}{=}\PY{n+nb}{len}\PY{p}{(}\PY{n}{t}\PY{p}{)}\PY{p}{,} \PY{n}{blit}\PY{o}{=}\PY{n+nb+bp}{True}\PY{p}{)}
         \PY{n}{anim}\PY{o}{.}\PY{n}{save}\PY{p}{(}\PY{l+s}{\PYZsq{}}\PY{l+s}{../data/animation.mp4}\PY{l+s}{\PYZsq{}}\PY{p}{,} \PY{n}{fps}\PY{o}{=}\PY{l+m+mi}{20}\PY{p}{)}
         \PY{n}{plt}\PY{o}{.}\PY{n}{close}\PY{p}{(}\PY{n}{fig}\PY{p}{)}
\end{Verbatim}

    \begin{Verbatim}[commandchars=\\\{\}]
{\color{incolor}In [{\color{incolor}22}]:} \PY{k+kn}{from} \PY{n+nn}{IPython.display} \PY{k+kn}{import} \PY{n}{HTML}
         \PY{k+kn}{from} \PY{n+nn}{tempfile} \PY{k+kn}{import} \PY{n}{NamedTemporaryFile}
         \PY{k+kn}{import} \PY{n+nn}{shutil}
         
         \PY{n}{WEBM\PYZus{}VIDEO\PYZus{}TAG} \PY{o}{=} \PY{l+s}{\PYZdq{}\PYZdq{}\PYZdq{}}\PY{l+s}{\PYZlt{}video controls\PYZgt{}}
         \PY{l+s}{ \PYZlt{}source src=}\PY{l+s}{\PYZdq{}}\PY{l+s}{data:video/x\PYZhy{}webm;base64,\PYZob{}0\PYZcb{}}\PY{l+s}{\PYZdq{}}\PY{l+s}{ type=}\PY{l+s}{\PYZdq{}}\PY{l+s}{video/webm}\PY{l+s}{\PYZdq{}}\PY{l+s}{\PYZgt{}}
         \PY{l+s}{ Your browser does not support the video tag.}
         \PY{l+s}{\PYZlt{}/video\PYZgt{}}\PY{l+s}{\PYZdq{}\PYZdq{}\PYZdq{}}
         
         \PY{n}{M4V\PYZus{}VIDEO\PYZus{}TAG} \PY{o}{=} \PY{l+s}{\PYZdq{}\PYZdq{}\PYZdq{}}\PY{l+s}{\PYZlt{}video controls\PYZgt{}}
         \PY{l+s}{ \PYZlt{}source src=}\PY{l+s}{\PYZdq{}}\PY{l+s}{data:video/x\PYZhy{}m4v;base64,\PYZob{}0\PYZcb{}}\PY{l+s}{\PYZdq{}}\PY{l+s}{ type=}\PY{l+s}{\PYZdq{}}\PY{l+s}{video/mp4}\PY{l+s}{\PYZdq{}}\PY{l+s}{\PYZgt{}}
         \PY{l+s}{ Your browser does not support the video tag.}
         \PY{l+s}{\PYZlt{}/video\PYZgt{}}\PY{l+s}{\PYZdq{}\PYZdq{}\PYZdq{}}
         
         \PY{n}{FPS} \PY{o}{=} \PY{l+m+mi}{20}         \PY{c}{\PYZsh{} Frames per second in the generated movie}
         
         \PY{k}{def} \PY{n+nf}{anim\PYZus{}to\PYZus{}html}\PY{p}{(}\PY{n}{anim}\PY{p}{,} \PY{n}{filename}\PY{o}{=}\PY{n+nb+bp}{None}\PY{p}{)}\PY{p}{:}
             \PY{k}{if} \PY{o+ow}{not} \PY{n+nb}{hasattr}\PY{p}{(}\PY{n}{anim}\PY{p}{,} \PY{l+s}{\PYZsq{}}\PY{l+s}{\PYZus{}encoded\PYZus{}video}\PY{l+s}{\PYZsq{}}\PY{p}{)}\PY{p}{:}
                 \PY{k}{with} \PY{n}{NamedTemporaryFile}\PY{p}{(}\PY{n}{suffix}\PY{o}{=}\PY{l+s}{\PYZsq{}}\PY{l+s}{.webm}\PY{l+s}{\PYZsq{}}\PY{p}{)} \PY{k}{as} \PY{n}{f}\PY{p}{:}
                     \PY{n}{webm\PYZus{}writer} \PY{o}{=} \PY{n}{animation}\PY{o}{.}\PY{n}{FFMpegWriter}\PY{p}{(}\PY{n}{fps}\PY{o}{=}\PY{n}{FPS}\PY{p}{,} \PY{n}{codec}\PY{o}{=}\PY{l+s}{\PYZdq{}}\PY{l+s}{libvpx}\PY{l+s}{\PYZdq{}}\PY{p}{)}  \PY{c}{\PYZsh{} you\PYZsq{}ll need libvpx to encode .webm videos}
                     \PY{n}{vpx\PYZus{}args} \PY{o}{=} \PY{p}{[}\PY{l+s}{\PYZdq{}}\PY{l+s}{\PYZhy{}quality}\PY{l+s}{\PYZdq{}}\PY{p}{,} \PY{l+s}{\PYZdq{}}\PY{l+s}{good}\PY{l+s}{\PYZdq{}}\PY{p}{,}    \PY{c}{\PYZsh{} many arguments are not needed in this example, but I left them for reference}
                                 \PY{l+s}{\PYZdq{}}\PY{l+s}{\PYZhy{}cpu\PYZhy{}used}\PY{l+s}{\PYZdq{}}\PY{p}{,} \PY{l+s}{\PYZdq{}}\PY{l+s}{0}\PY{l+s}{\PYZdq{}}\PY{p}{,}
                                 \PY{l+s}{\PYZdq{}}\PY{l+s}{\PYZhy{}b:v}\PY{l+s}{\PYZdq{}}\PY{p}{,} \PY{l+s}{\PYZdq{}}\PY{l+s}{500k}\PY{l+s}{\PYZdq{}}\PY{p}{,}
                                 \PY{l+s}{\PYZdq{}}\PY{l+s}{\PYZhy{}qmin}\PY{l+s}{\PYZdq{}}\PY{p}{,} \PY{l+s}{\PYZdq{}}\PY{l+s}{10}\PY{l+s}{\PYZdq{}}\PY{p}{,}
                                 \PY{l+s}{\PYZdq{}}\PY{l+s}{\PYZhy{}qmax}\PY{l+s}{\PYZdq{}}\PY{p}{,} \PY{l+s}{\PYZdq{}}\PY{l+s}{42}\PY{l+s}{\PYZdq{}}\PY{p}{,}
                                 \PY{l+s}{\PYZdq{}}\PY{l+s}{\PYZhy{}maxrate}\PY{l+s}{\PYZdq{}}\PY{p}{,} \PY{l+s}{\PYZdq{}}\PY{l+s}{500k}\PY{l+s}{\PYZdq{}}\PY{p}{,}
                                 \PY{l+s}{\PYZdq{}}\PY{l+s}{\PYZhy{}bufsize}\PY{l+s}{\PYZdq{}}\PY{p}{,} \PY{l+s}{\PYZdq{}}\PY{l+s}{1000k}\PY{l+s}{\PYZdq{}}\PY{p}{,}
                                 \PY{l+s}{\PYZdq{}}\PY{l+s}{\PYZhy{}threads}\PY{l+s}{\PYZdq{}}\PY{p}{,} \PY{l+s}{\PYZdq{}}\PY{l+s}{4}\PY{l+s}{\PYZdq{}}\PY{p}{,}
                                 \PY{l+s}{\PYZdq{}}\PY{l+s}{\PYZhy{}vf}\PY{l+s}{\PYZdq{}}\PY{p}{,} \PY{l+s}{\PYZdq{}}\PY{l+s}{scale=\PYZhy{}1:240}\PY{l+s}{\PYZdq{}}\PY{p}{,}
                                 \PY{l+s}{\PYZdq{}}\PY{l+s}{\PYZhy{}codec:a}\PY{l+s}{\PYZdq{}}\PY{p}{,} \PY{l+s}{\PYZdq{}}\PY{l+s}{libvorbis}\PY{l+s}{\PYZdq{}}\PY{p}{,}
                                 \PY{l+s}{\PYZdq{}}\PY{l+s}{\PYZhy{}b:a}\PY{l+s}{\PYZdq{}}\PY{p}{,} \PY{l+s}{\PYZdq{}}\PY{l+s}{128k}\PY{l+s}{\PYZdq{}}\PY{p}{]}
                     \PY{n}{anim}\PY{o}{.}\PY{n}{save}\PY{p}{(}\PY{n}{f}\PY{o}{.}\PY{n}{name}\PY{p}{,} \PY{n}{writer}\PY{o}{=}\PY{n}{webm\PYZus{}writer}\PY{p}{,} \PY{n}{extra\PYZus{}args}\PY{o}{=}\PY{n}{vpx\PYZus{}args}\PY{p}{)}
                     \PY{k}{if} \PY{n}{filename} \PY{o+ow}{is} \PY{o+ow}{not} \PY{n+nb+bp}{None}\PY{p}{:}  \PY{c}{\PYZsh{} in case you want to keep a copy of the generated movie}
                         \PY{n}{shutil}\PY{o}{.}\PY{n}{copyfile}\PY{p}{(}\PY{n}{f}\PY{o}{.}\PY{n}{name}\PY{p}{,} \PY{n}{filename}\PY{p}{)}
                     \PY{n}{video} \PY{o}{=} \PY{n+nb}{open}\PY{p}{(}\PY{n}{f}\PY{o}{.}\PY{n}{name}\PY{p}{,} \PY{l+s}{\PYZdq{}}\PY{l+s}{rb}\PY{l+s}{\PYZdq{}}\PY{p}{)}\PY{o}{.}\PY{n}{read}\PY{p}{(}\PY{p}{)}
                 \PY{n}{anim}\PY{o}{.}\PY{n}{\PYZus{}encoded\PYZus{}video} \PY{o}{=} \PY{n}{video}\PY{o}{.}\PY{n}{encode}\PY{p}{(}\PY{l+s}{\PYZdq{}}\PY{l+s}{base64}\PY{l+s}{\PYZdq{}}\PY{p}{)}
                 
             \PY{k}{return} \PY{n}{WEBM\PYZus{}VIDEO\PYZus{}TAG}\PY{o}{.}\PY{n}{format}\PY{p}{(}\PY{n}{anim}\PY{o}{.}\PY{n}{\PYZus{}encoded\PYZus{}video}\PY{p}{)}
         
         \PY{k}{def} \PY{n+nf}{display\PYZus{}animation}\PY{p}{(}\PY{n}{anim}\PY{p}{,} \PY{n}{filename}\PY{p}{)}\PY{p}{:}
             \PY{n}{plt}\PY{o}{.}\PY{n}{close}\PY{p}{(}\PY{n}{anim}\PY{o}{.}\PY{n}{\PYZus{}fig}\PY{p}{)}
             \PY{k}{return} \PY{n}{HTML}\PY{p}{(}\PY{n}{anim\PYZus{}to\PYZus{}html}\PY{p}{(}\PY{n}{anim}\PY{p}{,} \PY{n}{filename}\PY{p}{)}\PY{p}{)}
\end{Verbatim}

    \begin{Verbatim}[commandchars=\\\{\}]
{\color{incolor}In [{\color{incolor}24}]:} \PY{n}{display\PYZus{}animation}\PY{p}{(}\PY{n}{anim}\PY{p}{,}\PY{n}{filename}\PY{o}{=}\PY{l+s}{\PYZsq{}}\PY{l+s}{../data/animation.mp4}\PY{l+s}{\PYZsq{}}\PY{p}{)}
\end{Verbatim}

            \begin{Verbatim}[commandchars=\\\{\}]
{\color{outcolor}Out[{\color{outcolor}24}]:} <IPython.core.display.HTML at 0x7fce11bc5a90>
\end{Verbatim}
        
    \begin{Verbatim}[commandchars=\\\{\}]
{\color{incolor}In [{\color{incolor}25}]:} \PY{o}{\PYZpc{}}\PY{k}{load\PYZus{}ext} \PY{n}{version\PYZus{}information}
         \PY{o}{\PYZpc{}}\PY{k}{version\PYZus{}information} \PY{n}{matplotlib}
\end{Verbatim}
\texttt{\color{outcolor}Out[{\color{outcolor}25}]:}
    
    \begin{tabular}{|l|l|}\hline
{\bf Software} & {\bf Version} \\ \hline\hline
Python & 2.7.8 |Anaconda 2.1.0 (64-bit)| (default, Aug 21 2014, 18:22:21) [GCC 4.4.7 20120313 (Red Hat 4.4.7-1)] \\ \hline
IPython & 2.3.1 \\ \hline
OS & posix [linux2] \\ \hline
matplotlib & 1.4.2 \\ \hline
\hline \multicolumn{2}{|l|}{Fri Dec 05 10:17:37 2014 CET} \\ \hline
\end{tabular}


    

    \begin{Verbatim}[commandchars=\\\{\}]
{\color{incolor}In [{\color{incolor}2}]:} \PY{k+kn}{from} \PY{n+nn}{IPython.core.display} \PY{k+kn}{import} \PY{n}{HTML}
        \PY{k}{def} \PY{n+nf}{css\PYZus{}styling}\PY{p}{(}\PY{p}{)}\PY{p}{:}
            \PY{n}{styles} \PY{o}{=} \PY{n+nb}{open}\PY{p}{(}\PY{l+s}{\PYZdq{}}\PY{l+s}{./styles/custom.css}\PY{l+s}{\PYZdq{}}\PY{p}{,} \PY{l+s}{\PYZdq{}}\PY{l+s}{r}\PY{l+s}{\PYZdq{}}\PY{p}{)}\PY{o}{.}\PY{n}{read}\PY{p}{(}\PY{p}{)}
            \PY{k}{return} \PY{n}{HTML}\PY{p}{(}\PY{n}{styles}\PY{p}{)}
        \PY{n}{css\PYZus{}styling}\PY{p}{(}\PY{p}{)}
\end{Verbatim}

            \begin{Verbatim}[commandchars=\\\{\}]
{\color{outcolor}Out[{\color{outcolor}2}]:} <IPython.core.display.HTML at 0x7f35f0194a90>
\end{Verbatim}
        
    \hyperref[Top]{Back to top}


    % Add a bibliography block to the postdoc
    
    
    
    \end{document}


%\newpage
%
% Default to the notebook output style

    


% Inherit from the specified cell style.




    
\documentclass{article}

    
    
    \usepackage{graphicx} % Used to insert images
    \usepackage{adjustbox} % Used to constrain images to a maximum size 
    \usepackage{color} % Allow colors to be defined
    \usepackage{enumerate} % Needed for markdown enumerations to work
    \usepackage{geometry} % Used to adjust the document margins
    \usepackage{amsmath} % Equations
    \usepackage{amssymb} % Equations
    \usepackage[mathletters]{ucs} % Extended unicode (utf-8) support
    \usepackage[utf8x]{inputenc} % Allow utf-8 characters in the tex document
    \usepackage{fancyvrb} % verbatim replacement that allows latex
    \usepackage{grffile} % extends the file name processing of package graphics 
                         % to support a larger range 
    % The hyperref package gives us a pdf with properly built
    % internal navigation ('pdf bookmarks' for the table of contents,
    % internal cross-reference links, web links for URLs, etc.)
    \usepackage{hyperref}
    \usepackage{longtable} % longtable support required by pandoc >1.10
    \usepackage{booktabs}  % table support for pandoc > 1.12.2
    

    
    
    \definecolor{orange}{cmyk}{0,0.4,0.8,0.2}
    \definecolor{darkorange}{rgb}{.71,0.21,0.01}
    \definecolor{darkgreen}{rgb}{.12,.54,.11}
    \definecolor{myteal}{rgb}{.26, .44, .56}
    \definecolor{gray}{gray}{0.45}
    \definecolor{lightgray}{gray}{.95}
    \definecolor{mediumgray}{gray}{.8}
    \definecolor{inputbackground}{rgb}{.95, .95, .85}
    \definecolor{outputbackground}{rgb}{.95, .95, .95}
    \definecolor{traceback}{rgb}{1, .95, .95}
    % ansi colors
    \definecolor{red}{rgb}{.6,0,0}
    \definecolor{green}{rgb}{0,.65,0}
    \definecolor{brown}{rgb}{0.6,0.6,0}
    \definecolor{blue}{rgb}{0,.145,.698}
    \definecolor{purple}{rgb}{.698,.145,.698}
    \definecolor{cyan}{rgb}{0,.698,.698}
    \definecolor{lightgray}{gray}{0.5}
    
    % bright ansi colors
    \definecolor{darkgray}{gray}{0.25}
    \definecolor{lightred}{rgb}{1.0,0.39,0.28}
    \definecolor{lightgreen}{rgb}{0.48,0.99,0.0}
    \definecolor{lightblue}{rgb}{0.53,0.81,0.92}
    \definecolor{lightpurple}{rgb}{0.87,0.63,0.87}
    \definecolor{lightcyan}{rgb}{0.5,1.0,0.83}
    
    % commands and environments needed by pandoc snippets
    % extracted from the output of `pandoc -s`
    \DefineVerbatimEnvironment{Highlighting}{Verbatim}{commandchars=\\\{\}}
    % Add ',fontsize=\small' for more characters per line
    \newenvironment{Shaded}{}{}
    \newcommand{\KeywordTok}[1]{\textcolor[rgb]{0.00,0.44,0.13}{\textbf{{#1}}}}
    \newcommand{\DataTypeTok}[1]{\textcolor[rgb]{0.56,0.13,0.00}{{#1}}}
    \newcommand{\DecValTok}[1]{\textcolor[rgb]{0.25,0.63,0.44}{{#1}}}
    \newcommand{\BaseNTok}[1]{\textcolor[rgb]{0.25,0.63,0.44}{{#1}}}
    \newcommand{\FloatTok}[1]{\textcolor[rgb]{0.25,0.63,0.44}{{#1}}}
    \newcommand{\CharTok}[1]{\textcolor[rgb]{0.25,0.44,0.63}{{#1}}}
    \newcommand{\StringTok}[1]{\textcolor[rgb]{0.25,0.44,0.63}{{#1}}}
    \newcommand{\CommentTok}[1]{\textcolor[rgb]{0.38,0.63,0.69}{\textit{{#1}}}}
    \newcommand{\OtherTok}[1]{\textcolor[rgb]{0.00,0.44,0.13}{{#1}}}
    \newcommand{\AlertTok}[1]{\textcolor[rgb]{1.00,0.00,0.00}{\textbf{{#1}}}}
    \newcommand{\FunctionTok}[1]{\textcolor[rgb]{0.02,0.16,0.49}{{#1}}}
    \newcommand{\RegionMarkerTok}[1]{{#1}}
    \newcommand{\ErrorTok}[1]{\textcolor[rgb]{1.00,0.00,0.00}{\textbf{{#1}}}}
    \newcommand{\NormalTok}[1]{{#1}}
    
    % Define a nice break command that doesn't care if a line doesn't already
    % exist.
    \def\br{\hspace*{\fill} \\* }
    % Math Jax compatability definitions
    \def\gt{>}
    \def\lt{<}
    % Document parameters
    \title{VI-Pandas}
    
    
    

    % Pygments definitions
    
\makeatletter
\def\PY@reset{\let\PY@it=\relax \let\PY@bf=\relax%
    \let\PY@ul=\relax \let\PY@tc=\relax%
    \let\PY@bc=\relax \let\PY@ff=\relax}
\def\PY@tok#1{\csname PY@tok@#1\endcsname}
\def\PY@toks#1+{\ifx\relax#1\empty\else%
    \PY@tok{#1}\expandafter\PY@toks\fi}
\def\PY@do#1{\PY@bc{\PY@tc{\PY@ul{%
    \PY@it{\PY@bf{\PY@ff{#1}}}}}}}
\def\PY#1#2{\PY@reset\PY@toks#1+\relax+\PY@do{#2}}

\expandafter\def\csname PY@tok@gd\endcsname{\def\PY@tc##1{\textcolor[rgb]{0.63,0.00,0.00}{##1}}}
\expandafter\def\csname PY@tok@gu\endcsname{\let\PY@bf=\textbf\def\PY@tc##1{\textcolor[rgb]{0.50,0.00,0.50}{##1}}}
\expandafter\def\csname PY@tok@gt\endcsname{\def\PY@tc##1{\textcolor[rgb]{0.00,0.27,0.87}{##1}}}
\expandafter\def\csname PY@tok@gs\endcsname{\let\PY@bf=\textbf}
\expandafter\def\csname PY@tok@gr\endcsname{\def\PY@tc##1{\textcolor[rgb]{1.00,0.00,0.00}{##1}}}
\expandafter\def\csname PY@tok@cm\endcsname{\let\PY@it=\textit\def\PY@tc##1{\textcolor[rgb]{0.25,0.50,0.50}{##1}}}
\expandafter\def\csname PY@tok@vg\endcsname{\def\PY@tc##1{\textcolor[rgb]{0.10,0.09,0.49}{##1}}}
\expandafter\def\csname PY@tok@m\endcsname{\def\PY@tc##1{\textcolor[rgb]{0.40,0.40,0.40}{##1}}}
\expandafter\def\csname PY@tok@mh\endcsname{\def\PY@tc##1{\textcolor[rgb]{0.40,0.40,0.40}{##1}}}
\expandafter\def\csname PY@tok@go\endcsname{\def\PY@tc##1{\textcolor[rgb]{0.53,0.53,0.53}{##1}}}
\expandafter\def\csname PY@tok@ge\endcsname{\let\PY@it=\textit}
\expandafter\def\csname PY@tok@vc\endcsname{\def\PY@tc##1{\textcolor[rgb]{0.10,0.09,0.49}{##1}}}
\expandafter\def\csname PY@tok@il\endcsname{\def\PY@tc##1{\textcolor[rgb]{0.40,0.40,0.40}{##1}}}
\expandafter\def\csname PY@tok@cs\endcsname{\let\PY@it=\textit\def\PY@tc##1{\textcolor[rgb]{0.25,0.50,0.50}{##1}}}
\expandafter\def\csname PY@tok@cp\endcsname{\def\PY@tc##1{\textcolor[rgb]{0.74,0.48,0.00}{##1}}}
\expandafter\def\csname PY@tok@gi\endcsname{\def\PY@tc##1{\textcolor[rgb]{0.00,0.63,0.00}{##1}}}
\expandafter\def\csname PY@tok@gh\endcsname{\let\PY@bf=\textbf\def\PY@tc##1{\textcolor[rgb]{0.00,0.00,0.50}{##1}}}
\expandafter\def\csname PY@tok@ni\endcsname{\let\PY@bf=\textbf\def\PY@tc##1{\textcolor[rgb]{0.60,0.60,0.60}{##1}}}
\expandafter\def\csname PY@tok@nl\endcsname{\def\PY@tc##1{\textcolor[rgb]{0.63,0.63,0.00}{##1}}}
\expandafter\def\csname PY@tok@nn\endcsname{\let\PY@bf=\textbf\def\PY@tc##1{\textcolor[rgb]{0.00,0.00,1.00}{##1}}}
\expandafter\def\csname PY@tok@no\endcsname{\def\PY@tc##1{\textcolor[rgb]{0.53,0.00,0.00}{##1}}}
\expandafter\def\csname PY@tok@na\endcsname{\def\PY@tc##1{\textcolor[rgb]{0.49,0.56,0.16}{##1}}}
\expandafter\def\csname PY@tok@nb\endcsname{\def\PY@tc##1{\textcolor[rgb]{0.00,0.50,0.00}{##1}}}
\expandafter\def\csname PY@tok@nc\endcsname{\let\PY@bf=\textbf\def\PY@tc##1{\textcolor[rgb]{0.00,0.00,1.00}{##1}}}
\expandafter\def\csname PY@tok@nd\endcsname{\def\PY@tc##1{\textcolor[rgb]{0.67,0.13,1.00}{##1}}}
\expandafter\def\csname PY@tok@ne\endcsname{\let\PY@bf=\textbf\def\PY@tc##1{\textcolor[rgb]{0.82,0.25,0.23}{##1}}}
\expandafter\def\csname PY@tok@nf\endcsname{\def\PY@tc##1{\textcolor[rgb]{0.00,0.00,1.00}{##1}}}
\expandafter\def\csname PY@tok@si\endcsname{\let\PY@bf=\textbf\def\PY@tc##1{\textcolor[rgb]{0.73,0.40,0.53}{##1}}}
\expandafter\def\csname PY@tok@s2\endcsname{\def\PY@tc##1{\textcolor[rgb]{0.73,0.13,0.13}{##1}}}
\expandafter\def\csname PY@tok@vi\endcsname{\def\PY@tc##1{\textcolor[rgb]{0.10,0.09,0.49}{##1}}}
\expandafter\def\csname PY@tok@nt\endcsname{\let\PY@bf=\textbf\def\PY@tc##1{\textcolor[rgb]{0.00,0.50,0.00}{##1}}}
\expandafter\def\csname PY@tok@nv\endcsname{\def\PY@tc##1{\textcolor[rgb]{0.10,0.09,0.49}{##1}}}
\expandafter\def\csname PY@tok@s1\endcsname{\def\PY@tc##1{\textcolor[rgb]{0.73,0.13,0.13}{##1}}}
\expandafter\def\csname PY@tok@kd\endcsname{\let\PY@bf=\textbf\def\PY@tc##1{\textcolor[rgb]{0.00,0.50,0.00}{##1}}}
\expandafter\def\csname PY@tok@sh\endcsname{\def\PY@tc##1{\textcolor[rgb]{0.73,0.13,0.13}{##1}}}
\expandafter\def\csname PY@tok@sc\endcsname{\def\PY@tc##1{\textcolor[rgb]{0.73,0.13,0.13}{##1}}}
\expandafter\def\csname PY@tok@sx\endcsname{\def\PY@tc##1{\textcolor[rgb]{0.00,0.50,0.00}{##1}}}
\expandafter\def\csname PY@tok@bp\endcsname{\def\PY@tc##1{\textcolor[rgb]{0.00,0.50,0.00}{##1}}}
\expandafter\def\csname PY@tok@c1\endcsname{\let\PY@it=\textit\def\PY@tc##1{\textcolor[rgb]{0.25,0.50,0.50}{##1}}}
\expandafter\def\csname PY@tok@kc\endcsname{\let\PY@bf=\textbf\def\PY@tc##1{\textcolor[rgb]{0.00,0.50,0.00}{##1}}}
\expandafter\def\csname PY@tok@c\endcsname{\let\PY@it=\textit\def\PY@tc##1{\textcolor[rgb]{0.25,0.50,0.50}{##1}}}
\expandafter\def\csname PY@tok@mf\endcsname{\def\PY@tc##1{\textcolor[rgb]{0.40,0.40,0.40}{##1}}}
\expandafter\def\csname PY@tok@err\endcsname{\def\PY@bc##1{\setlength{\fboxsep}{0pt}\fcolorbox[rgb]{1.00,0.00,0.00}{1,1,1}{\strut ##1}}}
\expandafter\def\csname PY@tok@mb\endcsname{\def\PY@tc##1{\textcolor[rgb]{0.40,0.40,0.40}{##1}}}
\expandafter\def\csname PY@tok@ss\endcsname{\def\PY@tc##1{\textcolor[rgb]{0.10,0.09,0.49}{##1}}}
\expandafter\def\csname PY@tok@sr\endcsname{\def\PY@tc##1{\textcolor[rgb]{0.73,0.40,0.53}{##1}}}
\expandafter\def\csname PY@tok@mo\endcsname{\def\PY@tc##1{\textcolor[rgb]{0.40,0.40,0.40}{##1}}}
\expandafter\def\csname PY@tok@kn\endcsname{\let\PY@bf=\textbf\def\PY@tc##1{\textcolor[rgb]{0.00,0.50,0.00}{##1}}}
\expandafter\def\csname PY@tok@mi\endcsname{\def\PY@tc##1{\textcolor[rgb]{0.40,0.40,0.40}{##1}}}
\expandafter\def\csname PY@tok@gp\endcsname{\let\PY@bf=\textbf\def\PY@tc##1{\textcolor[rgb]{0.00,0.00,0.50}{##1}}}
\expandafter\def\csname PY@tok@o\endcsname{\def\PY@tc##1{\textcolor[rgb]{0.40,0.40,0.40}{##1}}}
\expandafter\def\csname PY@tok@kr\endcsname{\let\PY@bf=\textbf\def\PY@tc##1{\textcolor[rgb]{0.00,0.50,0.00}{##1}}}
\expandafter\def\csname PY@tok@s\endcsname{\def\PY@tc##1{\textcolor[rgb]{0.73,0.13,0.13}{##1}}}
\expandafter\def\csname PY@tok@kp\endcsname{\def\PY@tc##1{\textcolor[rgb]{0.00,0.50,0.00}{##1}}}
\expandafter\def\csname PY@tok@w\endcsname{\def\PY@tc##1{\textcolor[rgb]{0.73,0.73,0.73}{##1}}}
\expandafter\def\csname PY@tok@kt\endcsname{\def\PY@tc##1{\textcolor[rgb]{0.69,0.00,0.25}{##1}}}
\expandafter\def\csname PY@tok@ow\endcsname{\let\PY@bf=\textbf\def\PY@tc##1{\textcolor[rgb]{0.67,0.13,1.00}{##1}}}
\expandafter\def\csname PY@tok@sb\endcsname{\def\PY@tc##1{\textcolor[rgb]{0.73,0.13,0.13}{##1}}}
\expandafter\def\csname PY@tok@k\endcsname{\let\PY@bf=\textbf\def\PY@tc##1{\textcolor[rgb]{0.00,0.50,0.00}{##1}}}
\expandafter\def\csname PY@tok@se\endcsname{\let\PY@bf=\textbf\def\PY@tc##1{\textcolor[rgb]{0.73,0.40,0.13}{##1}}}
\expandafter\def\csname PY@tok@sd\endcsname{\let\PY@it=\textit\def\PY@tc##1{\textcolor[rgb]{0.73,0.13,0.13}{##1}}}

\def\PYZbs{\char`\\}
\def\PYZus{\char`\_}
\def\PYZob{\char`\{}
\def\PYZcb{\char`\}}
\def\PYZca{\char`\^}
\def\PYZam{\char`\&}
\def\PYZlt{\char`\<}
\def\PYZgt{\char`\>}
\def\PYZsh{\char`\#}
\def\PYZpc{\char`\%}
\def\PYZdl{\char`\$}
\def\PYZhy{\char`\-}
\def\PYZsq{\char`\'}
\def\PYZdq{\char`\"}
\def\PYZti{\char`\~}
% for compatibility with earlier versions
\def\PYZat{@}
\def\PYZlb{[}
\def\PYZrb{]}
\makeatother


    % Exact colors from NB
    \definecolor{incolor}{rgb}{0.0, 0.0, 0.5}
    \definecolor{outcolor}{rgb}{0.545, 0.0, 0.0}



    
    % Prevent overflowing lines due to hard-to-break entities
    \sloppy 
    % Setup hyperref package
    \hypersetup{
      breaklinks=true,  % so long urls are correctly broken across lines
      colorlinks=true,
      urlcolor=blue,
      linkcolor=darkorange,
      citecolor=darkgreen,
      }
    % Slightly bigger margins than the latex defaults
    
    \geometry{verbose,tmargin=1in,bmargin=1in,lmargin=1in,rmargin=1in}
    
    

    \begin{document}
    
    
    \maketitle
    
    

    
    

    \section{VI-\href{http://pandas.pydata.org}{Pandas}}\label{vi-pandas}

    \subsubsection{Index}\label{index}

\begin{itemize}
\itemsep1pt\parskip0pt\parsep0pt
\item
  \hyperref[timeseries]{Time Series}
\item
  \hyperref[energymarkets]{Energy Markets}
\end{itemize}

    From their website ``pandas is an open source, BSD-licensed library
providing high-performance, easy-to-use data structures and data
analysis tools for the Python programming language.''

    \begin{Verbatim}[commandchars=\\\{\}]
{\color{incolor}In [{\color{incolor}2}]:} \PY{k+kn}{from} \PY{n+nn}{IPython.display} \PY{k+kn}{import} \PY{n}{VimeoVideo}
        
        \PY{n}{VimeoVideo}\PY{p}{(}\PY{l+s}{\PYZsq{}}\PY{l+s}{59324550}\PY{l+s}{\PYZsq{}}\PY{p}{,}\PY{n}{width}\PY{o}{=}\PY{l+m+mi}{900}\PY{p}{,}\PY{n}{height}\PY{o}{=}\PY{l+m+mi}{768}\PY{p}{)}
\end{Verbatim}

            \begin{Verbatim}[commandchars=\\\{\}]
{\color{outcolor}Out[{\color{outcolor}2}]:} <IPython.lib.display.VimeoVideo at 0x7fb44199b610>
\end{Verbatim}
        
    


    \section{Time Series Analysis}


    \begin{Verbatim}[commandchars=\\\{\}]
{\color{incolor}In [{\color{incolor}308}]:} \PY{k+kn}{import} \PY{n+nn}{os}
          
          \PY{k+kn}{from} \PY{n+nn}{pandas} \PY{k+kn}{import} \PY{o}{*}
          \PY{k+kn}{import} \PY{n+nn}{pandas} \PY{k+kn}{as} \PY{n+nn}{pd}
          \PY{k+kn}{import} \PY{n+nn}{pandas.io.data} \PY{k+kn}{as} \PY{n+nn}{web}
          \PY{k+kn}{import} \PY{n+nn}{datetime}
          \PY{k+kn}{import} \PY{n+nn}{matplotlib}
          \PY{k+kn}{import} \PY{n+nn}{matplotlib.pyplot} \PY{k+kn}{as} \PY{n+nn}{plt}
          \PY{k+kn}{import} \PY{n+nn}{statsmodels} \PY{k+kn}{as} \PY{n+nn}{sm}
          \PY{k+kn}{import} \PY{n+nn}{seaborn}
          \PY{n}{seaborn}\PY{o}{.}\PY{n}{set}\PY{p}{(}\PY{p}{)}
          
          \PY{n}{pd}\PY{o}{.}\PY{n}{\PYZus{}\PYZus{}version\PYZus{}\PYZus{}}
\end{Verbatim}

            \begin{Verbatim}[commandchars=\\\{\}]
{\color{outcolor}Out[{\color{outcolor}308}]:} '0.15.0'
\end{Verbatim}
        
    \begin{Verbatim}[commandchars=\\\{\}]
{\color{incolor}In [{\color{incolor}309}]:} \PY{n}{startdate} \PY{o}{=} \PY{n}{datetime}\PY{o}{.}\PY{n}{datetime}\PY{p}{(}\PY{l+m+mi}{2000}\PY{p}{,}\PY{l+m+mi}{1}\PY{p}{,}\PY{l+m+mi}{1}\PY{p}{)}
          \PY{n}{enddate} \PY{o}{=} \PY{n}{datetime}\PY{o}{.}\PY{n}{datetime}\PY{o}{.}\PY{n}{today}\PY{p}{(}\PY{p}{)}
          \PY{n}{df} \PY{o}{=} \PY{n}{web}\PY{o}{.}\PY{n}{DataReader}\PY{p}{(}\PY{p}{[}\PY{l+s}{\PYZsq{}}\PY{l+s}{AAPL}\PY{l+s}{\PYZsq{}}\PY{p}{,}\PY{l+s}{\PYZsq{}}\PY{l+s}{GOOGL}\PY{l+s}{\PYZsq{}}\PY{p}{,}\PY{l+s}{\PYZsq{}}\PY{l+s}{TSLA}\PY{l+s}{\PYZsq{}}\PY{p}{,}\PY{l+s}{\PYZsq{}}\PY{l+s}{YNDX}\PY{l+s}{\PYZsq{}}\PY{p}{]}\PY{p}{,}\PY{l+s}{\PYZsq{}}\PY{l+s}{yahoo}\PY{l+s}{\PYZsq{}}\PY{p}{,}\PY{n}{start}\PY{o}{=}\PY{n}{startdate}\PY{p}{,}\PY{n}{end}\PY{o}{=}\PY{n}{enddate}\PY{p}{)}
\end{Verbatim}

    \begin{Verbatim}[commandchars=\\\{\}]
{\color{incolor}In [{\color{incolor}310}]:} \PY{n}{normvol} \PY{o}{=} \PY{n}{df}\PY{o}{.}\PY{n}{Volume}\PY{o}{/}\PY{n}{df}\PY{o}{.}\PY{n}{Volume}\PY{o}{.}\PY{n}{max}\PY{p}{(}\PY{p}{)}
\end{Verbatim}

    \begin{Verbatim}[commandchars=\\\{\}]
{\color{incolor}In [{\color{incolor}311}]:} \PY{n}{normvol}\PY{o}{.}\PY{n}{plot}\PY{p}{(}\PY{p}{)}
\end{Verbatim}

            \begin{Verbatim}[commandchars=\\\{\}]
{\color{outcolor}Out[{\color{outcolor}311}]:} <matplotlib.axes.\_subplots.AxesSubplot at 0x7f2754675350>
\end{Verbatim}
        
    \begin{Verbatim}[commandchars=\\\{\}]
/home/jpsilva/anaconda/lib/python2.7/site-packages/matplotlib/font\_manager.py:1279: UserWarning: findfont: Font family [u'Arial'] not found. Falling back to Bitstream Vera Sans
  (prop.get\_family(), self.defaultFamily[fontext]))
    \end{Verbatim}

    \begin{center}
    \adjustimage{max size={0.9\linewidth}{0.9\paperheight}}{VI-Pandas_files/VI-Pandas_10_2.pdf}
    \end{center}
    { \hspace*{\fill} \\}
    
    \begin{Verbatim}[commandchars=\\\{\}]
{\color{incolor}In [{\color{incolor}312}]:} \PY{n}{goog} \PY{o}{=} \PY{n}{df}\PY{o}{.}\PY{n}{Close}\PY{o}{.}\PY{n}{GOOGL}\PY{o}{.}\PY{n}{dropna}\PY{p}{(}\PY{p}{)}
          \PY{n}{yndx} \PY{o}{=} \PY{n}{df}\PY{o}{.}\PY{n}{Close}\PY{o}{.}\PY{n}{YNDX}\PY{o}{.}\PY{n}{dropna}\PY{p}{(}\PY{p}{)}
\end{Verbatim}

    \begin{Verbatim}[commandchars=\\\{\}]
{\color{incolor}In [{\color{incolor}313}]:} \PY{n}{googret} \PY{o}{=} \PY{n}{goog}\PY{o}{.}\PY{n}{pct\PYZus{}change}\PY{p}{(}\PY{p}{)}
          \PY{n}{yndxret} \PY{o}{=} \PY{n}{yndx}\PY{o}{.}\PY{n}{pct\PYZus{}change}\PY{p}{(}\PY{p}{)}
\end{Verbatim}

    \begin{Verbatim}[commandchars=\\\{\}]
{\color{incolor}In [{\color{incolor}314}]:} \PY{n}{pd}\PY{o}{.}\PY{n}{rolling\PYZus{}corr}\PY{p}{(}\PY{n}{googret}\PY{p}{,}\PY{n}{yndxret}\PY{p}{,}\PY{l+m+mi}{20}\PY{p}{)}\PY{o}{.}\PY{n}{dropna}\PY{p}{(}\PY{p}{)}\PY{o}{.}\PY{n}{plot}\PY{p}{(}\PY{p}{)}
          \PY{n}{pd}\PY{o}{.}\PY{n}{rolling\PYZus{}corr}\PY{p}{(}\PY{n}{googret}\PY{p}{,}\PY{n}{yndxret}\PY{p}{,}\PY{l+m+mi}{250}\PY{p}{)}\PY{o}{.}\PY{n}{dropna}\PY{p}{(}\PY{p}{)}\PY{o}{.}\PY{n}{plot}\PY{p}{(}\PY{p}{)}
\end{Verbatim}

            \begin{Verbatim}[commandchars=\\\{\}]
{\color{outcolor}Out[{\color{outcolor}314}]:} <matplotlib.axes.\_subplots.AxesSubplot at 0x7f2760d7cad0>
\end{Verbatim}
        
    \begin{center}
    \adjustimage{max size={0.9\linewidth}{0.9\paperheight}}{VI-Pandas_files/VI-Pandas_13_1.pdf}
    \end{center}
    { \hspace*{\fill} \\}
    
    \begin{Verbatim}[commandchars=\\\{\}]
{\color{incolor}In [{\color{incolor}316}]:} \PY{k+kn}{from} \PY{n+nn}{matplotlib.pyplot} \PY{k+kn}{import} \PY{o}{*}
\end{Verbatim}

    \begin{Verbatim}[commandchars=\\\{\}]
{\color{incolor}In [{\color{incolor}318}]:} \PY{n}{goog}\PY{o}{.}\PY{n}{plot}\PY{p}{(}\PY{n}{alpha}\PY{o}{=}\PY{l+m+mf}{0.45}\PY{p}{)}
          \PY{n}{pd}\PY{o}{.}\PY{n}{rolling\PYZus{}mean}\PY{p}{(}\PY{n}{goog}\PY{p}{,}\PY{l+m+mi}{50}\PY{p}{)}\PY{o}{.}\PY{n}{plot}\PY{p}{(}\PY{n}{color}\PY{o}{=}\PY{l+s}{\PYZsq{}}\PY{l+s}{k}\PY{l+s}{\PYZsq{}}\PY{p}{)}
          \PY{n}{pd}\PY{o}{.}\PY{n}{rolling\PYZus{}mean}\PY{p}{(}\PY{n}{goog}\PY{p}{,}\PY{l+m+mi}{250}\PY{p}{)}\PY{o}{.}\PY{n}{plot}\PY{p}{(}\PY{n}{color}\PY{o}{=}\PY{l+s}{\PYZsq{}}\PY{l+s}{k}\PY{l+s}{\PYZsq{}}\PY{p}{,}\PY{n}{linestyle}\PY{o}{=}\PY{l+s}{\PYZsq{}}\PY{l+s}{\PYZhy{}\PYZhy{}}\PY{l+s}{\PYZsq{}}\PY{p}{)}
          \PY{n}{ax} \PY{o}{=} \PY{n}{twinx}\PY{p}{(}\PY{p}{)}
          \PY{n}{pd}\PY{o}{.}\PY{n}{rolling\PYZus{}kurt}\PY{p}{(}\PY{n}{goog}\PY{p}{,}\PY{l+m+mi}{250}\PY{p}{)}\PY{o}{.}\PY{n}{plot}\PY{p}{(}\PY{n}{color}\PY{o}{=}\PY{l+s}{\PYZsq{}}\PY{l+s}{r}\PY{l+s}{\PYZsq{}}\PY{p}{)}
\end{Verbatim}

            \begin{Verbatim}[commandchars=\\\{\}]
{\color{outcolor}Out[{\color{outcolor}318}]:} <matplotlib.axes.\_subplots.AxesSubplot at 0x7f2760a91b10>
\end{Verbatim}
        
    \begin{center}
    \adjustimage{max size={0.9\linewidth}{0.9\paperheight}}{VI-Pandas_files/VI-Pandas_15_1.pdf}
    \end{center}
    { \hspace*{\fill} \\}
    
    


    \section{Energy Markets}


    \begin{Verbatim}[commandchars=\\\{\}]
{\color{incolor}In [{\color{incolor}227}]:} \PY{o}{\PYZpc{}}\PY{k}{matplotlib} \PY{n}{inline}
          \PY{n}{matplotlib}\PY{o}{.}\PY{n}{rcParams}\PY{p}{[}\PY{l+s}{\PYZsq{}}\PY{l+s}{figure.dpi}\PY{l+s}{\PYZsq{}}\PY{p}{]} \PY{o}{=} \PY{l+m+mi}{300}
\end{Verbatim}

    Let's list all the data contained in the folder
\emph{data/sicherung\_eex\_daten/energiespot}

    \begin{Verbatim}[commandchars=\\\{\}]
{\color{incolor}In [{\color{incolor}228}]:} \PY{n}{data\PYZus{}dir} \PY{o}{=} \PY{l+s}{\PYZsq{}}\PY{l+s}{./data/sicherung\PYZus{}eex\PYZus{}daten/energiespot/}\PY{l+s}{\PYZsq{}}
          \PY{k}{for} \PY{n}{filename} \PY{o+ow}{in} \PY{n}{os}\PY{o}{.}\PY{n}{listdir}\PY{p}{(}\PY{n}{data\PYZus{}dir}\PY{p}{)}\PY{p}{:}
              \PY{k}{print} \PY{n}{filename}
\end{Verbatim}

    \begin{Verbatim}[commandchars=\\\{\}]
energy\_spot\_historie\_2010.xls
energy\_spot\_historie\_2005.xls
energy\_spot\_historie\_2003.xls
energy\_intraday\_history\_2009.xls
energy\_spot\_historie\_2012.xls
energy\_spot\_historie\_2008.xls
energy\_intraday\_history\_2007.xls
swiss\_power\_spot\_market\_2011.xls
energy\_intraday\_history\_2006.xls
energy\_spot\_historie\_2006.xls
swiss\_power\_spot\_market\_2008.xls
energy\_intraday\_history\_2010.xls
energy\_spot\_historie\_end\_20020731\_xetra.xls
energy\_spot\_historie\_2004.xls
swiss\_power\_spot\_market\_2009.xls
energy\_intraday\_history\_2012.xls
energy\_intraday\_history\_2011 - Konflikt.xls
swiss\_power\_spot\_market\_2007.xls
Phelix\_Quarterly.xls
energy\_spot\_historie\_2011.xls
energy\_spot\_historie\_2012 - Konflikt.xls
swiss\_power\_spot\_market\_2012.xls
energy\_spot\_historie\_2002.xls
swiss\_power\_spot\_market\_2006.xls
energy\_spot\_historie\_2007.xls
swiss\_power\_spot\_market\_2010.xls
energy\_intraday\_history\_2011.xls
energy\_intraday\_history\_2008.xls
energy\_spot\_historie\_2009.xls
    \end{Verbatim}

    We now read the \emph{xls} file which contains intraday data from 2012
for energy prices. We use the read\_excel method from \emph{pandas} to
read xls files

    \begin{Verbatim}[commandchars=\\\{\}]
{\color{incolor}In [{\color{incolor}229}]:} \PY{n}{df} \PY{o}{=} \PY{n}{pd}\PY{o}{.}\PY{n}{read\PYZus{}excel}\PY{p}{(}\PY{n}{data\PYZus{}dir}\PY{o}{+}\PY{l+s}{\PYZsq{}}\PY{l+s}{energy\PYZus{}intraday\PYZus{}history\PYZus{}2012.xls}\PY{l+s}{\PYZsq{}}\PY{p}{,}\PY{n}{sheetname}\PY{o}{=}\PY{l+s}{\PYZsq{}}\PY{l+s}{Intraday\PYZhy{}Spot}\PY{l+s}{\PYZsq{}}\PY{p}{)}
\end{Verbatim}

    \begin{Verbatim}[commandchars=\\\{\}]
{\color{incolor}In [{\color{incolor}230}]:} \PY{n}{df}\PY{o}{.}\PY{n}{head}\PY{p}{(}\PY{p}{)}
\end{Verbatim}

            \begin{Verbatim}[commandchars=\\\{\}]
{\color{outcolor}Out[{\color{outcolor}230}]:}   EPEX Spot Intraday-Strom-Handel / EPEX Spot Intraday-Energy-Trading  \textbackslash{}
          0                                       Delivery Day                    
          1                                2012-12-27 00:00:00                    
          2                                2012-12-27 00:00:00                    
          3                                2012-12-27 00:00:00                    
          4                                2012-12-27 00:00:00                    
          
             Unnamed: 1 Unnamed: 2  Unnamed: 3        Unnamed: 4      Unnamed: 5  \textbackslash{}
          0  Hour\textbackslash{}nfrom   Hour\textbackslash{}nto  Volume\textbackslash{}nMW  Volume (OTC)\textbackslash{}nMW  Low Price\textbackslash{}nEUR   
          1       23:00      00:00       968.5               NaN               1   
          2       22:00      23:00      1640.2               NaN               1   
          3       21:00      22:00      1072.3               NaN               1   
          4       20:00      21:00      1011.3               NaN               1   
          
                  Unnamed: 6       Unnamed: 7          Unnamed: 8  
          0  High Price\textbackslash{}nEUR  Last Price\textbackslash{}nEUR  Average Price\textbackslash{}nEUR  
          1               35               12               21.11  
          2               45               25               30.16  
          3             42.5               11               27.41  
          4               43               26               35.96  
\end{Verbatim}
        
    \begin{Verbatim}[commandchars=\\\{\}]
{\color{incolor}In [{\color{incolor}231}]:} \PY{n}{df} \PY{o}{=} \PY{n}{pd}\PY{o}{.}\PY{n}{read\PYZus{}excel}\PY{p}{(}\PY{n}{data\PYZus{}dir}\PY{o}{+}\PY{l+s}{\PYZsq{}}\PY{l+s}{energy\PYZus{}intraday\PYZus{}history\PYZus{}2012.xls}\PY{l+s}{\PYZsq{}}\PY{p}{,}\PY{n}{sheetname}\PY{o}{=}\PY{l+s}{\PYZsq{}}\PY{l+s}{Intraday\PYZhy{}Spot}\PY{l+s}{\PYZsq{}}\PY{p}{,}\PY{n}{header}\PY{o}{=}\PY{l+m+mi}{1}\PY{p}{,}\PY{n}{index\PYZus{}col}\PY{o}{=}\PY{l+m+mi}{0}\PY{p}{)}        
\end{Verbatim}

    \begin{Verbatim}[commandchars=\\\{\}]
{\color{incolor}In [{\color{incolor}232}]:} \PY{n}{df}\PY{o}{.}\PY{n}{head}\PY{p}{(}\PY{p}{)}
\end{Verbatim}

            \begin{Verbatim}[commandchars=\\\{\}]
{\color{outcolor}Out[{\color{outcolor}232}]:}              Hour\textbackslash{}nfrom Hour\textbackslash{}nto  Volume\textbackslash{}nMW  Volume (OTC)\textbackslash{}nMW  \textbackslash{}
          Delivery Day                                                     
          2012-12-27        23:00    00:00       968.5               NaN   
          2012-12-27        22:00    23:00      1640.2               NaN   
          2012-12-27        21:00    22:00      1072.3               NaN   
          2012-12-27        20:00    21:00      1011.3               NaN   
          2012-12-27        19:00    20:00      2207.2               NaN   
          
                        Low Price\textbackslash{}nEUR  High Price\textbackslash{}nEUR  Last Price\textbackslash{}nEUR  \textbackslash{}
          Delivery Day                                                     
          2012-12-27                 1             35.0             12.0   
          2012-12-27                 1             45.0             25.0   
          2012-12-27                 1             42.5             11.0   
          2012-12-27                 1             43.0             26.0   
          2012-12-27                 1             56.0             40.5   
          
                        Average Price\textbackslash{}nEUR  
          Delivery Day                      
          2012-12-27                 21.11  
          2012-12-27                 30.16  
          2012-12-27                 27.41  
          2012-12-27                 35.96  
          2012-12-27                 42.25  
\end{Verbatim}
        
    \begin{Verbatim}[commandchars=\\\{\}]
{\color{incolor}In [{\color{incolor}233}]:} \PY{n}{df}\PY{o}{.}\PY{n}{columns}
\end{Verbatim}

            \begin{Verbatim}[commandchars=\\\{\}]
{\color{outcolor}Out[{\color{outcolor}233}]:} Index([u'Hour\textbackslash{}nfrom', u'Hour\textbackslash{}nto', u'Volume\textbackslash{}nMW', u'Volume (OTC)\textbackslash{}nMW', u'Low Price\textbackslash{}nEUR', u'High Price\textbackslash{}nEUR', u'Last Price\textbackslash{}nEUR', u'Average Price\textbackslash{}nEUR'], dtype='object')
\end{Verbatim}
        
    \begin{Verbatim}[commandchars=\\\{\}]
{\color{incolor}In [{\color{incolor}234}]:} \PY{n}{df}\PY{o}{.}\PY{n}{columns} \PY{o}{=} \PY{p}{[}\PY{n}{column}\PY{o}{.}\PY{n}{replace}\PY{p}{(}\PY{l+s}{\PYZsq{}}\PY{l+s}{ }\PY{l+s}{\PYZsq{}}\PY{p}{,}\PY{l+s}{\PYZsq{}}\PY{l+s}{\PYZsq{}}\PY{p}{)}\PY{o}{.}\PY{n}{replace}\PY{p}{(}\PY{l+s}{\PYZsq{}}\PY{l+s+se}{\PYZbs{}n}\PY{l+s}{\PYZsq{}}\PY{p}{,}\PY{l+s}{\PYZsq{}}\PY{l+s}{\PYZsq{}}\PY{p}{)} \PY{k}{for} \PY{n}{column} \PY{o+ow}{in} \PY{n}{df}\PY{o}{.}\PY{n}{columns}\PY{p}{]}
\end{Verbatim}

    \begin{Verbatim}[commandchars=\\\{\}]
{\color{incolor}In [{\color{incolor}235}]:} \PY{n}{df}\PY{o}{.}\PY{n}{columns}
\end{Verbatim}

            \begin{Verbatim}[commandchars=\\\{\}]
{\color{outcolor}Out[{\color{outcolor}235}]:} Index([u'Hourfrom', u'Hourto', u'VolumeMW', u'Volume(OTC)MW', u'LowPriceEUR', u'HighPriceEUR', u'LastPriceEUR', u'AveragePriceEUR'], dtype='object')
\end{Verbatim}
        
    \begin{Verbatim}[commandchars=\\\{\}]
{\color{incolor}In [{\color{incolor}236}]:} \PY{n}{df} \PY{o}{=} \PY{n}{pd}\PY{o}{.}\PY{n}{read\PYZus{}excel}\PY{p}{(}\PY{n}{data\PYZus{}dir}\PY{o}{+}\PY{l+s}{\PYZsq{}}\PY{l+s}{energy\PYZus{}intraday\PYZus{}history\PYZus{}2012.xls}\PY{l+s}{\PYZsq{}}\PY{p}{,}\PY{n}{sheetname}\PY{o}{=}\PY{l+s}{\PYZsq{}}\PY{l+s}{Intraday\PYZhy{}Spot}\PY{l+s}{\PYZsq{}}\PY{p}{,}\PYZbs{}
                             \PY{n}{header}\PY{o}{=}\PY{l+m+mi}{1}\PY{p}{,} \PY{n}{parse\PYZus{}dates} \PY{o}{=} \PY{p}{[}\PY{p}{[}\PY{l+s}{\PYZsq{}}\PY{l+s}{Delivery Day}\PY{l+s}{\PYZsq{}}\PY{p}{,}\PY{l+s}{\PYZsq{}}\PY{l+s}{Hour}\PY{l+s+se}{\PYZbs{}n}\PY{l+s}{from}\PY{l+s}{\PYZsq{}}\PY{p}{]}\PY{p}{]}\PY{p}{,}\PY{n}{index\PYZus{}col}\PY{o}{=}\PY{l+m+mi}{0}\PY{p}{)}        
\end{Verbatim}

    \begin{Verbatim}[commandchars=\\\{\}]
{\color{incolor}In [{\color{incolor}237}]:} \PY{k}{try}\PY{p}{:}
              \PY{k}{del} \PY{n}{df}\PY{p}{[}\PY{l+s}{\PYZsq{}}\PY{l+s}{Hour}\PY{l+s+se}{\PYZbs{}n}\PY{l+s}{to}\PY{l+s}{\PYZsq{}}\PY{p}{]}
          \PY{k}{except}\PY{p}{:}
              \PY{k}{pass}
          \PY{n}{df}\PY{o}{.}\PY{n}{columns} \PY{o}{=} \PY{p}{[}\PY{n}{column}\PY{o}{.}\PY{n}{replace}\PY{p}{(}\PY{l+s}{\PYZsq{}}\PY{l+s}{ }\PY{l+s}{\PYZsq{}}\PY{p}{,}\PY{l+s}{\PYZsq{}}\PY{l+s}{\PYZsq{}}\PY{p}{)}\PY{o}{.}\PY{n}{replace}\PY{p}{(}\PY{l+s}{\PYZsq{}}\PY{l+s+se}{\PYZbs{}n}\PY{l+s}{\PYZsq{}}\PY{p}{,}\PY{l+s}{\PYZsq{}}\PY{l+s}{\PYZsq{}}\PY{p}{)} \PY{k}{for} \PY{n}{column} \PY{o+ow}{in} \PY{n}{df}\PY{o}{.}\PY{n}{columns}\PY{p}{]}
\end{Verbatim}

    \begin{Verbatim}[commandchars=\\\{\}]
{\color{incolor}In [{\color{incolor}238}]:} \PY{n}{df}\PY{o}{.}\PY{n}{head}\PY{p}{(}\PY{p}{)}
\end{Verbatim}

            \begin{Verbatim}[commandchars=\\\{\}]
{\color{outcolor}Out[{\color{outcolor}238}]:}                          VolumeMW  Volume(OTC)MW  LowPriceEUR  HighPriceEUR  \textbackslash{}
          Delivery Day\_Hour\textbackslash{}nfrom                                                       
          2012-12-27 23:00:00         968.5            NaN            1          35.0   
          2012-12-27 22:00:00        1640.2            NaN            1          45.0   
          2012-12-27 21:00:00        1072.3            NaN            1          42.5   
          2012-12-27 20:00:00        1011.3            NaN            1          43.0   
          2012-12-27 19:00:00        2207.2            NaN            1          56.0   
          
                                   LastPriceEUR  AveragePriceEUR  
          Delivery Day\_Hour\textbackslash{}nfrom                                 
          2012-12-27 23:00:00              12.0            21.11  
          2012-12-27 22:00:00              25.0            30.16  
          2012-12-27 21:00:00              11.0            27.41  
          2012-12-27 20:00:00              26.0            35.96  
          2012-12-27 19:00:00              40.5            42.25  
\end{Verbatim}
        
    \begin{Verbatim}[commandchars=\\\{\}]
{\color{incolor}In [{\color{incolor}239}]:} \PY{n}{df}\PY{o}{.}\PY{n}{VolumeMW}\PY{o}{.}\PY{n}{plot}\PY{p}{(}\PY{p}{)}
\end{Verbatim}

            \begin{Verbatim}[commandchars=\\\{\}]
{\color{outcolor}Out[{\color{outcolor}239}]:} <matplotlib.axes.\_subplots.AxesSubplot at 0x7f27621c9550>
\end{Verbatim}
        
    \begin{center}
    \adjustimage{max size={0.9\linewidth}{0.9\paperheight}}{VI-Pandas_files/VI-Pandas_32_1.pdf}
    \end{center}
    { \hspace*{\fill} \\}
    
    \begin{Verbatim}[commandchars=\\\{\}]
{\color{incolor}In [{\color{incolor}240}]:} \PY{n}{df}\PY{o}{.}\PY{n}{index}\PY{o}{.}\PY{n}{get\PYZus{}duplicates}\PY{p}{(}\PY{p}{)}
\end{Verbatim}

            \begin{Verbatim}[commandchars=\\\{\}]
{\color{outcolor}Out[{\color{outcolor}240}]:} <class 'pandas.tseries.index.DatetimeIndex'>
          [2012-10-28 02:00:00]
          Length: 1, Freq: None, Timezone: None
\end{Verbatim}
        
    \begin{Verbatim}[commandchars=\\\{\}]
{\color{incolor}In [{\color{incolor}241}]:} \PY{n}{df}\PY{o}{.}\PY{n}{ix}\PY{p}{[}\PY{n}{df}\PY{o}{.}\PY{n}{index}\PY{o}{.}\PY{n}{get\PYZus{}duplicates}\PY{p}{(}\PY{p}{)}\PY{p}{]}
\end{Verbatim}

            \begin{Verbatim}[commandchars=\\\{\}]
{\color{outcolor}Out[{\color{outcolor}241}]:}                          VolumeMW  Volume(OTC)MW  LowPriceEUR  HighPriceEUR  \textbackslash{}
          Delivery Day\_Hour\textbackslash{}nfrom                                                       
          2012-10-28 02:00:00           625            150           25            40   
          2012-10-28 02:00:00           752            150           18            36   
          
                                   LastPriceEUR  AveragePriceEUR  
          Delivery Day\_Hour\textbackslash{}nfrom                                 
          2012-10-28 02:00:00                40            30.30  
          2012-10-28 02:00:00                33            26.67  
\end{Verbatim}
        
    \begin{Verbatim}[commandchars=\\\{\}]
{\color{incolor}In [{\color{incolor}242}]:} \PY{n}{dfgby} \PY{o}{=} \PY{n}{df}\PY{o}{.}\PY{n}{groupby}\PY{p}{(}\PY{n}{df}\PY{o}{.}\PY{n}{index}\PY{p}{)}\PY{o}{.}\PY{n}{first}\PY{p}{(}\PY{p}{)}
          \PY{n}{dfgby}\PY{o}{.}\PY{n}{ix}\PY{p}{[}\PY{l+s}{\PYZsq{}}\PY{l+s}{2012\PYZhy{}10\PYZhy{}28 02:00}\PY{l+s}{\PYZsq{}}\PY{p}{]}
\end{Verbatim}

            \begin{Verbatim}[commandchars=\\\{\}]
{\color{outcolor}Out[{\color{outcolor}242}]:} VolumeMW           625.0
          Volume(OTC)MW      150.0
          LowPriceEUR         25.0
          HighPriceEUR        40.0
          LastPriceEUR        40.0
          AveragePriceEUR     30.3
          Name: 2012-10-28 02:00:00, dtype: float64
\end{Verbatim}
        
    \begin{Verbatim}[commandchars=\\\{\}]
{\color{incolor}In [{\color{incolor}243}]:} \PY{k}{def} \PY{n+nf}{wavg}\PY{p}{(}\PY{n}{group}\PY{p}{)}\PY{p}{:}
              \PY{n}{w} \PY{o}{=} \PY{n}{group}\PY{p}{[}\PY{l+s}{\PYZsq{}}\PY{l+s}{VolumeMW}\PY{l+s}{\PYZsq{}}\PY{p}{]}\PY{o}{*}\PY{n}{group}\PY{p}{[}\PY{l+s}{\PYZsq{}}\PY{l+s}{AveragePriceEUR}\PY{l+s}{\PYZsq{}}\PY{p}{]}
              \PY{n}{d} \PY{o}{=} \PY{n}{group}
              \PY{k}{return} \PY{p}{(}\PY{n}{d}\PY{o}{*}\PY{n}{w}\PY{p}{)}\PY{o}{.}\PY{n}{sum}\PY{p}{(}\PY{p}{)}\PY{o}{/}\PY{n}{w}\PY{o}{.}\PY{n}{sum}\PY{p}{(}\PY{p}{)}
              
          \PY{n}{grouped} \PY{o}{=} \PY{n}{df}\PY{o}{.}\PY{n}{groupby}\PY{p}{(}\PY{n}{df}\PY{o}{.}\PY{n}{index}\PY{p}{)}\PY{o}{.}\PY{n}{apply}\PY{p}{(}\PY{n}{wavg}\PY{p}{)}
          \PY{n}{grouped}\PY{o}{.}\PY{n}{ix}\PY{p}{[}\PY{l+s}{\PYZsq{}}\PY{l+s}{2012\PYZhy{}10\PYZhy{}28 02:00}\PY{l+s}{\PYZsq{}}\PY{p}{]}
\end{Verbatim}

            \begin{Verbatim}[commandchars=\\\{\}]
{\color{outcolor}Out[{\color{outcolor}243}]:} VolumeMW           690.321198
          Volume(OTC)MW      150.000000
          LowPriceEUR         21.399619
          HighPriceEUR        37.942639
          LastPriceEUR        36.399619
          AveragePriceEUR     28.432945
          Name: 2012-10-28 02:00:00, dtype: float64
\end{Verbatim}
        
    \begin{Verbatim}[commandchars=\\\{\}]
{\color{incolor}In [{\color{incolor}244}]:} \PY{n}{df} \PY{o}{=} \PY{n}{grouped}
          \PY{n}{df}\PY{o}{.}\PY{n}{VolumeMW}\PY{o}{.}\PY{n}{plot}\PY{p}{(}\PY{p}{)}
\end{Verbatim}

            \begin{Verbatim}[commandchars=\\\{\}]
{\color{outcolor}Out[{\color{outcolor}244}]:} <matplotlib.axes.\_subplots.AxesSubplot at 0x7f2762f19310>
\end{Verbatim}
        
    \begin{center}
    \adjustimage{max size={0.9\linewidth}{0.9\paperheight}}{VI-Pandas_files/VI-Pandas_37_1.pdf}
    \end{center}
    { \hspace*{\fill} \\}
    
    \begin{Verbatim}[commandchars=\\\{\}]
{\color{incolor}In [{\color{incolor}245}]:} \PY{n}{df}\PY{o}{.}\PY{n}{AveragePriceEUR}\PY{o}{.}\PY{n}{plot}\PY{p}{(}\PY{p}{)}
\end{Verbatim}

            \begin{Verbatim}[commandchars=\\\{\}]
{\color{outcolor}Out[{\color{outcolor}245}]:} <matplotlib.axes.\_subplots.AxesSubplot at 0x7f27618721d0>
\end{Verbatim}
        
    \begin{center}
    \adjustimage{max size={0.9\linewidth}{0.9\paperheight}}{VI-Pandas_files/VI-Pandas_38_1.pdf}
    \end{center}
    { \hspace*{\fill} \\}
    
    \begin{Verbatim}[commandchars=\\\{\}]
{\color{incolor}In [{\color{incolor}246}]:} \PY{n}{ts} \PY{o}{=} \PY{n}{df}\PY{o}{.}\PY{n}{VolumeMW}
\end{Verbatim}

    \begin{Verbatim}[commandchars=\\\{\}]
{\color{incolor}In [{\color{incolor}247}]:} \PY{n}{ts}\PY{p}{[}\PY{l+s}{\PYZsq{}}\PY{l+s}{10/2012}\PY{l+s}{\PYZsq{}}\PY{p}{]}\PY{o}{.}\PY{n}{plot}\PY{p}{(}\PY{p}{)}
\end{Verbatim}

            \begin{Verbatim}[commandchars=\\\{\}]
{\color{outcolor}Out[{\color{outcolor}247}]:} <matplotlib.axes.\_subplots.AxesSubplot at 0x7f27681926d0>
\end{Verbatim}
        
    \begin{center}
    \adjustimage{max size={0.9\linewidth}{0.9\paperheight}}{VI-Pandas_files/VI-Pandas_40_1.pdf}
    \end{center}
    { \hspace*{\fill} \\}
    
    \begin{Verbatim}[commandchars=\\\{\}]
{\color{incolor}In [{\color{incolor}248}]:} \PY{n}{ts}\PY{p}{[}\PY{l+s}{\PYZsq{}}\PY{l+s}{10\PYZhy{}2012}\PY{l+s}{\PYZsq{}}\PY{p}{]}\PY{o}{.}\PY{n}{plot}\PY{p}{(}\PY{p}{)}
\end{Verbatim}

            \begin{Verbatim}[commandchars=\\\{\}]
{\color{outcolor}Out[{\color{outcolor}248}]:} <matplotlib.axes.\_subplots.AxesSubplot at 0x7f2761fe5350>
\end{Verbatim}
        
    \begin{center}
    \adjustimage{max size={0.9\linewidth}{0.9\paperheight}}{VI-Pandas_files/VI-Pandas_41_1.pdf}
    \end{center}
    { \hspace*{\fill} \\}
    
    \begin{Verbatim}[commandchars=\\\{\}]
{\color{incolor}In [{\color{incolor}249}]:} \PY{n}{ts}\PY{p}{[}\PY{l+s}{\PYZsq{}}\PY{l+s}{09\PYZhy{}09\PYZhy{}2012}\PY{l+s}{\PYZsq{}}\PY{p}{]}\PY{o}{.}\PY{n}{plot}\PY{p}{(}\PY{p}{)}
\end{Verbatim}

            \begin{Verbatim}[commandchars=\\\{\}]
{\color{outcolor}Out[{\color{outcolor}249}]:} <matplotlib.axes.\_subplots.AxesSubplot at 0x7f27618ce150>
\end{Verbatim}
        
    \begin{center}
    \adjustimage{max size={0.9\linewidth}{0.9\paperheight}}{VI-Pandas_files/VI-Pandas_42_1.pdf}
    \end{center}
    { \hspace*{\fill} \\}
    
    \begin{Verbatim}[commandchars=\\\{\}]
{\color{incolor}In [{\color{incolor}250}]:} \PY{n}{df}\PY{o}{.}\PY{n}{sort\PYZus{}index}\PY{p}{(}\PY{n}{inplace}\PY{o}{=}\PY{n+nb+bp}{True}\PY{p}{)}
\end{Verbatim}

    \begin{Verbatim}[commandchars=\\\{\}]
{\color{incolor}In [{\color{incolor}251}]:} \PY{n}{df}\PY{o}{.}\PY{n}{VolumeMW}\PY{o}{.}\PY{n}{plot}\PY{p}{(}\PY{p}{)}
\end{Verbatim}

            \begin{Verbatim}[commandchars=\\\{\}]
{\color{outcolor}Out[{\color{outcolor}251}]:} <matplotlib.axes.\_subplots.AxesSubplot at 0x7f276218c890>
\end{Verbatim}
        
    \begin{center}
    \adjustimage{max size={0.9\linewidth}{0.9\paperheight}}{VI-Pandas_files/VI-Pandas_44_1.pdf}
    \end{center}
    { \hspace*{\fill} \\}
    
    \begin{Verbatim}[commandchars=\\\{\}]
{\color{incolor}In [{\color{incolor}252}]:} \PY{n}{ts} \PY{o}{=} \PY{n}{df}\PY{o}{.}\PY{n}{VolumeMW}
          \PY{k}{with} \PY{n}{plt}\PY{o}{.}\PY{n}{xkcd}\PY{p}{(}\PY{p}{)}\PY{p}{:}
              \PY{n}{ts}\PY{o}{.}\PY{n}{asfreq}\PY{p}{(}\PY{n}{freq}\PY{o}{=}\PY{l+s}{\PYZsq{}}\PY{l+s}{W}\PY{l+s}{\PYZsq{}}\PY{p}{)}\PY{o}{.}\PY{n}{plot}\PY{p}{(}\PY{p}{)}
\end{Verbatim}

    \begin{center}
    \adjustimage{max size={0.9\linewidth}{0.9\paperheight}}{VI-Pandas_files/VI-Pandas_45_0.pdf}
    \end{center}
    { \hspace*{\fill} \\}
    
    \begin{Verbatim}[commandchars=\\\{\}]
{\color{incolor}In [{\color{incolor}253}]:} \PY{n}{ts}\PY{o}{.}\PY{n}{asfreq}\PY{p}{(}\PY{n}{freq}\PY{o}{=}\PY{l+s}{\PYZsq{}}\PY{l+s}{W}\PY{l+s}{\PYZsq{}}\PY{p}{)}\PY{o}{.}\PY{n}{plot}\PY{p}{(}\PY{p}{)}
\end{Verbatim}

            \begin{Verbatim}[commandchars=\\\{\}]
{\color{outcolor}Out[{\color{outcolor}253}]:} <matplotlib.axes.\_subplots.AxesSubplot at 0x7f2762a793d0>
\end{Verbatim}
        
    \begin{center}
    \adjustimage{max size={0.9\linewidth}{0.9\paperheight}}{VI-Pandas_files/VI-Pandas_46_1.pdf}
    \end{center}
    { \hspace*{\fill} \\}
    
    \begin{Verbatim}[commandchars=\\\{\}]
{\color{incolor}In [{\color{incolor}254}]:} \PY{n}{ts}\PY{o}{.}\PY{n}{describe}\PY{p}{(}\PY{p}{)}
\end{Verbatim}

            \begin{Verbatim}[commandchars=\\\{\}]
{\color{outcolor}Out[{\color{outcolor}254}]:} count    8687.000000
          mean     1510.502834
          std       911.039358
          min        43.000000
          25\%       840.000000
          50\%      1325.000000
          75\%      1960.000000
          max      7369.000000
          Name: VolumeMW, dtype: float64
\end{Verbatim}
        
    \begin{Verbatim}[commandchars=\\\{\}]
{\color{incolor}In [{\color{incolor}255}]:} \PY{n}{ts}\PY{o}{.}\PY{n}{resample}\PY{p}{(}\PY{l+s}{\PYZsq{}}\PY{l+s}{W\PYZhy{}FRI}\PY{l+s}{\PYZsq{}}\PY{p}{)}\PY{o}{.}\PY{n}{plot}\PY{p}{(}\PY{p}{)}
\end{Verbatim}

            \begin{Verbatim}[commandchars=\\\{\}]
{\color{outcolor}Out[{\color{outcolor}255}]:} <matplotlib.axes.\_subplots.AxesSubplot at 0x7f27626b8a50>
\end{Verbatim}
        
    \begin{center}
    \adjustimage{max size={0.9\linewidth}{0.9\paperheight}}{VI-Pandas_files/VI-Pandas_48_1.pdf}
    \end{center}
    { \hspace*{\fill} \\}
    
    \begin{Verbatim}[commandchars=\\\{\}]
{\color{incolor}In [{\color{incolor}256}]:} \PY{n}{ts}\PY{o}{.}\PY{n}{resample}\PY{p}{(}\PY{l+s}{\PYZsq{}}\PY{l+s}{W\PYZhy{}SUN}\PY{l+s}{\PYZsq{}}\PY{p}{,}\PY{n}{how}\PY{o}{=}\PY{p}{[}\PY{l+s}{\PYZsq{}}\PY{l+s}{mean}\PY{l+s}{\PYZsq{}}\PY{p}{,}\PY{l+s}{\PYZsq{}}\PY{l+s}{max}\PY{l+s}{\PYZsq{}}\PY{p}{,}\PY{l+s}{\PYZsq{}}\PY{l+s}{min}\PY{l+s}{\PYZsq{}}\PY{p}{]}\PY{p}{)}\PY{o}{.}\PY{n}{plot}\PY{p}{(}\PY{p}{)}
\end{Verbatim}

            \begin{Verbatim}[commandchars=\\\{\}]
{\color{outcolor}Out[{\color{outcolor}256}]:} <matplotlib.axes.\_subplots.AxesSubplot at 0x7f2761aade50>
\end{Verbatim}
        
    \begin{center}
    \adjustimage{max size={0.9\linewidth}{0.9\paperheight}}{VI-Pandas_files/VI-Pandas_49_1.pdf}
    \end{center}
    { \hspace*{\fill} \\}
    
    \begin{Verbatim}[commandchars=\\\{\}]
{\color{incolor}In [{\color{incolor}257}]:} \PY{n}{resampled} \PY{o}{=} \PY{n}{ts}\PY{o}{.}\PY{n}{resample}\PY{p}{(}\PY{l+s}{\PYZsq{}}\PY{l+s}{30t}\PY{l+s}{\PYZsq{}}\PY{p}{)}
          \PY{n}{resampled}
\end{Verbatim}

            \begin{Verbatim}[commandchars=\\\{\}]
{\color{outcolor}Out[{\color{outcolor}257}]:} Delivery Day\_Hour\textbackslash{}nfrom
          2012-01-01 00:00:00        1161.0
          2012-01-01 00:30:00           NaN
          2012-01-01 01:00:00         791.0
          2012-01-01 01:30:00           NaN
          2012-01-01 02:00:00         911.0
          2012-01-01 02:30:00           NaN
          2012-01-01 03:00:00         666.0
          2012-01-01 03:30:00           NaN
          2012-01-01 04:00:00         694.0
          2012-01-01 04:30:00           NaN
          2012-01-01 05:00:00         730.0
          2012-01-01 05:30:00           NaN
          2012-01-01 06:00:00         587.9
          2012-01-01 06:30:00           NaN
          2012-01-01 07:00:00        1077.7
          \ldots
          2012-12-27 16:00:00        4034.7
          2012-12-27 16:30:00           NaN
          2012-12-27 17:00:00        3861.6
          2012-12-27 17:30:00           NaN
          2012-12-27 18:00:00        4029.4
          2012-12-27 18:30:00           NaN
          2012-12-27 19:00:00        2207.2
          2012-12-27 19:30:00           NaN
          2012-12-27 20:00:00        1011.3
          2012-12-27 20:30:00           NaN
          2012-12-27 21:00:00        1072.3
          2012-12-27 21:30:00           NaN
          2012-12-27 22:00:00        1640.2
          2012-12-27 22:30:00           NaN
          2012-12-27 23:00:00         968.5
          Freq: 30T, Name: VolumeMW, Length: 17375
\end{Verbatim}
        
    \begin{Verbatim}[commandchars=\\\{\}]
{\color{incolor}In [{\color{incolor}258}]:} \PY{n}{resampled}\PY{o}{.}\PY{n}{interpolate}\PY{p}{(}\PY{p}{)}
\end{Verbatim}

            \begin{Verbatim}[commandchars=\\\{\}]
{\color{outcolor}Out[{\color{outcolor}258}]:} Delivery Day\_Hour\textbackslash{}nfrom
          2012-01-01 00:00:00        1161.00
          2012-01-01 00:30:00         976.00
          2012-01-01 01:00:00         791.00
          2012-01-01 01:30:00         851.00
          2012-01-01 02:00:00         911.00
          2012-01-01 02:30:00         788.50
          2012-01-01 03:00:00         666.00
          2012-01-01 03:30:00         680.00
          2012-01-01 04:00:00         694.00
          2012-01-01 04:30:00         712.00
          2012-01-01 05:00:00         730.00
          2012-01-01 05:30:00         658.95
          2012-01-01 06:00:00         587.90
          2012-01-01 06:30:00         832.80
          2012-01-01 07:00:00        1077.70
          \ldots
          2012-12-27 16:00:00        4034.70
          2012-12-27 16:30:00        3948.15
          2012-12-27 17:00:00        3861.60
          2012-12-27 17:30:00        3945.50
          2012-12-27 18:00:00        4029.40
          2012-12-27 18:30:00        3118.30
          2012-12-27 19:00:00        2207.20
          2012-12-27 19:30:00        1609.25
          2012-12-27 20:00:00        1011.30
          2012-12-27 20:30:00        1041.80
          2012-12-27 21:00:00        1072.30
          2012-12-27 21:30:00        1356.25
          2012-12-27 22:00:00        1640.20
          2012-12-27 22:30:00        1304.35
          2012-12-27 23:00:00         968.50
          Freq: 30T, Name: VolumeMW, Length: 17375
\end{Verbatim}
        
    \begin{Verbatim}[commandchars=\\\{\}]
{\color{incolor}In [{\color{incolor}259}]:} \PY{n}{df}\PY{o}{.}\PY{n}{resample}\PY{p}{(}\PY{l+s}{\PYZsq{}}\PY{l+s}{D}\PY{l+s}{\PYZsq{}}\PY{p}{)}\PY{o}{.}\PY{n}{AveragePriceEUR}\PY{o}{.}\PY{n}{plot}\PY{p}{(}\PY{n}{style}\PY{o}{=}\PY{l+s}{\PYZsq{}}\PY{l+s}{b}\PY{l+s}{\PYZsq{}}\PY{p}{)}
          \PY{n}{df}\PY{o}{.}\PY{n}{resample}\PY{p}{(}\PY{l+s}{\PYZsq{}}\PY{l+s}{D}\PY{l+s}{\PYZsq{}}\PY{p}{)}\PY{o}{.}\PY{n}{VolumeMW}\PY{o}{.}\PY{n}{plot}\PY{p}{(}\PY{n}{secondary\PYZus{}y}\PY{o}{=}\PY{n+nb+bp}{True}\PY{p}{,} \PY{n}{style}\PY{o}{=}\PY{l+s}{\PYZsq{}}\PY{l+s}{r}\PY{l+s}{\PYZsq{}}\PY{p}{,} \PY{n}{alpha} \PY{o}{=} \PY{l+m+mf}{0.4}\PY{p}{)}
\end{Verbatim}

            \begin{Verbatim}[commandchars=\\\{\}]
{\color{outcolor}Out[{\color{outcolor}259}]:} <matplotlib.axes.\_subplots.AxesSubplot at 0x7f27617e1cd0>
\end{Verbatim}
        
    \begin{center}
    \adjustimage{max size={0.9\linewidth}{0.9\paperheight}}{VI-Pandas_files/VI-Pandas_52_1.pdf}
    \end{center}
    { \hspace*{\fill} \\}
    
    \begin{Verbatim}[commandchars=\\\{\}]
{\color{incolor}In [{\color{incolor}260}]:} \PY{n}{ax1} \PY{o}{=} \PY{n}{df}\PY{o}{.}\PY{n}{resample}\PY{p}{(}\PY{l+s}{\PYZsq{}}\PY{l+s}{D}\PY{l+s}{\PYZsq{}}\PY{p}{)}\PY{o}{.}\PY{n}{AveragePriceEUR}\PY{o}{.}\PY{n}{plot}\PY{p}{(}\PY{n}{style}\PY{o}{=}\PY{l+s}{\PYZsq{}}\PY{l+s}{b}\PY{l+s}{\PYZsq{}}\PY{p}{)}
          \PY{n}{ax2} \PY{o}{=} \PY{n}{df}\PY{o}{.}\PY{n}{resample}\PY{p}{(}\PY{l+s}{\PYZsq{}}\PY{l+s}{D}\PY{l+s}{\PYZsq{}}\PY{p}{)}\PY{o}{.}\PY{n}{VolumeMW}\PY{o}{.}\PY{n}{plot}\PY{p}{(}\PY{n}{secondary\PYZus{}y}\PY{o}{=}\PY{n+nb+bp}{True}\PY{p}{,} \PY{n}{style}\PY{o}{=}\PY{l+s}{\PYZsq{}}\PY{l+s}{r}\PY{l+s}{\PYZsq{}}\PY{p}{,} \PY{n}{alpha} \PY{o}{=} \PY{l+m+mf}{0.4}\PY{p}{)}
          \PY{n}{ax1}\PY{o}{.}\PY{n}{set\PYZus{}ylabel}\PY{p}{(}\PY{l+s}{\PYZsq{}}\PY{l+s}{Average Price (EUR)}\PY{l+s}{\PYZsq{}}\PY{p}{)}
          \PY{n}{ax2}\PY{o}{.}\PY{n}{set\PYZus{}ylabel}\PY{p}{(}\PY{l+s}{\PYZsq{}}\PY{l+s}{Volume (MW)}\PY{l+s}{\PYZsq{}}\PY{p}{)}
\end{Verbatim}

            \begin{Verbatim}[commandchars=\\\{\}]
{\color{outcolor}Out[{\color{outcolor}260}]:} <matplotlib.text.Text at 0x7f276125c310>
\end{Verbatim}
        
    \begin{center}
    \adjustimage{max size={0.9\linewidth}{0.9\paperheight}}{VI-Pandas_files/VI-Pandas_53_1.pdf}
    \end{center}
    { \hspace*{\fill} \\}
    

    \paragraph{\textbf{Rolling windows}}


    \begin{Verbatim}[commandchars=\\\{\}]
{\color{incolor}In [{\color{incolor}261}]:} \PY{n}{rolling\PYZus{}std}\PY{p}{(}\PY{n}{df}\PY{o}{.}\PY{n}{AveragePriceEUR}\PY{o}{.}\PY{n}{resample}\PY{p}{(}\PY{l+s}{\PYZsq{}}\PY{l+s}{1D}\PY{l+s}{\PYZsq{}}\PY{p}{)}\PY{p}{,} \PY{n}{window} \PY{o}{=} \PY{l+m+mi}{20}\PY{p}{)}\PY{o}{.}\PY{n}{plot}\PY{p}{(}\PY{p}{)}
\end{Verbatim}

            \begin{Verbatim}[commandchars=\\\{\}]
{\color{outcolor}Out[{\color{outcolor}261}]:} <matplotlib.axes.\_subplots.AxesSubplot at 0x7f276115fcd0>
\end{Verbatim}
        
    \begin{center}
    \adjustimage{max size={0.9\linewidth}{0.9\paperheight}}{VI-Pandas_files/VI-Pandas_55_1.pdf}
    \end{center}
    { \hspace*{\fill} \\}
    
    \begin{Verbatim}[commandchars=\\\{\}]
{\color{incolor}In [{\color{incolor}262}]:} \PY{n}{corr\PYZus{}vol\PYZus{}avgp20} \PY{o}{=} \PY{n}{rolling\PYZus{}corr}\PY{p}{(}\PY{n}{df}\PY{o}{.}\PY{n}{VolumeMW}\PY{p}{,} \PY{n}{df}\PY{o}{.}\PY{n}{AveragePriceEUR}\PY{p}{,} \PY{n}{window}\PY{o}{=}\PY{l+m+mi}{20}\PY{p}{)}
          \PY{n}{corr\PYZus{}vol\PYZus{}avgp200} \PY{o}{=} \PY{n}{rolling\PYZus{}corr}\PY{p}{(}\PY{n}{df}\PY{o}{.}\PY{n}{VolumeMW}\PY{p}{,} \PY{n}{df}\PY{o}{.}\PY{n}{AveragePriceEUR}\PY{p}{,} \PY{n}{window}\PY{o}{=}\PY{l+m+mi}{200}\PY{p}{)}
          \PY{n}{corr\PYZus{}vol\PYZus{}avgp20}\PY{o}{.}\PY{n}{plot}\PY{p}{(}\PY{n}{alpha}\PY{o}{=}\PY{l+m+mf}{0.7}\PY{p}{,}\PY{n}{label}\PY{o}{=}\PY{l+s}{\PYZsq{}}\PY{l+s}{20day correlation}\PY{l+s}{\PYZsq{}}\PY{p}{)}
          \PY{n}{corr\PYZus{}vol\PYZus{}avgp200}\PY{o}{.}\PY{n}{plot}\PY{p}{(}\PY{n}{style}\PY{o}{=}\PY{l+s}{\PYZsq{}}\PY{l+s}{red}\PY{l+s}{\PYZsq{}}\PY{p}{,}\PY{n}{label}\PY{o}{=}\PY{l+s}{\PYZsq{}}\PY{l+s}{200day correlation}\PY{l+s}{\PYZsq{}}\PY{p}{)}
\end{Verbatim}

            \begin{Verbatim}[commandchars=\\\{\}]
{\color{outcolor}Out[{\color{outcolor}262}]:} <matplotlib.axes.\_subplots.AxesSubplot at 0x7f2761066550>
\end{Verbatim}
        
    \begin{center}
    \adjustimage{max size={0.9\linewidth}{0.9\paperheight}}{VI-Pandas_files/VI-Pandas_56_1.pdf}
    \end{center}
    { \hspace*{\fill} \\}
    
    \begin{Verbatim}[commandchars=\\\{\}]
{\color{incolor}In [{\color{incolor}263}]:} \PY{n}{rolling\PYZus{}mean}\PY{p}{(}\PY{n}{corr\PYZus{}vol\PYZus{}avgp20}\PY{p}{,}\PY{n}{window}\PY{o}{=}\PY{l+m+mi}{100}\PY{p}{)}\PY{o}{.}\PY{n}{plot}\PY{p}{(}\PY{p}{)}
\end{Verbatim}

            \begin{Verbatim}[commandchars=\\\{\}]
{\color{outcolor}Out[{\color{outcolor}263}]:} <matplotlib.axes.\_subplots.AxesSubplot at 0x7f2761078110>
\end{Verbatim}
        
    \begin{center}
    \adjustimage{max size={0.9\linewidth}{0.9\paperheight}}{VI-Pandas_files/VI-Pandas_57_1.pdf}
    \end{center}
    { \hspace*{\fill} \\}
    
    \textbf{Merging Series}

    \begin{Verbatim}[commandchars=\\\{\}]
{\color{incolor}In [{\color{incolor}264}]:} \PY{o}{!}ls \PY{n+nv}{\PYZdl{}data\PYZus{}dir}
\end{Verbatim}

    \begin{Verbatim}[commandchars=\\\{\}]
energy\_intraday\_history\_2006.xls	     energy\_spot\_historie\_2004.xls	       energy\_spot\_historie\_end\_20020731\_xetra.xls
energy\_intraday\_history\_2007.xls	     energy\_spot\_historie\_2005.xls	       Phelix\_Quarterly.xls
energy\_intraday\_history\_2008.xls	     energy\_spot\_historie\_2006.xls	       swiss\_power\_spot\_market\_2006.xls
energy\_intraday\_history\_2009.xls	     energy\_spot\_historie\_2007.xls	       swiss\_power\_spot\_market\_2007.xls
energy\_intraday\_history\_2010.xls	     energy\_spot\_historie\_2008.xls	       swiss\_power\_spot\_market\_2008.xls
energy\_intraday\_history\_2011 - Konflikt.xls  energy\_spot\_historie\_2009.xls	       swiss\_power\_spot\_market\_2009.xls
energy\_intraday\_history\_2011.xls	     energy\_spot\_historie\_2010.xls	       swiss\_power\_spot\_market\_2010.xls
energy\_intraday\_history\_2012.xls	     energy\_spot\_historie\_2011.xls	       swiss\_power\_spot\_market\_2011.xls
energy\_spot\_historie\_2002.xls		     energy\_spot\_historie\_2012 - Konflikt.xls  swiss\_power\_spot\_market\_2012.xls
energy\_spot\_historie\_2003.xls		     energy\_spot\_historie\_2012.xls
    \end{Verbatim}

    \begin{Verbatim}[commandchars=\\\{\}]
{\color{incolor}In [{\color{incolor}265}]:} \PY{n}{dseries} \PY{o}{=} \PY{p}{\PYZob{}}\PY{p}{\PYZcb{}}
          \PY{k}{for} \PY{n}{filename} \PY{o+ow}{in} \PY{n}{os}\PY{o}{.}\PY{n}{listdir}\PY{p}{(}\PY{n}{data\PYZus{}dir}\PY{p}{)}\PY{p}{:}
              \PY{k}{if} \PY{l+s}{\PYZsq{}}\PY{l+s}{Konflikt}\PY{l+s}{\PYZsq{}} \PY{o+ow}{not} \PY{o+ow}{in} \PY{n}{filename} \PY{o+ow}{and} \PY{l+s}{\PYZsq{}}\PY{l+s}{energy\PYZus{}intraday}\PY{l+s}{\PYZsq{}} \PY{o+ow}{in} \PY{n}{filename}\PY{p}{:}
                  \PY{n}{dseries}\PY{p}{[}\PY{n}{filename}\PY{o}{.}\PY{n}{split}\PY{p}{(}\PY{l+s}{\PYZsq{}}\PY{l+s}{\PYZus{}}\PY{l+s}{\PYZsq{}}\PY{p}{)}\PY{p}{[}\PY{o}{\PYZhy{}}\PY{l+m+mi}{1}\PY{p}{]}\PY{p}{[}\PY{p}{:}\PY{l+m+mi}{4}\PY{p}{]}\PY{p}{]} \PY{o}{=} \PY{n}{pd}\PY{o}{.}\PY{n}{read\PYZus{}excel}\PY{p}{(}\PY{n}{data\PYZus{}dir}\PY{o}{+}\PY{n}{filename}\PY{p}{,}\PY{n}{sheetname}\PY{o}{=}\PY{l+s}{\PYZsq{}}\PY{l+s}{Intraday\PYZhy{}Spot}\PY{l+s}{\PYZsq{}}\PY{p}{,}\PYZbs{}
                                                                       \PY{n}{header}\PY{o}{=}\PY{l+m+mi}{1}\PY{p}{,} \PY{n}{parse\PYZus{}dates} \PY{o}{=} \PY{p}{[}\PY{p}{[}\PY{l+s}{\PYZsq{}}\PY{l+s}{Delivery Day}\PY{l+s}{\PYZsq{}}\PY{p}{,}\PY{l+s}{\PYZsq{}}\PY{l+s}{Hour}\PY{l+s+se}{\PYZbs{}n}\PY{l+s}{from}\PY{l+s}{\PYZsq{}}\PY{p}{]}\PY{p}{]}\PY{p}{,}\PY{n}{index\PYZus{}col}\PY{o}{=}\PY{l+m+mi}{0}\PY{p}{)}      
\end{Verbatim}

    \begin{Verbatim}[commandchars=\\\{\}]
{\color{incolor}In [{\color{incolor}266}]:} \PY{n}{df} \PY{o}{=} \PY{n}{concat}\PY{p}{(}\PY{n}{dseries}\PY{o}{.}\PY{n}{values}\PY{p}{(}\PY{p}{)}\PY{p}{)}
\end{Verbatim}

    \begin{Verbatim}[commandchars=\\\{\}]
{\color{incolor}In [{\color{incolor}267}]:} \PY{n+nb}{list}\PY{p}{(}\PY{n}{df}\PY{o}{.}\PY{n}{index}\PY{o}{.}\PY{n}{get\PYZus{}duplicates}\PY{p}{(}\PY{p}{)}\PY{p}{)}
\end{Verbatim}

            \begin{Verbatim}[commandchars=\\\{\}]
{\color{outcolor}Out[{\color{outcolor}267}]:} [Timestamp('2006-10-29 02:00:00'),
           Timestamp('2007-10-28 02:00:00'),
           Timestamp('2008-10-26 02:00:00'),
           Timestamp('2009-10-25 02:00:00'),
           Timestamp('2010-10-31 02:00:00'),
           Timestamp('2011-10-30 02:00:00'),
           Timestamp('2012-10-28 02:00:00')]
\end{Verbatim}
        
    \begin{Verbatim}[commandchars=\\\{\}]
{\color{incolor}In [{\color{incolor}268}]:} \PY{n}{df} \PY{o}{=} \PY{n}{df}\PY{o}{.}\PY{n}{groupby}\PY{p}{(}\PY{n}{df}\PY{o}{.}\PY{n}{index}\PY{p}{)}\PY{o}{.}\PY{n}{first}\PY{p}{(}\PY{p}{)}
          \PY{n}{df}\PY{o}{.}\PY{n}{columns} \PY{o}{=} \PY{p}{[}\PY{n}{column}\PY{o}{.}\PY{n}{replace}\PY{p}{(}\PY{l+s}{\PYZsq{}}\PY{l+s}{ }\PY{l+s}{\PYZsq{}}\PY{p}{,}\PY{l+s}{\PYZsq{}}\PY{l+s}{\PYZsq{}}\PY{p}{)}\PY{o}{.}\PY{n}{replace}\PY{p}{(}\PY{l+s}{\PYZsq{}}\PY{l+s+se}{\PYZbs{}n}\PY{l+s}{\PYZsq{}}\PY{p}{,}\PY{l+s}{\PYZsq{}}\PY{l+s}{\PYZsq{}}\PY{p}{)} \PY{k}{for} \PY{n}{column} \PY{o+ow}{in} \PY{n}{df}\PY{o}{.}\PY{n}{columns}\PY{p}{]}
\end{Verbatim}

    \begin{Verbatim}[commandchars=\\\{\}]
{\color{incolor}In [{\color{incolor}269}]:} \PY{k+kn}{from} \PY{n+nn}{matplotlib.pyplot} \PY{k+kn}{import} \PY{o}{*}
\end{Verbatim}

    \begin{Verbatim}[commandchars=\\\{\}]
{\color{incolor}In [{\color{incolor}270}]:} \PY{n}{df}\PY{o}{.}\PY{n}{VolumeMW}\PY{o}{.}\PY{n}{asfreq}\PY{p}{(}\PY{l+s}{\PYZsq{}}\PY{l+s}{W\PYZhy{}FRI}\PY{l+s}{\PYZsq{}}\PY{p}{)}\PY{o}{.}\PY{n}{plot}\PY{p}{(}\PY{p}{)}
          \PY{n}{df}\PY{o}{.}\PY{n}{VolumeMW}\PY{o}{.}\PY{n}{asfreq}\PY{p}{(}\PY{l+s}{\PYZsq{}}\PY{l+s}{W\PYZhy{}MON}\PY{l+s}{\PYZsq{}}\PY{p}{)}\PY{o}{.}\PY{n}{plot}\PY{p}{(}\PY{p}{)}
          \PY{n}{df}\PY{o}{.}\PY{n}{VolumeMW}\PY{o}{.}\PY{n}{asfreq}\PY{p}{(}\PY{l+s}{\PYZsq{}}\PY{l+s}{W\PYZhy{}SUN}\PY{l+s}{\PYZsq{}}\PY{p}{)}\PY{o}{.}\PY{n}{plot}\PY{p}{(}\PY{p}{)}
          \PY{n}{legend}\PY{p}{(}\PY{p}{[}\PY{l+s}{\PYZsq{}}\PY{l+s}{FRI}\PY{l+s}{\PYZsq{}}\PY{p}{,}\PY{l+s}{\PYZsq{}}\PY{l+s}{MON}\PY{l+s}{\PYZsq{}}\PY{p}{,}\PY{l+s}{\PYZsq{}}\PY{l+s}{SUN}\PY{l+s}{\PYZsq{}}\PY{p}{]}\PY{p}{,}\PY{n}{loc}\PY{o}{=}\PY{l+s}{\PYZsq{}}\PY{l+s}{best}\PY{l+s}{\PYZsq{}}\PY{p}{)}
\end{Verbatim}

            \begin{Verbatim}[commandchars=\\\{\}]
{\color{outcolor}Out[{\color{outcolor}270}]:} <matplotlib.legend.Legend at 0x7f2760872e90>
\end{Verbatim}
        
    \begin{center}
    \adjustimage{max size={0.9\linewidth}{0.9\paperheight}}{VI-Pandas_files/VI-Pandas_65_1.pdf}
    \end{center}
    { \hspace*{\fill} \\}
    
    \begin{Verbatim}[commandchars=\\\{\}]
{\color{incolor}In [{\color{incolor}271}]:} \PY{n}{df}\PY{o}{.}\PY{n}{ix}\PY{p}{[}\PY{p}{:}\PY{p}{,}\PY{p}{[}\PY{l+s}{\PYZsq{}}\PY{l+s}{VolumeMW}\PY{l+s}{\PYZsq{}}\PY{p}{,}\PY{l+s}{\PYZsq{}}\PY{l+s}{HighPriceEUR}\PY{l+s}{\PYZsq{}}\PY{p}{]}\PY{p}{]}\PY{o}{.}\PY{n}{plot}\PY{p}{(}\PY{n}{subplots}\PY{o}{=}\PY{n+nb+bp}{True}\PY{p}{)}
\end{Verbatim}

            \begin{Verbatim}[commandchars=\\\{\}]
{\color{outcolor}Out[{\color{outcolor}271}]:} array([<matplotlib.axes.\_subplots.AxesSubplot object at 0x7f2760802190>,
                 <matplotlib.axes.\_subplots.AxesSubplot object at 0x7f2760748650>], dtype=object)
\end{Verbatim}
        
    \begin{center}
    \adjustimage{max size={0.9\linewidth}{0.9\paperheight}}{VI-Pandas_files/VI-Pandas_66_1.pdf}
    \end{center}
    { \hspace*{\fill} \\}
    

    \subsubsection{\textbf{Regression}}


    Meteorological historical data for Düsseldorf

    \begin{Verbatim}[commandchars=\\\{\}]
{\color{incolor}In [{\color{incolor}272}]:} \PY{o}{!}head \PY{l+s+s1}{\PYZsq{}./data/kl\PYZus{}10400\PYZus{}00\PYZus{}akt\PYZus{}txt.txt\PYZsq{}}
\end{Verbatim}

    \begin{Verbatim}[commandchars=\\\{\}]
KL01192200001010000101701101861102201101924  691  251 444  38 1  511  641  271  424  41 6  60 6  24 6 754 914 714 794 864 944 954 924 861 951 95116 3124 2120 11 204 81-99-99 81-99-99 81-99-99 804  001 71 41 21 11 11 11 11 11 11   001  2161   161  2261  001  001 601-9999-99999-99999
KL01192200001020000102371102331102251102324  791  271 524   2 1  391  731  741  654  36 6  59 6  63 6 774 834 884 834 954 814 854 874 961 801 85116 3116 2116 31 274 41-99-99 71-99-99 71-99-99 604  711 61 81 81 11 11 11 01 11 01   001   001   001   001  001  001 701-9999-99999-99999
KL01192200001030000101981101771101581101784  911  691 224  67 1  781  871  711  774  61 6  70 6  54 6 834 894 784 834 784 794 784 784 781 791 78116 3120 5116 31 374 81-99-99 81-99-99 61-99-99 734  001 81 81 81 11 12 11 11 01 01   001   061   001  2861  001  0011401-9999-99999-99999
KL01192200001040000101301100891100971101054  921  661 264  58 1  681  771  831  784  63 6  72 6  64 6 924 984 834 914 934 934 764 874 931 931 76120 3116 3120 41 334 81-99-99 81-99-99 61-99-99 734  001 61 61 81 21 21 21 11 01 31  2861  6861  2161  9761  001  0011701-9999-99999-99999
KL01192200001050000101651101581101421101554  861   41 824 -22 1   71  801  611  524   2 6  55 6  47 6 594 744 764 704 914 684 814 804 921 681 80120 3116 3116 31 304 11-99-99 71-99-99 31-99-99 374 6511 81 81 81 11 11 11 01 11 01   861   001   001   001  001  0011101-9999-99999-99999
KL01192200001060000101221101091101351101224 1041  431 614  25 1  501  971  781  764  40 6  74 6  70 6 754 884 954 864 864 734 904 834 851 731 89116 4116 4116 31 374 31-99-99 71-99-99 81-99-99 604 2811 81 81 61 11 11 11 11 11 01   001   001  1561  1861  001  0011201-9999-99999-99999
KL01192200001070000101771101731101611101704  871  471 404  42 1  551  871  611  664  49 6  64 6  43 6 834 814 714 784 914 724 754 794 911 721 75120 3120 3116 31 304 71-99-99 51-99-99 71-99-99 634 2511 61 81 81 11 11 11 11 11 01   361   001   001   001  001  001 901-9999-99999-99999
KL01192200001080000101341101361101531101414  791  541 254  45 1  591  681  681  664  39 6  61 6  62 6 674 904 914 834 724 914 924 854 731 901 92116 4116 3120 21 304 71-99-99 81-99-99 81-99-99 774  001 83 53 73 11 11 11 11 01 12   001   261  1061  1261  001  0011001-9999-99999-99999
KL01192200001090000101701102051102541102104  681   51 634   4 1  261  431  161  254  23 6  40 6  12 6 704 794 644 714 954 954 934 944 951 951 94120 1132 21 0 01 104 41-99-99 71-99-99 71-99-99 604  001 51 51 31 11 11 11 11 11 11   061   001   001   061  001  001 501-9999-99999-99999
KL01192200001100000102991103171103261103144  241   41 204  10 1   81  121  121  114   6 6  10 6   9 6 624 644 634 634 964 974 954 964 961 961 95116 1124 1120 21 134 81-99-99 81-99-99 81-99-99 804  001 21 31 61 11 11 11 11 01 11   061   001   001   001  001  001 401-9999-99999-99999
    \end{Verbatim}

    \begin{Verbatim}[commandchars=\\\{\}]
{\color{incolor}In [{\color{incolor}273}]:} \PY{k+kn}{from} \PY{n+nn}{IPython.display} \PY{k+kn}{import} \PY{n}{display}\PY{p}{,} \PY{n}{HTML} 
          \PY{n}{HTML}\PY{p}{(}\PY{l+s}{\PYZsq{}}\PY{l+s}{http://www.dwd.de/bvbw/generator/DWDWWW/Content/Oeffentlichkeit/KU/KU2/KU21/klimadaten/german/format\PYZus{}\PYZus{}kl,templateId=raw,property=publicationFile.html/format\PYZus{}kl.html}\PY{l+s}{\PYZsq{}}\PY{p}{)}
\end{Verbatim}

            \begin{Verbatim}[commandchars=\\\{\}]
{\color{outcolor}Out[{\color{outcolor}273}]:} <IPython.core.display.HTML at 0x7f2761083710>
\end{Verbatim}
        
    \begin{Verbatim}[commandchars=\\\{\}]
{\color{incolor}In [{\color{incolor}274}]:} \PY{n}{table\PYZus{}description} \PY{o}{=} \PY{n}{pd}\PY{o}{.}\PY{n}{read\PYZus{}html}\PY{p}{(}\PY{l+s}{\PYZsq{}}\PY{l+s}{http://www.dwd.de/bvbw/generator/DWDWWW/Content/Oeffentlichkeit/KU/KU2/KU21/klimadaten/german/format\PYZus{}\PYZus{}kl,templateId=raw,property=publicationFile.html/format\PYZus{}kl.html}\PY{l+s}{\PYZsq{}}\PY{p}{,}\PY{n}{skiprows}\PY{o}{=}\PY{l+m+mi}{1}\PY{p}{,}\PY{n}{header}\PY{o}{=}\PY{l+m+mi}{1}\PY{p}{)}\PY{p}{[}\PY{l+m+mi}{0}\PY{p}{]}
\end{Verbatim}

    \begin{Verbatim}[commandchars=\\\{\}]
{\color{incolor}In [{\color{incolor}275}]:} \PY{n}{table\PYZus{}description}\PY{o}{.}\PY{n}{head}\PY{p}{(}\PY{p}{)}
\end{Verbatim}

            \begin{Verbatim}[commandchars=\\\{\}]
{\color{outcolor}Out[{\color{outcolor}275}]:}    KL    KE KE.1 Kennung fuer das Datenkollektiv Unnamed: 4     Unnamed: 5  \textbackslash{}
          0  KL  STAT   ST                  Stationsnummer       CODE  STATIONSLISTE   
          1  KL    JA   JA                            Jahr        NaN            NaN   
          2  KL    MO   MO                           Monat        NaN            NaN   
          3  KL    TA   TA                             Tag        NaN            NaN   
          4  KL   NaN  NaN        numerisches Leerfeld (0)        NaN            NaN   
          
            Standardformat: Klimadaten aus Klimaroutine des DWD (3 Termine: 07,14,21 MOZ, ab 01.01.1987 07:30,14:30,21:30 MEZ) und Tageswerte jeweils nach Beobachteranleitung fuer Klimastationen (BAK). nebenamtl.Stationen auch teilweise nach 03.2001). Satzlaenge: 282 Zeichen  \textbackslash{}
          0                                                NaN                                                                                                                                                                                                                        
          1                                                NaN                                                                                                                                                                                                                        
          2                                                NaN                                                                                                                                                                                                                        
          3                                                NaN                                                                                                                                                                                                                        
          4                                                NaN                                                                                                                                                                                                                        
          
             X(2)   1 siehe KE\_IND  Unnamed: 10  
          0  9(5)   3  00001-99999          NaN  
          1  9(4)   8    1800-2100          NaN  
          2  9(2)  12        01-12          NaN  
          3  9(2)  14        01-31          NaN  
          4  9(4)  16         0000          NaN  
\end{Verbatim}
        
    \begin{Verbatim}[commandchars=\\\{\}]
{\color{incolor}In [{\color{incolor}276}]:} \PY{n}{widths} \PY{o}{=} \PY{n}{table\PYZus{}description}\PY{o}{.}\PY{n}{iloc}\PY{p}{[}\PY{p}{:}\PY{p}{,}\PY{l+m+mi}{8}\PY{p}{]}\PY{o}{.}\PY{n}{diff}\PY{p}{(}\PY{p}{)}\PY{o}{.}\PY{n}{values}
          \PY{n}{widths}
\end{Verbatim}

            \begin{Verbatim}[commandchars=\\\{\}]
{\color{outcolor}Out[{\color{outcolor}276}]:} array([ nan,   5.,   4.,   2.,   2.,   4.,   5.,   1.,   5.,   1.,   5.,
                   1.,   5.,   1.,   4.,   1.,   4.,   1.,   3.,   1.,   4.,   1.,
                   1.,   4.,   1.,   4.,   1.,   4.,   1.,   4.,   1.,   4.,   1.,
                   1.,   4.,   1.,   1.,   4.,   1.,   1.,   3.,   1.,   3.,   1.,
                   3.,   1.,   3.,   1.,   3.,   1.,   3.,   1.,   3.,   1.,   3.,
                   1.,   3.,   1.,   3.,   1.,   3.,   1.,   2.,   2.,   1.,   2.,
                   2.,   1.,   2.,   2.,   1.,   3.,   1.,   2.,   1.,   2.,   1.,
                   2.,   1.,   2.,   1.,   2.,   1.,   2.,   1.,   2.,   1.,   2.,
                   1.,   2.,   1.,   3.,   1.,   3.,   1.,   1.,   2.,   1.,   2.,
                   1.,   2.,   1.,   2.,   1.,   2.,   1.,   2.,   1.,   2.,   1.,
                   2.,   1.,   2.,   1.,   4.,   1.,   1.,   4.,   1.,   1.,   4.,
                   1.,   1.,   4.,   1.,   1.,   3.,   1.,   1.,   3.,   1.,   1.,
                   3.,   1.,   4.,   1.,   5.,   1.,   5.])
\end{Verbatim}
        
    \begin{Verbatim}[commandchars=\\\{\}]
{\color{incolor}In [{\color{incolor}277}]:} \PY{n}{col\PYZus{}names} \PY{o}{=} \PY{n}{table\PYZus{}description}\PY{o}{.}\PY{n}{ix}\PY{p}{[}\PY{p}{:}\PY{p}{,}\PY{l+m+mi}{3}\PY{p}{]}
          \PY{n}{widths}\PY{o}{=}\PY{p}{[}\PY{l+m+mi}{2}\PY{p}{,}\PY{l+m+mi}{5}\PY{p}{,}\PY{l+m+mi}{4}\PY{p}{,}\PY{l+m+mi}{2}\PY{p}{,}\PY{l+m+mi}{2}\PY{p}{,}\PY{l+m+mi}{4}\PY{p}{,}\PY{l+m+mi}{5}\PY{p}{,}\PY{l+m+mi}{1}\PY{p}{,}\PY{l+m+mi}{5}\PY{p}{,}\PY{l+m+mi}{1}\PY{p}{,}\PY{l+m+mi}{5}\PY{p}{,}\PY{l+m+mi}{1}\PY{p}{,}\PY{l+m+mi}{5}\PY{p}{,}\PY{l+m+mi}{1}\PY{p}{,}\PY{l+m+mi}{4}\PY{p}{,}\PY{l+m+mi}{1}\PY{p}{,}\PY{l+m+mi}{4}\PY{p}{,}\PY{l+m+mi}{1}\PY{p}{,}\PY{l+m+mi}{3}\PY{p}{,}\PY{l+m+mi}{1}\PY{p}{,}\PY{l+m+mi}{4}\PY{p}{,}\PY{l+m+mi}{1}\PY{p}{,}\PY{l+m+mi}{1}\PY{p}{,}\PY{l+m+mi}{4}\PY{p}{,}\PY{l+m+mi}{1}\PY{p}{,}\PY{l+m+mi}{4}\PY{p}{,}\PY{l+m+mi}{1}\PY{p}{,}\PY{l+m+mi}{4}\PY{p}{,}\PY{l+m+mi}{1}\PY{p}{,}\PY{l+m+mi}{4}\PY{p}{,}\PY{l+m+mi}{1}\PY{p}{,}\PY{l+m+mi}{4}\PY{p}{,}\PY{l+m+mi}{1}\PY{p}{,}\PY{l+m+mi}{1}\PY{p}{,}\PY{l+m+mi}{4}\PY{p}{,}\PY{l+m+mi}{1}\PY{p}{,}\PY{l+m+mi}{1}\PY{p}{,}\PY{l+m+mi}{4}\PY{p}{,}\PY{l+m+mi}{1}\PY{p}{,}\PY{l+m+mi}{1}\PY{p}{,}\PY{l+m+mi}{3}\PY{p}{,}\PY{l+m+mi}{1}\PY{p}{,}\PY{l+m+mi}{3}\PY{p}{,}\PY{l+m+mi}{1}\PY{p}{,}\PY{l+m+mi}{3}\PY{p}{,}\PY{l+m+mi}{1}\PY{p}{,}\PY{l+m+mi}{3}\PY{p}{,}\PY{l+m+mi}{1}\PY{p}{,}\PY{l+m+mi}{3}\PY{p}{,}\PY{l+m+mi}{1}\PY{p}{,}\PY{l+m+mi}{3}\PY{p}{,}\PY{l+m+mi}{1}\PY{p}{,}\PY{l+m+mi}{3}\PY{p}{,}\PY{l+m+mi}{1}\PY{p}{,}\PY{l+m+mi}{3}\PY{p}{,}\PY{l+m+mi}{1}\PY{p}{,}\PY{l+m+mi}{3}\PY{p}{,}\PY{l+m+mi}{1}\PY{p}{,}\PY{l+m+mi}{3}\PY{p}{,}\PY{l+m+mi}{1}\PY{p}{,}\PY{l+m+mi}{3}\PY{p}{,}\PY{l+m+mi}{1}\PY{p}{]}
          \PY{n}{df\PYZus{}temp} \PY{o}{=} \PY{n}{pd}\PY{o}{.}\PY{n}{read\PYZus{}fwf}\PY{p}{(}\PY{l+s}{\PYZsq{}}\PY{l+s}{./data/kl\PYZus{}10400\PYZus{}00\PYZus{}akt\PYZus{}txt.txt}\PY{l+s}{\PYZsq{}}\PY{p}{,}\PY{n}{widths}\PY{o}{=}\PY{n}{widths}\PY{p}{)}
\end{Verbatim}

    \begin{Verbatim}[commandchars=\\\{\}]
{\color{incolor}In [{\color{incolor}278}]:} \PY{n}{df\PYZus{}temp}\PY{o}{.}\PY{n}{head}\PY{p}{(}\PY{p}{)}
\end{Verbatim}

            \begin{Verbatim}[commandchars=\\\{\}]
{\color{outcolor}Out[{\color{outcolor}278}]:}    KL  01192  2000  01  01.1  0000  10170  1  10186  1.1  \ldots   95  4.9  92  \textbackslash{}
          0  KL   1192  2000   1     2     0  10237  1  10233    1  \ldots   85    4  87   
          1  KL   1192  2000   1     3     0  10198  1  10177    1  \ldots   78    4  78   
          2  KL   1192  2000   1     4     0  10130  1  10089    1  \ldots   76    4  87   
          3  KL   1192  2000   1     5     0  10165  1  10158    1  \ldots   81    4  80   
          4  KL   1192  2000   1     6     0  10122  1  10109    1  \ldots   90    4  83   
          
             4.10  86.1  1.9  95.1  1.10  95.2  1.11  
          0     4    96    1    80     1    85     1  
          1     4    78    1    79     1    78     1  
          2     4    93    1    93     1    76     1  
          3     4    92    1    68     1    80     1  
          4     4    85    1    73     1    89     1  
          
          [5 rows x 62 columns]
\end{Verbatim}
        
    \begin{Verbatim}[commandchars=\\\{\}]
{\color{incolor}In [{\color{incolor}279}]:} \PY{n}{df\PYZus{}temp} \PY{o}{=} \PY{n}{pd}\PY{o}{.}\PY{n}{read\PYZus{}fwf}\PY{p}{(}\PY{l+s}{\PYZsq{}}\PY{l+s}{./data/kl\PYZus{}10400\PYZus{}00\PYZus{}akt\PYZus{}txt.txt}\PY{l+s}{\PYZsq{}}\PY{p}{,}\PY{n}{widths}\PY{o}{=}\PY{n}{widths}\PY{p}{,}\PY{n}{header}\PY{o}{=}\PY{n+nb+bp}{None}\PY{p}{,}\PY{n}{parse\PYZus{}dates} \PY{o}{=} \PY{p}{[}\PY{p}{[}\PY{l+m+mi}{2}\PY{p}{,}\PY{l+m+mi}{3}\PY{p}{,}\PY{l+m+mi}{4}\PY{p}{]}\PY{p}{]}\PY{p}{,} \PY{n}{index\PYZus{}col} \PY{o}{=} \PY{l+m+mi}{0}\PY{p}{)}
          \PY{n}{df\PYZus{}temp}\PY{o}{.}\PY{n}{ix}\PY{p}{[}\PY{p}{:}\PY{l+m+mi}{5}\PY{p}{,}\PY{p}{[}\PY{l+m+mi}{14}\PY{p}{,}\PY{l+m+mi}{16}\PY{p}{,}\PY{l+m+mi}{29}\PY{p}{]}\PY{p}{]}
\end{Verbatim}

            \begin{Verbatim}[commandchars=\\\{\}]
{\color{outcolor}Out[{\color{outcolor}279}]:}             14  16  29
          2\_3\_4                 
          2000-01-01  69  25  42
          2000-01-02  79  27  65
          2000-01-03  91  69  77
          2000-01-04  92  66  78
          2000-01-05  86   4  52
\end{Verbatim}
        
    \begin{Verbatim}[commandchars=\\\{\}]
{\color{incolor}In [{\color{incolor}280}]:} \PY{n}{df\PYZus{}temp} \PY{o}{=} \PY{n}{df\PYZus{}temp}\PY{p}{[}\PY{p}{[}\PY{l+m+mi}{14}\PY{p}{,}\PY{l+m+mi}{16}\PY{p}{,}\PY{l+m+mi}{29}\PY{p}{]}\PY{p}{]}\PY{o}{.}\PY{n}{apply}\PY{p}{(}\PY{k}{lambda} \PY{n}{x}\PY{p}{:} \PY{n}{x}\PY{o}{/}\PY{l+m+mf}{10.}\PY{p}{)}
          \PY{n}{df\PYZus{}temp}\PY{o}{.}\PY{n}{head}\PY{p}{(}\PY{p}{)}
\end{Verbatim}

            \begin{Verbatim}[commandchars=\\\{\}]
{\color{outcolor}Out[{\color{outcolor}280}]:}              14   16   29
          2\_3\_4                    
          2000-01-01  6.9  2.5  4.2
          2000-01-02  7.9  2.7  6.5
          2000-01-03  9.1  6.9  7.7
          2000-01-04  9.2  6.6  7.8
          2000-01-05  8.6  0.4  5.2
\end{Verbatim}
        
    \begin{Verbatim}[commandchars=\\\{\}]
{\color{incolor}In [{\color{incolor}281}]:} \PY{n}{df\PYZus{}temp}\PY{o}{.}\PY{n}{columns} \PY{o}{=} \PY{p}{[}\PY{l+s}{\PYZsq{}}\PY{l+s}{HighTemp}\PY{l+s}{\PYZsq{}}\PY{p}{,}\PY{l+s}{\PYZsq{}}\PY{l+s}{LowTemp}\PY{l+s}{\PYZsq{}}\PY{p}{,}\PY{l+s}{\PYZsq{}}\PY{l+s}{MeanTemp}\PY{l+s}{\PYZsq{}}\PY{p}{]}
          \PY{n}{df\PYZus{}temp}\PY{o}{.}\PY{n}{head}\PY{p}{(}\PY{p}{)}
\end{Verbatim}

            \begin{Verbatim}[commandchars=\\\{\}]
{\color{outcolor}Out[{\color{outcolor}281}]:}             HighTemp  LowTemp  MeanTemp
          2\_3\_4                                  
          2000-01-01       6.9      2.5       4.2
          2000-01-02       7.9      2.7       6.5
          2000-01-03       9.1      6.9       7.7
          2000-01-04       9.2      6.6       7.8
          2000-01-05       8.6      0.4       5.2
\end{Verbatim}
        
    \begin{Verbatim}[commandchars=\\\{\}]
{\color{incolor}In [{\color{incolor}282}]:} \PY{n}{df\PYZus{}temp}\PY{o}{.}\PY{n}{index}\PY{o}{.}\PY{n}{name} \PY{o}{=} \PY{l+s}{\PYZsq{}}\PY{l+s}{Date}\PY{l+s}{\PYZsq{}}
          \PY{n}{df\PYZus{}temp}\PY{o}{.}\PY{n}{head}\PY{p}{(}\PY{p}{)}
\end{Verbatim}

            \begin{Verbatim}[commandchars=\\\{\}]
{\color{outcolor}Out[{\color{outcolor}282}]:}             HighTemp  LowTemp  MeanTemp
          Date                                   
          2000-01-01       6.9      2.5       4.2
          2000-01-02       7.9      2.7       6.5
          2000-01-03       9.1      6.9       7.7
          2000-01-04       9.2      6.6       7.8
          2000-01-05       8.6      0.4       5.2
\end{Verbatim}
        
    \begin{Verbatim}[commandchars=\\\{\}]
{\color{incolor}In [{\color{incolor}283}]:} \PY{n}{df2} \PY{o}{=} \PY{n}{df}\PY{o}{.}\PY{n}{join}\PY{p}{(}\PY{n}{df\PYZus{}temp}\PY{p}{,} \PY{n}{how}\PY{o}{=}\PY{l+s}{\PYZsq{}}\PY{l+s}{left}\PY{l+s}{\PYZsq{}}\PY{p}{)}
\end{Verbatim}

    \begin{Verbatim}[commandchars=\\\{\}]
{\color{incolor}In [{\color{incolor}284}]:} \PY{k+kn}{from} \PY{n+nn}{statsmodels.tsa.api} \PY{k+kn}{import} \PY{o}{*}
\end{Verbatim}

    \begin{Verbatim}[commandchars=\\\{\}]
{\color{incolor}In [{\color{incolor}285}]:} \PY{n}{df2}\PY{p}{[}\PY{l+s}{\PYZsq{}}\PY{l+s}{AmpTemp}\PY{l+s}{\PYZsq{}}\PY{p}{]} \PY{o}{=} \PY{n}{df2}\PY{o}{.}\PY{n}{HighTemp}\PY{o}{\PYZhy{}}\PY{n}{df2}\PY{o}{.}\PY{n}{LowTemp}
          \PY{n}{data} \PY{o}{=} \PY{n}{df2}\PY{p}{[}\PY{p}{[}\PY{l+s}{\PYZsq{}}\PY{l+s}{AveragePriceEUR}\PY{l+s}{\PYZsq{}}\PY{p}{,}\PY{l+s}{\PYZsq{}}\PY{l+s}{VolumeMW}\PY{l+s}{\PYZsq{}}\PY{p}{,}\PY{l+s}{\PYZsq{}}\PY{l+s}{AmpTemp}\PY{l+s}{\PYZsq{}}\PY{p}{]}\PY{p}{]}\PY{o}{.}\PY{n}{asfreq}\PY{p}{(}\PY{l+s}{\PYZsq{}}\PY{l+s}{D}\PY{l+s}{\PYZsq{}}\PY{p}{)}
          \PY{n}{model} \PY{o}{=} \PY{n}{VAR}\PY{p}{(}\PY{n}{data}\PY{p}{,}\PY{n}{missing}\PY{o}{=}\PY{l+s}{\PYZsq{}}\PY{l+s}{drop}\PY{l+s}{\PYZsq{}}\PY{p}{)}     \PY{c}{\PYZsh{} NaN will produce LinalgError, hence the missing=\PYZsq{}drop\PYZsq{}}
\end{Verbatim}

    \begin{Verbatim}[commandchars=\\\{\}]
{\color{incolor}In [{\color{incolor}286}]:} \PY{n}{results}\PY{o}{=} \PY{n}{model}\PY{o}{.}\PY{n}{fit}\PY{p}{(}\PY{p}{)}
          \PY{n}{results}\PY{o}{.}\PY{n}{plot}\PY{p}{(}\PY{p}{)}
\end{Verbatim}

    \begin{center}
    \adjustimage{max size={0.9\linewidth}{0.9\paperheight}}{VI-Pandas_files/VI-Pandas_83_0.pdf}
    \end{center}
    { \hspace*{\fill} \\}
    
    \begin{Verbatim}[commandchars=\\\{\}]
{\color{incolor}In [{\color{incolor}287}]:} \PY{n}{results}\PY{o}{.}\PY{n}{plot\PYZus{}acorr}\PY{p}{(}\PY{p}{)}
\end{Verbatim}

    \begin{center}
    \adjustimage{max size={0.9\linewidth}{0.9\paperheight}}{VI-Pandas_files/VI-Pandas_84_0.pdf}
    \end{center}
    { \hspace*{\fill} \\}
    
    \begin{Verbatim}[commandchars=\\\{\}]
{\color{incolor}In [{\color{incolor}288}]:} \PY{n}{lag\PYZus{}order} \PY{o}{=} \PY{n}{results}\PY{o}{.}\PY{n}{k\PYZus{}ar}
          \PY{n}{results}\PY{o}{.}\PY{n}{forecast}\PY{p}{(}\PY{n}{data}\PY{o}{.}\PY{n}{values}\PY{p}{[}\PY{o}{\PYZhy{}}\PY{n}{lag\PYZus{}order}\PY{p}{:}\PY{p}{]}\PY{p}{,}\PY{l+m+mi}{2}\PY{p}{)}
\end{Verbatim}

            \begin{Verbatim}[commandchars=\\\{\}]
{\color{outcolor}Out[{\color{outcolor}288}]:} array([[  25.47955327,  844.24536267,    5.46105252],
                 [  31.77881602,  731.28635887,    6.682358  ]])
\end{Verbatim}
        
    \begin{Verbatim}[commandchars=\\\{\}]
{\color{incolor}In [{\color{incolor}289}]:} \PY{n}{results}\PY{o}{.}\PY{n}{plot\PYZus{}forecast}\PY{p}{(}\PY{l+m+mi}{2}\PY{p}{)}
          \PY{n}{legend}\PY{p}{(}\PY{n}{loc}\PY{o}{=}\PY{l+s}{\PYZsq{}}\PY{l+s}{best}\PY{l+s}{\PYZsq{}}\PY{p}{)}
\end{Verbatim}

            \begin{Verbatim}[commandchars=\\\{\}]
{\color{outcolor}Out[{\color{outcolor}289}]:} <matplotlib.legend.Legend at 0x7f274ffa3a10>
\end{Verbatim}
        
    \begin{center}
    \adjustimage{max size={0.9\linewidth}{0.9\paperheight}}{VI-Pandas_files/VI-Pandas_86_1.pdf}
    \end{center}
    { \hspace*{\fill} \\}
    
    \begin{Verbatim}[commandchars=\\\{\}]
{\color{incolor}In [{\color{incolor}290}]:} \PY{n}{model} \PY{o}{=} \PY{n}{pd}\PY{o}{.}\PY{n}{ols}\PY{p}{(}\PY{n}{y}\PY{o}{=}\PY{n}{df2}\PY{o}{.}\PY{n}{AveragePriceEUR}\PY{p}{,} \PY{n}{x} \PY{o}{=} \PY{n}{df2}\PY{p}{[}\PY{p}{[}\PY{l+s}{\PYZsq{}}\PY{l+s}{AmpTemp}\PY{l+s}{\PYZsq{}}\PY{p}{,}\PY{l+s}{\PYZsq{}}\PY{l+s}{VolumeMW}\PY{l+s}{\PYZsq{}}\PY{p}{]}\PY{p}{]}\PY{p}{)}
\end{Verbatim}

    \begin{Verbatim}[commandchars=\\\{\}]
{\color{incolor}In [{\color{incolor}291}]:} \PY{k}{print}\PY{p}{(}\PY{n}{model}\PY{p}{)}
\end{Verbatim}

    \begin{Verbatim}[commandchars=\\\{\}]
-------------------------Summary of Regression Analysis-------------------------

Formula: Y \textasciitilde{} <AmpTemp> + <VolumeMW> + <intercept>

Number of Observations:         2011
Number of Degrees of Freedom:   3

R-squared:         0.0013
Adj R-squared:     0.0003

Rmse:             17.0065

F-stat (2, 2008):     1.3320, p-value:     0.2642

Degrees of Freedom: model 2, resid 2008

-----------------------Summary of Estimated Coefficients------------------------
      Variable       Coef    Std Err     t-stat    p-value    CI 2.5\%   CI 97.5\%
--------------------------------------------------------------------------------
       AmpTemp     0.1553     0.0954       1.63     0.1039    -0.0318     0.3423
      VolumeMW     0.0002     0.0006       0.27     0.7891    -0.0010     0.0013
     intercept    35.0410     0.9885      35.45     0.0000    33.1035    36.9784
---------------------------------End of Summary---------------------------------
    \end{Verbatim}

    \begin{Verbatim}[commandchars=\\\{\}]
{\color{incolor}In [{\color{incolor}292}]:} \PY{n}{model}\PY{o}{.}\PY{n}{y\PYZus{}fitted}\PY{o}{.}\PY{n}{plot}\PY{p}{(}\PY{p}{)}
          \PY{n}{model}\PY{o}{.}\PY{n}{y}\PY{o}{.}\PY{n}{plot}\PY{p}{(}\PY{n}{style}\PY{o}{=}\PY{l+s}{\PYZsq{}}\PY{l+s}{k}\PY{l+s}{\PYZsq{}}\PY{p}{)}
          \PY{n}{model}\PY{o}{.}\PY{n}{y\PYZus{}predict}\PY{o}{.}\PY{n}{plot}\PY{p}{(}\PY{n}{style}\PY{o}{=}\PY{l+s}{\PYZsq{}}\PY{l+s}{red}\PY{l+s}{\PYZsq{}}\PY{p}{)}
\end{Verbatim}

            \begin{Verbatim}[commandchars=\\\{\}]
{\color{outcolor}Out[{\color{outcolor}292}]:} <matplotlib.axes.\_subplots.AxesSubplot at 0x7f275416e710>
\end{Verbatim}
        
    \begin{center}
    \adjustimage{max size={0.9\linewidth}{0.9\paperheight}}{VI-Pandas_files/VI-Pandas_89_1.pdf}
    \end{center}
    { \hspace*{\fill} \\}
    
    \begin{Verbatim}[commandchars=\\\{\}]
{\color{incolor}In [{\color{incolor}3}]:} \PY{o}{\PYZpc{}}\PY{k}{reload\PYZus{}ext} \PY{n}{version\PYZus{}information}
        
        \PY{o}{\PYZpc{}}\PY{k}{version\PYZus{}information} \PY{n}{numpy}\PY{p}{,} \PY{n}{scipy}\PY{p}{,} \PY{n}{matplotlib}\PY{p}{,} \PY{n}{pandas}\PY{p}{,} \PY{n}{statsmodels}
\end{Verbatim}
\texttt{\color{outcolor}Out[{\color{outcolor}3}]:}
    
    \begin{tabular}{|l|l|}\hline
{\bf Software} & {\bf Version} \\ \hline\hline
Python & 2.7.8 |Anaconda 2.1.0 (64-bit)| (default, Aug 21 2014, 18:22:21) [GCC 4.4.7 20120313 (Red Hat 4.4.7-1)] \\ \hline
IPython & 2.3.0 \\ \hline
OS & posix [linux2] \\ \hline
numpy & 1.9.1 \\ \hline
scipy & 0.14.0 \\ \hline
matplotlib & 1.4.2 \\ \hline
pandas & 0.15.0 \\ \hline
statsmodels & 0.5.0 \\ \hline
\hline \multicolumn{2}{|l|}{Thu Nov 13 10:39:56 2014 CET} \\ \hline
\end{tabular}


    

    \emph{The full notebook can be downloaded}
\href{https://raw.github.com/PoeticCapybara/Python-Introduction-Zittau/master/Lecture-7-Pandas.ipynb}{\emph{here}},
\emph{or viewed statically on}
\href{http://nbviewer.ipython.org/urls/raw.github.com/PoeticCapybara/Python-Introduction-Zittau/master/Lecture-7-Pandas.ipynb}{\emph{nbviewer}}

    \begin{Verbatim}[commandchars=\\\{\}]
{\color{incolor}In [{\color{incolor}148}]:} \PY{n}{df} \PY{o}{=} \PY{n}{pd}\PY{o}{.}\PY{n}{read\PYZus{}html}\PY{p}{(}\PY{l+s}{\PYZsq{}}\PY{l+s}{lista.html}\PY{l+s}{\PYZsq{}}\PY{p}{)}\PY{p}{[}\PY{l+m+mi}{6}\PY{p}{]}
          \PY{n}{df}\PY{o}{.}\PY{n}{head}\PY{p}{(}\PY{p}{)}
\end{Verbatim}

            \begin{Verbatim}[commandchars=\\\{\}]
{\color{outcolor}Out[{\color{outcolor}148}]:}   Stations-Kennziffer  Klima-Kennung ICAO-Kennung    Stationsname  \textbackslash{}
          0               10501           2205          NaN          Aachen   
          1               10505           2206          NaN  Aachen-Orsbach   
          2               10291           3058          NaN      Angermünde   
          3               10091           3005          NaN          Arkona   
          4               10852           4128         EDMA        Augsburg   
          
             Stationshöhe in Metern geogr. Breite geogr. Länge  \textbackslash{}
          0                     202       50° 47'      06° 05'   
          1                     231       50° 47'      06° 01'   
          2                      54       53° 01'      13° 59'   
          3                      42       54° 40'      13° 26'   
          4                     462       48° 25'      10° 56'   
          
            Automat für Lufttemperatur\textbackslash{}n\textbackslash{}nseit:\textbackslash{}n  Beginn Klimareihe  
          0                            01.07.1993               1891  
          1                                   NaN               2011  
          2                                   NaN               1947  
          3                                   NaN               1947  
          4                            10.11.1996               1947  
\end{Verbatim}
        
    \begin{Verbatim}[commandchars=\\\{\}]
{\color{incolor}In [{\color{incolor}1}]:} \PY{k+kn}{from} \PY{n+nn}{IPython.core.display} \PY{k+kn}{import} \PY{n}{HTML}
        \PY{k}{def} \PY{n+nf}{css\PYZus{}styling}\PY{p}{(}\PY{p}{)}\PY{p}{:}
            \PY{n}{styles} \PY{o}{=} \PY{n+nb}{open}\PY{p}{(}\PY{l+s}{\PYZdq{}}\PY{l+s}{./styles/custom.css}\PY{l+s}{\PYZdq{}}\PY{p}{,} \PY{l+s}{\PYZdq{}}\PY{l+s}{r}\PY{l+s}{\PYZdq{}}\PY{p}{)}\PY{o}{.}\PY{n}{read}\PY{p}{(}\PY{p}{)}
            \PY{k}{return} \PY{n}{HTML}\PY{p}{(}\PY{n}{styles}\PY{p}{)}
        \PY{n}{css\PYZus{}styling}\PY{p}{(}\PY{p}{)}
\end{Verbatim}

            \begin{Verbatim}[commandchars=\\\{\}]
{\color{outcolor}Out[{\color{outcolor}1}]:} <IPython.core.display.HTML at 0x7fc1784fd7d0>
\end{Verbatim}
        
    \hyperref[Top]{Back to top}

    \begin{Verbatim}[commandchars=\\\{\}]
{\color{incolor}In [{\color{incolor}}]:} 
\end{Verbatim}


    % Add a bibliography block to the postdoc
    
    
    
    \end{document}


%\newpage
%
% Default to the notebook output style

    


% Inherit from the specified cell style.




    
\documentclass{article}

    
    
    \usepackage{graphicx} % Used to insert images
    \usepackage{adjustbox} % Used to constrain images to a maximum size 
    \usepackage{color} % Allow colors to be defined
    \usepackage{enumerate} % Needed for markdown enumerations to work
    \usepackage{geometry} % Used to adjust the document margins
    \usepackage{amsmath} % Equations
    \usepackage{amssymb} % Equations
    \usepackage[mathletters]{ucs} % Extended unicode (utf-8) support
    \usepackage[utf8x]{inputenc} % Allow utf-8 characters in the tex document
    \usepackage{fancyvrb} % verbatim replacement that allows latex
    \usepackage{grffile} % extends the file name processing of package graphics 
                         % to support a larger range 
    % The hyperref package gives us a pdf with properly built
    % internal navigation ('pdf bookmarks' for the table of contents,
    % internal cross-reference links, web links for URLs, etc.)
    \usepackage{hyperref}
    \usepackage{longtable} % longtable support required by pandoc >1.10
    \usepackage{booktabs}  % table support for pandoc > 1.12.2
    

    
    
    \definecolor{orange}{cmyk}{0,0.4,0.8,0.2}
    \definecolor{darkorange}{rgb}{.71,0.21,0.01}
    \definecolor{darkgreen}{rgb}{.12,.54,.11}
    \definecolor{myteal}{rgb}{.26, .44, .56}
    \definecolor{gray}{gray}{0.45}
    \definecolor{lightgray}{gray}{.95}
    \definecolor{mediumgray}{gray}{.8}
    \definecolor{inputbackground}{rgb}{.95, .95, .85}
    \definecolor{outputbackground}{rgb}{.95, .95, .95}
    \definecolor{traceback}{rgb}{1, .95, .95}
    % ansi colors
    \definecolor{red}{rgb}{.6,0,0}
    \definecolor{green}{rgb}{0,.65,0}
    \definecolor{brown}{rgb}{0.6,0.6,0}
    \definecolor{blue}{rgb}{0,.145,.698}
    \definecolor{purple}{rgb}{.698,.145,.698}
    \definecolor{cyan}{rgb}{0,.698,.698}
    \definecolor{lightgray}{gray}{0.5}
    
    % bright ansi colors
    \definecolor{darkgray}{gray}{0.25}
    \definecolor{lightred}{rgb}{1.0,0.39,0.28}
    \definecolor{lightgreen}{rgb}{0.48,0.99,0.0}
    \definecolor{lightblue}{rgb}{0.53,0.81,0.92}
    \definecolor{lightpurple}{rgb}{0.87,0.63,0.87}
    \definecolor{lightcyan}{rgb}{0.5,1.0,0.83}
    
    % commands and environments needed by pandoc snippets
    % extracted from the output of `pandoc -s`
    \DefineVerbatimEnvironment{Highlighting}{Verbatim}{commandchars=\\\{\}}
    % Add ',fontsize=\small' for more characters per line
    \newenvironment{Shaded}{}{}
    \newcommand{\KeywordTok}[1]{\textcolor[rgb]{0.00,0.44,0.13}{\textbf{{#1}}}}
    \newcommand{\DataTypeTok}[1]{\textcolor[rgb]{0.56,0.13,0.00}{{#1}}}
    \newcommand{\DecValTok}[1]{\textcolor[rgb]{0.25,0.63,0.44}{{#1}}}
    \newcommand{\BaseNTok}[1]{\textcolor[rgb]{0.25,0.63,0.44}{{#1}}}
    \newcommand{\FloatTok}[1]{\textcolor[rgb]{0.25,0.63,0.44}{{#1}}}
    \newcommand{\CharTok}[1]{\textcolor[rgb]{0.25,0.44,0.63}{{#1}}}
    \newcommand{\StringTok}[1]{\textcolor[rgb]{0.25,0.44,0.63}{{#1}}}
    \newcommand{\CommentTok}[1]{\textcolor[rgb]{0.38,0.63,0.69}{\textit{{#1}}}}
    \newcommand{\OtherTok}[1]{\textcolor[rgb]{0.00,0.44,0.13}{{#1}}}
    \newcommand{\AlertTok}[1]{\textcolor[rgb]{1.00,0.00,0.00}{\textbf{{#1}}}}
    \newcommand{\FunctionTok}[1]{\textcolor[rgb]{0.02,0.16,0.49}{{#1}}}
    \newcommand{\RegionMarkerTok}[1]{{#1}}
    \newcommand{\ErrorTok}[1]{\textcolor[rgb]{1.00,0.00,0.00}{\textbf{{#1}}}}
    \newcommand{\NormalTok}[1]{{#1}}
    
    % Define a nice break command that doesn't care if a line doesn't already
    % exist.
    \def\br{\hspace*{\fill} \\* }
    % Math Jax compatability definitions
    \def\gt{>}
    \def\lt{<}
    % Document parameters
    \title{VII-Tips-and-Tricks}
    
    
    

    % Pygments definitions
    
\makeatletter
\def\PY@reset{\let\PY@it=\relax \let\PY@bf=\relax%
    \let\PY@ul=\relax \let\PY@tc=\relax%
    \let\PY@bc=\relax \let\PY@ff=\relax}
\def\PY@tok#1{\csname PY@tok@#1\endcsname}
\def\PY@toks#1+{\ifx\relax#1\empty\else%
    \PY@tok{#1}\expandafter\PY@toks\fi}
\def\PY@do#1{\PY@bc{\PY@tc{\PY@ul{%
    \PY@it{\PY@bf{\PY@ff{#1}}}}}}}
\def\PY#1#2{\PY@reset\PY@toks#1+\relax+\PY@do{#2}}

\expandafter\def\csname PY@tok@gd\endcsname{\def\PY@tc##1{\textcolor[rgb]{0.63,0.00,0.00}{##1}}}
\expandafter\def\csname PY@tok@gu\endcsname{\let\PY@bf=\textbf\def\PY@tc##1{\textcolor[rgb]{0.50,0.00,0.50}{##1}}}
\expandafter\def\csname PY@tok@gt\endcsname{\def\PY@tc##1{\textcolor[rgb]{0.00,0.27,0.87}{##1}}}
\expandafter\def\csname PY@tok@gs\endcsname{\let\PY@bf=\textbf}
\expandafter\def\csname PY@tok@gr\endcsname{\def\PY@tc##1{\textcolor[rgb]{1.00,0.00,0.00}{##1}}}
\expandafter\def\csname PY@tok@cm\endcsname{\let\PY@it=\textit\def\PY@tc##1{\textcolor[rgb]{0.25,0.50,0.50}{##1}}}
\expandafter\def\csname PY@tok@vg\endcsname{\def\PY@tc##1{\textcolor[rgb]{0.10,0.09,0.49}{##1}}}
\expandafter\def\csname PY@tok@m\endcsname{\def\PY@tc##1{\textcolor[rgb]{0.40,0.40,0.40}{##1}}}
\expandafter\def\csname PY@tok@mh\endcsname{\def\PY@tc##1{\textcolor[rgb]{0.40,0.40,0.40}{##1}}}
\expandafter\def\csname PY@tok@go\endcsname{\def\PY@tc##1{\textcolor[rgb]{0.53,0.53,0.53}{##1}}}
\expandafter\def\csname PY@tok@ge\endcsname{\let\PY@it=\textit}
\expandafter\def\csname PY@tok@vc\endcsname{\def\PY@tc##1{\textcolor[rgb]{0.10,0.09,0.49}{##1}}}
\expandafter\def\csname PY@tok@il\endcsname{\def\PY@tc##1{\textcolor[rgb]{0.40,0.40,0.40}{##1}}}
\expandafter\def\csname PY@tok@cs\endcsname{\let\PY@it=\textit\def\PY@tc##1{\textcolor[rgb]{0.25,0.50,0.50}{##1}}}
\expandafter\def\csname PY@tok@cp\endcsname{\def\PY@tc##1{\textcolor[rgb]{0.74,0.48,0.00}{##1}}}
\expandafter\def\csname PY@tok@gi\endcsname{\def\PY@tc##1{\textcolor[rgb]{0.00,0.63,0.00}{##1}}}
\expandafter\def\csname PY@tok@gh\endcsname{\let\PY@bf=\textbf\def\PY@tc##1{\textcolor[rgb]{0.00,0.00,0.50}{##1}}}
\expandafter\def\csname PY@tok@ni\endcsname{\let\PY@bf=\textbf\def\PY@tc##1{\textcolor[rgb]{0.60,0.60,0.60}{##1}}}
\expandafter\def\csname PY@tok@nl\endcsname{\def\PY@tc##1{\textcolor[rgb]{0.63,0.63,0.00}{##1}}}
\expandafter\def\csname PY@tok@nn\endcsname{\let\PY@bf=\textbf\def\PY@tc##1{\textcolor[rgb]{0.00,0.00,1.00}{##1}}}
\expandafter\def\csname PY@tok@no\endcsname{\def\PY@tc##1{\textcolor[rgb]{0.53,0.00,0.00}{##1}}}
\expandafter\def\csname PY@tok@na\endcsname{\def\PY@tc##1{\textcolor[rgb]{0.49,0.56,0.16}{##1}}}
\expandafter\def\csname PY@tok@nb\endcsname{\def\PY@tc##1{\textcolor[rgb]{0.00,0.50,0.00}{##1}}}
\expandafter\def\csname PY@tok@nc\endcsname{\let\PY@bf=\textbf\def\PY@tc##1{\textcolor[rgb]{0.00,0.00,1.00}{##1}}}
\expandafter\def\csname PY@tok@nd\endcsname{\def\PY@tc##1{\textcolor[rgb]{0.67,0.13,1.00}{##1}}}
\expandafter\def\csname PY@tok@ne\endcsname{\let\PY@bf=\textbf\def\PY@tc##1{\textcolor[rgb]{0.82,0.25,0.23}{##1}}}
\expandafter\def\csname PY@tok@nf\endcsname{\def\PY@tc##1{\textcolor[rgb]{0.00,0.00,1.00}{##1}}}
\expandafter\def\csname PY@tok@si\endcsname{\let\PY@bf=\textbf\def\PY@tc##1{\textcolor[rgb]{0.73,0.40,0.53}{##1}}}
\expandafter\def\csname PY@tok@s2\endcsname{\def\PY@tc##1{\textcolor[rgb]{0.73,0.13,0.13}{##1}}}
\expandafter\def\csname PY@tok@vi\endcsname{\def\PY@tc##1{\textcolor[rgb]{0.10,0.09,0.49}{##1}}}
\expandafter\def\csname PY@tok@nt\endcsname{\let\PY@bf=\textbf\def\PY@tc##1{\textcolor[rgb]{0.00,0.50,0.00}{##1}}}
\expandafter\def\csname PY@tok@nv\endcsname{\def\PY@tc##1{\textcolor[rgb]{0.10,0.09,0.49}{##1}}}
\expandafter\def\csname PY@tok@s1\endcsname{\def\PY@tc##1{\textcolor[rgb]{0.73,0.13,0.13}{##1}}}
\expandafter\def\csname PY@tok@kd\endcsname{\let\PY@bf=\textbf\def\PY@tc##1{\textcolor[rgb]{0.00,0.50,0.00}{##1}}}
\expandafter\def\csname PY@tok@sh\endcsname{\def\PY@tc##1{\textcolor[rgb]{0.73,0.13,0.13}{##1}}}
\expandafter\def\csname PY@tok@sc\endcsname{\def\PY@tc##1{\textcolor[rgb]{0.73,0.13,0.13}{##1}}}
\expandafter\def\csname PY@tok@sx\endcsname{\def\PY@tc##1{\textcolor[rgb]{0.00,0.50,0.00}{##1}}}
\expandafter\def\csname PY@tok@bp\endcsname{\def\PY@tc##1{\textcolor[rgb]{0.00,0.50,0.00}{##1}}}
\expandafter\def\csname PY@tok@c1\endcsname{\let\PY@it=\textit\def\PY@tc##1{\textcolor[rgb]{0.25,0.50,0.50}{##1}}}
\expandafter\def\csname PY@tok@kc\endcsname{\let\PY@bf=\textbf\def\PY@tc##1{\textcolor[rgb]{0.00,0.50,0.00}{##1}}}
\expandafter\def\csname PY@tok@c\endcsname{\let\PY@it=\textit\def\PY@tc##1{\textcolor[rgb]{0.25,0.50,0.50}{##1}}}
\expandafter\def\csname PY@tok@mf\endcsname{\def\PY@tc##1{\textcolor[rgb]{0.40,0.40,0.40}{##1}}}
\expandafter\def\csname PY@tok@err\endcsname{\def\PY@bc##1{\setlength{\fboxsep}{0pt}\fcolorbox[rgb]{1.00,0.00,0.00}{1,1,1}{\strut ##1}}}
\expandafter\def\csname PY@tok@mb\endcsname{\def\PY@tc##1{\textcolor[rgb]{0.40,0.40,0.40}{##1}}}
\expandafter\def\csname PY@tok@ss\endcsname{\def\PY@tc##1{\textcolor[rgb]{0.10,0.09,0.49}{##1}}}
\expandafter\def\csname PY@tok@sr\endcsname{\def\PY@tc##1{\textcolor[rgb]{0.73,0.40,0.53}{##1}}}
\expandafter\def\csname PY@tok@mo\endcsname{\def\PY@tc##1{\textcolor[rgb]{0.40,0.40,0.40}{##1}}}
\expandafter\def\csname PY@tok@kn\endcsname{\let\PY@bf=\textbf\def\PY@tc##1{\textcolor[rgb]{0.00,0.50,0.00}{##1}}}
\expandafter\def\csname PY@tok@mi\endcsname{\def\PY@tc##1{\textcolor[rgb]{0.40,0.40,0.40}{##1}}}
\expandafter\def\csname PY@tok@gp\endcsname{\let\PY@bf=\textbf\def\PY@tc##1{\textcolor[rgb]{0.00,0.00,0.50}{##1}}}
\expandafter\def\csname PY@tok@o\endcsname{\def\PY@tc##1{\textcolor[rgb]{0.40,0.40,0.40}{##1}}}
\expandafter\def\csname PY@tok@kr\endcsname{\let\PY@bf=\textbf\def\PY@tc##1{\textcolor[rgb]{0.00,0.50,0.00}{##1}}}
\expandafter\def\csname PY@tok@s\endcsname{\def\PY@tc##1{\textcolor[rgb]{0.73,0.13,0.13}{##1}}}
\expandafter\def\csname PY@tok@kp\endcsname{\def\PY@tc##1{\textcolor[rgb]{0.00,0.50,0.00}{##1}}}
\expandafter\def\csname PY@tok@w\endcsname{\def\PY@tc##1{\textcolor[rgb]{0.73,0.73,0.73}{##1}}}
\expandafter\def\csname PY@tok@kt\endcsname{\def\PY@tc##1{\textcolor[rgb]{0.69,0.00,0.25}{##1}}}
\expandafter\def\csname PY@tok@ow\endcsname{\let\PY@bf=\textbf\def\PY@tc##1{\textcolor[rgb]{0.67,0.13,1.00}{##1}}}
\expandafter\def\csname PY@tok@sb\endcsname{\def\PY@tc##1{\textcolor[rgb]{0.73,0.13,0.13}{##1}}}
\expandafter\def\csname PY@tok@k\endcsname{\let\PY@bf=\textbf\def\PY@tc##1{\textcolor[rgb]{0.00,0.50,0.00}{##1}}}
\expandafter\def\csname PY@tok@se\endcsname{\let\PY@bf=\textbf\def\PY@tc##1{\textcolor[rgb]{0.73,0.40,0.13}{##1}}}
\expandafter\def\csname PY@tok@sd\endcsname{\let\PY@it=\textit\def\PY@tc##1{\textcolor[rgb]{0.73,0.13,0.13}{##1}}}

\def\PYZbs{\char`\\}
\def\PYZus{\char`\_}
\def\PYZob{\char`\{}
\def\PYZcb{\char`\}}
\def\PYZca{\char`\^}
\def\PYZam{\char`\&}
\def\PYZlt{\char`\<}
\def\PYZgt{\char`\>}
\def\PYZsh{\char`\#}
\def\PYZpc{\char`\%}
\def\PYZdl{\char`\$}
\def\PYZhy{\char`\-}
\def\PYZsq{\char`\'}
\def\PYZdq{\char`\"}
\def\PYZti{\char`\~}
% for compatibility with earlier versions
\def\PYZat{@}
\def\PYZlb{[}
\def\PYZrb{]}
\makeatother


    % Exact colors from NB
    \definecolor{incolor}{rgb}{0.0, 0.0, 0.5}
    \definecolor{outcolor}{rgb}{0.545, 0.0, 0.0}



    
    % Prevent overflowing lines due to hard-to-break entities
    \sloppy 
    % Setup hyperref package
    \hypersetup{
      breaklinks=true,  % so long urls are correctly broken across lines
      colorlinks=true,
      urlcolor=blue,
      linkcolor=darkorange,
      citecolor=darkgreen,
      }
    % Slightly bigger margins than the latex defaults
    
    \geometry{verbose,tmargin=1in,bmargin=1in,lmargin=1in,rmargin=1in}
    
    

    \begin{document}
    
    
    \maketitle
    
    

    
    \section{VII-Tips and Tricks}\label{vii-tips-and-tricks}

    \begin{Verbatim}[commandchars=\\\{\}]
{\color{incolor}In [{\color{incolor}}]:} 
\end{Verbatim}

    \begin{Verbatim}[commandchars=\\\{\}]
{\color{incolor}In [{\color{incolor}1}]:} \PY{o}{\PYZpc{}}\PY{k}{load\PYZus{}ext} \PY{n}{memory\PYZus{}profiler}
\end{Verbatim}

    \begin{Verbatim}[commandchars=\\\{\}]

        ---------------------------------------------------------------------------
    ImportError                               Traceback (most recent call last)

        <ipython-input-1-98a82a26c530> in <module>()
    ----> 1 get\_ipython().magic(u'load\_ext memory\_profiler')
    

        /home/jpsilva/anaconda/lib/python2.7/site-packages/IPython/core/interactiveshell.pyc in magic(self, arg\_s)
       2203         magic\_name, \_, magic\_arg\_s = arg\_s.partition(' ')
       2204         magic\_name = magic\_name.lstrip(prefilter.ESC\_MAGIC)
    -> 2205         return self.run\_line\_magic(magic\_name, magic\_arg\_s)
       2206 
       2207     \#-------------------------------------------------------------------------


        /home/jpsilva/anaconda/lib/python2.7/site-packages/IPython/core/interactiveshell.pyc in run\_line\_magic(self, magic\_name, line)
       2124                 kwargs['local\_ns'] = sys.\_getframe(stack\_depth).f\_locals
       2125             with self.builtin\_trap:
    -> 2126                 result = fn(*args,**kwargs)
       2127             return result
       2128 


        /home/jpsilva/anaconda/lib/python2.7/site-packages/IPython/core/magics/extension.pyc in load\_ext(self, module\_str)


        /home/jpsilva/anaconda/lib/python2.7/site-packages/IPython/core/magic.pyc in <lambda>(f, *a, **k)
        191     \# but it's overkill for just that one bit of state.
        192     def magic\_deco(arg):
    --> 193         call = lambda f, *a, **k: f(*a, **k)
        194 
        195         if callable(arg):


        /home/jpsilva/anaconda/lib/python2.7/site-packages/IPython/core/magics/extension.pyc in load\_ext(self, module\_str)
         61         if not module\_str:
         62             raise UsageError('Missing module name.')
    ---> 63         res = self.shell.extension\_manager.load\_extension(module\_str)
         64 
         65         if res == 'already loaded':


        /home/jpsilva/anaconda/lib/python2.7/site-packages/IPython/core/extensions.pyc in load\_extension(self, module\_str)
         96             if module\_str not in sys.modules:
         97                 with prepended\_to\_syspath(self.ipython\_extension\_dir):
    ---> 98                     \_\_import\_\_(module\_str)
         99             mod = sys.modules[module\_str]
        100             if self.\_call\_load\_ipython\_extension(mod):


        ImportError: No module named memory\_profiler

    \end{Verbatim}

    \textbf{LOOPS}

    \begin{Verbatim}[commandchars=\\\{\}]
{\color{incolor}In [{\color{incolor}2}]:} \PY{k}{for} \PY{n}{i} \PY{o+ow}{in} \PY{p}{[}\PY{l+m+mi}{0}\PY{p}{,}\PY{l+m+mi}{1}\PY{p}{,}\PY{l+m+mi}{2}\PY{p}{,}\PY{l+m+mi}{3}\PY{p}{,}\PY{l+m+mi}{4}\PY{p}{,}\PY{l+m+mi}{5}\PY{p}{]}\PY{p}{:}
            \PY{k}{print} \PY{n}{i}\PY{o}{*}\PY{o}{*}\PY{l+m+mi}{2}
\end{Verbatim}

    \begin{Verbatim}[commandchars=\\\{\}]
0
1
4
9
16
25
    \end{Verbatim}

    \begin{Verbatim}[commandchars=\\\{\}]
{\color{incolor}In [{\color{incolor}3}]:} \PY{k}{for} \PY{n}{i} \PY{o+ow}{in} \PY{n+nb}{range}\PY{p}{(}\PY{l+m+mi}{6}\PY{p}{)}\PY{p}{:}
            \PY{k}{print} \PY{n}{i}\PY{o}{*}\PY{o}{*}\PY{l+m+mi}{2}
\end{Verbatim}

    \begin{Verbatim}[commandchars=\\\{\}]
0
1
4
9
16
25
    \end{Verbatim}

    \begin{Verbatim}[commandchars=\\\{\}]
{\color{incolor}In [{\color{incolor}4}]:} \PY{k}{for} \PY{n}{i} \PY{o+ow}{in} \PY{n+nb}{xrange}\PY{p}{(}\PY{l+m+mi}{6}\PY{p}{)}\PY{p}{:}
            \PY{k}{print} \PY{n}{i}\PY{o}{*}\PY{o}{*}\PY{l+m+mi}{2}
\end{Verbatim}

    \begin{Verbatim}[commandchars=\\\{\}]
0
1
4
9
16
25
    \end{Verbatim}

    \begin{Verbatim}[commandchars=\\\{\}]
{\color{incolor}In [{\color{incolor}5}]:} \PY{o}{\PYZpc{}\PYZpc{}}\PY{k}{timeit}
        \PY{k}{for} \PY{n}{i} \PY{o+ow}{in} \PY{n+nb}{range}\PY{p}{(}\PY{l+m+mi}{10000}\PY{p}{)}\PY{p}{:}
            \PY{k}{pass}
\end{Verbatim}

    \begin{Verbatim}[commandchars=\\\{\}]
1000 loops, best of 3: 210 µs per loop
    \end{Verbatim}

    \begin{Verbatim}[commandchars=\\\{\}]
{\color{incolor}In [{\color{incolor}6}]:} \PY{o}{\PYZpc{}\PYZpc{}}\PY{k}{timeit}
        \PY{k}{for} \PY{n}{i} \PY{o+ow}{in} \PY{n+nb}{xrange}\PY{p}{(}\PY{l+m+mi}{10000}\PY{p}{)}\PY{p}{:}
            \PY{k}{pass}
\end{Verbatim}

    \begin{Verbatim}[commandchars=\\\{\}]
10000 loops, best of 3: 159 µs per loop
    \end{Verbatim}

    xrange is now range in Python3

    \textbf{Looping over collections}

    \begin{Verbatim}[commandchars=\\\{\}]
{\color{incolor}In [{\color{incolor}8}]:} \PY{n}{colors} \PY{o}{=} \PY{p}{[}\PY{l+s}{\PYZsq{}}\PY{l+s}{red}\PY{l+s}{\PYZsq{}}\PY{p}{,}\PY{l+s}{\PYZsq{}}\PY{l+s}{green}\PY{l+s}{\PYZsq{}}\PY{p}{,}\PY{l+s}{\PYZsq{}}\PY{l+s}{blue}\PY{l+s}{\PYZsq{}}\PY{p}{,}\PY{l+s}{\PYZsq{}}\PY{l+s}{yellow}\PY{l+s}{\PYZsq{}}\PY{p}{]}
\end{Verbatim}

    \begin{Verbatim}[commandchars=\\\{\}]
{\color{incolor}In [{\color{incolor}9}]:} \PY{k}{for} \PY{n}{i} \PY{o+ow}{in} \PY{n+nb}{range}\PY{p}{(}\PY{n+nb}{len}\PY{p}{(}\PY{n}{colors}\PY{p}{)}\PY{p}{)}\PY{p}{:}
            \PY{k}{print} \PY{n}{colors}\PY{p}{[}\PY{n}{i}\PY{p}{]}
\end{Verbatim}

    \begin{Verbatim}[commandchars=\\\{\}]
red
green
blue
yellow
    \end{Verbatim}

    \begin{Verbatim}[commandchars=\\\{\}]
{\color{incolor}In [{\color{incolor}10}]:} \PY{k}{for} \PY{n}{color} \PY{o+ow}{in} \PY{n}{colors}\PY{p}{:}
             \PY{k}{print} \PY{n}{color}
\end{Verbatim}

    \begin{Verbatim}[commandchars=\\\{\}]
red
green
blue
yellow
    \end{Verbatim}

    \textbf{Looping backwards}

    \begin{Verbatim}[commandchars=\\\{\}]
{\color{incolor}In [{\color{incolor}11}]:} \PY{k}{for} \PY{n}{i} \PY{o+ow}{in} \PY{n+nb}{range}\PY{p}{(}\PY{n+nb}{len}\PY{p}{(}\PY{n}{colors}\PY{p}{)}\PY{o}{\PYZhy{}}\PY{l+m+mi}{1}\PY{p}{,}\PY{o}{\PYZhy{}}\PY{l+m+mi}{1}\PY{p}{,}\PY{o}{\PYZhy{}}\PY{l+m+mi}{1}\PY{p}{)}\PY{p}{:}
             \PY{k}{print} \PY{n}{colors}\PY{p}{[}\PY{n}{i}\PY{p}{]}
\end{Verbatim}

    \begin{Verbatim}[commandchars=\\\{\}]
yellow
blue
green
red
    \end{Verbatim}

    \begin{Verbatim}[commandchars=\\\{\}]
{\color{incolor}In [{\color{incolor}12}]:} \PY{k}{for} \PY{n}{color} \PY{o+ow}{in} \PY{n+nb}{reversed}\PY{p}{(}\PY{n}{colors}\PY{p}{)}\PY{p}{:}
             \PY{k}{print} \PY{n}{color}
\end{Verbatim}

    \begin{Verbatim}[commandchars=\\\{\}]
yellow
blue
green
red
    \end{Verbatim}

    \textbf{Looping with indices}

    \begin{Verbatim}[commandchars=\\\{\}]
{\color{incolor}In [{\color{incolor}13}]:} \PY{k}{for} \PY{n}{i} \PY{o+ow}{in} \PY{n+nb}{range}\PY{p}{(}\PY{n+nb}{len}\PY{p}{(}\PY{n}{colors}\PY{p}{)}\PY{p}{)}\PY{p}{:}
             \PY{k}{print} \PY{n}{i}\PY{p}{,} \PY{l+s}{\PYZsq{}}\PY{l+s}{\PYZhy{}\PYZhy{}\PYZgt{}}\PY{l+s}{\PYZsq{}}\PY{p}{,} \PY{n}{colors}\PY{p}{[}\PY{n}{i}\PY{p}{]}
\end{Verbatim}

    \begin{Verbatim}[commandchars=\\\{\}]
0 --> red
1 --> green
2 --> blue
3 --> yellow
    \end{Verbatim}

    \begin{Verbatim}[commandchars=\\\{\}]
{\color{incolor}In [{\color{incolor}14}]:} \PY{k}{for} \PY{n}{i}\PY{p}{,}\PY{n}{color} \PY{o+ow}{in} \PY{n+nb}{enumerate}\PY{p}{(}\PY{n}{colors}\PY{p}{)}\PY{p}{:}
             \PY{k}{print} \PY{n}{i}\PY{p}{,} \PY{l+s}{\PYZsq{}}\PY{l+s}{\PYZhy{}\PYZhy{}\PYZgt{}}\PY{l+s}{\PYZsq{}}\PY{p}{,} \PY{n}{color}
\end{Verbatim}

    \begin{Verbatim}[commandchars=\\\{\}]
0 --> red
1 --> green
2 --> blue
3 --> yellow
    \end{Verbatim}

    \begin{Verbatim}[commandchars=\\\{\}]
{\color{incolor}In [{\color{incolor}17}]:} \PY{o}{\PYZpc{}\PYZpc{}}\PY{k}{timeit}
         \PY{k}{for} \PY{n}{i} \PY{o+ow}{in} \PY{n+nb}{range}\PY{p}{(}\PY{n+nb}{len}\PY{p}{(}\PY{n}{colors}\PY{p}{)}\PY{p}{)}\PY{p}{:}
             \PY{k}{pass}
\end{Verbatim}

    \begin{Verbatim}[commandchars=\\\{\}]
1000000 loops, best of 3: 359 ns per loop
    \end{Verbatim}

    \begin{Verbatim}[commandchars=\\\{\}]
{\color{incolor}In [{\color{incolor}18}]:} \PY{o}{\PYZpc{}\PYZpc{}}\PY{k}{timeit}
         \PY{k}{for} \PY{n}{i}\PY{p}{,}\PY{n}{color} \PY{o+ow}{in} \PY{n+nb}{enumerate}\PY{p}{(}\PY{n}{colors}\PY{p}{)}\PY{p}{:}
             \PY{k}{pass}
\end{Verbatim}

    \begin{Verbatim}[commandchars=\\\{\}]
1000000 loops, best of 3: 403 ns per loop
    \end{Verbatim}

    \textbf{Looping over two collection}

    \begin{Verbatim}[commandchars=\\\{\}]
{\color{incolor}In [{\color{incolor}19}]:} \PY{n}{names} \PY{o}{=} \PY{p}{[}\PY{l+s}{\PYZsq{}}\PY{l+s}{Long}\PY{l+s}{\PYZsq{}}\PY{p}{,}\PY{l+s}{\PYZsq{}}\PY{l+s}{Dmitri}\PY{l+s}{\PYZsq{}}\PY{p}{,}\PY{l+s}{\PYZsq{}}\PY{l+s}{Zuzana}\PY{l+s}{\PYZsq{}}\PY{p}{,}\PY{l+s}{\PYZsq{}}\PY{l+s}{Jose}\PY{l+s}{\PYZsq{}}\PY{p}{,}\PY{l+s}{\PYZsq{}}\PY{l+s}{Daniel}\PY{l+s}{\PYZsq{}}\PY{p}{]}
\end{Verbatim}

    \begin{Verbatim}[commandchars=\\\{\}]
{\color{incolor}In [{\color{incolor}20}]:} \PY{n}{n} \PY{o}{=} \PY{n+nb}{min}\PY{p}{(}\PY{n+nb}{len}\PY{p}{(}\PY{n}{names}\PY{p}{)}\PY{p}{,} \PY{n+nb}{len}\PY{p}{(}\PY{n}{colors}\PY{p}{)}\PY{p}{)}
         \PY{k}{for} \PY{n}{i} \PY{o+ow}{in} \PY{n+nb}{range}\PY{p}{(}\PY{n}{n}\PY{p}{)}\PY{p}{:}
             \PY{k}{print} \PY{n}{names}\PY{p}{[}\PY{n}{i}\PY{p}{]}\PY{p}{,} \PY{l+s}{\PYZsq{}}\PY{l+s}{\PYZhy{}\PYZhy{}\PYZgt{}}\PY{l+s}{\PYZsq{}}\PY{p}{,} \PY{n}{colors}\PY{p}{[}\PY{n}{i}\PY{p}{]}
\end{Verbatim}

    \begin{Verbatim}[commandchars=\\\{\}]
Long --> red
Dmitri --> green
Zuzana --> blue
Jose --> yellow
    \end{Verbatim}

    \begin{Verbatim}[commandchars=\\\{\}]
{\color{incolor}In [{\color{incolor}21}]:} \PY{k}{for} \PY{n}{name}\PY{p}{,}\PY{n}{color} \PY{o+ow}{in} \PY{n+nb}{zip}\PY{p}{(}\PY{n}{names}\PY{p}{,}\PY{n}{colors}\PY{p}{)}\PY{p}{:}
             \PY{k}{print} \PY{n}{name}\PY{p}{,}\PY{l+s}{\PYZsq{}}\PY{l+s}{\PYZhy{}\PYZhy{}\PYZgt{}}\PY{l+s}{\PYZsq{}}\PY{p}{,} \PY{n}{color}
\end{Verbatim}

    \begin{Verbatim}[commandchars=\\\{\}]
Long --> red
Dmitri --> green
Zuzana --> blue
Jose --> yellow
    \end{Verbatim}

    \begin{Verbatim}[commandchars=\\\{\}]
{\color{incolor}In [{\color{incolor}22}]:} \PY{k+kn}{from} \PY{n+nn}{itertools} \PY{k+kn}{import} \PY{n}{izip}
         \PY{k}{for} \PY{n}{name}\PY{p}{,}\PY{n}{color} \PY{o+ow}{in} \PY{n}{izip}\PY{p}{(}\PY{n}{names}\PY{p}{,}\PY{n}{colors}\PY{p}{)}\PY{p}{:}
             \PY{k}{print} \PY{n}{name}\PY{p}{,} \PY{l+s}{\PYZsq{}}\PY{l+s}{\PYZhy{}\PYZhy{}\PYZgt{}}\PY{l+s}{\PYZsq{}}\PY{p}{,} \PY{n}{color}
\end{Verbatim}

    \begin{Verbatim}[commandchars=\\\{\}]
Long --> red
Dmitri --> green
Zuzana --> blue
Jose --> yellow
    \end{Verbatim}

    \begin{Verbatim}[commandchars=\\\{\}]
{\color{incolor}In [{\color{incolor}23}]:} \PY{k}{for} \PY{n}{color} \PY{o+ow}{in} \PY{n+nb}{sorted}\PY{p}{(}\PY{n}{colors}\PY{p}{)}\PY{p}{:}
             \PY{k}{print} \PY{n}{color}
\end{Verbatim}

    \begin{Verbatim}[commandchars=\\\{\}]
blue
green
red
yellow
    \end{Verbatim}

    \begin{Verbatim}[commandchars=\\\{\}]
{\color{incolor}In [{\color{incolor}24}]:} \PY{n}{partial}\PY{err}{?}
\end{Verbatim}

    \begin{Verbatim}[commandchars=\\\{\}]
Object `partial` not found.
    \end{Verbatim}

    \begin{Verbatim}[commandchars=\\\{\}]
{\color{incolor}In [{\color{incolor}25}]:} \PY{n}{d} \PY{o}{=} \PY{p}{\PYZob{}}\PY{l+s}{\PYZsq{}}\PY{l+s}{Daniel}\PY{l+s}{\PYZsq{}}\PY{p}{:}\PY{l+s}{\PYZsq{}}\PY{l+s}{I}\PY{l+s}{\PYZsq{}}\PY{p}{,}\PY{l+s}{\PYZsq{}}\PY{l+s}{Christian}\PY{l+s}{\PYZsq{}}\PY{p}{:}\PY{l+s}{\PYZsq{}}\PY{l+s}{II}\PY{l+s}{\PYZsq{}}\PY{p}{,}\PY{l+s}{\PYZsq{}}\PY{l+s}{Timo}\PY{l+s}{\PYZsq{}}\PY{p}{:}\PY{l+s}{\PYZsq{}}\PY{l+s}{III}\PY{l+s}{\PYZsq{}}\PY{p}{,}\PY{l+s}{\PYZsq{}}\PY{l+s}{Radoslav}\PY{l+s}{\PYZsq{}}\PY{p}{:}\PY{l+s}{\PYZsq{}}\PY{l+s}{IV}\PY{l+s}{\PYZsq{}}\PY{p}{\PYZcb{}}
\end{Verbatim}

    \begin{Verbatim}[commandchars=\\\{\}]
{\color{incolor}In [{\color{incolor}26}]:} \PY{k}{for} \PY{n}{k} \PY{o+ow}{in} \PY{n}{d}\PY{p}{:}
             \PY{k}{print} \PY{n}{k}
\end{Verbatim}

    \begin{Verbatim}[commandchars=\\\{\}]
Radoslav
Daniel
Christian
Timo
    \end{Verbatim}

    \begin{Verbatim}[commandchars=\\\{\}]
{\color{incolor}In [{\color{incolor}27}]:} \PY{k}{for} \PY{n}{k} \PY{o+ow}{in} \PY{n}{d}\PY{p}{:}
             \PY{k}{print} \PY{n}{k}\PY{p}{,} \PY{l+s}{\PYZsq{}}\PY{l+s}{\PYZhy{}\PYZhy{}\PYZgt{}}\PY{l+s}{\PYZsq{}}\PY{p}{,} \PY{n}{d}\PY{p}{[}\PY{n}{k}\PY{p}{]}
\end{Verbatim}

    \begin{Verbatim}[commandchars=\\\{\}]
Radoslav --> IV
Daniel --> I
Christian --> II
Timo --> III
    \end{Verbatim}

    \begin{Verbatim}[commandchars=\\\{\}]
{\color{incolor}In [{\color{incolor}28}]:} \PY{k}{for} \PY{n}{k}\PY{p}{,}\PY{n}{v} \PY{o+ow}{in} \PY{n}{d}\PY{o}{.}\PY{n}{items}\PY{p}{(}\PY{p}{)}\PY{p}{:}
             \PY{k}{print} \PY{n}{k}\PY{p}{,} \PY{l+s}{\PYZsq{}}\PY{l+s}{\PYZhy{}\PYZhy{}\PYZgt{}}\PY{l+s}{\PYZsq{}}\PY{p}{,} \PY{n}{v}
\end{Verbatim}

    \begin{Verbatim}[commandchars=\\\{\}]
Radoslav --> IV
Daniel --> I
Christian --> II
Timo --> III
    \end{Verbatim}

    \begin{Verbatim}[commandchars=\\\{\}]
{\color{incolor}In [{\color{incolor}29}]:} \PY{k}{for} \PY{n}{k}\PY{p}{,}\PY{n}{v} \PY{o+ow}{in} \PY{n}{d}\PY{o}{.}\PY{n}{iteritems}\PY{p}{(}\PY{p}{)}\PY{p}{:}
             \PY{k}{print} \PY{n}{k}\PY{p}{,} \PY{l+s}{\PYZsq{}}\PY{l+s}{\PYZhy{}\PYZhy{}\PYZgt{}}\PY{l+s}{\PYZsq{}}\PY{p}{,} \PY{n}{v}
\end{Verbatim}

    \begin{Verbatim}[commandchars=\\\{\}]
Radoslav --> IV
Daniel --> I
Christian --> II
Timo --> III
    \end{Verbatim}

    \begin{Verbatim}[commandchars=\\\{\}]
{\color{incolor}In [{\color{incolor}32}]:} \PY{n}{d2} \PY{o}{=} \PY{n+nb}{dict}\PY{p}{(}\PY{n}{izip}\PY{p}{(}\PY{n}{names}\PY{p}{,}\PY{n}{colors}\PY{p}{)}\PY{p}{)}
         \PY{n}{d2}
\end{Verbatim}

            \begin{Verbatim}[commandchars=\\\{\}]
{\color{outcolor}Out[{\color{outcolor}32}]:} \{'Dmitri': 'green', 'Jose': 'yellow', 'Long': 'red', 'Zuzana': 'blue'\}
\end{Verbatim}
        
    \textbf{Counting with dict}

    \begin{Verbatim}[commandchars=\\\{\}]
{\color{incolor}In [{\color{incolor}39}]:} \PY{n}{colors} \PY{o}{=} \PY{p}{[}\PY{l+s}{\PYZsq{}}\PY{l+s}{red}\PY{l+s}{\PYZsq{}}\PY{p}{,}\PY{l+s}{\PYZsq{}}\PY{l+s}{blue}\PY{l+s}{\PYZsq{}}\PY{p}{,}\PY{l+s}{\PYZsq{}}\PY{l+s}{green}\PY{l+s}{\PYZsq{}}\PY{p}{,}\PY{l+s}{\PYZsq{}}\PY{l+s}{red}\PY{l+s}{\PYZsq{}}\PY{p}{,}\PY{l+s}{\PYZsq{}}\PY{l+s}{yellow}\PY{l+s}{\PYZsq{}}\PY{p}{,}\PY{l+s}{\PYZsq{}}\PY{l+s}{green}\PY{l+s}{\PYZsq{}}\PY{p}{]}
\end{Verbatim}

    \begin{Verbatim}[commandchars=\\\{\}]
{\color{incolor}In [{\color{incolor}35}]:} \PY{n}{d3} \PY{o}{=} \PY{p}{\PYZob{}}\PY{p}{\PYZcb{}}
         \PY{k}{for} \PY{n}{color} \PY{o+ow}{in} \PY{n}{colors}\PY{p}{:}
             \PY{k}{if} \PY{n}{color} \PY{o+ow}{not} \PY{o+ow}{in} \PY{n}{d3}\PY{p}{:}
                 \PY{n}{d3}\PY{p}{[}\PY{n}{color}\PY{p}{]} \PY{o}{=} \PY{l+m+mi}{0}
             \PY{n}{d3}\PY{p}{[}\PY{n}{color}\PY{p}{]} \PY{o}{+}\PY{o}{=} \PY{l+m+mi}{1}
         \PY{n}{d3}
\end{Verbatim}

            \begin{Verbatim}[commandchars=\\\{\}]
{\color{outcolor}Out[{\color{outcolor}35}]:} \{'blue': 1, 'green': 2, 'red': 2, 'yellow': 1\}
\end{Verbatim}
        
    \begin{Verbatim}[commandchars=\\\{\}]
{\color{incolor}In [{\color{incolor}52}]:} \PY{n}{d3} \PY{o}{=} \PY{p}{\PYZob{}}\PY{p}{\PYZcb{}}
         \PY{k}{for} \PY{n}{color} \PY{o+ow}{in} \PY{n}{colors}\PY{p}{:}
             \PY{n}{d3}\PY{p}{[}\PY{n}{color}\PY{p}{]} \PY{o}{=} \PY{n}{d3}\PY{o}{.}\PY{n}{get}\PY{p}{(}\PY{n}{color}\PY{p}{,}\PY{l+m+mi}{0}\PY{p}{)} \PY{o}{+} \PY{l+m+mi}{1}
         \PY{n}{d3}
\end{Verbatim}

            \begin{Verbatim}[commandchars=\\\{\}]
{\color{outcolor}Out[{\color{outcolor}52}]:} \{'blue': 1, 'green': 2, 'red': 2, 'yellow': 1\}
\end{Verbatim}
        
    \begin{Verbatim}[commandchars=\\\{\}]
{\color{incolor}In [{\color{incolor}48}]:} \PY{k+kn}{from} \PY{n+nn}{collections} \PY{k+kn}{import} \PY{n}{defaultdict}
\end{Verbatim}

    \begin{Verbatim}[commandchars=\\\{\}]
{\color{incolor}In [{\color{incolor}51}]:} \PY{n}{d3} \PY{o}{=} \PY{n}{defaultdict}\PY{p}{(}\PY{n+nb}{int}\PY{p}{)}
         \PY{k}{for} \PY{n}{color} \PY{o+ow}{in} \PY{n}{colors}\PY{p}{:}
             \PY{n}{d3}\PY{p}{[}\PY{n}{color}\PY{p}{]} \PY{o}{+}\PY{o}{=} \PY{l+m+mi}{1}
         \PY{n}{d3} \PY{o}{=} \PY{n+nb}{dict}\PY{p}{(}\PY{n}{d3}\PY{p}{)}
         \PY{n}{d3}
\end{Verbatim}

            \begin{Verbatim}[commandchars=\\\{\}]
{\color{outcolor}Out[{\color{outcolor}51}]:} \{'blue': 1, 'green': 2, 'red': 2, 'yellow': 1\}
\end{Verbatim}
        
    \textbf{Group with dict}

    \begin{Verbatim}[commandchars=\\\{\}]
{\color{incolor}In [{\color{incolor}53}]:} \PY{n}{names} \PY{o}{=} \PY{p}{[}\PY{l+s}{\PYZsq{}}\PY{l+s}{Dmitri}\PY{l+s}{\PYZsq{}}\PY{p}{,}\PY{l+s}{\PYZsq{}}\PY{l+s}{Long}\PY{l+s}{\PYZsq{}}\PY{p}{,}\PY{l+s}{\PYZsq{}}\PY{l+s}{Zuzana}\PY{l+s}{\PYZsq{}}\PY{p}{,}\PY{l+s}{\PYZsq{}}\PY{l+s}{Jose}\PY{l+s}{\PYZsq{}}\PY{p}{,}\PY{l+s}{\PYZsq{}}\PY{l+s}{Radoslav}\PY{l+s}{\PYZsq{}}\PY{p}{,}\PY{l+s}{\PYZsq{}}\PY{l+s}{Timo}\PY{l+s}{\PYZsq{}}\PY{p}{,}\PY{l+s}{\PYZsq{}}\PY{l+s}{Christian}\PY{l+s}{\PYZsq{}}\PY{p}{,}\PY{l+s}{\PYZsq{}}\PY{l+s}{Daniel}\PY{l+s}{\PYZsq{}}\PY{p}{]}
\end{Verbatim}

    \begin{Verbatim}[commandchars=\\\{\}]
{\color{incolor}In [{\color{incolor}54}]:} \PY{n}{d} \PY{o}{=} \PY{p}{\PYZob{}}\PY{p}{\PYZcb{}}
         \PY{k}{for} \PY{n}{name} \PY{o+ow}{in} \PY{n}{names}\PY{p}{:}
             \PY{n}{key} \PY{o}{=} \PY{n+nb}{len}\PY{p}{(}\PY{n}{name}\PY{p}{)}
             \PY{k}{if} \PY{n}{key} \PY{o+ow}{not} \PY{o+ow}{in} \PY{n}{d}\PY{p}{:}
                 \PY{n}{d}\PY{p}{[}\PY{n}{key}\PY{p}{]} \PY{o}{=} \PY{p}{[}\PY{p}{]}
             \PY{n}{d}\PY{p}{[}\PY{n}{key}\PY{p}{]}\PY{o}{.}\PY{n}{append}\PY{p}{(}\PY{n}{name}\PY{p}{)}
         \PY{n}{d}
\end{Verbatim}

            \begin{Verbatim}[commandchars=\\\{\}]
{\color{outcolor}Out[{\color{outcolor}54}]:} \{4: ['Long', 'Jose', 'Timo'],
          6: ['Dmitri', 'Zuzana', 'Daniel'],
          8: ['Radoslav'],
          9: ['Christian']\}
\end{Verbatim}
        
    \begin{Verbatim}[commandchars=\\\{\}]
{\color{incolor}In [{\color{incolor}55}]:} \PY{n}{d} \PY{o}{=} \PY{p}{\PYZob{}}\PY{p}{\PYZcb{}}
         \PY{k}{for} \PY{n}{name} \PY{o+ow}{in} \PY{n}{names}\PY{p}{:}
             \PY{n}{key} \PY{o}{=} \PY{n+nb}{len}\PY{p}{(}\PY{n}{name}\PY{p}{)}
             \PY{n}{d}\PY{o}{.}\PY{n}{setdefault}\PY{p}{(}\PY{n}{key}\PY{p}{,} \PY{p}{[}\PY{p}{]}\PY{p}{)}\PY{o}{.}\PY{n}{append}\PY{p}{(}\PY{n}{name}\PY{p}{)}
         \PY{n}{d}
\end{Verbatim}

            \begin{Verbatim}[commandchars=\\\{\}]
{\color{outcolor}Out[{\color{outcolor}55}]:} \{4: ['Long', 'Jose', 'Timo'],
          6: ['Dmitri', 'Zuzana', 'Daniel'],
          8: ['Radoslav'],
          9: ['Christian']\}
\end{Verbatim}
        
    \begin{Verbatim}[commandchars=\\\{\}]
{\color{incolor}In [{\color{incolor}60}]:} \PY{n}{d} \PY{o}{=} \PY{n}{defaultdict}\PY{p}{(}\PY{n+nb}{list}\PY{p}{)}
         \PY{k}{for} \PY{n}{name} \PY{o+ow}{in} \PY{n}{names}\PY{p}{:}
             \PY{n}{key} \PY{o}{=} \PY{n+nb}{len}\PY{p}{(}\PY{n}{name}\PY{p}{)}
             \PY{n}{d}\PY{p}{[}\PY{n}{key}\PY{p}{]}\PY{o}{.}\PY{n}{append}\PY{p}{(}\PY{n}{name}\PY{p}{)}
         \PY{n+nb}{dict}\PY{p}{(}\PY{n}{d}\PY{p}{)}
\end{Verbatim}

            \begin{Verbatim}[commandchars=\\\{\}]
{\color{outcolor}Out[{\color{outcolor}60}]:} \{4: ['Long', 'Jose', 'Timo'],
          6: ['Dmitri', 'Zuzana', 'Daniel'],
          8: ['Radoslav'],
          9: ['Christian']\}
\end{Verbatim}
        
    \textbf{Unpacking sequences}

    \begin{Verbatim}[commandchars=\\\{\}]
{\color{incolor}In [{\color{incolor}61}]:} \PY{n}{data} \PY{o}{=} \PY{l+s}{\PYZsq{}}\PY{l+s}{Jose}\PY{l+s}{\PYZsq{}}\PY{p}{,}\PY{l+s}{\PYZsq{}}\PY{l+s}{Silva}\PY{l+s}{\PYZsq{}}\PY{p}{,}\PY{l+m+mi}{27}\PY{p}{,}\PY{l+s}{\PYZsq{}}\PY{l+s}{silva@math.uni\PYZhy{}wuppertal.de}\PY{l+s}{\PYZsq{}}
         \PY{n}{fname} \PY{o}{=} \PY{n}{data}\PY{p}{[}\PY{l+m+mi}{0}\PY{p}{]}
         \PY{n}{lname} \PY{o}{=} \PY{n}{data}\PY{p}{[}\PY{l+m+mi}{1}\PY{p}{]}
         \PY{n}{age} \PY{o}{=} \PY{n}{data}\PY{p}{[}\PY{l+m+mi}{2}\PY{p}{]}
         \PY{n}{email} \PY{o}{=} \PY{n}{data}\PY{p}{[}\PY{l+m+mi}{3}\PY{p}{]}
         \PY{k}{print} \PY{n}{fname}\PY{p}{,} \PY{n}{lname}\PY{p}{,} \PY{n}{age}\PY{p}{,} \PY{n}{email}
\end{Verbatim}

    \begin{Verbatim}[commandchars=\\\{\}]
Jose Silva 27 silva@math.uni-wuppertal.de
    \end{Verbatim}

    \begin{Verbatim}[commandchars=\\\{\}]
{\color{incolor}In [{\color{incolor}62}]:} \PY{n}{fname}\PY{p}{,} \PY{n}{lname}\PY{p}{,} \PY{n}{age}\PY{p}{,} \PY{n}{email} \PY{o}{=} \PY{n}{data}
         \PY{k}{print} \PY{n}{fname}\PY{p}{,} \PY{n}{lname}\PY{p}{,} \PY{n}{age}\PY{p}{,} \PY{n}{email}
\end{Verbatim}

    \begin{Verbatim}[commandchars=\\\{\}]
Jose Silva 27 silva@math.uni-wuppertal.de
    \end{Verbatim}

    Works specially well with points, for example

    \begin{Verbatim}[commandchars=\\\{\}]
{\color{incolor}In [{\color{incolor}64}]:} \PY{k+kn}{from} \PY{n+nn}{numpy} \PY{k+kn}{import} \PY{o}{*}
\end{Verbatim}

    \begin{Verbatim}[commandchars=\\\{\}]
{\color{incolor}In [{\color{incolor}67}]:} \PY{n}{p1} \PY{o}{=} \PY{p}{(}\PY{l+m+mi}{1}\PY{p}{,}\PY{l+m+mi}{1}\PY{p}{)}
         \PY{n}{x}\PY{p}{,} \PY{n}{y} \PY{o}{=} \PY{n}{p1}
         \PY{n}{r}\PY{p}{,} \PY{n}{theta} \PY{o}{=} \PY{n}{sqrt}\PY{p}{(}\PY{n}{x}\PY{o}{*}\PY{o}{*}\PY{l+m+mi}{2}\PY{o}{+}\PY{n}{y}\PY{o}{*}\PY{o}{*}\PY{l+m+mi}{2}\PY{p}{)}\PY{p}{,} \PY{n}{arctan}\PY{p}{(}\PY{n}{y}\PY{o}{/}\PY{n}{x}\PY{p}{)}
         \PY{k}{print} \PY{n}{r}\PY{p}{,}\PY{n}{theta}
\end{Verbatim}

    \begin{Verbatim}[commandchars=\\\{\}]
1.41421356237 0.785398163397
    \end{Verbatim}

    The ugly, the map, and the list comprehensions

    \begin{Verbatim}[commandchars=\\\{\}]
{\color{incolor}In [{\color{incolor}58}]:} \PY{n}{oldlist} \PY{o}{=} \PY{p}{[}\PY{l+s}{\PYZsq{}}\PY{l+s}{eu}\PY{l+s}{\PYZsq{}}\PY{p}{,}\PY{l+s}{\PYZsq{}}\PY{l+s}{ich}\PY{l+s}{\PYZsq{}}\PY{p}{,}\PY{l+s}{\PYZsq{}}\PY{l+s}{i}\PY{l+s}{\PYZsq{}}\PY{p}{,}\PY{l+s}{\PYZsq{}}\PY{l+s}{io}\PY{l+s}{\PYZsq{}}\PY{p}{,}\PY{l+s}{\PYZsq{}}\PY{l+s}{yo}\PY{l+s}{\PYZsq{}}\PY{p}{,}\PY{l+s}{\PYZsq{}}\PY{l+s}{я}\PY{l+s}{\PYZsq{}}\PY{p}{]}
\end{Verbatim}

    \begin{Verbatim}[commandchars=\\\{\}]
{\color{incolor}In [{\color{incolor}47}]:} \PY{n}{newlist} \PY{o}{=} \PY{p}{[}\PY{p}{]}
         \PY{k}{for} \PY{n}{word} \PY{o+ow}{in} \PY{n}{oldlist}\PY{p}{:}
             \PY{n}{newlist}\PY{o}{.}\PY{n}{append}\PY{p}{(}\PY{n}{word}\PY{o}{.}\PY{n}{capitalize}\PY{p}{(}\PY{p}{)}\PY{p}{)}
         \PY{k}{for} \PY{n}{word} \PY{o+ow}{in} \PY{n}{newlist}\PY{p}{:}
             \PY{k}{print} \PY{n}{word}\PY{p}{,}
\end{Verbatim}

    \begin{Verbatim}[commandchars=\\\{\}]
Eu Ich I Io Yo я
    \end{Verbatim}

    \begin{Verbatim}[commandchars=\\\{\}]
{\color{incolor}In [{\color{incolor}48}]:} \PY{n}{newlist} \PY{o}{=} \PY{n+nb}{map}\PY{p}{(}\PY{n+nb}{str}\PY{o}{.}\PY{n}{capitalize}\PY{p}{,}\PY{n}{oldlist}\PY{p}{)}
         \PY{k}{for} \PY{n}{word} \PY{o+ow}{in} \PY{n}{newlist}\PY{p}{:}
             \PY{k}{print} \PY{n}{word}\PY{p}{,}
\end{Verbatim}

    \begin{Verbatim}[commandchars=\\\{\}]
Eu Ich I Io Yo я
    \end{Verbatim}

    \begin{Verbatim}[commandchars=\\\{\}]
{\color{incolor}In [{\color{incolor}49}]:} \PY{n}{newlist} \PY{o}{=} \PY{p}{[}\PY{n}{word}\PY{o}{.}\PY{n}{capitalize}\PY{p}{(}\PY{p}{)} \PY{k}{for} \PY{n}{word} \PY{o+ow}{in} \PY{n}{oldlist}\PY{p}{]}
         \PY{k}{for} \PY{n}{word} \PY{o+ow}{in} \PY{n}{newlist}\PY{p}{:}
             \PY{k}{print} \PY{n}{word}\PY{p}{,}
\end{Verbatim}

    \begin{Verbatim}[commandchars=\\\{\}]
Eu Ich I Io Yo я
    \end{Verbatim}

    \begin{Verbatim}[commandchars=\\\{\}]
{\color{incolor}In [{\color{incolor}59}]:} \PY{o}{\PYZpc{}\PYZpc{}}\PY{k}{timeit}
         \PY{n}{newlist} \PY{o}{=} \PY{p}{[}\PY{p}{]}
         \PY{k}{for} \PY{n}{word} \PY{o+ow}{in} \PY{n}{oldlist}\PY{p}{:}
             \PY{n}{newlist}\PY{o}{.}\PY{n}{append}\PY{p}{(}\PY{n}{word}\PY{o}{.}\PY{n}{capitalize}\PY{p}{(}\PY{p}{)}\PY{p}{)}
\end{Verbatim}

    \begin{Verbatim}[commandchars=\\\{\}]
1000000 loops, best of 3: 1.25 µs per loop
    \end{Verbatim}

    \begin{Verbatim}[commandchars=\\\{\}]
{\color{incolor}In [{\color{incolor}60}]:} \PY{o}{\PYZpc{}}\PY{k}{timeit} \PY{n}{newlist} \PY{o}{=} \PY{n+nb}{map}\PY{p}{(}\PY{n+nb}{str}\PY{o}{.}\PY{n}{capitalize}\PY{p}{,}\PY{n}{oldlist}\PY{p}{)}
         \PY{o}{\PYZpc{}}\PY{k}{timeit} \PY{n}{newlist} \PY{o}{=} \PY{p}{[}\PY{n}{word}\PY{o}{.}\PY{n}{capitalize}\PY{p}{(}\PY{p}{)} \PY{k}{for} \PY{n}{word} \PY{o+ow}{in} \PY{n}{oldlist}\PY{p}{]}
\end{Verbatim}

    \begin{Verbatim}[commandchars=\\\{\}]
1000000 loops, best of 3: 1.04 µs per loop
1000000 loops, best of 3: 846 ns per loop
    \end{Verbatim}

    \begin{Verbatim}[commandchars=\\\{\}]
{\color{incolor}In [{\color{incolor}63}]:} \PY{n}{oldlist} \PY{o}{=} \PY{p}{[}\PY{l+s}{u\PYZsq{}}\PY{l+s}{eu}\PY{l+s}{\PYZsq{}}\PY{p}{,}\PY{l+s}{u\PYZsq{}}\PY{l+s}{ich}\PY{l+s}{\PYZsq{}}\PY{p}{,}\PY{l+s}{u\PYZsq{}}\PY{l+s}{i}\PY{l+s}{\PYZsq{}}\PY{p}{,}\PY{l+s}{u\PYZsq{}}\PY{l+s}{io}\PY{l+s}{\PYZsq{}}\PY{p}{,}\PY{l+s}{u\PYZsq{}}\PY{l+s}{yo}\PY{l+s}{\PYZsq{}}\PY{p}{,}\PY{l+s}{u\PYZsq{}}\PY{l+s}{я}\PY{l+s}{\PYZsq{}}\PY{p}{]}
         \PY{n}{newlist} \PY{o}{=} \PY{p}{[}\PY{p}{]}
         \PY{k}{for} \PY{n}{word} \PY{o+ow}{in} \PY{n}{oldlist}\PY{p}{:}
             \PY{n}{newlist}\PY{o}{.}\PY{n}{append}\PY{p}{(}\PY{n}{word}\PY{o}{.}\PY{n}{capitalize}\PY{p}{(}\PY{p}{)}\PY{p}{)}
         \PY{k}{for} \PY{n}{word} \PY{o+ow}{in} \PY{n}{newlist}\PY{p}{:}
             \PY{k}{print} \PY{n}{word}\PY{p}{,}
         \PY{k}{print}
         \PY{n}{newlist} \PY{o}{=} \PY{p}{[}\PY{n}{word}\PY{o}{.}\PY{n}{capitalize}\PY{p}{(}\PY{p}{)} \PY{k}{for} \PY{n}{word} \PY{o+ow}{in} \PY{n}{oldlist}\PY{p}{]}
         \PY{k}{for} \PY{n}{word} \PY{o+ow}{in} \PY{n}{newlist}\PY{p}{:}
             \PY{k}{print} \PY{n}{word}\PY{p}{,}
         \PY{n}{newlist} \PY{o}{=} \PY{n+nb}{map}\PY{p}{(}\PY{n+nb}{unicode}\PY{o}{.}\PY{n}{capitalize}\PY{p}{,}\PY{n}{oldlist}\PY{p}{)}
         \PY{k}{for} \PY{n}{word} \PY{o+ow}{in} \PY{n}{newlist}\PY{p}{:}
             \PY{k}{print} \PY{n}{word}\PY{p}{,}
\end{Verbatim}

    \begin{Verbatim}[commandchars=\\\{\}]
Eu Ich I Io Yo Я
Eu Ich I Io Yo Я Eu Ich I Io Yo Я
    \end{Verbatim}

    \textbf{Sum of squares up to N}

    \begin{Verbatim}[commandchars=\\\{\}]
{\color{incolor}In [{\color{incolor}30}]:} \PY{n}{result} \PY{o}{=} \PY{p}{[}\PY{p}{]}
         \PY{n}{N} \PY{o}{=} \PY{l+m+mi}{10}
         \PY{k}{for} \PY{n}{i} \PY{o+ow}{in} \PY{n+nb}{range}\PY{p}{(}\PY{n}{N}\PY{p}{)}\PY{p}{:}
             \PY{n}{s} \PY{o}{=} \PY{n}{i}\PY{o}{*}\PY{o}{*}\PY{l+m+mi}{2}
             \PY{n}{result}\PY{o}{.}\PY{n}{append}\PY{p}{(}\PY{n}{s}\PY{p}{)}
         \PY{k}{print} \PY{n+nb}{sum}\PY{p}{(}\PY{n}{result}\PY{p}{)}
\end{Verbatim}

    \begin{Verbatim}[commandchars=\\\{\}]
285
    \end{Verbatim}

    \begin{Verbatim}[commandchars=\\\{\}]
{\color{incolor}In [{\color{incolor}31}]:} \PY{o}{\PYZpc{}\PYZpc{}}\PY{k}{timeit}
         \PY{n+nb}{sum}\PY{p}{(}\PY{p}{[}\PY{n}{i}\PY{o}{*}\PY{o}{*}\PY{l+m+mi}{2} \PY{k}{for} \PY{n}{i} \PY{o+ow}{in} \PY{n+nb}{range}\PY{p}{(}\PY{n}{N}\PY{p}{)}\PY{p}{]}\PY{p}{)}
\end{Verbatim}

    \begin{Verbatim}[commandchars=\\\{\}]
1000000 loops, best of 3: 1.38 µs per loop
    \end{Verbatim}

    \begin{Verbatim}[commandchars=\\\{\}]
{\color{incolor}In [{\color{incolor}32}]:} \PY{o}{\PYZpc{}\PYZpc{}}\PY{k}{timeit}
         \PY{n+nb}{sum}\PY{p}{(}\PY{p}{[}\PY{n}{i}\PY{o}{*}\PY{o}{*}\PY{l+m+mi}{2} \PY{k}{for} \PY{n}{i} \PY{o+ow}{in} \PY{n+nb}{xrange}\PY{p}{(}\PY{n}{N}\PY{p}{)}\PY{p}{]}\PY{p}{)}
\end{Verbatim}

    \begin{Verbatim}[commandchars=\\\{\}]
1000000 loops, best of 3: 1.29 µs per loop
    \end{Verbatim}

    \begin{Verbatim}[commandchars=\\\{\}]
{\color{incolor}In [{\color{incolor}33}]:} \PY{o}{\PYZpc{}\PYZpc{}}\PY{k}{timeit}
         \PY{n+nb}{sum}\PY{p}{(}\PY{n}{i}\PY{o}{*}\PY{o}{*}\PY{l+m+mi}{2} \PY{k}{for} \PY{n}{i} \PY{o+ow}{in} \PY{n+nb}{xrange}\PY{p}{(}\PY{n}{N}\PY{p}{)}\PY{p}{)}
\end{Verbatim}

    \begin{Verbatim}[commandchars=\\\{\}]
1000000 loops, best of 3: 1.37 µs per loop
    \end{Verbatim}

    \begin{Verbatim}[commandchars=\\\{\}]
{\color{incolor}In [{\color{incolor}34}]:} \PY{k+kn}{import} \PY{n+nn}{numpy} \PY{k+kn}{as} \PY{n+nn}{np}
\end{Verbatim}

    \begin{Verbatim}[commandchars=\\\{\}]
{\color{incolor}In [{\color{incolor}35}]:} \PY{o}{\PYZpc{}\PYZpc{}}\PY{k}{timeit}
         \PY{p}{(}\PY{n}{np}\PY{o}{.}\PY{n}{arange}\PY{p}{(}\PY{n}{N}\PY{p}{,}\PY{n}{dtype}\PY{o}{=}\PY{n+nb}{int}\PY{p}{)}\PY{o}{*}\PY{o}{*}\PY{l+m+mi}{2}\PY{p}{)}\PY{o}{.}\PY{n}{sum}\PY{p}{(}\PY{p}{)}
\end{Verbatim}

    \begin{Verbatim}[commandchars=\\\{\}]
100000 loops, best of 3: 3.12 µs per loop
    \end{Verbatim}

    \textbf{Column Order}

    \begin{Verbatim}[commandchars=\\\{\}]
{\color{incolor}In [{\color{incolor}39}]:} \PY{n}{arr\PYZus{}c} \PY{o}{=} \PY{n}{np}\PY{o}{.}\PY{n}{ones}\PY{p}{(}\PY{p}{(}\PY{l+m+mi}{1000}\PY{p}{,}\PY{l+m+mi}{1000}\PY{p}{)}\PY{p}{,}\PY{n}{order}\PY{o}{=}\PY{l+s}{\PYZsq{}}\PY{l+s}{C}\PY{l+s}{\PYZsq{}}\PY{p}{)}
         \PY{n}{arr\PYZus{}f} \PY{o}{=} \PY{n}{np}\PY{o}{.}\PY{n}{ones}\PY{p}{(}\PY{p}{(}\PY{l+m+mi}{1000}\PY{p}{,}\PY{l+m+mi}{1000}\PY{p}{)}\PY{p}{,}\PY{n}{order}\PY{o}{=}\PY{l+s}{\PYZsq{}}\PY{l+s}{F}\PY{l+s}{\PYZsq{}}\PY{p}{)}
\end{Verbatim}

    \begin{Verbatim}[commandchars=\\\{\}]
{\color{incolor}In [{\color{incolor}40}]:} \PY{o}{\PYZpc{}}\PY{k}{timeit} \PY{n}{arr\PYZus{}c}\PY{o}{.}\PY{n}{sum}\PY{p}{(}\PY{l+m+mi}{1}\PY{p}{)}
\end{Verbatim}

    \begin{Verbatim}[commandchars=\\\{\}]
1000 loops, best of 3: 625 µs per loop
    \end{Verbatim}

    \begin{Verbatim}[commandchars=\\\{\}]
{\color{incolor}In [{\color{incolor}41}]:} \PY{o}{\PYZpc{}}\PY{k}{timeit} \PY{n}{arr\PYZus{}f}\PY{o}{.}\PY{n}{sum}\PY{p}{(}\PY{l+m+mi}{1}\PY{p}{)}
\end{Verbatim}

    \begin{Verbatim}[commandchars=\\\{\}]
1000 loops, best of 3: 611 µs per loop
    \end{Verbatim}

    \begin{Verbatim}[commandchars=\\\{\}]
{\color{incolor}In [{\color{incolor}42}]:} \PY{n}{arr\PYZus{}f}\PY{o}{.}\PY{n}{flags}
\end{Verbatim}

            \begin{Verbatim}[commandchars=\\\{\}]
{\color{outcolor}Out[{\color{outcolor}42}]:}   C\_CONTIGUOUS : False
           F\_CONTIGUOUS : True
           OWNDATA : True
           WRITEABLE : True
           ALIGNED : True
           UPDATEIFCOPY : False
\end{Verbatim}
        
    \begin{Verbatim}[commandchars=\\\{\}]
{\color{incolor}In [{\color{incolor}43}]:} \PY{n}{arr\PYZus{}c}\PY{o}{.}\PY{n}{flags}
\end{Verbatim}

            \begin{Verbatim}[commandchars=\\\{\}]
{\color{outcolor}Out[{\color{outcolor}43}]:}   C\_CONTIGUOUS : True
           F\_CONTIGUOUS : False
           OWNDATA : True
           WRITEABLE : True
           ALIGNED : True
           UPDATEIFCOPY : False
\end{Verbatim}
        
    \begin{Verbatim}[commandchars=\\\{\}]
{\color{incolor}In [{\color{incolor}2}]:} \PY{o}{\PYZpc{}}\PY{k}{reload\PYZus{}ext} \PY{n}{version\PYZus{}information}
        
        \PY{o}{\PYZpc{}}\PY{k}{version\PYZus{}information} \PY{n}{numpy}
\end{Verbatim}
\texttt{\color{outcolor}Out[{\color{outcolor}2}]:}
    
    \begin{tabular}{|l|l|}\hline
{\bf Software} & {\bf Version} \\ \hline\hline
Python & 2.7.4 (default, Sep 26 2013, 03:20:56) [GCC 4.7.3] \\ \hline
IPython & 1.1.0 \\ \hline
OS & posix [linux2] \\ \hline
numpy & 1.7.1 \\ \hline
\hline \multicolumn{2}{|l|}{So Okt 06 15:27:18 2013 CEST} \\ \hline
\end{tabular}


    

    \begin{Verbatim}[commandchars=\\\{\}]
{\color{incolor}In [{\color{incolor}1}]:} \PY{k+kn}{from} \PY{n+nn}{IPython.core.display} \PY{k+kn}{import} \PY{n}{HTML}
        \PY{k}{def} \PY{n+nf}{css\PYZus{}styling}\PY{p}{(}\PY{p}{)}\PY{p}{:}
            \PY{n}{styles} \PY{o}{=} \PY{n+nb}{open}\PY{p}{(}\PY{l+s}{\PYZdq{}}\PY{l+s}{./styles/custom.css}\PY{l+s}{\PYZdq{}}\PY{p}{,} \PY{l+s}{\PYZdq{}}\PY{l+s}{r}\PY{l+s}{\PYZdq{}}\PY{p}{)}\PY{o}{.}\PY{n}{read}\PY{p}{(}\PY{p}{)}
            \PY{k}{return} \PY{n}{HTML}\PY{p}{(}\PY{n}{styles}\PY{p}{)}
        \PY{n}{css\PYZus{}styling}\PY{p}{(}\PY{p}{)}
\end{Verbatim}

            \begin{Verbatim}[commandchars=\\\{\}]
{\color{outcolor}Out[{\color{outcolor}1}]:} <IPython.core.display.HTML at 0x7f10fc2d0850>
\end{Verbatim}
        
    \begin{Verbatim}[commandchars=\\\{\}]
{\color{incolor}In [{\color{incolor}}]:} 
\end{Verbatim}


    % Add a bibliography block to the postdoc
    
    
    
    \end{document}


%\newpage
%
% Default to the notebook output style

    


% Inherit from the specified cell style.




    
\documentclass{article}

    
    
    \usepackage{graphicx} % Used to insert images
    \usepackage{adjustbox} % Used to constrain images to a maximum size 
    \usepackage{color} % Allow colors to be defined
    \usepackage{enumerate} % Needed for markdown enumerations to work
    \usepackage{geometry} % Used to adjust the document margins
    \usepackage{amsmath} % Equations
    \usepackage{amssymb} % Equations
    \usepackage[mathletters]{ucs} % Extended unicode (utf-8) support
    \usepackage[utf8x]{inputenc} % Allow utf-8 characters in the tex document
    \usepackage{fancyvrb} % verbatim replacement that allows latex
    \usepackage{grffile} % extends the file name processing of package graphics 
                         % to support a larger range 
    % The hyperref package gives us a pdf with properly built
    % internal navigation ('pdf bookmarks' for the table of contents,
    % internal cross-reference links, web links for URLs, etc.)
    \usepackage{hyperref}
    \usepackage{longtable} % longtable support required by pandoc >1.10
    \usepackage{booktabs}  % table support for pandoc > 1.12.2
    

    
    
    \definecolor{orange}{cmyk}{0,0.4,0.8,0.2}
    \definecolor{darkorange}{rgb}{.71,0.21,0.01}
    \definecolor{darkgreen}{rgb}{.12,.54,.11}
    \definecolor{myteal}{rgb}{.26, .44, .56}
    \definecolor{gray}{gray}{0.45}
    \definecolor{lightgray}{gray}{.95}
    \definecolor{mediumgray}{gray}{.8}
    \definecolor{inputbackground}{rgb}{.95, .95, .85}
    \definecolor{outputbackground}{rgb}{.95, .95, .95}
    \definecolor{traceback}{rgb}{1, .95, .95}
    % ansi colors
    \definecolor{red}{rgb}{.6,0,0}
    \definecolor{green}{rgb}{0,.65,0}
    \definecolor{brown}{rgb}{0.6,0.6,0}
    \definecolor{blue}{rgb}{0,.145,.698}
    \definecolor{purple}{rgb}{.698,.145,.698}
    \definecolor{cyan}{rgb}{0,.698,.698}
    \definecolor{lightgray}{gray}{0.5}
    
    % bright ansi colors
    \definecolor{darkgray}{gray}{0.25}
    \definecolor{lightred}{rgb}{1.0,0.39,0.28}
    \definecolor{lightgreen}{rgb}{0.48,0.99,0.0}
    \definecolor{lightblue}{rgb}{0.53,0.81,0.92}
    \definecolor{lightpurple}{rgb}{0.87,0.63,0.87}
    \definecolor{lightcyan}{rgb}{0.5,1.0,0.83}
    
    % commands and environments needed by pandoc snippets
    % extracted from the output of `pandoc -s`
    \DefineVerbatimEnvironment{Highlighting}{Verbatim}{commandchars=\\\{\}}
    % Add ',fontsize=\small' for more characters per line
    \newenvironment{Shaded}{}{}
    \newcommand{\KeywordTok}[1]{\textcolor[rgb]{0.00,0.44,0.13}{\textbf{{#1}}}}
    \newcommand{\DataTypeTok}[1]{\textcolor[rgb]{0.56,0.13,0.00}{{#1}}}
    \newcommand{\DecValTok}[1]{\textcolor[rgb]{0.25,0.63,0.44}{{#1}}}
    \newcommand{\BaseNTok}[1]{\textcolor[rgb]{0.25,0.63,0.44}{{#1}}}
    \newcommand{\FloatTok}[1]{\textcolor[rgb]{0.25,0.63,0.44}{{#1}}}
    \newcommand{\CharTok}[1]{\textcolor[rgb]{0.25,0.44,0.63}{{#1}}}
    \newcommand{\StringTok}[1]{\textcolor[rgb]{0.25,0.44,0.63}{{#1}}}
    \newcommand{\CommentTok}[1]{\textcolor[rgb]{0.38,0.63,0.69}{\textit{{#1}}}}
    \newcommand{\OtherTok}[1]{\textcolor[rgb]{0.00,0.44,0.13}{{#1}}}
    \newcommand{\AlertTok}[1]{\textcolor[rgb]{1.00,0.00,0.00}{\textbf{{#1}}}}
    \newcommand{\FunctionTok}[1]{\textcolor[rgb]{0.02,0.16,0.49}{{#1}}}
    \newcommand{\RegionMarkerTok}[1]{{#1}}
    \newcommand{\ErrorTok}[1]{\textcolor[rgb]{1.00,0.00,0.00}{\textbf{{#1}}}}
    \newcommand{\NormalTok}[1]{{#1}}
    
    % Define a nice break command that doesn't care if a line doesn't already
    % exist.
    \def\br{\hspace*{\fill} \\* }
    % Math Jax compatability definitions
    \def\gt{>}
    \def\lt{<}
    % Document parameters
    \title{VIII-Benchmarks}
    
    
    

    % Pygments definitions
    
\makeatletter
\def\PY@reset{\let\PY@it=\relax \let\PY@bf=\relax%
    \let\PY@ul=\relax \let\PY@tc=\relax%
    \let\PY@bc=\relax \let\PY@ff=\relax}
\def\PY@tok#1{\csname PY@tok@#1\endcsname}
\def\PY@toks#1+{\ifx\relax#1\empty\else%
    \PY@tok{#1}\expandafter\PY@toks\fi}
\def\PY@do#1{\PY@bc{\PY@tc{\PY@ul{%
    \PY@it{\PY@bf{\PY@ff{#1}}}}}}}
\def\PY#1#2{\PY@reset\PY@toks#1+\relax+\PY@do{#2}}

\expandafter\def\csname PY@tok@gd\endcsname{\def\PY@tc##1{\textcolor[rgb]{0.63,0.00,0.00}{##1}}}
\expandafter\def\csname PY@tok@gu\endcsname{\let\PY@bf=\textbf\def\PY@tc##1{\textcolor[rgb]{0.50,0.00,0.50}{##1}}}
\expandafter\def\csname PY@tok@gt\endcsname{\def\PY@tc##1{\textcolor[rgb]{0.00,0.27,0.87}{##1}}}
\expandafter\def\csname PY@tok@gs\endcsname{\let\PY@bf=\textbf}
\expandafter\def\csname PY@tok@gr\endcsname{\def\PY@tc##1{\textcolor[rgb]{1.00,0.00,0.00}{##1}}}
\expandafter\def\csname PY@tok@cm\endcsname{\let\PY@it=\textit\def\PY@tc##1{\textcolor[rgb]{0.25,0.50,0.50}{##1}}}
\expandafter\def\csname PY@tok@vg\endcsname{\def\PY@tc##1{\textcolor[rgb]{0.10,0.09,0.49}{##1}}}
\expandafter\def\csname PY@tok@m\endcsname{\def\PY@tc##1{\textcolor[rgb]{0.40,0.40,0.40}{##1}}}
\expandafter\def\csname PY@tok@mh\endcsname{\def\PY@tc##1{\textcolor[rgb]{0.40,0.40,0.40}{##1}}}
\expandafter\def\csname PY@tok@go\endcsname{\def\PY@tc##1{\textcolor[rgb]{0.53,0.53,0.53}{##1}}}
\expandafter\def\csname PY@tok@ge\endcsname{\let\PY@it=\textit}
\expandafter\def\csname PY@tok@vc\endcsname{\def\PY@tc##1{\textcolor[rgb]{0.10,0.09,0.49}{##1}}}
\expandafter\def\csname PY@tok@il\endcsname{\def\PY@tc##1{\textcolor[rgb]{0.40,0.40,0.40}{##1}}}
\expandafter\def\csname PY@tok@cs\endcsname{\let\PY@it=\textit\def\PY@tc##1{\textcolor[rgb]{0.25,0.50,0.50}{##1}}}
\expandafter\def\csname PY@tok@cp\endcsname{\def\PY@tc##1{\textcolor[rgb]{0.74,0.48,0.00}{##1}}}
\expandafter\def\csname PY@tok@gi\endcsname{\def\PY@tc##1{\textcolor[rgb]{0.00,0.63,0.00}{##1}}}
\expandafter\def\csname PY@tok@gh\endcsname{\let\PY@bf=\textbf\def\PY@tc##1{\textcolor[rgb]{0.00,0.00,0.50}{##1}}}
\expandafter\def\csname PY@tok@ni\endcsname{\let\PY@bf=\textbf\def\PY@tc##1{\textcolor[rgb]{0.60,0.60,0.60}{##1}}}
\expandafter\def\csname PY@tok@nl\endcsname{\def\PY@tc##1{\textcolor[rgb]{0.63,0.63,0.00}{##1}}}
\expandafter\def\csname PY@tok@nn\endcsname{\let\PY@bf=\textbf\def\PY@tc##1{\textcolor[rgb]{0.00,0.00,1.00}{##1}}}
\expandafter\def\csname PY@tok@no\endcsname{\def\PY@tc##1{\textcolor[rgb]{0.53,0.00,0.00}{##1}}}
\expandafter\def\csname PY@tok@na\endcsname{\def\PY@tc##1{\textcolor[rgb]{0.49,0.56,0.16}{##1}}}
\expandafter\def\csname PY@tok@nb\endcsname{\def\PY@tc##1{\textcolor[rgb]{0.00,0.50,0.00}{##1}}}
\expandafter\def\csname PY@tok@nc\endcsname{\let\PY@bf=\textbf\def\PY@tc##1{\textcolor[rgb]{0.00,0.00,1.00}{##1}}}
\expandafter\def\csname PY@tok@nd\endcsname{\def\PY@tc##1{\textcolor[rgb]{0.67,0.13,1.00}{##1}}}
\expandafter\def\csname PY@tok@ne\endcsname{\let\PY@bf=\textbf\def\PY@tc##1{\textcolor[rgb]{0.82,0.25,0.23}{##1}}}
\expandafter\def\csname PY@tok@nf\endcsname{\def\PY@tc##1{\textcolor[rgb]{0.00,0.00,1.00}{##1}}}
\expandafter\def\csname PY@tok@si\endcsname{\let\PY@bf=\textbf\def\PY@tc##1{\textcolor[rgb]{0.73,0.40,0.53}{##1}}}
\expandafter\def\csname PY@tok@s2\endcsname{\def\PY@tc##1{\textcolor[rgb]{0.73,0.13,0.13}{##1}}}
\expandafter\def\csname PY@tok@vi\endcsname{\def\PY@tc##1{\textcolor[rgb]{0.10,0.09,0.49}{##1}}}
\expandafter\def\csname PY@tok@nt\endcsname{\let\PY@bf=\textbf\def\PY@tc##1{\textcolor[rgb]{0.00,0.50,0.00}{##1}}}
\expandafter\def\csname PY@tok@nv\endcsname{\def\PY@tc##1{\textcolor[rgb]{0.10,0.09,0.49}{##1}}}
\expandafter\def\csname PY@tok@s1\endcsname{\def\PY@tc##1{\textcolor[rgb]{0.73,0.13,0.13}{##1}}}
\expandafter\def\csname PY@tok@kd\endcsname{\let\PY@bf=\textbf\def\PY@tc##1{\textcolor[rgb]{0.00,0.50,0.00}{##1}}}
\expandafter\def\csname PY@tok@sh\endcsname{\def\PY@tc##1{\textcolor[rgb]{0.73,0.13,0.13}{##1}}}
\expandafter\def\csname PY@tok@sc\endcsname{\def\PY@tc##1{\textcolor[rgb]{0.73,0.13,0.13}{##1}}}
\expandafter\def\csname PY@tok@sx\endcsname{\def\PY@tc##1{\textcolor[rgb]{0.00,0.50,0.00}{##1}}}
\expandafter\def\csname PY@tok@bp\endcsname{\def\PY@tc##1{\textcolor[rgb]{0.00,0.50,0.00}{##1}}}
\expandafter\def\csname PY@tok@c1\endcsname{\let\PY@it=\textit\def\PY@tc##1{\textcolor[rgb]{0.25,0.50,0.50}{##1}}}
\expandafter\def\csname PY@tok@kc\endcsname{\let\PY@bf=\textbf\def\PY@tc##1{\textcolor[rgb]{0.00,0.50,0.00}{##1}}}
\expandafter\def\csname PY@tok@c\endcsname{\let\PY@it=\textit\def\PY@tc##1{\textcolor[rgb]{0.25,0.50,0.50}{##1}}}
\expandafter\def\csname PY@tok@mf\endcsname{\def\PY@tc##1{\textcolor[rgb]{0.40,0.40,0.40}{##1}}}
\expandafter\def\csname PY@tok@err\endcsname{\def\PY@bc##1{\setlength{\fboxsep}{0pt}\fcolorbox[rgb]{1.00,0.00,0.00}{1,1,1}{\strut ##1}}}
\expandafter\def\csname PY@tok@mb\endcsname{\def\PY@tc##1{\textcolor[rgb]{0.40,0.40,0.40}{##1}}}
\expandafter\def\csname PY@tok@ss\endcsname{\def\PY@tc##1{\textcolor[rgb]{0.10,0.09,0.49}{##1}}}
\expandafter\def\csname PY@tok@sr\endcsname{\def\PY@tc##1{\textcolor[rgb]{0.73,0.40,0.53}{##1}}}
\expandafter\def\csname PY@tok@mo\endcsname{\def\PY@tc##1{\textcolor[rgb]{0.40,0.40,0.40}{##1}}}
\expandafter\def\csname PY@tok@kn\endcsname{\let\PY@bf=\textbf\def\PY@tc##1{\textcolor[rgb]{0.00,0.50,0.00}{##1}}}
\expandafter\def\csname PY@tok@mi\endcsname{\def\PY@tc##1{\textcolor[rgb]{0.40,0.40,0.40}{##1}}}
\expandafter\def\csname PY@tok@gp\endcsname{\let\PY@bf=\textbf\def\PY@tc##1{\textcolor[rgb]{0.00,0.00,0.50}{##1}}}
\expandafter\def\csname PY@tok@o\endcsname{\def\PY@tc##1{\textcolor[rgb]{0.40,0.40,0.40}{##1}}}
\expandafter\def\csname PY@tok@kr\endcsname{\let\PY@bf=\textbf\def\PY@tc##1{\textcolor[rgb]{0.00,0.50,0.00}{##1}}}
\expandafter\def\csname PY@tok@s\endcsname{\def\PY@tc##1{\textcolor[rgb]{0.73,0.13,0.13}{##1}}}
\expandafter\def\csname PY@tok@kp\endcsname{\def\PY@tc##1{\textcolor[rgb]{0.00,0.50,0.00}{##1}}}
\expandafter\def\csname PY@tok@w\endcsname{\def\PY@tc##1{\textcolor[rgb]{0.73,0.73,0.73}{##1}}}
\expandafter\def\csname PY@tok@kt\endcsname{\def\PY@tc##1{\textcolor[rgb]{0.69,0.00,0.25}{##1}}}
\expandafter\def\csname PY@tok@ow\endcsname{\let\PY@bf=\textbf\def\PY@tc##1{\textcolor[rgb]{0.67,0.13,1.00}{##1}}}
\expandafter\def\csname PY@tok@sb\endcsname{\def\PY@tc##1{\textcolor[rgb]{0.73,0.13,0.13}{##1}}}
\expandafter\def\csname PY@tok@k\endcsname{\let\PY@bf=\textbf\def\PY@tc##1{\textcolor[rgb]{0.00,0.50,0.00}{##1}}}
\expandafter\def\csname PY@tok@se\endcsname{\let\PY@bf=\textbf\def\PY@tc##1{\textcolor[rgb]{0.73,0.40,0.13}{##1}}}
\expandafter\def\csname PY@tok@sd\endcsname{\let\PY@it=\textit\def\PY@tc##1{\textcolor[rgb]{0.73,0.13,0.13}{##1}}}

\def\PYZbs{\char`\\}
\def\PYZus{\char`\_}
\def\PYZob{\char`\{}
\def\PYZcb{\char`\}}
\def\PYZca{\char`\^}
\def\PYZam{\char`\&}
\def\PYZlt{\char`\<}
\def\PYZgt{\char`\>}
\def\PYZsh{\char`\#}
\def\PYZpc{\char`\%}
\def\PYZdl{\char`\$}
\def\PYZhy{\char`\-}
\def\PYZsq{\char`\'}
\def\PYZdq{\char`\"}
\def\PYZti{\char`\~}
% for compatibility with earlier versions
\def\PYZat{@}
\def\PYZlb{[}
\def\PYZrb{]}
\makeatother


    % Exact colors from NB
    \definecolor{incolor}{rgb}{0.0, 0.0, 0.5}
    \definecolor{outcolor}{rgb}{0.545, 0.0, 0.0}



    
    % Prevent overflowing lines due to hard-to-break entities
    \sloppy 
    % Setup hyperref package
    \hypersetup{
      breaklinks=true,  % so long urls are correctly broken across lines
      colorlinks=true,
      urlcolor=blue,
      linkcolor=darkorange,
      citecolor=darkgreen,
      }
    % Slightly bigger margins than the latex defaults
    
    \geometry{verbose,tmargin=1in,bmargin=1in,lmargin=1in,rmargin=1in}
    
    

    \begin{document}
    
    
    \maketitle
    
    

    
    

    \section{VIII-Benchmarks}\label{viii-benchmarks}

    \emph{The first benchmark in this notebook first appeared as a}
\href{http://jakevdp.github.io/blog/2012/08/24/numba-vs-cython/}{\emph{post}}
\emph{by Jake Vanderplas on the blog}
\href{http://jakevdp.github.io}{\emph{Pythonic Perambulations}}

    \subsubsection{Index}\label{index}

\begin{itemize}
\itemsep1pt\parskip0pt\parsep0pt
\item
  \hyperref[Pairwise-Distance]{Pairwise Distance}

  \begin{itemize}
  \itemsep1pt\parskip0pt\parsep0pt
  \item
    \hyperref[Numpy-broadcasting]{NumPy Broadcasting}
  \item
    \hyperref[Pure-python]{Pure Python}
  \item
    \hyperref[Numba]{Numba}
  \item
    \hyperref[Cython]{Cython}
  \item
    \hyperref[Parakeet]{Parakeet}
  \item
    \hyperref[Fortran-f2py]{Fortran/F2Py}
  \item
    \hyperref[Scipy-pairwise-distances]{SciPy Euclidean Distance}
  \item
    \hyperref[Scikitlearn-pairwise-distances]{Scikit-learn Distance}
  \end{itemize}
\item
  \hyperref[1Dux5cux2520Black-Scholes]{1D Black-Scholes}

  \begin{itemize}
  \itemsep1pt\parskip0pt\parsep0pt
  \item
    \hyperref[1DBS-PurePython]{Pure Python}
  \item
    \hyperref[1DBS-Numba]{Numba}
  \item
    \hyperref[1DBS-Cython]{Cython}
  \item
    \hyperref[1DBS-MATLAB]{Matlab}
  \end{itemize}
\end{itemize}

    


    \subsection{Pairwise distance}


    \begin{Verbatim}[commandchars=\\\{\}]
{\color{incolor}In [{\color{incolor}1}]:} \PY{k+kn}{import} \PY{n+nn}{numpy} \PY{k+kn}{as} \PY{n+nn}{np}
        \PY{k+kn}{import} \PY{n+nn}{matplotlib} \PY{k+kn}{as} \PY{n+nn}{mpl}
        \PY{k+kn}{import} \PY{n+nn}{matplotlib.pyplot} \PY{k+kn}{as} \PY{n+nn}{plt}
        \PY{n}{X} \PY{o}{=} \PY{n}{np}\PY{o}{.}\PY{n}{random}\PY{o}{.}\PY{n}{random}\PY{p}{(}\PY{p}{(}\PY{l+m+mi}{1000}\PY{p}{,} \PY{l+m+mi}{3}\PY{p}{)}\PY{p}{)}
\end{Verbatim}

    \begin{Verbatim}[commandchars=\\\{\}]
{\color{incolor}In [{\color{incolor}2}]:} \PY{k}{def} \PY{n+nf}{parse\PYZus{}cap}\PY{p}{(}\PY{n}{s}\PY{p}{)}\PY{p}{:}
            \PY{k}{return} \PY{n}{cap}\PY{o}{.}\PY{n}{stdout}\PY{o}{.}\PY{n}{split}\PY{p}{(}\PY{l+s}{\PYZsq{}}\PY{l+s}{ }\PY{l+s}{\PYZsq{}}\PY{p}{)}\PY{p}{[}\PY{o}{\PYZhy{}}\PY{l+m+mi}{4}\PY{p}{:}\PY{o}{\PYZhy{}}\PY{l+m+mi}{2}\PY{p}{]}
        \PY{n}{Benchs} \PY{o}{=} \PY{p}{[}\PY{l+s}{\PYZsq{}}\PY{l+s}{python}\PY{l+s+se}{\PYZbs{}n}\PY{l+s}{loop}\PY{l+s}{\PYZsq{}}\PY{p}{,} \PY{l+s}{\PYZsq{}}\PY{l+s}{numpy}\PY{l+s+se}{\PYZbs{}n}\PY{l+s}{broadc.}\PY{l+s}{\PYZsq{}}\PY{p}{,} \PY{l+s}{\PYZsq{}}\PY{l+s}{sklearn}\PY{l+s}{\PYZsq{}}\PY{p}{,} \PY{l+s}{\PYZsq{}}\PY{l+s}{fortran/}\PY{l+s+se}{\PYZbs{}n}\PY{l+s}{f2py}\PY{l+s}{\PYZsq{}}\PY{p}{,} \PY{l+s}{\PYZsq{}}\PY{l+s}{scipy}\PY{l+s}{\PYZsq{}}\PY{p}{,} \PY{l+s}{\PYZsq{}}\PY{l+s}{cython}\PY{l+s}{\PYZsq{}}\PY{p}{,} \PY{l+s}{\PYZsq{}}\PY{l+s}{numba}\PY{l+s}{\PYZsq{}}\PY{p}{,}\PY{l+s}{\PYZsq{}}\PY{l+s}{parakeet}\PY{l+s}{\PYZsq{}}\PY{p}{]}
        \PY{n}{Benchmarks} \PY{o}{=} \PY{n+nb}{dict}\PY{p}{(}\PY{p}{(}\PY{n}{bench}\PY{p}{,}\PY{l+m+mi}{0}\PY{p}{)} \PY{k}{for} \PY{n}{bench} \PY{o+ow}{in} \PY{n}{Benchs}\PY{p}{)}
\end{Verbatim}

    We'll start by defining the array which we'll use for the benchmarks:
one thousand points in three dimensions.

    \begin{Verbatim}[commandchars=\\\{\}]
{\color{incolor}In [{\color{incolor}3}]:} \PY{n}{np}\PY{o}{.}\PY{n}{show\PYZus{}config}\PY{p}{(}\PY{p}{)}
\end{Verbatim}

    \begin{Verbatim}[commandchars=\\\{\}]
lapack\_opt\_info:
    libraries = ['mkl\_lapack95\_lp64', 'mkl\_intel\_lp64', 'mkl\_intel\_thread', 'mkl\_core', 'iomp5', 'pthread']
    library\_dirs = ['/home/jpsilva/anaconda/lib']
    define\_macros = [('SCIPY\_MKL\_H', None)]
    include\_dirs = ['/home/jpsilva/anaconda/include']
blas\_opt\_info:
    libraries = ['mkl\_intel\_lp64', 'mkl\_intel\_thread', 'mkl\_core', 'iomp5', 'pthread']
    library\_dirs = ['/home/jpsilva/anaconda/lib']
    define\_macros = [('SCIPY\_MKL\_H', None)]
    include\_dirs = ['/home/jpsilva/anaconda/include']
openblas\_lapack\_info:
  NOT AVAILABLE
lapack\_mkl\_info:
    libraries = ['mkl\_lapack95\_lp64', 'mkl\_intel\_lp64', 'mkl\_intel\_thread', 'mkl\_core', 'iomp5', 'pthread']
    library\_dirs = ['/home/jpsilva/anaconda/lib']
    define\_macros = [('SCIPY\_MKL\_H', None)]
    include\_dirs = ['/home/jpsilva/anaconda/include']
blas\_mkl\_info:
    libraries = ['mkl\_intel\_lp64', 'mkl\_intel\_thread', 'mkl\_core', 'iomp5', 'pthread']
    library\_dirs = ['/home/jpsilva/anaconda/lib']
    define\_macros = [('SCIPY\_MKL\_H', None)]
    include\_dirs = ['/home/jpsilva/anaconda/include']
mkl\_info:
    libraries = ['mkl\_intel\_lp64', 'mkl\_intel\_thread', 'mkl\_core', 'iomp5', 'pthread']
    library\_dirs = ['/home/jpsilva/anaconda/lib']
    define\_macros = [('SCIPY\_MKL\_H', None)]
    include\_dirs = ['/home/jpsilva/anaconda/include']
    \end{Verbatim}

    


    \subsection{Numpy Function With Broadcasting}


    We'll start with a typical numpy broadcasting approach to this problem.
Numpy broadcasting is an abstraction that allows loops over array
indices to be executed in compiled C. For many applications, this is
extremely fast and efficient. Unfortunately, there is a problem with
broadcasting approaches that comes up here: it ends up allocating hidden
temporary arrays which can eat up memory and cause computational
overhead. Nevertheless, it's a good comparison to have. The function
looks like this:

    \begin{Verbatim}[commandchars=\\\{\}]
{\color{incolor}In [{\color{incolor}4}]:} \PY{o}{\PYZpc{}\PYZpc{}}\PY{k}{capture} \PY{n}{cap}
        
        \PY{k}{def} \PY{n+nf}{pairwise\PYZus{}numpy}\PY{p}{(}\PY{n}{X}\PY{p}{)}\PY{p}{:}
            \PY{k}{return} \PY{n}{np}\PY{o}{.}\PY{n}{sqrt}\PY{p}{(}\PY{p}{(}\PY{p}{(}\PY{n}{X}\PY{p}{[}\PY{p}{:}\PY{p}{,} \PY{n+nb+bp}{None}\PY{p}{,} \PY{p}{:}\PY{p}{]} \PY{o}{\PYZhy{}} \PY{n}{X}\PY{p}{)} \PY{o}{*}\PY{o}{*} \PY{l+m+mi}{2}\PY{p}{)}\PY{o}{.}\PY{n}{sum}\PY{p}{(}\PY{o}{\PYZhy{}}\PY{l+m+mi}{1}\PY{p}{)}\PY{p}{)}
        \PY{o}{\PYZpc{}}\PY{k}{timeit} \PY{n}{pairwise\PYZus{}numpy}\PY{p}{(}\PY{n}{X}\PY{p}{)}
\end{Verbatim}

    \begin{Verbatim}[commandchars=\\\{\}]
{\color{incolor}In [{\color{incolor}5}]:} \PY{n}{Benchmarks}\PY{p}{[}\PY{l+s}{\PYZsq{}}\PY{l+s}{numpy}\PY{l+s+se}{\PYZbs{}n}\PY{l+s}{broadc.}\PY{l+s}{\PYZsq{}}\PY{p}{]} \PY{o}{=} \PY{n}{parse\PYZus{}cap}\PY{p}{(}\PY{n}{cap}\PY{p}{)}
        \PY{k}{print} \PY{n}{cap}\PY{o}{.}\PY{n}{stdout}
\end{Verbatim}

    \begin{Verbatim}[commandchars=\\\{\}]
10 loops, best of 3: 40 ms per loop
    \end{Verbatim}

    


    \subsection{Pure Python Function}


    A loop-based solution avoids the overhead associated with temporary
arrays, and can be written like this:

    \begin{Verbatim}[commandchars=\\\{\}]
{\color{incolor}In [{\color{incolor}6}]:} \PY{k}{def} \PY{n+nf}{pairwise\PYZus{}python}\PY{p}{(}\PY{n}{X}\PY{p}{)}\PY{p}{:}
            \PY{n}{M} \PY{o}{=} \PY{n}{X}\PY{o}{.}\PY{n}{shape}\PY{p}{[}\PY{l+m+mi}{0}\PY{p}{]}
            \PY{n}{N} \PY{o}{=} \PY{n}{X}\PY{o}{.}\PY{n}{shape}\PY{p}{[}\PY{l+m+mi}{1}\PY{p}{]}
            \PY{n}{D} \PY{o}{=} \PY{n}{np}\PY{o}{.}\PY{n}{empty}\PY{p}{(}\PY{p}{(}\PY{n}{M}\PY{p}{,} \PY{n}{M}\PY{p}{)}\PY{p}{,} \PY{n}{dtype}\PY{o}{=}\PY{n}{np}\PY{o}{.}\PY{n}{float}\PY{p}{)}
            \PY{k}{for} \PY{n}{i} \PY{o+ow}{in} \PY{n+nb}{range}\PY{p}{(}\PY{n}{M}\PY{p}{)}\PY{p}{:}
                \PY{k}{for} \PY{n}{j} \PY{o+ow}{in} \PY{n+nb}{range}\PY{p}{(}\PY{n}{M}\PY{p}{)}\PY{p}{:}
                    \PY{n}{d} \PY{o}{=} \PY{l+m+mf}{0.0}
                    \PY{k}{for} \PY{n}{k} \PY{o+ow}{in} \PY{n+nb}{range}\PY{p}{(}\PY{n}{N}\PY{p}{)}\PY{p}{:}
                        \PY{n}{tmp} \PY{o}{=} \PY{n}{X}\PY{p}{[}\PY{n}{i}\PY{p}{,} \PY{n}{k}\PY{p}{]} \PY{o}{\PYZhy{}} \PY{n}{X}\PY{p}{[}\PY{n}{j}\PY{p}{,} \PY{n}{k}\PY{p}{]}
                        \PY{n}{d} \PY{o}{+}\PY{o}{=} \PY{n}{tmp} \PY{o}{*} \PY{n}{tmp}
                    \PY{n}{D}\PY{p}{[}\PY{n}{i}\PY{p}{,} \PY{n}{j}\PY{p}{]} \PY{o}{=} \PY{n}{np}\PY{o}{.}\PY{n}{sqrt}\PY{p}{(}\PY{n}{d}\PY{p}{)}
            \PY{k}{return} \PY{n}{D}
\end{Verbatim}

    \begin{Verbatim}[commandchars=\\\{\}]
{\color{incolor}In [{\color{incolor}7}]:} \PY{o}{\PYZpc{}\PYZpc{}}\PY{k}{capture} \PY{n}{cap}
        \PY{o}{\PYZpc{}}\PY{k}{timeit} \PY{n}{pairwise\PYZus{}python}\PY{p}{(}\PY{n}{X}\PY{p}{)}
\end{Verbatim}

    \begin{Verbatim}[commandchars=\\\{\}]
{\color{incolor}In [{\color{incolor}8}]:} \PY{n}{Benchmarks}\PY{p}{[}\PY{l+s}{\PYZsq{}}\PY{l+s}{python}\PY{l+s+se}{\PYZbs{}n}\PY{l+s}{loop}\PY{l+s}{\PYZsq{}}\PY{p}{]} \PY{o}{=} \PY{n}{parse\PYZus{}cap}\PY{p}{(}\PY{n}{cap}\PY{p}{)}
        \PY{k}{print} \PY{n}{cap}\PY{o}{.}\PY{n}{stdout}
\end{Verbatim}

    \begin{Verbatim}[commandchars=\\\{\}]
1 loops, best of 3: 2.7 s per loop
    \end{Verbatim}

    As we see, it is over 100 times slower than the numpy broadcasting
approach! This is due to Python's dynamic type checking, which can
drastically slow down nested loops. With these two solutions, we're left
with a tradeoff between efficiency of computation and efficiency of
memory usage. This is where tools like Numba and Cython become vital

    


    \subsection{Numba Wrapper}


    \href{http://numba.pydata.org/}{Numba} is an LLVM compiler for python
code, which allows code written in Python to be converted to highly
efficient compiled code in real-time.\\Numba is extremely simple to use.
We just wrap our python function with \texttt{autojit} (JIT stands for
``just in time'' compilation) to automatically create an efficient,
compiled version of the function:

    \textbf{Note} Somehow, importing pylab breaks numba

    \begin{Verbatim}[commandchars=\\\{\}]
{\color{incolor}In [{\color{incolor}12}]:} \PY{o}{\PYZpc{}\PYZpc{}}\PY{k}{capture} \PY{n}{cap}
         
         \PY{k+kn}{from} \PY{n+nn}{numba} \PY{k+kn}{import} \PY{n}{double}\PY{p}{,} \PY{n}{jit}\PY{p}{,}\PY{n}{autojit}
         
         \PY{n+nd}{@jit}
         \PY{k}{def} \PY{n+nf}{pairwise\PYZus{}numba}\PY{p}{(}\PY{n}{X}\PY{p}{)}\PY{p}{:}
             \PY{n}{M} \PY{o}{=} \PY{n}{X}\PY{o}{.}\PY{n}{shape}\PY{p}{[}\PY{l+m+mi}{0}\PY{p}{]}
             \PY{n}{N} \PY{o}{=} \PY{n}{X}\PY{o}{.}\PY{n}{shape}\PY{p}{[}\PY{l+m+mi}{1}\PY{p}{]}
             \PY{n}{D} \PY{o}{=} \PY{n}{np}\PY{o}{.}\PY{n}{empty}\PY{p}{(}\PY{p}{(}\PY{n}{M}\PY{p}{,} \PY{n}{M}\PY{p}{)}\PY{p}{)}\PY{c}{\PYZsh{}, dtype=np.float)}
             \PY{k}{for} \PY{n}{i} \PY{o+ow}{in} \PY{n+nb}{range}\PY{p}{(}\PY{n}{M}\PY{p}{)}\PY{p}{:}
                 \PY{k}{for} \PY{n}{j} \PY{o+ow}{in} \PY{n+nb}{range}\PY{p}{(}\PY{n}{M}\PY{p}{)}\PY{p}{:}
                     \PY{n}{d} \PY{o}{=} \PY{l+m+mf}{0.0}
                     \PY{k}{for} \PY{n}{k} \PY{o+ow}{in} \PY{n+nb}{range}\PY{p}{(}\PY{n}{N}\PY{p}{)}\PY{p}{:}
                         \PY{n}{tmp} \PY{o}{=} \PY{n}{X}\PY{p}{[}\PY{n}{i}\PY{p}{,} \PY{n}{k}\PY{p}{]} \PY{o}{\PYZhy{}} \PY{n}{X}\PY{p}{[}\PY{n}{j}\PY{p}{,} \PY{n}{k}\PY{p}{]}
                         \PY{n}{d} \PY{o}{+}\PY{o}{=} \PY{n}{tmp} \PY{o}{*} \PY{n}{tmp}
                     \PY{n}{D}\PY{p}{[}\PY{n}{i}\PY{p}{,} \PY{n}{j}\PY{p}{]} \PY{o}{=} \PY{n}{np}\PY{o}{.}\PY{n}{sqrt}\PY{p}{(}\PY{n}{d}\PY{p}{)}
             \PY{k}{return} \PY{n}{D}
         
         \PY{o}{\PYZpc{}}\PY{k}{timeit} \PY{n}{pairwise\PYZus{}numba}\PY{p}{(}\PY{n}{X}\PY{p}{)}
\end{Verbatim}

    \begin{Verbatim}[commandchars=\\\{\}]
{\color{incolor}In [{\color{incolor}13}]:} \PY{n}{Benchmarks}\PY{p}{[}\PY{l+s}{\PYZsq{}}\PY{l+s}{numba}\PY{l+s}{\PYZsq{}}\PY{p}{]} \PY{o}{=} \PY{n}{parse\PYZus{}cap}\PY{p}{(}\PY{n}{cap}\PY{p}{)}
         \PY{k}{print} \PY{n}{cap}\PY{o}{.}\PY{n}{stdout}
\end{Verbatim}

    \begin{Verbatim}[commandchars=\\\{\}]
1 loops, best of 3: 3.06 s per loop
    \end{Verbatim}

    


    \subsection{Optimized Cython Function}


    \href{http://cython.org}{Cython} is another package which is built to
convert Python-like statements into compiled code. The language is
actually a superset of Python which acts as a sort of hybrid between
Python and C. By adding type annotations to Python code and running it
through the Cython interpreter, we obtain fast compiled code. Here is a
highly-optimized Cython version of the pairwise distance function, which
we compile using IPython's Cython magic:

    \begin{Verbatim}[commandchars=\\\{\}]
{\color{incolor}In [{\color{incolor}14}]:} \PY{o}{\PYZpc{}}\PY{k}{load\PYZus{}ext} \PY{n}{cythonmagic}
\end{Verbatim}

    \begin{Verbatim}[commandchars=\\\{\}]
{\color{incolor}In [{\color{incolor}15}]:} \PY{o}{\PYZpc{}\PYZpc{}}\PY{k}{cython}
         \PY{k+kn}{import} \PY{n+nn}{numpy} \PY{k+kn}{as} \PY{n+nn}{np}
         \PY{n}{cimport} \PY{n}{cython}
         \PY{k+kn}{from} \PY{n+nn}{libc.math} \PY{n+nn}{cimport} \PY{n+nn}{sqrt}
         
         \PY{n+nd}{@cython.boundscheck}\PY{p}{(}\PY{n+nb+bp}{False}\PY{p}{)}
         \PY{n+nd}{@cython.wraparound}\PY{p}{(}\PY{n+nb+bp}{False}\PY{p}{)}
         \PY{k}{def} \PY{n+nf}{pairwise\PYZus{}cython}\PY{p}{(}\PY{n}{double}\PY{p}{[}\PY{p}{:}\PY{p}{,} \PY{p}{:}\PY{p}{:}\PY{l+m+mi}{1}\PY{p}{]} \PY{n}{X}\PY{p}{)}\PY{p}{:}
             \PY{n}{cdef} \PY{n+nb}{int} \PY{n}{M} \PY{o}{=} \PY{n}{X}\PY{o}{.}\PY{n}{shape}\PY{p}{[}\PY{l+m+mi}{0}\PY{p}{]}
             \PY{n}{cdef} \PY{n+nb}{int} \PY{n}{N} \PY{o}{=} \PY{n}{X}\PY{o}{.}\PY{n}{shape}\PY{p}{[}\PY{l+m+mi}{1}\PY{p}{]}
             \PY{n}{cdef} \PY{n}{double} \PY{n}{tmp}\PY{p}{,} \PY{n}{d}
             \PY{n}{cdef} \PY{n}{double}\PY{p}{[}\PY{p}{:}\PY{p}{,} \PY{p}{:}\PY{p}{:}\PY{l+m+mi}{1}\PY{p}{]} \PY{n}{D} \PY{o}{=} \PY{n}{np}\PY{o}{.}\PY{n}{empty}\PY{p}{(}\PY{p}{(}\PY{n}{M}\PY{p}{,} \PY{n}{M}\PY{p}{)}\PY{p}{,} \PY{n}{dtype}\PY{o}{=}\PY{n}{np}\PY{o}{.}\PY{n}{float64}\PY{p}{)}
             \PY{k}{for} \PY{n}{i} \PY{o+ow}{in} \PY{n+nb}{range}\PY{p}{(}\PY{n}{M}\PY{p}{)}\PY{p}{:}
                 \PY{k}{for} \PY{n}{j} \PY{o+ow}{in} \PY{n+nb}{range}\PY{p}{(}\PY{n}{M}\PY{p}{)}\PY{p}{:}
                     \PY{n}{d} \PY{o}{=} \PY{l+m+mf}{0.0}
                     \PY{k}{for} \PY{n}{k} \PY{o+ow}{in} \PY{n+nb}{range}\PY{p}{(}\PY{n}{N}\PY{p}{)}\PY{p}{:}
                         \PY{n}{tmp} \PY{o}{=} \PY{n}{X}\PY{p}{[}\PY{n}{i}\PY{p}{,} \PY{n}{k}\PY{p}{]} \PY{o}{\PYZhy{}} \PY{n}{X}\PY{p}{[}\PY{n}{j}\PY{p}{,} \PY{n}{k}\PY{p}{]}
                         \PY{n}{d} \PY{o}{+}\PY{o}{=} \PY{n}{tmp} \PY{o}{*} \PY{n}{tmp}
                     \PY{n}{D}\PY{p}{[}\PY{n}{i}\PY{p}{,} \PY{n}{j}\PY{p}{]} \PY{o}{=} \PY{n}{sqrt}\PY{p}{(}\PY{n}{d}\PY{p}{)}
             \PY{k}{return} \PY{n}{np}\PY{o}{.}\PY{n}{asarray}\PY{p}{(}\PY{n}{D}\PY{p}{)}
\end{Verbatim}

    \begin{Verbatim}[commandchars=\\\{\}]
{\color{incolor}In [{\color{incolor}16}]:} \PY{o}{\PYZpc{}\PYZpc{}}\PY{k}{capture} \PY{n}{cap}
         \PY{o}{\PYZpc{}}\PY{k}{timeit} \PY{n}{pairwise\PYZus{}cython}\PY{p}{(}\PY{n}{X}\PY{p}{)}
\end{Verbatim}

    \begin{Verbatim}[commandchars=\\\{\}]
{\color{incolor}In [{\color{incolor}17}]:} \PY{n}{Benchmarks}\PY{p}{[}\PY{l+s}{\PYZsq{}}\PY{l+s}{cython}\PY{l+s}{\PYZsq{}}\PY{p}{]} \PY{o}{=} \PY{n}{parse\PYZus{}cap}\PY{p}{(}\PY{n}{cap}\PY{p}{)}
         \PY{k}{print} \PY{n}{cap}\PY{o}{.}\PY{n}{stdout}
\end{Verbatim}

    \begin{Verbatim}[commandchars=\\\{\}]
100 loops, best of 3: 7.12 ms per loop
    \end{Verbatim}

    The Cython version, despite all the optimization, is more or less the
same as the result of the simple Numba decorator!\\By comparison, the
Numba version is a simple, unadorned wrapper around plainly-written
Python code.

    


    \section{Parakeet}


    \textbf{\href{http://www.parakeetpython.com/}{Parakeet}} is a runtime
accelerator for an array-oriented subset of Python. If you're doing a
lot of number crunching in Python, Parakeet may be able to significantly
speed up your code.

To accelerate a function, wrap it with Parakeet's @jit decorator:

    \begin{Verbatim}[commandchars=\\\{\}]
{\color{incolor}In [{\color{incolor}18}]:} \PY{o}{\PYZpc{}\PYZpc{}}\PY{k}{capture} \PY{n}{cap}
         
         \PY{k+kn}{from} \PY{n+nn}{parakeet} \PY{k+kn}{import} \PY{n}{jit}
         
         \PY{n}{pairwise\PYZus{}parakeet} \PY{o}{=} \PY{n}{jit}\PY{p}{(}\PY{n}{pairwise\PYZus{}python}\PY{p}{)}
         
         \PY{o}{\PYZpc{}}\PY{k}{timeit} \PY{n}{pairwise\PYZus{}parakeet}\PY{p}{(}\PY{n}{X}\PY{p}{)}
\end{Verbatim}

    \begin{Verbatim}[commandchars=\\\{\}]
{\color{incolor}In [{\color{incolor}19}]:} \PY{n}{Benchmarks}\PY{p}{[}\PY{l+s}{\PYZsq{}}\PY{l+s}{parakeet}\PY{l+s}{\PYZsq{}}\PY{p}{]} \PY{o}{=} \PY{n}{parse\PYZus{}cap}\PY{p}{(}\PY{n}{cap}\PY{p}{)}
         \PY{k}{print} \PY{n}{cap}\PY{o}{.}\PY{n}{stdout}
\end{Verbatim}

    \begin{Verbatim}[commandchars=\\\{\}]
100 loops, best of 3: 6.17 ms per loop
    \end{Verbatim}

    


    \subsection{Fortran/F2Py}


    Another option for fast computation is to write a Fortran function
directly, and use the \texttt{f2py} package to interface with the
function. We can write the function as follows:

    \begin{Verbatim}[commandchars=\\\{\}]
{\color{incolor}In [{\color{incolor}20}]:} \PY{o}{\PYZpc{}\PYZpc{}}\PY{k}{file} \PY{n}{pairwise\PYZus{}fort}\PY{o}{.}\PY{n}{f}
         
               \PY{n}{subroutine} \PY{n}{pairwise\PYZus{}fort}\PY{p}{(}\PY{n}{X}\PY{p}{,}\PY{n}{D}\PY{p}{,}\PY{n}{m}\PY{p}{,}\PY{n}{n}\PY{p}{)}
                   \PY{n}{integer} \PY{p}{:}\PY{p}{:} \PY{n}{n}\PY{p}{,}\PY{n}{m}
                   \PY{n}{double} \PY{n}{precision}\PY{p}{,} \PY{n}{intent}\PY{p}{(}\PY{o+ow}{in}\PY{p}{)} \PY{p}{:}\PY{p}{:} \PY{n}{X}\PY{p}{(}\PY{n}{m}\PY{p}{,}\PY{n}{n}\PY{p}{)}
                   \PY{n}{double} \PY{n}{precision}\PY{p}{,} \PY{n}{intent}\PY{p}{(}\PY{n}{out}\PY{p}{)} \PY{p}{:}\PY{p}{:} \PY{n}{D}\PY{p}{(}\PY{n}{m}\PY{p}{,}\PY{n}{m}\PY{p}{)} 
                   \PY{n}{integer} \PY{p}{:}\PY{p}{:} \PY{n}{i}\PY{p}{,}\PY{n}{j}\PY{p}{,}\PY{n}{k}
                   \PY{n}{double} \PY{n}{precision} \PY{p}{:}\PY{p}{:} \PY{n}{r} 
                   \PY{n}{do} \PY{n}{i} \PY{o}{=} \PY{l+m+mi}{1}\PY{p}{,}\PY{n}{m} 
                       \PY{n}{do} \PY{n}{j} \PY{o}{=} \PY{l+m+mi}{1}\PY{p}{,}\PY{n}{m} 
                           \PY{n}{r} \PY{o}{=} \PY{l+m+mi}{0}
                           \PY{n}{do} \PY{n}{k} \PY{o}{=} \PY{l+m+mi}{1}\PY{p}{,}\PY{n}{n} 
                               \PY{n}{r} \PY{o}{=} \PY{n}{r} \PY{o}{+} \PY{p}{(}\PY{n}{X}\PY{p}{(}\PY{n}{i}\PY{p}{,}\PY{n}{k}\PY{p}{)} \PY{o}{\PYZhy{}} \PY{n}{X}\PY{p}{(}\PY{n}{j}\PY{p}{,}\PY{n}{k}\PY{p}{)}\PY{p}{)} \PY{o}{*} \PY{p}{(}\PY{n}{X}\PY{p}{(}\PY{n}{i}\PY{p}{,}\PY{n}{k}\PY{p}{)} \PY{o}{\PYZhy{}} \PY{n}{X}\PY{p}{(}\PY{n}{j}\PY{p}{,}\PY{n}{k}\PY{p}{)}\PY{p}{)} 
                           \PY{n}{end} \PY{n}{do} 
                           \PY{n}{D}\PY{p}{(}\PY{n}{i}\PY{p}{,}\PY{n}{j}\PY{p}{)} \PY{o}{=} \PY{n}{sqrt}\PY{p}{(}\PY{n}{r}\PY{p}{)} 
                       \PY{n}{end} \PY{n}{do} 
                   \PY{n}{end} \PY{n}{do} 
               \PY{n}{end} \PY{n}{subroutine} \PY{n}{pairwise\PYZus{}fort}
\end{Verbatim}

    \begin{Verbatim}[commandchars=\\\{\}]
Overwriting pairwise\_fort.f
    \end{Verbatim}

    We can then use the shell interface to compile the Fortran function. In
order to hide the output of this operation, we direct it into
\texttt{/dev/null} (note: I tested this on Linux, and it may have to be
modified for Mac or Windows).

    \begin{Verbatim}[commandchars=\\\{\}]
{\color{incolor}In [{\color{incolor}21}]:} \PY{o}{!}f2py \PYZhy{}c \PYZhy{}\PYZhy{}help\PYZhy{}fcompiler
\end{Verbatim}

    \begin{Verbatim}[commandchars=\\\{\}]
Gnu95FCompiler instance properties:
  archiver        = ['/usr/bin/gfortran', '-cr']
  compile\_switch  = '-c'
  compiler\_f77    = ['/usr/bin/gfortran', '-Wall', '-g', '-ffixed-form', '-
                    fno-second-underscore', '-fPIC', '-O3', '-funroll-loops']
  compiler\_f90    = ['/usr/bin/gfortran', '-Wall', '-g', '-fno-second-
                    underscore', '-fPIC', '-O3', '-funroll-loops']
  compiler\_fix    = ['/usr/bin/gfortran', '-Wall', '-g', '-ffixed-form', '-
                    fno-second-underscore', '-Wall', '-g', '-fno-second-
                    underscore', '-fPIC', '-O3', '-funroll-loops']
  libraries       = ['gfortran']
  library\_dirs    = []
  linker\_exe      = ['/usr/bin/gfortran', '-Wall', '-Wall']
  linker\_so       = ['/usr/bin/gfortran', '-Wall', '-g', '-Wall', '-g', '-
                    shared']
  object\_switch   = '-o '
  ranlib          = ['/usr/bin/gfortran']
  version         = LooseVersion ('4.9.1-16')
  version\_cmd     = ['/usr/bin/gfortran', '--version']
Fortran compilers found:
  --fcompiler=gnu95  GNU Fortran 95 compiler (4.9.1-16)
Compilers available for this platform, but not found:
  --fcompiler=absoft   Absoft Corp Fortran Compiler
  --fcompiler=compaq   Compaq Fortran Compiler
  --fcompiler=g95      G95 Fortran Compiler
  --fcompiler=gnu      GNU Fortran 77 compiler
  --fcompiler=intel    Intel Fortran Compiler for 32-bit apps
  --fcompiler=intele   Intel Fortran Compiler for Itanium apps
  --fcompiler=intelem  Intel Fortran Compiler for 64-bit apps
  --fcompiler=lahey    Lahey/Fujitsu Fortran 95 Compiler
  --fcompiler=nag      NAGWare Fortran 95 Compiler
  --fcompiler=pathf95  PathScale Fortran Compiler
  --fcompiler=pg       Portland Group Fortran Compiler
  --fcompiler=vast     Pacific-Sierra Research Fortran 90 Compiler
Compilers not available on this platform:
  --fcompiler=hpux      HP Fortran 90 Compiler
  --fcompiler=ibm       IBM XL Fortran Compiler
  --fcompiler=intelev   Intel Visual Fortran Compiler for Itanium apps
  --fcompiler=intelv    Intel Visual Fortran Compiler for 32-bit apps
  --fcompiler=intelvem  Intel Visual Fortran Compiler for 64-bit apps
  --fcompiler=mips      MIPSpro Fortran Compiler
  --fcompiler=none      Fake Fortran compiler
  --fcompiler=sun       Sun or Forte Fortran 95 Compiler
For compiler details, run 'config\_fc --verbose' setup command.
Removing build directory /tmp/tmpiygPNS
    \end{Verbatim}

    \begin{Verbatim}[commandchars=\\\{\}]
{\color{incolor}In [{\color{incolor}22}]:} \PY{o}{!}f2py \PYZhy{}c \PYZhy{}\PYZhy{}help\PYZhy{}compiler
\end{Verbatim}

    \begin{Verbatim}[commandchars=\\\{\}]
List of available compilers:
  --compiler=bcpp     Borland C++ Compiler
  --compiler=cygwin   Cygwin port of GNU C Compiler for Win32
  --compiler=emx      EMX port of GNU C Compiler for OS/2
  --compiler=intel    Intel C Compiler for 32-bit applications
  --compiler=intele   Intel C Itanium Compiler for Itanium-based applications
  --compiler=intelem  Intel C Compiler for 64-bit applications
  --compiler=mingw32  Mingw32 port of GNU C Compiler for Win32
  --compiler=msvc     Microsoft Visual C++
  --compiler=pathcc   PathScale Compiler for SiCortex-based applications
  --compiler=unix     standard UNIX-style compiler
Removing build directory /tmp/tmp9lM8Xb
    \end{Verbatim}

    \begin{Verbatim}[commandchars=\\\{\}]
{\color{incolor}In [{\color{incolor}23}]:} \PY{c}{\PYZsh{} Compile the Fortran with f2py.}
         \PY{c}{\PYZsh{} We\PYZsq{}ll direct the output into /dev/null so it doesn\PYZsq{}t fill the screen}
         \PY{c}{\PYZsh{}!f2py \PYZhy{}c \PYZhy{}\PYZhy{}compiler=intel pairwise\PYZus{}fort.f \PYZhy{}m pairwise\PYZus{}fort \PYZgt{} /dev/null}
         \PY{o}{!}f2py \PYZhy{}c \PYZhy{}\PYZhy{}fcompiler\PY{o}{=}gnu95 pairwise\PYZus{}fort.f \PYZhy{}m pairwise\PYZus{}fort \PYZgt{} /dev/null
\end{Verbatim}

    We can import the resulting code into Python to time the execution of
the function. To make sure we're being fair, we'll first convert the
test array to Fortran-ordering so that no conversion needs to happen in
the background:

    \begin{Verbatim}[commandchars=\\\{\}]
{\color{incolor}In [{\color{incolor}24}]:} \PY{o}{\PYZpc{}\PYZpc{}}\PY{k}{capture} \PY{n}{cap}
         
         \PY{k+kn}{from} \PY{n+nn}{pairwise\PYZus{}fort} \PY{k+kn}{import} \PY{n}{pairwise\PYZus{}fort}
         \PY{n}{XF} \PY{o}{=} \PY{n}{np}\PY{o}{.}\PY{n}{asarray}\PY{p}{(}\PY{n}{X}\PY{p}{,} \PY{n}{order}\PY{o}{=}\PY{l+s}{\PYZsq{}}\PY{l+s}{F}\PY{l+s}{\PYZsq{}}\PY{p}{)}
         \PY{o}{\PYZpc{}}\PY{k}{timeit} \PY{n}{pairwise\PYZus{}fort}\PY{p}{(}\PY{n}{XF}\PY{p}{)}
\end{Verbatim}

    \begin{Verbatim}[commandchars=\\\{\}]
{\color{incolor}In [{\color{incolor}25}]:} \PY{n}{Benchmarks}\PY{p}{[}\PY{l+s}{\PYZsq{}}\PY{l+s}{fortran/}\PY{l+s+se}{\PYZbs{}n}\PY{l+s}{f2py}\PY{l+s}{\PYZsq{}}\PY{p}{]} \PY{o}{=} \PY{n}{parse\PYZus{}cap}\PY{p}{(}\PY{n}{cap}\PY{p}{)}
         \PY{k}{print} \PY{n}{cap}\PY{o}{.}\PY{n}{stdout}
\end{Verbatim}

    \begin{Verbatim}[commandchars=\\\{\}]
100 loops, best of 3: 6.19 ms per loop
    \end{Verbatim}

    The result is nearly a factor of two slower than the Cython and Numba
versions.

    


    \subsection{Scipy Pairwise Distances}


    \begin{Verbatim}[commandchars=\\\{\}]
{\color{incolor}In [{\color{incolor}26}]:} \PY{o}{\PYZpc{}\PYZpc{}}\PY{k}{capture} \PY{n}{cap}
         
         \PY{k+kn}{from} \PY{n+nn}{scipy.spatial.distance} \PY{k+kn}{import} \PY{n}{cdist}
         \PY{o}{\PYZpc{}}\PY{k}{timeit} \PY{n}{cdist}\PY{p}{(}\PY{n}{X}\PY{p}{,} \PY{n}{X}\PY{p}{)}
\end{Verbatim}

    \begin{Verbatim}[commandchars=\\\{\}]
{\color{incolor}In [{\color{incolor}27}]:} \PY{n}{Benchmarks}\PY{p}{[}\PY{l+s}{\PYZsq{}}\PY{l+s}{scipy}\PY{l+s}{\PYZsq{}}\PY{p}{]} \PY{o}{=} \PY{n}{parse\PYZus{}cap}\PY{p}{(}\PY{n}{cap}\PY{p}{)}
         \PY{k}{print} \PY{n}{cap}\PY{o}{.}\PY{n}{stdout}
\end{Verbatim}

    \begin{Verbatim}[commandchars=\\\{\}]
100 loops, best of 3: 5.79 ms per loop
    \end{Verbatim}

    


    \subsection{Scikit-learn Pairwise Distances}


    Scikit-learn contains the \texttt{euclidean\_distances} function, works
on sparse matrices as well as numpy arrays, and is implemented in
Cython:

    \begin{Verbatim}[commandchars=\\\{\}]
{\color{incolor}In [{\color{incolor}28}]:} \PY{o}{\PYZpc{}\PYZpc{}}\PY{k}{capture} \PY{n}{cap}
         
         \PY{k+kn}{from} \PY{n+nn}{sklearn.metrics} \PY{k+kn}{import} \PY{n}{euclidean\PYZus{}distances}
         \PY{o}{\PYZpc{}}\PY{k}{timeit} \PY{n}{euclidean\PYZus{}distances}\PY{p}{(}\PY{n}{X}\PY{p}{,} \PY{n}{X}\PY{p}{)}
\end{Verbatim}

    \begin{Verbatim}[commandchars=\\\{\}]
{\color{incolor}In [{\color{incolor}29}]:} \PY{n}{Benchmarks}\PY{p}{[}\PY{l+s}{\PYZsq{}}\PY{l+s}{sklearn}\PY{l+s}{\PYZsq{}}\PY{p}{]} \PY{o}{=} \PY{n}{parse\PYZus{}cap}\PY{p}{(}\PY{n}{cap}\PY{p}{)}
         \PY{k}{print} \PY{n}{cap}\PY{o}{.}\PY{n}{stdout}
\end{Verbatim}

    \begin{Verbatim}[commandchars=\\\{\}]
100 loops, best of 3: 9.7 ms per loop
    \end{Verbatim}

    \texttt{euclidean\_distances} is several times slower than the Numba
pairwise function on dense arrays.


    \subsection{Comparing the Results}


    As a summary of the results, we'll create a bar-chart to visualize the
timings:

    \begin{Verbatim}[commandchars=\\\{\}]
{\color{incolor}In [{\color{incolor}30}]:} \PY{o}{\PYZpc{}}\PY{k}{matplotlib} \PY{n}{inline}
\end{Verbatim}

    \begin{Verbatim}[commandchars=\\\{\}]
{\color{incolor}In [{\color{incolor}31}]:} \PY{n}{labels} \PY{o}{=} \PY{n}{Benchmarks}\PY{o}{.}\PY{n}{keys}\PY{p}{(}\PY{p}{)}
         \PY{n}{timings} \PY{o}{=} \PY{p}{[}\PY{n}{np}\PY{o}{.}\PY{n}{timedelta64}\PY{p}{(}\PY{n+nb}{int}\PY{p}{(}\PY{n+nb}{float}\PY{p}{(}\PY{n}{value}\PY{p}{[}\PY{l+m+mi}{0}\PY{p}{]}\PY{p}{)}\PY{p}{)}\PY{p}{,}\PY{n+nb}{str}\PY{p}{(}\PY{n}{value}\PY{p}{[}\PY{l+m+mi}{1}\PY{p}{]}\PY{p}{)}\PY{p}{)}\PY{o}{/}\PY{n}{np}\PY{o}{.}\PY{n}{timedelta64}\PY{p}{(}\PY{l+m+mi}{1}\PY{p}{,} \PY{l+s}{\PYZsq{}}\PY{l+s}{s}\PY{l+s}{\PYZsq{}}\PY{p}{)} \PY{k}{for} \PY{n}{value} \PY{o+ow}{in} \PY{n}{Benchmarks}\PY{o}{.}\PY{n}{values}\PY{p}{(}\PY{p}{)}\PY{p}{]}
         \PY{n}{x} \PY{o}{=} \PY{n}{np}\PY{o}{.}\PY{n}{arange}\PY{p}{(}\PY{n+nb}{len}\PY{p}{(}\PY{n}{labels}\PY{p}{)}\PY{p}{)}
         \PY{n}{ax} \PY{o}{=} \PY{n}{plt}\PY{o}{.}\PY{n}{axes}\PY{p}{(}\PY{n}{xticks}\PY{o}{=}\PY{n}{x}\PY{p}{,} \PY{n}{yscale}\PY{o}{=}\PY{l+s}{\PYZsq{}}\PY{l+s}{log}\PY{l+s}{\PYZsq{}}\PY{p}{)}
         \PY{n}{ax}\PY{o}{.}\PY{n}{bar}\PY{p}{(}\PY{n}{x} \PY{o}{\PYZhy{}} \PY{l+m+mf}{0.3}\PY{p}{,} \PY{n}{timings}\PY{p}{,} \PY{n}{width}\PY{o}{=}\PY{l+m+mf}{0.6}\PY{p}{,} \PY{n}{alpha}\PY{o}{=}\PY{l+m+mf}{0.4}\PY{p}{,} \PY{n}{bottom}\PY{o}{=}\PY{l+m+mf}{1E\PYZhy{}6}\PY{p}{)}
         \PY{n}{ax}\PY{o}{.}\PY{n}{grid}\PY{p}{(}\PY{p}{)}
         \PY{n}{ax}\PY{o}{.}\PY{n}{set\PYZus{}xlim}\PY{p}{(}\PY{o}{\PYZhy{}}\PY{l+m+mf}{0.5}\PY{p}{,} \PY{n+nb}{len}\PY{p}{(}\PY{n}{labels}\PY{p}{)} \PY{o}{\PYZhy{}} \PY{l+m+mf}{0.5}\PY{p}{)}
         \PY{n}{ax}\PY{o}{.}\PY{n}{set\PYZus{}ylim}\PY{p}{(}\PY{l+m+mf}{0.5}\PY{o}{*}\PY{n+nb}{min}\PY{p}{(}\PY{n}{timings}\PY{p}{)}\PY{p}{,} \PY{l+m+mf}{1.1}\PY{o}{*}\PY{n+nb}{max}\PY{p}{(}\PY{n}{timings}\PY{p}{)}\PY{p}{)}
         \PY{n}{ax}\PY{o}{.}\PY{n}{xaxis}\PY{o}{.}\PY{n}{set\PYZus{}major\PYZus{}formatter}\PY{p}{(}\PY{n}{plt}\PY{o}{.}\PY{n}{FuncFormatter}\PY{p}{(}\PY{k}{lambda} \PY{n}{i}\PY{p}{,} \PY{n}{loc}\PY{p}{:} \PY{n}{labels}\PY{p}{[}\PY{n+nb}{int}\PY{p}{(}\PY{n}{i}\PY{p}{)}\PY{p}{]}\PY{p}{)}\PY{p}{)}
         \PY{n}{ax}\PY{o}{.}\PY{n}{set\PYZus{}ylabel}\PY{p}{(}\PY{l+s}{\PYZsq{}}\PY{l+s}{time (s)}\PY{l+s}{\PYZsq{}}\PY{p}{)}
         \PY{n}{ax}\PY{o}{.}\PY{n}{set\PYZus{}title}\PY{p}{(}\PY{l+s}{\PYZdq{}}\PY{l+s}{Pairwise Distance Timings}\PY{l+s}{\PYZdq{}}\PY{p}{)}
\end{Verbatim}

            \begin{Verbatim}[commandchars=\\\{\}]
{\color{outcolor}Out[{\color{outcolor}31}]:} <matplotlib.text.Text at 0x7f7c57063f50>
\end{Verbatim}
        
    \begin{center}
    \adjustimage{max size={0.9\linewidth}{0.9\paperheight}}{VIII-Benchmarks_files/VIII-Benchmarks_66_1.pdf}
    \end{center}
    { \hspace*{\fill} \\}
    
    Note that this is log-scaled, so the vertical space between two grid
lines indicates a factor of 10 difference in computation time. As Pure
Python takes more time by two orders of magnitude we'll now remove it
from the benchmarks for a better comparison.

    \begin{Verbatim}[commandchars=\\\{\}]
{\color{incolor}In [{\color{incolor}33}]:} \PY{n}{labels} \PY{o}{=} \PY{p}{[}\PY{n}{val} \PY{k}{for} \PY{n}{val} \PY{o+ow}{in} \PY{n}{Benchmarks}\PY{o}{.}\PY{n}{iterkeys}\PY{p}{(}\PY{p}{)} \PY{k}{if} \PY{n}{val} \PY{o}{!=} \PY{l+s}{\PYZsq{}}\PY{l+s}{python}\PY{l+s+se}{\PYZbs{}n}\PY{l+s}{loop}\PY{l+s}{\PYZsq{}}\PY{p}{]}
         \PY{n}{timings} \PY{o}{=} \PY{p}{[}\PY{n}{np}\PY{o}{.}\PY{n}{timedelta64}\PY{p}{(}\PY{n+nb}{int}\PY{p}{(}\PY{n+nb}{float}\PY{p}{(}\PY{n}{value}\PY{p}{[}\PY{l+m+mi}{0}\PY{p}{]}\PY{p}{)}\PY{p}{)}\PY{p}{,}\PY{n+nb}{str}\PY{p}{(}\PY{n}{value}\PY{p}{[}\PY{l+m+mi}{1}\PY{p}{]}\PY{p}{)}\PY{p}{)}\PY{o}{/}\PY{n}{np}\PY{o}{.}\PY{n}{timedelta64}\PY{p}{(}\PY{l+m+mi}{1}\PY{p}{,} \PY{l+s}{\PYZsq{}}\PY{l+s}{s}\PY{l+s}{\PYZsq{}}\PY{p}{)}\PYZbs{}
                    \PY{k}{for} \PY{n}{k}\PY{p}{,} \PY{n}{value} \PY{o+ow}{in} \PY{n}{Benchmarks}\PY{o}{.}\PY{n}{items}\PY{p}{(}\PY{p}{)} \PY{k}{if} \PY{n}{k} \PY{o}{!=} \PY{l+s}{\PYZsq{}}\PY{l+s}{python}\PY{l+s+se}{\PYZbs{}n}\PY{l+s}{loop}\PY{l+s}{\PYZsq{}}\PY{p}{]}
         \PY{n}{x} \PY{o}{=} \PY{n}{np}\PY{o}{.}\PY{n}{arange}\PY{p}{(}\PY{n+nb}{len}\PY{p}{(}\PY{n}{labels}\PY{p}{)}\PY{p}{)}
         \PY{n}{ax} \PY{o}{=} \PY{n}{plt}\PY{o}{.}\PY{n}{axes}\PY{p}{(}\PY{n}{xticks}\PY{o}{=}\PY{n}{x}\PY{p}{,} \PY{n}{yscale}\PY{o}{=}\PY{l+s}{\PYZsq{}}\PY{l+s}{log}\PY{l+s}{\PYZsq{}}\PY{p}{)}
         \PY{n}{ax}\PY{o}{.}\PY{n}{bar}\PY{p}{(}\PY{n}{x} \PY{o}{\PYZhy{}} \PY{l+m+mf}{0.3}\PY{p}{,} \PY{n}{timings}\PY{p}{,} \PY{n}{width}\PY{o}{=}\PY{l+m+mf}{0.6}\PY{p}{,} \PY{n}{alpha}\PY{o}{=}\PY{l+m+mf}{0.4}\PY{p}{,} \PY{n}{bottom}\PY{o}{=}\PY{l+m+mf}{1E\PYZhy{}6}\PY{p}{)}
         \PY{n}{ax}\PY{o}{.}\PY{n}{grid}\PY{p}{(}\PY{p}{)}
         \PY{n}{ax}\PY{o}{.}\PY{n}{set\PYZus{}xlim}\PY{p}{(}\PY{o}{\PYZhy{}}\PY{l+m+mf}{0.5}\PY{p}{,} \PY{n+nb}{len}\PY{p}{(}\PY{n}{labels}\PY{p}{)} \PY{o}{\PYZhy{}} \PY{l+m+mf}{0.5}\PY{p}{)}
         \PY{n}{ax}\PY{o}{.}\PY{n}{set\PYZus{}ylim}\PY{p}{(}\PY{l+m+mf}{0.5}\PY{o}{*}\PY{n+nb}{min}\PY{p}{(}\PY{n}{timings}\PY{p}{)}\PY{p}{,} \PY{l+m+mf}{1.1}\PY{o}{*}\PY{n+nb}{max}\PY{p}{(}\PY{n}{timings}\PY{p}{)}\PY{p}{)}
         \PY{n}{ax}\PY{o}{.}\PY{n}{xaxis}\PY{o}{.}\PY{n}{set\PYZus{}major\PYZus{}formatter}\PY{p}{(}\PY{n}{plt}\PY{o}{.}\PY{n}{FuncFormatter}\PY{p}{(}\PY{k}{lambda} \PY{n}{i}\PY{p}{,} \PY{n}{loc}\PY{p}{:} \PY{n}{labels}\PY{p}{[}\PY{n+nb}{int}\PY{p}{(}\PY{n}{i}\PY{p}{)}\PY{p}{]}\PY{p}{)}\PY{p}{)}
         \PY{n}{ax}\PY{o}{.}\PY{n}{set\PYZus{}ylabel}\PY{p}{(}\PY{l+s}{\PYZsq{}}\PY{l+s}{time (s)}\PY{l+s}{\PYZsq{}}\PY{p}{)}
         \PY{n}{ax}\PY{o}{.}\PY{n}{set\PYZus{}title}\PY{p}{(}\PY{l+s}{\PYZdq{}}\PY{l+s}{Pairwise Distance Timings}\PY{l+s}{\PYZdq{}}\PY{p}{)}
\end{Verbatim}

            \begin{Verbatim}[commandchars=\\\{\}]
{\color{outcolor}Out[{\color{outcolor}33}]:} <matplotlib.text.Text at 0x7fb6ba1d5b10>
\end{Verbatim}
        
    \begin{center}
    \adjustimage{max size={0.9\linewidth}{0.9\paperheight}}{VIII-Benchmarks_files/VIII-Benchmarks_68_1.pdf}
    \end{center}
    { \hspace*{\fill} \\}
    
    


    \subsection{\textbf{1D European Option using Black-Scholes Formula}}


    \begin{Verbatim}[commandchars=\\\{\}]
{\color{incolor}In [{\color{incolor}37}]:} \PY{c}{\PYZsh{}Restart notebook!}
\end{Verbatim}

    We'll now benchmark a simple function which returns the price of a 1D
European Option using the closed form solution with the following
parameters:

\begin{itemize}
\itemsep1pt\parskip0pt\parsep0pt
\item
  $S_0$ = 100
\item
  $K$ = 100
\item
  $T$ = 1 (years)
\item
  $\sigma$ = 0.3
\item
  $r$ = 0.03
\item
  dividend = 0
\item
  option type = -1 (+1 for call, -1 for put)
\end{itemize}

    \begin{Verbatim}[commandchars=\\\{\}]
{\color{incolor}In [{\color{incolor}34}]:} \PY{n}{S0} \PY{o}{=} \PY{l+m+mi}{100}
         \PY{n}{K} \PY{o}{=} \PY{l+m+mi}{100}
         \PY{n}{T} \PY{o}{=} \PY{l+m+mi}{1}
         \PY{n}{vol} \PY{o}{=} \PY{l+m+mf}{0.3}
         \PY{n}{r} \PY{o}{=} \PY{l+m+mf}{0.03}
         \PY{n}{q} \PY{o}{=} \PY{l+m+mi}{0}
         \PY{n}{optype} \PY{o}{=} \PY{o}{\PYZhy{}}\PY{l+m+mi}{1}
\end{Verbatim}

    \begin{Verbatim}[commandchars=\\\{\}]
{\color{incolor}In [{\color{incolor}35}]:} \PY{k+kn}{from} \PY{n+nn}{numpy} \PY{k+kn}{import} \PY{n}{log}\PY{p}{,} \PY{n}{sqrt}\PY{p}{,} \PY{n}{zeros}\PY{p}{,} \PY{n}{exp}\PY{p}{,} \PY{n}{arange}
         \PY{k+kn}{from} \PY{n+nn}{scipy.special} \PY{k+kn}{import} \PY{n}{erf}
         
         \PY{k}{def} \PY{n+nf}{parse\PYZus{}cap}\PY{p}{(}\PY{n}{s}\PY{p}{)}\PY{p}{:}
             \PY{k}{return} \PY{n}{cap}\PY{o}{.}\PY{n}{stdout}\PY{o}{.}\PY{n}{split}\PY{p}{(}\PY{l+s}{\PYZsq{}}\PY{l+s}{ }\PY{l+s}{\PYZsq{}}\PY{p}{)}\PY{p}{[}\PY{o}{\PYZhy{}}\PY{l+m+mi}{4}\PY{p}{:}\PY{o}{\PYZhy{}}\PY{l+m+mi}{2}\PY{p}{]}
         
         \PY{n}{benchs} \PY{o}{=} \PY{p}{[}\PY{l+s}{\PYZsq{}}\PY{l+s}{python}\PY{l+s}{\PYZsq{}}\PY{p}{,}\PY{l+s}{\PYZsq{}}\PY{l+s}{numba}\PY{l+s}{\PYZsq{}}\PY{p}{,}\PY{l+s}{\PYZsq{}}\PY{l+s}{cython}\PY{l+s}{\PYZsq{}}\PY{p}{,}\PY{l+s}{\PYZsq{}}\PY{l+s}{matlab}\PY{l+s}{\PYZsq{}}\PY{p}{]}
         \PY{n}{BS\PYZus{}bench} \PY{o}{=} \PY{n+nb}{dict}\PY{p}{(}\PY{n+nb}{zip}\PY{p}{(}\PY{n}{benchs}\PY{p}{,}\PY{n}{zeros}\PY{p}{(}\PY{l+m+mi}{4}\PY{p}{)}\PY{p}{)}\PY{p}{)}
\end{Verbatim}

    


    \section{Pure Python}


    \begin{Verbatim}[commandchars=\\\{\}]
{\color{incolor}In [{\color{incolor}36}]:} \PY{k}{def} \PY{n+nf}{std\PYZus{}norm\PYZus{}cdf}\PY{p}{(}\PY{n}{x}\PY{p}{)}\PY{p}{:}
             \PY{k}{return} \PY{l+m+mf}{0.5}\PY{o}{*}\PY{p}{(}\PY{l+m+mi}{1}\PY{o}{+}\PY{n}{erf}\PY{p}{(}\PY{n}{x}\PY{o}{/}\PY{n}{sqrt}\PY{p}{(}\PY{l+m+mf}{2.0}\PY{p}{)}\PY{p}{)}\PY{p}{)}
         
         \PY{k}{def} \PY{n+nf}{black\PYZus{}scholes}\PY{p}{(}\PY{n}{s}\PY{p}{,} \PY{n}{k}\PY{p}{,} \PY{n}{t}\PY{p}{,} \PY{n}{v}\PY{p}{,} \PY{n}{rf}\PY{p}{,} \PY{n}{div}\PY{p}{,} \PY{n}{cp}\PY{p}{)}\PY{p}{:}
             \PY{l+s+sd}{\PYZdq{}\PYZdq{}\PYZdq{}Price an option using the Black\PYZhy{}Scholes model.}
         \PY{l+s+sd}{    }
         \PY{l+s+sd}{    s : initial stock price}
         \PY{l+s+sd}{    k : strike price}
         \PY{l+s+sd}{    t : expiration time}
         \PY{l+s+sd}{    v : volatility}
         \PY{l+s+sd}{    rf : risk\PYZhy{}free rate}
         \PY{l+s+sd}{    div : dividend}
         \PY{l+s+sd}{    cp : +1/\PYZhy{}1 for call/put}
         \PY{l+s+sd}{    \PYZdq{}\PYZdq{}\PYZdq{}}
             \PY{n}{d1} \PY{o}{=} \PY{p}{(}\PY{n}{log}\PY{p}{(}\PY{n}{s}\PY{o}{/}\PY{n}{k}\PY{p}{)}\PY{o}{+}\PY{p}{(}\PY{n}{rf}\PY{o}{\PYZhy{}}\PY{n}{div}\PY{o}{+}\PY{l+m+mf}{0.5}\PY{o}{*}\PY{n+nb}{pow}\PY{p}{(}\PY{n}{v}\PY{p}{,}\PY{l+m+mi}{2}\PY{p}{)}\PY{p}{)}\PY{o}{*}\PY{n}{t}\PY{p}{)}\PY{o}{/}\PY{p}{(}\PY{n}{v}\PY{o}{*}\PY{n}{sqrt}\PY{p}{(}\PY{n}{t}\PY{p}{)}\PY{p}{)}
             \PY{n}{d2} \PY{o}{=} \PY{n}{d1} \PY{o}{\PYZhy{}} \PY{n}{v}\PY{o}{*}\PY{n}{sqrt}\PY{p}{(}\PY{n}{t}\PY{p}{)}
             \PY{n}{optprice} \PY{o}{=} \PY{n}{cp}\PY{o}{*}\PY{n}{s}\PY{o}{*}\PY{n}{exp}\PY{p}{(}\PY{o}{\PYZhy{}}\PY{n}{div}\PY{o}{*}\PY{n}{t}\PY{p}{)}\PY{o}{*}\PY{n}{std\PYZus{}norm\PYZus{}cdf}\PY{p}{(}\PY{n}{cp}\PY{o}{*}\PY{n}{d1}\PY{p}{)} \PY{o}{\PYZhy{}} \PYZbs{}
                     \PY{n}{cp}\PY{o}{*}\PY{n}{k}\PY{o}{*}\PY{n}{exp}\PY{p}{(}\PY{o}{\PYZhy{}}\PY{n}{rf}\PY{o}{*}\PY{n}{t}\PY{p}{)}\PY{o}{*}\PY{n}{std\PYZus{}norm\PYZus{}cdf}\PY{p}{(}\PY{n}{cp}\PY{o}{*}\PY{n}{d2}\PY{p}{)}
             \PY{k}{return} \PY{n}{optprice}
\end{Verbatim}

    \begin{Verbatim}[commandchars=\\\{\}]
{\color{incolor}In [{\color{incolor}37}]:} \PY{o}{\PYZpc{}\PYZpc{}}\PY{k}{capture} \PY{n}{cap}
         \PY{o}{\PYZpc{}}\PY{k}{timeit} \PY{n}{black\PYZus{}scholes}\PY{p}{(}\PY{n}{S0}\PY{p}{,} \PY{n}{K}\PY{p}{,} \PY{n}{T}\PY{p}{,} \PY{n}{vol}\PY{p}{,} \PY{n}{r}\PY{p}{,} \PY{n}{q}\PY{p}{,} \PY{n}{optype}\PY{p}{)}
\end{Verbatim}

    \begin{Verbatim}[commandchars=\\\{\}]
{\color{incolor}In [{\color{incolor}38}]:} \PY{n}{BS\PYZus{}bench}\PY{p}{[}\PY{l+s}{\PYZsq{}}\PY{l+s}{python}\PY{l+s}{\PYZsq{}}\PY{p}{]} \PY{o}{=} \PY{n}{parse\PYZus{}cap}\PY{p}{(}\PY{n}{cap}\PY{p}{)}
         \PY{k}{print} \PY{n}{cap}\PY{o}{.}\PY{n}{stdout}
\end{Verbatim}

    \begin{Verbatim}[commandchars=\\\{\}]
100000 loops, best of 3: 11.6 us per loop
    \end{Verbatim}

    


    \section{Numba}


    \begin{Verbatim}[commandchars=\\\{\}]
{\color{incolor}In [{\color{incolor}39}]:} \PY{k+kn}{from} \PY{n+nn}{numba} \PY{k+kn}{import} \PY{n}{double}
         \PY{k+kn}{from} \PY{n+nn}{numba.decorators} \PY{k+kn}{import} \PY{n}{jit}\PY{p}{,} \PY{n}{autojit}
         \PY{k+kn}{import} \PY{n+nn}{math}
         
         \PY{n+nd}{@autojit}
         \PY{k}{def} \PY{n+nf}{std\PYZus{}norm\PYZus{}cdf\PYZus{}numba}\PY{p}{(}\PY{n}{x}\PY{p}{)}\PY{p}{:}
             \PY{k}{return} \PY{l+m+mf}{0.5}\PY{o}{*}\PY{p}{(}\PY{l+m+mi}{1}\PY{o}{+}\PY{n}{math}\PY{o}{.}\PY{n}{erf}\PY{p}{(}\PY{n}{x}\PY{o}{/}\PY{n}{math}\PY{o}{.}\PY{n}{sqrt}\PY{p}{(}\PY{l+m+mf}{2.0}\PY{p}{)}\PY{p}{)}\PY{p}{)}
         
         \PY{n+nd}{@autojit}
         \PY{k}{def} \PY{n+nf}{black\PYZus{}scholes\PYZus{}numba}\PY{p}{(}\PY{n}{s}\PY{p}{,} \PY{n}{k}\PY{p}{,} \PY{n}{t}\PY{p}{,} \PY{n}{v}\PY{p}{,} \PY{n}{rf}\PY{p}{,} \PY{n}{div}\PY{p}{,} \PY{n}{cp}\PY{p}{)}\PY{p}{:}
             \PY{l+s+sd}{\PYZdq{}\PYZdq{}\PYZdq{}Price an option using the Black\PYZhy{}Scholes model.}
         \PY{l+s+sd}{    }
         \PY{l+s+sd}{    s : initial stock price}
         \PY{l+s+sd}{    k : strike price}
         \PY{l+s+sd}{    t : expiration time}
         \PY{l+s+sd}{    v : volatility}
         \PY{l+s+sd}{    rf : risk\PYZhy{}free rate}
         \PY{l+s+sd}{    div : dividend}
         \PY{l+s+sd}{    cp : +1/\PYZhy{}1 for call/put}
         \PY{l+s+sd}{    \PYZdq{}\PYZdq{}\PYZdq{}}
             \PY{n}{d1} \PY{o}{=} \PY{p}{(}\PY{n}{math}\PY{o}{.}\PY{n}{log}\PY{p}{(}\PY{n}{s}\PY{o}{/}\PY{n}{k}\PY{p}{)}\PY{o}{+}\PY{p}{(}\PY{n}{rf}\PY{o}{\PYZhy{}}\PY{n}{div}\PY{o}{+}\PY{l+m+mf}{0.5}\PY{o}{*}\PY{n}{v}\PY{o}{*}\PY{n}{v}\PY{p}{)}\PY{o}{*}\PY{n}{t}\PY{p}{)}\PY{o}{/}\PY{p}{(}\PY{n}{v}\PY{o}{*}\PY{n}{math}\PY{o}{.}\PY{n}{sqrt}\PY{p}{(}\PY{n}{t}\PY{p}{)}\PY{p}{)}
             \PY{n}{d2} \PY{o}{=} \PY{n}{d1} \PY{o}{\PYZhy{}} \PY{n}{v}\PY{o}{*}\PY{n}{math}\PY{o}{.}\PY{n}{sqrt}\PY{p}{(}\PY{n}{t}\PY{p}{)}
             \PY{n}{optprice} \PY{o}{=} \PY{n}{cp}\PY{o}{*}\PY{n}{s}\PY{o}{*}\PY{n}{math}\PY{o}{.}\PY{n}{exp}\PY{p}{(}\PY{o}{\PYZhy{}}\PY{n}{div}\PY{o}{*}\PY{n}{t}\PY{p}{)}\PY{o}{*}\PY{n}{std\PYZus{}norm\PYZus{}cdf\PYZus{}numba}\PY{p}{(}\PY{n}{cp}\PY{o}{*}\PY{n}{d1}\PY{p}{)} \PY{o}{\PYZhy{}} \PYZbs{}
                     \PY{n}{cp}\PY{o}{*}\PY{n}{k}\PY{o}{*}\PY{n}{math}\PY{o}{.}\PY{n}{exp}\PY{p}{(}\PY{o}{\PYZhy{}}\PY{n}{rf}\PY{o}{*}\PY{n}{t}\PY{p}{)}\PY{o}{*}\PY{n}{std\PYZus{}norm\PYZus{}cdf\PYZus{}numba}\PY{p}{(}\PY{n}{cp}\PY{o}{*}\PY{n}{d2}\PY{p}{)}
             \PY{k}{return} \PY{n}{optprice}
\end{Verbatim}

    \begin{Verbatim}[commandchars=\\\{\}]
{\color{incolor}In [{\color{incolor}40}]:} \PY{o}{\PYZpc{}\PYZpc{}}\PY{k}{capture} \PY{n}{cap}
         \PY{o}{\PYZpc{}}\PY{k}{timeit} \PY{n}{black\PYZus{}scholes\PYZus{}numba}\PY{p}{(}\PY{n}{S0}\PY{p}{,} \PY{n}{K}\PY{p}{,} \PY{n}{T}\PY{p}{,} \PY{n}{vol}\PY{p}{,} \PY{n}{r}\PY{p}{,} \PY{n}{q}\PY{p}{,} \PY{n}{optype}\PY{p}{)}
\end{Verbatim}

    \begin{Verbatim}[commandchars=\\\{\}]
{\color{incolor}In [{\color{incolor}41}]:} \PY{n}{BS\PYZus{}bench}\PY{p}{[}\PY{l+s}{\PYZsq{}}\PY{l+s}{numba}\PY{l+s}{\PYZsq{}}\PY{p}{]} \PY{o}{=} \PY{n}{parse\PYZus{}cap}\PY{p}{(}\PY{n}{cap}\PY{p}{)}
         \PY{k}{print} \PY{n}{cap}\PY{o}{.}\PY{n}{stdout}
\end{Verbatim}

    \begin{Verbatim}[commandchars=\\\{\}]
1 loops, best of 3: 5.01 us per loop
    \end{Verbatim}

    


    \section{Cython}


    \begin{Verbatim}[commandchars=\\\{\}]
{\color{incolor}In [{\color{incolor}42}]:} \PY{o}{\PYZpc{}}\PY{k}{load\PYZus{}ext} \PY{n}{cythonmagic}
\end{Verbatim}

    \begin{Verbatim}[commandchars=\\\{\}]
The cythonmagic extension is already loaded. To reload it, use:
  \%reload\_ext cythonmagic
    \end{Verbatim}

    \begin{Verbatim}[commandchars=\\\{\}]
{\color{incolor}In [{\color{incolor}43}]:} \PY{o}{\PYZpc{}\PYZpc{}}\PY{k}{cython}
         \PY{n}{cimport} \PY{n}{cython}
         \PY{k+kn}{from} \PY{n+nn}{libc.math} \PY{n+nn}{cimport} \PY{n+nn}{exp}\PY{p}{,} \PY{n}{sqrt}\PY{p}{,} \PY{n+nb}{pow}\PY{p}{,} \PY{n}{log}\PY{p}{,} \PY{n}{erf}
         
         \PY{n+nd}{@cython.cdivision}\PY{p}{(}\PY{n+nb+bp}{True}\PY{p}{)}
         \PY{n}{cdef} \PY{n}{double} \PY{n}{std\PYZus{}norm\PYZus{}cdf}\PY{p}{(}\PY{n}{double} \PY{n}{x}\PY{p}{)} \PY{n}{nogil}\PY{p}{:}
             \PY{k}{return} \PY{l+m+mf}{0.5}\PY{o}{*}\PY{p}{(}\PY{l+m+mi}{1}\PY{o}{+}\PY{n}{erf}\PY{p}{(}\PY{n}{x}\PY{o}{/}\PY{n}{sqrt}\PY{p}{(}\PY{l+m+mf}{2.0}\PY{p}{)}\PY{p}{)}\PY{p}{)}
         
         \PY{n+nd}{@cython.cdivision}\PY{p}{(}\PY{n+nb+bp}{True}\PY{p}{)}
         \PY{k}{def} \PY{n+nf}{black\PYZus{}scholes}\PY{p}{(}\PY{n}{double} \PY{n}{s}\PY{p}{,} \PY{n}{double} \PY{n}{k}\PY{p}{,} \PY{n}{double} \PY{n}{t}\PY{p}{,} \PY{n}{double} \PY{n}{v}\PY{p}{,}
                          \PY{n}{double} \PY{n}{rf}\PY{p}{,} \PY{n}{double} \PY{n}{div}\PY{p}{,} \PY{n}{double} \PY{n}{cp}\PY{p}{)}\PY{p}{:}
             \PY{l+s+sd}{\PYZdq{}\PYZdq{}\PYZdq{}Price an option using the Black\PYZhy{}Scholes model.}
         \PY{l+s+sd}{    }
         \PY{l+s+sd}{    s : initial stock price}
         \PY{l+s+sd}{    k : strike price}
         \PY{l+s+sd}{    t : expiration time}
         \PY{l+s+sd}{    v : volatility}
         \PY{l+s+sd}{    rf : risk\PYZhy{}free rate}
         \PY{l+s+sd}{    div : dividend}
         \PY{l+s+sd}{    cp : +1/\PYZhy{}1 for call/put}
         \PY{l+s+sd}{    \PYZdq{}\PYZdq{}\PYZdq{}}
             \PY{n}{cdef} \PY{n}{double} \PY{n}{d1}\PY{p}{,} \PY{n}{d2}\PY{p}{,} \PY{n}{optprice}
             \PY{k}{with} \PY{n}{nogil}\PY{p}{:}
                 \PY{n}{d1} \PY{o}{=} \PY{p}{(}\PY{n}{log}\PY{p}{(}\PY{n}{s}\PY{o}{/}\PY{n}{k}\PY{p}{)}\PY{o}{+}\PY{p}{(}\PY{n}{rf}\PY{o}{\PYZhy{}}\PY{n}{div}\PY{o}{+}\PY{l+m+mf}{0.5}\PY{o}{*}\PY{n+nb}{pow}\PY{p}{(}\PY{n}{v}\PY{p}{,}\PY{l+m+mi}{2}\PY{p}{)}\PY{p}{)}\PY{o}{*}\PY{n}{t}\PY{p}{)}\PY{o}{/}\PY{p}{(}\PY{n}{v}\PY{o}{*}\PY{n}{sqrt}\PY{p}{(}\PY{n}{t}\PY{p}{)}\PY{p}{)}
                 \PY{n}{d2} \PY{o}{=} \PY{n}{d1} \PY{o}{\PYZhy{}} \PY{n}{v}\PY{o}{*}\PY{n}{sqrt}\PY{p}{(}\PY{n}{t}\PY{p}{)}
                 \PY{n}{optprice} \PY{o}{=} \PY{n}{cp}\PY{o}{*}\PY{n}{s}\PY{o}{*}\PY{n}{exp}\PY{p}{(}\PY{o}{\PYZhy{}}\PY{n}{div}\PY{o}{*}\PY{n}{t}\PY{p}{)}\PY{o}{*}\PY{n}{std\PYZus{}norm\PYZus{}cdf}\PY{p}{(}\PY{n}{cp}\PY{o}{*}\PY{n}{d1}\PY{p}{)} \PY{o}{\PYZhy{}} \PYZbs{}
                     \PY{n}{cp}\PY{o}{*}\PY{n}{k}\PY{o}{*}\PY{n}{exp}\PY{p}{(}\PY{o}{\PYZhy{}}\PY{n}{rf}\PY{o}{*}\PY{n}{t}\PY{p}{)}\PY{o}{*}\PY{n}{std\PYZus{}norm\PYZus{}cdf}\PY{p}{(}\PY{n}{cp}\PY{o}{*}\PY{n}{d2}\PY{p}{)}
             \PY{k}{return} \PY{n}{optprice}
\end{Verbatim}

    \begin{Verbatim}[commandchars=\\\{\}]
{\color{incolor}In [{\color{incolor}44}]:} \PY{o}{\PYZpc{}\PYZpc{}}\PY{k}{capture} \PY{n}{cap}
         \PY{o}{\PYZpc{}}\PY{k}{timeit} \PY{n}{black\PYZus{}scholes}\PY{p}{(}\PY{n}{S0}\PY{p}{,} \PY{n}{K}\PY{p}{,} \PY{n}{T}\PY{p}{,} \PY{n}{vol}\PY{p}{,} \PY{n}{r}\PY{p}{,} \PY{n}{q}\PY{p}{,} \PY{n}{optype}\PY{p}{)}
\end{Verbatim}

    \begin{Verbatim}[commandchars=\\\{\}]
{\color{incolor}In [{\color{incolor}45}]:} \PY{n}{BS\PYZus{}bench}\PY{p}{[}\PY{l+s}{\PYZsq{}}\PY{l+s}{cython}\PY{l+s}{\PYZsq{}}\PY{p}{]} \PY{o}{=} \PY{n}{parse\PYZus{}cap}\PY{p}{(}\PY{n}{cap}\PY{p}{)}
         \PY{k}{print} \PY{n}{cap}\PY{o}{.}\PY{n}{stdout}
\end{Verbatim}

    \begin{Verbatim}[commandchars=\\\{\}]
1000000 loops, best of 3: 400 ns per loop
    \end{Verbatim}

    


    \section{MATLAB}


    We now analyse the corresponding \textbf{MATLAB} code which we'll use
for comparison

    \begin{Verbatim}[commandchars=\\\{\}]
{\color{incolor}In [{\color{incolor}47}]:} \PY{o}{!}cat ../std\PYZus{}norm\PYZus{}cdf.m
\end{Verbatim}

    \begin{Verbatim}[commandchars=\\\{\}]
function z = std\_norm\_cdf(x)
z = 0.5*(1+erf(x/sqrt(2.0)));
    \end{Verbatim}

    \begin{Verbatim}[commandchars=\\\{\}]
{\color{incolor}In [{\color{incolor}48}]:} \PY{o}{!}cat ../black\PYZus{}scholes.m
\end{Verbatim}

    \begin{Verbatim}[commandchars=\\\{\}]
function z = black\_scholes(s, k, t, v, rf, div, cp)
    
d1 = (log(s/k)+(rf-div+0.5*v*v)*t)/(v*sqrt(t));
d2 = d1 - v*sqrt(t);
optprice = cp*s*exp(-div*t)*std\_norm\_cdf(cp*d1)-cp*k*exp(-rf*t)*std\_norm\_cdf(cp*d2);
z = optprice;
    \end{Verbatim}

    \begin{Verbatim}[commandchars=\\\{\}]
{\color{incolor}In [{\color{incolor}49}]:} \PY{o}{\PYZpc{}}\PY{k}{matplotlib} \PY{n}{inline}
\end{Verbatim}

    \begin{Verbatim}[commandchars=\\\{\}]
{\color{incolor}In [{\color{incolor}50}]:} \PY{n}{BS\PYZus{}bench}\PY{p}{[}\PY{l+s}{\PYZsq{}}\PY{l+s}{matlab}\PY{l+s}{\PYZsq{}}\PY{p}{]} \PY{o}{=} \PY{p}{[}\PY{l+s}{u\PYZsq{}}\PY{l+s}{55}\PY{l+s}{\PYZsq{}}\PY{p}{,}\PY{l+s}{u\PYZsq{}}\PY{l+s}{us}\PY{l+s}{\PYZsq{}}\PY{p}{]}
         \PY{n}{labels}\PY{o}{=} \PY{n}{BS\PYZus{}bench}\PY{o}{.}\PY{n}{keys}\PY{p}{(}\PY{p}{)}
         \PY{n}{timings} \PY{o}{=} \PY{p}{[}\PY{n}{np}\PY{o}{.}\PY{n}{timedelta64}\PY{p}{(}\PY{n+nb}{int}\PY{p}{(}\PY{n+nb}{float}\PY{p}{(}\PY{n}{value}\PY{p}{[}\PY{l+m+mi}{0}\PY{p}{]}\PY{p}{)}\PY{p}{)}\PY{p}{,}\PY{n+nb}{str}\PY{p}{(}\PY{n}{value}\PY{p}{[}\PY{l+m+mi}{1}\PY{p}{]}\PY{p}{)}\PY{p}{)}\PY{o}{/}\PY{n}{np}\PY{o}{.}\PY{n}{timedelta64}\PY{p}{(}\PY{l+m+mi}{1}\PY{p}{,} \PY{l+s}{\PYZsq{}}\PY{l+s}{s}\PY{l+s}{\PYZsq{}}\PY{p}{)} \PY{k}{for} \PY{n}{value} \PY{o+ow}{in} \PY{n}{BS\PYZus{}bench}\PY{o}{.}\PY{n}{values}\PY{p}{(}\PY{p}{)}\PY{p}{]}
         \PY{n}{x} \PY{o}{=} \PY{n}{arange}\PY{p}{(}\PY{n+nb}{len}\PY{p}{(}\PY{n}{labels}\PY{p}{)}\PY{p}{)}
         \PY{n}{ax} \PY{o}{=} \PY{n}{plt}\PY{o}{.}\PY{n}{axes}\PY{p}{(}\PY{n}{xticks}\PY{o}{=}\PY{n}{x}\PY{p}{,} \PY{n}{yscale}\PY{o}{=}\PY{l+s}{\PYZsq{}}\PY{l+s}{log}\PY{l+s}{\PYZsq{}}\PY{p}{)}
         \PY{n}{ax}\PY{o}{.}\PY{n}{bar}\PY{p}{(}\PY{n}{x} \PY{o}{\PYZhy{}} \PY{l+m+mf}{0.3}\PY{p}{,} \PY{n}{timings}\PY{p}{,} \PY{n}{width}\PY{o}{=}\PY{l+m+mf}{0.6}\PY{p}{,} \PY{n}{alpha}\PY{o}{=}\PY{l+m+mf}{0.4}\PY{p}{,} \PY{n}{bottom}\PY{o}{=}\PY{l+m+mf}{1E\PYZhy{}6}\PY{p}{)}
         \PY{n}{ax}\PY{o}{.}\PY{n}{grid}\PY{p}{(}\PY{p}{)}
         \PY{n}{ax}\PY{o}{.}\PY{n}{set\PYZus{}xlim}\PY{p}{(}\PY{o}{\PYZhy{}}\PY{l+m+mf}{0.5}\PY{p}{,} \PY{n+nb}{len}\PY{p}{(}\PY{n}{labels}\PY{p}{)} \PY{o}{\PYZhy{}} \PY{l+m+mf}{0.5}\PY{p}{)}
         \PY{n}{ax}\PY{o}{.}\PY{n}{set\PYZus{}ylim}\PY{p}{(}\PY{n+nb}{min}\PY{p}{(}\PY{n}{timings}\PY{p}{)}\PY{p}{,} \PY{l+m+mf}{1.1}\PY{o}{*}\PY{n+nb}{max}\PY{p}{(}\PY{n}{timings}\PY{p}{)}\PY{p}{)}
         \PY{n}{ax}\PY{o}{.}\PY{n}{xaxis}\PY{o}{.}\PY{n}{set\PYZus{}major\PYZus{}formatter}\PY{p}{(}\PY{n}{plt}\PY{o}{.}\PY{n}{FuncFormatter}\PY{p}{(}\PY{k}{lambda} \PY{n}{i}\PY{p}{,} \PY{n}{loc}\PY{p}{:} \PY{n}{labels}\PY{p}{[}\PY{n+nb}{int}\PY{p}{(}\PY{n}{i}\PY{p}{)}\PY{p}{]}\PY{p}{)}\PY{p}{)}
         \PY{n}{ax}\PY{o}{.}\PY{n}{set\PYZus{}ylabel}\PY{p}{(}\PY{l+s}{\PYZsq{}}\PY{l+s}{time (\PYZdl{}s\PYZdl{})}\PY{l+s}{\PYZsq{}}\PY{p}{)}
         \PY{n}{ax}\PY{o}{.}\PY{n}{set\PYZus{}title}\PY{p}{(}\PY{l+s}{\PYZdq{}}\PY{l+s}{Black\PYZhy{}Scholes Timings}\PY{l+s}{\PYZdq{}}\PY{p}{)}
\end{Verbatim}

            \begin{Verbatim}[commandchars=\\\{\}]
{\color{outcolor}Out[{\color{outcolor}50}]:} <matplotlib.text.Text at 0x7f7c55db4a50>
\end{Verbatim}
        
    \begin{center}
    \adjustimage{max size={0.9\linewidth}{0.9\paperheight}}{VIII-Benchmarks_files/VIII-Benchmarks_97_1.pdf}
    \end{center}
    { \hspace*{\fill} \\}
    
    \emph{This post was written entirely as an IPython notebook.} \emph{The
full notebook can be downloaded}
\href{https://raw.github.com/PoeticCapybara/Python-Introduction-Zittau/master/Lecture-8-Benchmarks.ipynb}{\emph{here}},
\emph{or viewed statically on}
\href{http://nbviewer.ipython.org/urls/raw.github.com/PoeticCapybara/Python-Introduction-Zittau/master/Lecture-8-Benchmarks.ipynb}{\emph{nbviewer}}

\hyperref[top]{Back to top}

    \begin{Verbatim}[commandchars=\\\{\}]
{\color{incolor}In [{\color{incolor}51}]:} \PY{o}{\PYZpc{}}\PY{k}{load\PYZus{}ext} \PY{n}{version\PYZus{}information}
         \PY{o}{\PYZpc{}}\PY{k}{version\PYZus{}information} \PY{n}{numpy}\PY{p}{,}\PY{n}{numba}\PY{p}{,}\PY{n}{parakeet}\PY{p}{,}\PY{n}{cython}\PY{p}{,}\PY{n}{matplotlib}\PY{p}{,}\PY{n}{scipy}\PY{p}{,}\PY{n}{sklearn}
\end{Verbatim}
\texttt{\color{outcolor}Out[{\color{outcolor}51}]:}
    
    \begin{tabular}{|l|l|}\hline
{\bf Software} & {\bf Version} \\ \hline\hline
Python & 2.7.8 |Anaconda 2.1.0 (64-bit)| (default, Aug 21 2014, 18:22:21) [GCC 4.4.7 20120313 (Red Hat 4.4.7-1)] \\ \hline
IPython & 2.3.1 \\ \hline
OS & posix [linux2] \\ \hline
numpy & 1.9.1 \\ \hline
numba & 0.15.1 \\ \hline
parakeet & 0.23.2 \\ \hline
cython & 0.21.1 \\ \hline
matplotlib & 1.4.2 \\ \hline
scipy & 0.14.0 \\ \hline
sklearn & 0.15.2 \\ \hline
\hline \multicolumn{2}{|l|}{Fri Dec 05 11:18:11 2014 CET} \\ \hline
\end{tabular}


    

    \begin{Verbatim}[commandchars=\\\{\}]
{\color{incolor}In [{\color{incolor}1}]:} \PY{k+kn}{from} \PY{n+nn}{IPython.core.display} \PY{k+kn}{import} \PY{n}{HTML}
        \PY{k}{def} \PY{n+nf}{css\PYZus{}styling}\PY{p}{(}\PY{p}{)}\PY{p}{:}
            \PY{n}{styles} \PY{o}{=} \PY{n+nb}{open}\PY{p}{(}\PY{l+s}{\PYZdq{}}\PY{l+s}{./styles/custom.css}\PY{l+s}{\PYZdq{}}\PY{p}{,} \PY{l+s}{\PYZdq{}}\PY{l+s}{r}\PY{l+s}{\PYZdq{}}\PY{p}{)}\PY{o}{.}\PY{n}{read}\PY{p}{(}\PY{p}{)}
            \PY{k}{return} \PY{n}{HTML}\PY{p}{(}\PY{n}{styles}\PY{p}{)}
        \PY{n}{css\PYZus{}styling}\PY{p}{(}\PY{p}{)}
\end{Verbatim}

            \begin{Verbatim}[commandchars=\\\{\}]
{\color{outcolor}Out[{\color{outcolor}1}]:} <IPython.core.display.HTML at 0x7fb6edb1b8d0>
\end{Verbatim}
        
    \begin{Verbatim}[commandchars=\\\{\}]
{\color{incolor}In [{\color{incolor}}]:} 
\end{Verbatim}


    % Add a bibliography block to the postdoc
    
    
    
    \end{document}


%\newpage
%
% Default to the notebook output style

    


% Inherit from the specified cell style.




    
\documentclass{article}

    
    
    \usepackage{graphicx} % Used to insert images
    \usepackage{adjustbox} % Used to constrain images to a maximum size 
    \usepackage{color} % Allow colors to be defined
    \usepackage{enumerate} % Needed for markdown enumerations to work
    \usepackage{geometry} % Used to adjust the document margins
    \usepackage{amsmath} % Equations
    \usepackage{amssymb} % Equations
    \usepackage[mathletters]{ucs} % Extended unicode (utf-8) support
    \usepackage[utf8x]{inputenc} % Allow utf-8 characters in the tex document
    \usepackage{fancyvrb} % verbatim replacement that allows latex
    \usepackage{grffile} % extends the file name processing of package graphics 
                         % to support a larger range 
    % The hyperref package gives us a pdf with properly built
    % internal navigation ('pdf bookmarks' for the table of contents,
    % internal cross-reference links, web links for URLs, etc.)
    \usepackage{hyperref}
    \usepackage{longtable} % longtable support required by pandoc >1.10
    \usepackage{booktabs}  % table support for pandoc > 1.12.2
    

    
    
    \definecolor{orange}{cmyk}{0,0.4,0.8,0.2}
    \definecolor{darkorange}{rgb}{.71,0.21,0.01}
    \definecolor{darkgreen}{rgb}{.12,.54,.11}
    \definecolor{myteal}{rgb}{.26, .44, .56}
    \definecolor{gray}{gray}{0.45}
    \definecolor{lightgray}{gray}{.95}
    \definecolor{mediumgray}{gray}{.8}
    \definecolor{inputbackground}{rgb}{.95, .95, .85}
    \definecolor{outputbackground}{rgb}{.95, .95, .95}
    \definecolor{traceback}{rgb}{1, .95, .95}
    % ansi colors
    \definecolor{red}{rgb}{.6,0,0}
    \definecolor{green}{rgb}{0,.65,0}
    \definecolor{brown}{rgb}{0.6,0.6,0}
    \definecolor{blue}{rgb}{0,.145,.698}
    \definecolor{purple}{rgb}{.698,.145,.698}
    \definecolor{cyan}{rgb}{0,.698,.698}
    \definecolor{lightgray}{gray}{0.5}
    
    % bright ansi colors
    \definecolor{darkgray}{gray}{0.25}
    \definecolor{lightred}{rgb}{1.0,0.39,0.28}
    \definecolor{lightgreen}{rgb}{0.48,0.99,0.0}
    \definecolor{lightblue}{rgb}{0.53,0.81,0.92}
    \definecolor{lightpurple}{rgb}{0.87,0.63,0.87}
    \definecolor{lightcyan}{rgb}{0.5,1.0,0.83}
    
    % commands and environments needed by pandoc snippets
    % extracted from the output of `pandoc -s`
    \DefineVerbatimEnvironment{Highlighting}{Verbatim}{commandchars=\\\{\}}
    % Add ',fontsize=\small' for more characters per line
    \newenvironment{Shaded}{}{}
    \newcommand{\KeywordTok}[1]{\textcolor[rgb]{0.00,0.44,0.13}{\textbf{{#1}}}}
    \newcommand{\DataTypeTok}[1]{\textcolor[rgb]{0.56,0.13,0.00}{{#1}}}
    \newcommand{\DecValTok}[1]{\textcolor[rgb]{0.25,0.63,0.44}{{#1}}}
    \newcommand{\BaseNTok}[1]{\textcolor[rgb]{0.25,0.63,0.44}{{#1}}}
    \newcommand{\FloatTok}[1]{\textcolor[rgb]{0.25,0.63,0.44}{{#1}}}
    \newcommand{\CharTok}[1]{\textcolor[rgb]{0.25,0.44,0.63}{{#1}}}
    \newcommand{\StringTok}[1]{\textcolor[rgb]{0.25,0.44,0.63}{{#1}}}
    \newcommand{\CommentTok}[1]{\textcolor[rgb]{0.38,0.63,0.69}{\textit{{#1}}}}
    \newcommand{\OtherTok}[1]{\textcolor[rgb]{0.00,0.44,0.13}{{#1}}}
    \newcommand{\AlertTok}[1]{\textcolor[rgb]{1.00,0.00,0.00}{\textbf{{#1}}}}
    \newcommand{\FunctionTok}[1]{\textcolor[rgb]{0.02,0.16,0.49}{{#1}}}
    \newcommand{\RegionMarkerTok}[1]{{#1}}
    \newcommand{\ErrorTok}[1]{\textcolor[rgb]{1.00,0.00,0.00}{\textbf{{#1}}}}
    \newcommand{\NormalTok}[1]{{#1}}
    
    % Define a nice break command that doesn't care if a line doesn't already
    % exist.
    \def\br{\hspace*{\fill} \\* }
    % Math Jax compatability definitions
    \def\gt{>}
    \def\lt{<}
    % Document parameters
    \title{IX-Randomness}
    
    
    

    % Pygments definitions
    
\makeatletter
\def\PY@reset{\let\PY@it=\relax \let\PY@bf=\relax%
    \let\PY@ul=\relax \let\PY@tc=\relax%
    \let\PY@bc=\relax \let\PY@ff=\relax}
\def\PY@tok#1{\csname PY@tok@#1\endcsname}
\def\PY@toks#1+{\ifx\relax#1\empty\else%
    \PY@tok{#1}\expandafter\PY@toks\fi}
\def\PY@do#1{\PY@bc{\PY@tc{\PY@ul{%
    \PY@it{\PY@bf{\PY@ff{#1}}}}}}}
\def\PY#1#2{\PY@reset\PY@toks#1+\relax+\PY@do{#2}}

\expandafter\def\csname PY@tok@gd\endcsname{\def\PY@tc##1{\textcolor[rgb]{0.63,0.00,0.00}{##1}}}
\expandafter\def\csname PY@tok@gu\endcsname{\let\PY@bf=\textbf\def\PY@tc##1{\textcolor[rgb]{0.50,0.00,0.50}{##1}}}
\expandafter\def\csname PY@tok@gt\endcsname{\def\PY@tc##1{\textcolor[rgb]{0.00,0.27,0.87}{##1}}}
\expandafter\def\csname PY@tok@gs\endcsname{\let\PY@bf=\textbf}
\expandafter\def\csname PY@tok@gr\endcsname{\def\PY@tc##1{\textcolor[rgb]{1.00,0.00,0.00}{##1}}}
\expandafter\def\csname PY@tok@cm\endcsname{\let\PY@it=\textit\def\PY@tc##1{\textcolor[rgb]{0.25,0.50,0.50}{##1}}}
\expandafter\def\csname PY@tok@vg\endcsname{\def\PY@tc##1{\textcolor[rgb]{0.10,0.09,0.49}{##1}}}
\expandafter\def\csname PY@tok@m\endcsname{\def\PY@tc##1{\textcolor[rgb]{0.40,0.40,0.40}{##1}}}
\expandafter\def\csname PY@tok@mh\endcsname{\def\PY@tc##1{\textcolor[rgb]{0.40,0.40,0.40}{##1}}}
\expandafter\def\csname PY@tok@go\endcsname{\def\PY@tc##1{\textcolor[rgb]{0.53,0.53,0.53}{##1}}}
\expandafter\def\csname PY@tok@ge\endcsname{\let\PY@it=\textit}
\expandafter\def\csname PY@tok@vc\endcsname{\def\PY@tc##1{\textcolor[rgb]{0.10,0.09,0.49}{##1}}}
\expandafter\def\csname PY@tok@il\endcsname{\def\PY@tc##1{\textcolor[rgb]{0.40,0.40,0.40}{##1}}}
\expandafter\def\csname PY@tok@cs\endcsname{\let\PY@it=\textit\def\PY@tc##1{\textcolor[rgb]{0.25,0.50,0.50}{##1}}}
\expandafter\def\csname PY@tok@cp\endcsname{\def\PY@tc##1{\textcolor[rgb]{0.74,0.48,0.00}{##1}}}
\expandafter\def\csname PY@tok@gi\endcsname{\def\PY@tc##1{\textcolor[rgb]{0.00,0.63,0.00}{##1}}}
\expandafter\def\csname PY@tok@gh\endcsname{\let\PY@bf=\textbf\def\PY@tc##1{\textcolor[rgb]{0.00,0.00,0.50}{##1}}}
\expandafter\def\csname PY@tok@ni\endcsname{\let\PY@bf=\textbf\def\PY@tc##1{\textcolor[rgb]{0.60,0.60,0.60}{##1}}}
\expandafter\def\csname PY@tok@nl\endcsname{\def\PY@tc##1{\textcolor[rgb]{0.63,0.63,0.00}{##1}}}
\expandafter\def\csname PY@tok@nn\endcsname{\let\PY@bf=\textbf\def\PY@tc##1{\textcolor[rgb]{0.00,0.00,1.00}{##1}}}
\expandafter\def\csname PY@tok@no\endcsname{\def\PY@tc##1{\textcolor[rgb]{0.53,0.00,0.00}{##1}}}
\expandafter\def\csname PY@tok@na\endcsname{\def\PY@tc##1{\textcolor[rgb]{0.49,0.56,0.16}{##1}}}
\expandafter\def\csname PY@tok@nb\endcsname{\def\PY@tc##1{\textcolor[rgb]{0.00,0.50,0.00}{##1}}}
\expandafter\def\csname PY@tok@nc\endcsname{\let\PY@bf=\textbf\def\PY@tc##1{\textcolor[rgb]{0.00,0.00,1.00}{##1}}}
\expandafter\def\csname PY@tok@nd\endcsname{\def\PY@tc##1{\textcolor[rgb]{0.67,0.13,1.00}{##1}}}
\expandafter\def\csname PY@tok@ne\endcsname{\let\PY@bf=\textbf\def\PY@tc##1{\textcolor[rgb]{0.82,0.25,0.23}{##1}}}
\expandafter\def\csname PY@tok@nf\endcsname{\def\PY@tc##1{\textcolor[rgb]{0.00,0.00,1.00}{##1}}}
\expandafter\def\csname PY@tok@si\endcsname{\let\PY@bf=\textbf\def\PY@tc##1{\textcolor[rgb]{0.73,0.40,0.53}{##1}}}
\expandafter\def\csname PY@tok@s2\endcsname{\def\PY@tc##1{\textcolor[rgb]{0.73,0.13,0.13}{##1}}}
\expandafter\def\csname PY@tok@vi\endcsname{\def\PY@tc##1{\textcolor[rgb]{0.10,0.09,0.49}{##1}}}
\expandafter\def\csname PY@tok@nt\endcsname{\let\PY@bf=\textbf\def\PY@tc##1{\textcolor[rgb]{0.00,0.50,0.00}{##1}}}
\expandafter\def\csname PY@tok@nv\endcsname{\def\PY@tc##1{\textcolor[rgb]{0.10,0.09,0.49}{##1}}}
\expandafter\def\csname PY@tok@s1\endcsname{\def\PY@tc##1{\textcolor[rgb]{0.73,0.13,0.13}{##1}}}
\expandafter\def\csname PY@tok@kd\endcsname{\let\PY@bf=\textbf\def\PY@tc##1{\textcolor[rgb]{0.00,0.50,0.00}{##1}}}
\expandafter\def\csname PY@tok@sh\endcsname{\def\PY@tc##1{\textcolor[rgb]{0.73,0.13,0.13}{##1}}}
\expandafter\def\csname PY@tok@sc\endcsname{\def\PY@tc##1{\textcolor[rgb]{0.73,0.13,0.13}{##1}}}
\expandafter\def\csname PY@tok@sx\endcsname{\def\PY@tc##1{\textcolor[rgb]{0.00,0.50,0.00}{##1}}}
\expandafter\def\csname PY@tok@bp\endcsname{\def\PY@tc##1{\textcolor[rgb]{0.00,0.50,0.00}{##1}}}
\expandafter\def\csname PY@tok@c1\endcsname{\let\PY@it=\textit\def\PY@tc##1{\textcolor[rgb]{0.25,0.50,0.50}{##1}}}
\expandafter\def\csname PY@tok@kc\endcsname{\let\PY@bf=\textbf\def\PY@tc##1{\textcolor[rgb]{0.00,0.50,0.00}{##1}}}
\expandafter\def\csname PY@tok@c\endcsname{\let\PY@it=\textit\def\PY@tc##1{\textcolor[rgb]{0.25,0.50,0.50}{##1}}}
\expandafter\def\csname PY@tok@mf\endcsname{\def\PY@tc##1{\textcolor[rgb]{0.40,0.40,0.40}{##1}}}
\expandafter\def\csname PY@tok@err\endcsname{\def\PY@bc##1{\setlength{\fboxsep}{0pt}\fcolorbox[rgb]{1.00,0.00,0.00}{1,1,1}{\strut ##1}}}
\expandafter\def\csname PY@tok@mb\endcsname{\def\PY@tc##1{\textcolor[rgb]{0.40,0.40,0.40}{##1}}}
\expandafter\def\csname PY@tok@ss\endcsname{\def\PY@tc##1{\textcolor[rgb]{0.10,0.09,0.49}{##1}}}
\expandafter\def\csname PY@tok@sr\endcsname{\def\PY@tc##1{\textcolor[rgb]{0.73,0.40,0.53}{##1}}}
\expandafter\def\csname PY@tok@mo\endcsname{\def\PY@tc##1{\textcolor[rgb]{0.40,0.40,0.40}{##1}}}
\expandafter\def\csname PY@tok@kn\endcsname{\let\PY@bf=\textbf\def\PY@tc##1{\textcolor[rgb]{0.00,0.50,0.00}{##1}}}
\expandafter\def\csname PY@tok@mi\endcsname{\def\PY@tc##1{\textcolor[rgb]{0.40,0.40,0.40}{##1}}}
\expandafter\def\csname PY@tok@gp\endcsname{\let\PY@bf=\textbf\def\PY@tc##1{\textcolor[rgb]{0.00,0.00,0.50}{##1}}}
\expandafter\def\csname PY@tok@o\endcsname{\def\PY@tc##1{\textcolor[rgb]{0.40,0.40,0.40}{##1}}}
\expandafter\def\csname PY@tok@kr\endcsname{\let\PY@bf=\textbf\def\PY@tc##1{\textcolor[rgb]{0.00,0.50,0.00}{##1}}}
\expandafter\def\csname PY@tok@s\endcsname{\def\PY@tc##1{\textcolor[rgb]{0.73,0.13,0.13}{##1}}}
\expandafter\def\csname PY@tok@kp\endcsname{\def\PY@tc##1{\textcolor[rgb]{0.00,0.50,0.00}{##1}}}
\expandafter\def\csname PY@tok@w\endcsname{\def\PY@tc##1{\textcolor[rgb]{0.73,0.73,0.73}{##1}}}
\expandafter\def\csname PY@tok@kt\endcsname{\def\PY@tc##1{\textcolor[rgb]{0.69,0.00,0.25}{##1}}}
\expandafter\def\csname PY@tok@ow\endcsname{\let\PY@bf=\textbf\def\PY@tc##1{\textcolor[rgb]{0.67,0.13,1.00}{##1}}}
\expandafter\def\csname PY@tok@sb\endcsname{\def\PY@tc##1{\textcolor[rgb]{0.73,0.13,0.13}{##1}}}
\expandafter\def\csname PY@tok@k\endcsname{\let\PY@bf=\textbf\def\PY@tc##1{\textcolor[rgb]{0.00,0.50,0.00}{##1}}}
\expandafter\def\csname PY@tok@se\endcsname{\let\PY@bf=\textbf\def\PY@tc##1{\textcolor[rgb]{0.73,0.40,0.13}{##1}}}
\expandafter\def\csname PY@tok@sd\endcsname{\let\PY@it=\textit\def\PY@tc##1{\textcolor[rgb]{0.73,0.13,0.13}{##1}}}

\def\PYZbs{\char`\\}
\def\PYZus{\char`\_}
\def\PYZob{\char`\{}
\def\PYZcb{\char`\}}
\def\PYZca{\char`\^}
\def\PYZam{\char`\&}
\def\PYZlt{\char`\<}
\def\PYZgt{\char`\>}
\def\PYZsh{\char`\#}
\def\PYZpc{\char`\%}
\def\PYZdl{\char`\$}
\def\PYZhy{\char`\-}
\def\PYZsq{\char`\'}
\def\PYZdq{\char`\"}
\def\PYZti{\char`\~}
% for compatibility with earlier versions
\def\PYZat{@}
\def\PYZlb{[}
\def\PYZrb{]}
\makeatother


    % Exact colors from NB
    \definecolor{incolor}{rgb}{0.0, 0.0, 0.5}
    \definecolor{outcolor}{rgb}{0.545, 0.0, 0.0}



    
    % Prevent overflowing lines due to hard-to-break entities
    \sloppy 
    % Setup hyperref package
    \hypersetup{
      breaklinks=true,  % so long urls are correctly broken across lines
      colorlinks=true,
      urlcolor=blue,
      linkcolor=darkorange,
      citecolor=darkgreen,
      }
    % Slightly bigger margins than the latex defaults
    
    \geometry{verbose,tmargin=1in,bmargin=1in,lmargin=1in,rmargin=1in}
    
    

    \begin{document}
    
    
    \maketitle
    
    

    
    \section{IX-\href{http://lmgtfy.com/?q=Randomness\&l=1}{Randomness}}\label{ix-randomness}

    \begin{Verbatim}[commandchars=\\\{\}]
{\color{incolor}In [{\color{incolor}1}]:} \PY{k+kn}{from} \PY{n+nn}{IPython.display} \PY{k+kn}{import} \PY{n}{Image}
\end{Verbatim}

    \begin{Verbatim}[commandchars=\\\{\}]
{\color{incolor}In [{\color{incolor}2}]:} \PY{n}{Image}\PY{p}{(}\PY{l+s}{\PYZsq{}}\PY{l+s}{http://imgs.xkcd.com/comics/random\PYZus{}number.png}\PY{l+s}{\PYZsq{}}\PY{p}{)}
\end{Verbatim}
\texttt{\color{outcolor}Out[{\color{outcolor}2}]:}
    
    \begin{center}
    \adjustimage{max size={0.9\linewidth}{0.9\paperheight}}{IX-Randomness_files/IX-Randomness_2_0.png}
    \end{center}
    { \hspace*{\fill} \\}
    

    \begin{Verbatim}[commandchars=\\\{\}]
{\color{incolor}In [{\color{incolor}3}]:} \PY{n}{Image}\PY{p}{(}\PY{l+s}{\PYZsq{}}\PY{l+s}{http://imgs.xkcd.com/comics/ayn\PYZus{}random.png}\PY{l+s}{\PYZsq{}}\PY{p}{)}
\end{Verbatim}
\texttt{\color{outcolor}Out[{\color{outcolor}3}]:}
    
    \begin{center}
    \adjustimage{max size={0.9\linewidth}{0.9\paperheight}}{IX-Randomness_files/IX-Randomness_3_0.png}
    \end{center}
    { \hspace*{\fill} \\}
    

    \begin{Verbatim}[commandchars=\\\{\}]
{\color{incolor}In [{\color{incolor}4}]:} \PY{n}{Image}\PY{p}{(}\PY{l+s}{\PYZsq{}}\PY{l+s}{http://www.random.org/analysis/dilbert.jpg}\PY{l+s}{\PYZsq{}}\PY{p}{)}
\end{Verbatim}
\texttt{\color{outcolor}Out[{\color{outcolor}4}]:}
    
    \begin{center}
    \adjustimage{max size={0.9\linewidth}{0.9\paperheight}}{IX-Randomness_files/IX-Randomness_4_0.jpeg}
    \end{center}
    { \hspace*{\fill} \\}
    

    \begin{Verbatim}[commandchars=\\\{\}]
{\color{incolor}In [{\color{incolor}15}]:} \PY{k+kn}{import} \PY{n+nn}{numpy} \PY{k+kn}{as} \PY{n+nn}{np}
\end{Verbatim}

    The \emph{random} subpackage is the package responsible for all random
functions in the numpy environment

    \begin{Verbatim}[commandchars=\\\{\}]
{\color{incolor}In [{\color{incolor}16}]:} \PY{c}{\PYZsh{}help(np.random)}
\end{Verbatim}

    Let's generate a couple of random paths

    \begin{Verbatim}[commandchars=\\\{\}]
{\color{incolor}In [{\color{incolor}26}]:} \PY{n}{Nt} \PY{o}{=} \PY{l+m+mi}{1000}
         \PY{n}{Npaths} \PY{o}{=} \PY{l+m+mi}{10}
         \PY{n}{rv} \PY{o}{=} \PY{n}{np}\PY{o}{.}\PY{n}{random}\PY{o}{.}\PY{n}{randn}\PY{p}{(}\PY{l+m+mi}{10}\PY{p}{,}\PY{l+m+mi}{1000}\PY{p}{)}
\end{Verbatim}

    \begin{Verbatim}[commandchars=\\\{\}]
{\color{incolor}In [{\color{incolor}27}]:} \PY{n}{rv}\PY{o}{.}\PY{n}{shape}
\end{Verbatim}

            \begin{Verbatim}[commandchars=\\\{\}]
{\color{outcolor}Out[{\color{outcolor}27}]:} (10, 1000)
\end{Verbatim}
        
    \begin{Verbatim}[commandchars=\\\{\}]
{\color{incolor}In [{\color{incolor}28}]:} \PY{n}{paths} \PY{o}{=} \PY{n}{rv}\PY{o}{.}\PY{n}{cumsum}\PY{p}{(}\PY{l+m+mi}{1}\PY{p}{)}
\end{Verbatim}

    \begin{Verbatim}[commandchars=\\\{\}]
{\color{incolor}In [{\color{incolor}29}]:} \PY{n}{paths}\PY{o}{.}\PY{n}{shape}
\end{Verbatim}

            \begin{Verbatim}[commandchars=\\\{\}]
{\color{outcolor}Out[{\color{outcolor}29}]:} (10, 1000)
\end{Verbatim}
        
    \begin{Verbatim}[commandchars=\\\{\}]
{\color{incolor}In [{\color{incolor}21}]:} \PY{o}{\PYZpc{}}\PY{k}{matplotlib} \PY{n}{inline}
         \PY{k+kn}{from} \PY{n+nn}{matplotlib.pyplot} \PY{k+kn}{import} \PY{o}{*}
         \PY{k+kn}{import} \PY{n+nn}{seaborn} \PY{k+kn}{as} \PY{n+nn}{sns}
         \PY{n}{sns}\PY{o}{.}\PY{n}{set}\PY{p}{(}\PY{n}{style}\PY{o}{=}\PY{l+s}{\PYZsq{}}\PY{l+s}{ticks}\PY{l+s}{\PYZsq{}}\PY{p}{,} \PY{n}{palette}\PY{o}{=}\PY{l+s}{\PYZsq{}}\PY{l+s}{Set2}\PY{l+s}{\PYZsq{}}\PY{p}{)}
\end{Verbatim}

    \begin{Verbatim}[commandchars=\\\{\}]
{\color{incolor}In [{\color{incolor}22}]:} \PY{n}{plot}\PY{p}{(}\PY{n}{paths}\PY{o}{.}\PY{n}{T}\PY{p}{)}
\end{Verbatim}

            \begin{Verbatim}[commandchars=\\\{\}]
{\color{outcolor}Out[{\color{outcolor}22}]:} [<matplotlib.lines.Line2D at 0x7f09d8b351d0>,
          <matplotlib.lines.Line2D at 0x7f09d8b23650>,
          <matplotlib.lines.Line2D at 0x7f09d8b238d0>,
          <matplotlib.lines.Line2D at 0x7f09d8b23110>,
          <matplotlib.lines.Line2D at 0x7f09d8a9f110>,
          <matplotlib.lines.Line2D at 0x7f09d8a9f2d0>,
          <matplotlib.lines.Line2D at 0x7f09d8b3e6d0>,
          <matplotlib.lines.Line2D at 0x7f09d8a9f650>,
          <matplotlib.lines.Line2D at 0x7f09d8a9f810>,
          <matplotlib.lines.Line2D at 0x7f09d8a9f9d0>]
\end{Verbatim}
        
    \begin{Verbatim}[commandchars=\\\{\}]
/home/jpsilva/anaconda/lib/python2.7/site-packages/matplotlib/font\_manager.py:1279: UserWarning: findfont: Font family [u'Arial'] not found. Falling back to Bitstream Vera Sans
  (prop.get\_family(), self.defaultFamily[fontext]))
    \end{Verbatim}

    \begin{center}
    \adjustimage{max size={0.9\linewidth}{0.9\paperheight}}{IX-Randomness_files/IX-Randomness_14_2.pdf}
    \end{center}
    { \hspace*{\fill} \\}
    
    \begin{Verbatim}[commandchars=\\\{\}]
{\color{incolor}In [{\color{incolor}38}]:} \PY{k+kn}{from} \PY{n+nn}{scipy} \PY{k+kn}{import} \PY{n}{stats}
\end{Verbatim}

    We can easily get a list of available distributions (continuous and
discrete) by using \emph{dir} or by checking on the online
\href{http://docs.scipy.org/doc/scipy/reference/stats.html\#continuous-distributions}{documentation}

    \begin{Verbatim}[commandchars=\\\{\}]
{\color{incolor}In [{\color{incolor}45}]:} \PY{n}{list\PYZus{}distributions} \PY{o}{=} \PY{n+nb}{dir}\PY{p}{(}\PY{n}{stats}\PY{p}{)}
         \PY{n+nb}{len}\PY{p}{(}\PY{n}{list\PYZus{}distributions}\PY{p}{)}
\end{Verbatim}

            \begin{Verbatim}[commandchars=\\\{\}]
{\color{outcolor}Out[{\color{outcolor}45}]:} 243
\end{Verbatim}
        
    Let's choose a Pareto distribution with parameter $\alpha=3$
\[ pdf_{pareto}(x) = \frac{\alpha}{x^{\alpha+1}} \]

    \begin{Verbatim}[commandchars=\\\{\}]
{\color{incolor}In [{\color{incolor}51}]:} \PY{n}{alpha} \PY{o}{=} \PY{l+m+mi}{3}
         \PY{n}{X} \PY{o}{=} \PY{n}{stats}\PY{o}{.}\PY{n}{pareto}\PY{p}{(}\PY{n}{alpha}\PY{p}{)}
\end{Verbatim}

    We can sample\ldots{}

    \begin{Verbatim}[commandchars=\\\{\}]
{\color{incolor}In [{\color{incolor}50}]:} \PY{n}{n} \PY{o}{=} \PY{l+m+mi}{100}
         \PY{n}{plot}\PY{p}{(}\PY{n}{X}\PY{o}{.}\PY{n}{rvs}\PY{p}{(}\PY{n}{n}\PY{p}{)}\PY{p}{,}\PY{l+s}{\PYZsq{}}\PY{l+s}{.}\PY{l+s}{\PYZsq{}}\PY{p}{)}
\end{Verbatim}

            \begin{Verbatim}[commandchars=\\\{\}]
{\color{outcolor}Out[{\color{outcolor}50}]:} [<matplotlib.lines.Line2D at 0x7f09d6ce03d0>]
\end{Verbatim}
        
    \begin{center}
    \adjustimage{max size={0.9\linewidth}{0.9\paperheight}}{IX-Randomness_files/IX-Randomness_21_1.pdf}
    \end{center}
    { \hspace*{\fill} \\}
    
    \ldots{}have a look at the probability density function\ldots{}

    \begin{Verbatim}[commandchars=\\\{\}]
{\color{incolor}In [{\color{incolor}58}]:} \PY{n}{x} \PY{o}{=} \PY{n}{np}\PY{o}{.}\PY{n}{linspace}\PY{p}{(}\PY{n}{X}\PY{o}{.}\PY{n}{ppf}\PY{p}{(}\PY{l+m+mf}{0.01}\PY{p}{)}\PY{p}{,} \PY{n}{X}\PY{o}{.}\PY{n}{ppf}\PY{p}{(}\PY{l+m+mf}{0.99}\PY{p}{)}\PY{p}{,} \PY{l+m+mi}{100}\PY{p}{)}
         \PY{n}{plot}\PY{p}{(}\PY{n}{x}\PY{p}{,} \PY{n}{X}\PY{o}{.}\PY{n}{pdf}\PY{p}{(}\PY{n}{x}\PY{p}{)}\PY{p}{,}\PY{n}{label}\PY{o}{=}\PY{l+s}{\PYZsq{}}\PY{l+s}{Pareto pdf}\PY{l+s}{\PYZsq{}}\PY{p}{)}
         \PY{n}{legend}\PY{p}{(}\PY{p}{)}
         \PY{n}{xlabel}\PY{p}{(}\PY{l+s}{\PYZsq{}}\PY{l+s}{\PYZdl{}x\PYZdl{}}\PY{l+s}{\PYZsq{}}\PY{p}{)}
\end{Verbatim}

            \begin{Verbatim}[commandchars=\\\{\}]
{\color{outcolor}Out[{\color{outcolor}58}]:} <matplotlib.text.Text at 0x7f09d691bf90>
\end{Verbatim}
        
    \begin{center}
    \adjustimage{max size={0.9\linewidth}{0.9\paperheight}}{IX-Randomness_files/IX-Randomness_23_1.pdf}
    \end{center}
    { \hspace*{\fill} \\}
    
    \ldots{}have a look at the cumulative density function\ldots{}

    \begin{Verbatim}[commandchars=\\\{\}]
{\color{incolor}In [{\color{incolor}60}]:} \PY{n}{plot}\PY{p}{(}\PY{n}{x}\PY{p}{,} \PY{n}{X}\PY{o}{.}\PY{n}{cdf}\PY{p}{(}\PY{n}{x}\PY{p}{)}\PY{p}{,}\PY{n}{label}\PY{o}{=}\PY{l+s}{\PYZsq{}}\PY{l+s}{Pareto cdf}\PY{l+s}{\PYZsq{}}\PY{p}{)}
         \PY{n}{legend}\PY{p}{(}\PY{p}{)}
         \PY{n}{xlabel}\PY{p}{(}\PY{l+s}{\PYZsq{}}\PY{l+s}{\PYZdl{}x\PYZdl{}}\PY{l+s}{\PYZsq{}}\PY{p}{)}
\end{Verbatim}

            \begin{Verbatim}[commandchars=\\\{\}]
{\color{outcolor}Out[{\color{outcolor}60}]:} <matplotlib.text.Text at 0x7f09d67ef390>
\end{Verbatim}
        
    \begin{center}
    \adjustimage{max size={0.9\linewidth}{0.9\paperheight}}{IX-Randomness_files/IX-Randomness_25_1.pdf}
    \end{center}
    { \hspace*{\fill} \\}
    
    \ldots{}compare the histogram of the previously generated sample with
the pdf\ldots{}

    \begin{Verbatim}[commandchars=\\\{\}]
{\color{incolor}In [{\color{incolor}68}]:} \PY{n}{fig}\PY{p}{,} \PY{n}{ax} \PY{o}{=} \PY{n}{subplots}\PY{p}{(}\PY{p}{)}
         \PY{n}{ax}\PY{o}{.}\PY{n}{hist}\PY{p}{(}\PY{n}{X}\PY{o}{.}\PY{n}{rvs}\PY{p}{(}\PY{l+m+mi}{10000}\PY{p}{)}\PY{p}{,}\PY{n}{normed}\PY{o}{=}\PY{n+nb+bp}{True}\PY{p}{,}\PY{n}{bins}\PY{o}{=}\PY{l+m+mi}{100}\PY{p}{)}
         \PY{n}{ax}\PY{o}{.}\PY{n}{plot}\PY{p}{(}\PY{n}{x}\PY{p}{,}\PY{n}{X}\PY{o}{.}\PY{n}{pdf}\PY{p}{(}\PY{n}{x}\PY{p}{)}\PY{p}{)}
\end{Verbatim}

            \begin{Verbatim}[commandchars=\\\{\}]
{\color{outcolor}Out[{\color{outcolor}68}]:} [<matplotlib.lines.Line2D at 0x7f09d6016590>]
\end{Verbatim}
        
    \begin{center}
    \adjustimage{max size={0.9\linewidth}{0.9\paperheight}}{IX-Randomness_files/IX-Randomness_27_1.pdf}
    \end{center}
    { \hspace*{\fill} \\}
    
    (let's just see the influence of the number of bins in the
non-parametric estimation)

    \begin{Verbatim}[commandchars=\\\{\}]
{\color{incolor}In [{\color{incolor}86}]:} \PY{o}{\PYZpc{}}\PY{k}{matplotlib}
         \PY{k+kn}{from} \PY{n+nn}{matplotlib.widgets} \PY{k+kn}{import} \PY{n}{Slider}
         
         \PY{n}{samples} \PY{o}{=} \PY{n}{X}\PY{o}{.}\PY{n}{rvs}\PY{p}{(}\PY{l+m+mi}{10000}\PY{p}{)}
         
         \PY{n}{ax} \PY{o}{=} \PY{n}{axes}\PY{p}{(}\PY{p}{[}\PY{l+m+mf}{0.1}\PY{p}{,}\PY{l+m+mf}{0.25}\PY{p}{,}\PY{l+m+mf}{0.8}\PY{p}{,}\PY{l+m+mf}{0.6}\PY{p}{]}\PY{p}{)}
         \PY{n}{ax}\PY{o}{.}\PY{n}{hist}\PY{p}{(}\PY{n}{samples}\PY{p}{,}\PY{n}{normed}\PY{o}{=}\PY{n+nb+bp}{True}\PY{p}{)}
         \PY{n}{ax}\PY{o}{.}\PY{n}{plot}\PY{p}{(}\PY{n}{x}\PY{p}{,}\PY{n}{X}\PY{o}{.}\PY{n}{pdf}\PY{p}{(}\PY{n}{x}\PY{p}{)}\PY{p}{)}
         \PY{n}{sl} \PY{o}{=} \PY{n}{Slider}\PY{p}{(}\PY{n}{axes}\PY{p}{(}\PY{p}{[}\PY{l+m+mf}{0.1}\PY{p}{,}\PY{l+m+mf}{0.1}\PY{p}{,}\PY{l+m+mf}{0.8}\PY{p}{,}\PY{l+m+mf}{0.1}\PY{p}{]}\PY{p}{)}\PY{p}{,}\PY{l+s}{\PYZsq{}}\PY{l+s}{Bins}\PY{l+s}{\PYZsq{}}\PY{p}{,}\PY{l+m+mi}{1}\PY{p}{,}\PY{l+m+mi}{200}\PY{p}{,}\PY{n}{valinit}\PY{o}{=}\PY{l+m+mi}{10}\PY{p}{,}\PY{n}{valfmt}\PY{o}{=}\PY{l+s}{\PYZsq{}}\PY{l+s+si}{\PYZpc{}d}\PY{l+s}{\PYZsq{}}\PY{p}{)}
         
         \PY{k}{def} \PY{n+nf}{update}\PY{p}{(}\PY{n}{data}\PY{p}{)}\PY{p}{:}
             \PY{n}{data} \PY{o}{=} \PY{n+nb}{int}\PY{p}{(}\PY{n}{data}\PY{p}{)}
             \PY{n}{ax}\PY{o}{.}\PY{n}{cla}\PY{p}{(}\PY{p}{)}
             \PY{n}{ax}\PY{o}{.}\PY{n}{hist}\PY{p}{(}\PY{n}{samples}\PY{p}{,}\PY{n}{normed}\PY{o}{=}\PY{n+nb+bp}{True}\PY{p}{,}\PY{n}{bins}\PY{o}{=}\PY{n}{data}\PY{p}{)}
             \PY{n}{ax}\PY{o}{.}\PY{n}{plot}\PY{p}{(}\PY{n}{x}\PY{p}{,}\PY{n}{X}\PY{o}{.}\PY{n}{pdf}\PY{p}{(}\PY{n}{x}\PY{p}{)}\PY{p}{)}
             \PY{n}{ax}\PY{o}{.}\PY{n}{draw}\PY{p}{(}\PY{p}{)}
         
         \PY{n}{sl}\PY{o}{.}\PY{n}{on\PYZus{}changed}\PY{p}{(}\PY{n}{update}\PY{p}{)}
\end{Verbatim}

    \begin{Verbatim}[commandchars=\\\{\}]
Using matplotlib backend: Qt4Agg
    \end{Verbatim}

            \begin{Verbatim}[commandchars=\\\{\}]
{\color{outcolor}Out[{\color{outcolor}86}]:} 0
\end{Verbatim}
        
    \begin{Verbatim}[commandchars=\\\{\}]
{\color{incolor}In [{\color{incolor}87}]:} \PY{n}{X}\PY{o}{.}\PY{n}{stats}\PY{p}{(}\PY{n}{moments}\PY{o}{=}\PY{l+s}{\PYZsq{}}\PY{l+s}{mv}\PY{l+s}{\PYZsq{}}\PY{p}{)}
\end{Verbatim}

            \begin{Verbatim}[commandchars=\\\{\}]
{\color{outcolor}Out[{\color{outcolor}87}]:} (array(1.5), array(0.75))
\end{Verbatim}
        
    \begin{Verbatim}[commandchars=\\\{\}]
{\color{incolor}In [{\color{incolor}23}]:} \PY{k+kn}{from} \PY{n+nn}{scipy.stats} \PY{k+kn}{import} \PY{n}{kstest}
\end{Verbatim}

    \begin{Verbatim}[commandchars=\\\{\}]
{\color{incolor}In [{\color{incolor}30}]:} \PY{n}{kstest}\PY{p}{(}\PY{n}{rv}\PY{p}{,}\PY{l+s}{\PYZsq{}}\PY{l+s}{norm}\PY{l+s}{\PYZsq{}}\PY{p}{)}
\end{Verbatim}

    \begin{Verbatim}[commandchars=\\\{\}]

        ---------------------------------------------------------------------------
    ValueError                                Traceback (most recent call last)

        <ipython-input-30-f62508e716a2> in <module>()
    ----> 1 kstest(rv,'norm')
    

        /home/jpsilva/anaconda/lib/python2.7/site-packages/scipy/stats/stats.pyc in kstest(rvs, cdf, args, N, alternative, mode)
       3433 
       3434     if alternative in ['two-sided', 'greater']:
    -> 3435         Dplus = (np.arange(1.0, N+1)/N - cdfvals).max()
       3436         if alternative == 'greater':
       3437             return Dplus, distributions.ksone.sf(Dplus,N)


        ValueError: operands could not be broadcast together with shapes (10,) (10,1000) 

    \end{Verbatim}

    \begin{Verbatim}[commandchars=\\\{\}]
{\color{incolor}In [{\color{incolor}31}]:} \PY{n}{kstest}\PY{p}{(}\PY{n}{rv}\PY{p}{,}\PY{l+s}{\PYZsq{}}\PY{l+s}{norm}\PY{l+s}{\PYZsq{}}\PY{p}{)}
\end{Verbatim}

    \begin{Verbatim}[commandchars=\\\{\}]

        ---------------------------------------------------------------------------
    ValueError                                Traceback (most recent call last)

        <ipython-input-31-f62508e716a2> in <module>()
    ----> 1 kstest(rv,'norm')
    

        /home/jpsilva/anaconda/lib/python2.7/site-packages/scipy/stats/stats.pyc in kstest(rvs, cdf, args, N, alternative, mode)
       3433 
       3434     if alternative in ['two-sided', 'greater']:
    -> 3435         Dplus = (np.arange(1.0, N+1)/N - cdfvals).max()
       3436         if alternative == 'greater':
       3437             return Dplus, distributions.ksone.sf(Dplus,N)


        ValueError: operands could not be broadcast together with shapes (10,) (10,1000) 

    \end{Verbatim}

    \begin{Verbatim}[commandchars=\\\{\}]
{\color{incolor}In [{\color{incolor}33}]:} \PY{n}{rv}\PY{o}{.}\PY{n}{apply}\PY{p}{(}\PY{n}{kstest}\PY{p}{,}\PY{n}{axis}\PY{o}{=}\PY{l+m+mi}{1}\PY{p}{)}
\end{Verbatim}

    \begin{Verbatim}[commandchars=\\\{\}]

        ---------------------------------------------------------------------------
    AttributeError                            Traceback (most recent call last)

        <ipython-input-33-0e6a60028561> in <module>()
    ----> 1 rv.apply(kstest,axis=1)
    

        AttributeError: 'numpy.ndarray' object has no attribute 'apply'

    \end{Verbatim}

    \begin{Verbatim}[commandchars=\\\{\}]
{\color{incolor}In [{\color{incolor}34}]:} \PY{n}{np}\PY{o}{.}\PY{n}{apply\PYZus{}along\PYZus{}axis}\PY{p}{(}\PY{k}{lambda} \PY{n}{x}\PY{p}{:} \PY{n}{kstest}\PY{p}{(}\PY{n}{x}\PY{p}{,}\PY{l+s}{\PYZsq{}}\PY{l+s}{norm}\PY{l+s}{\PYZsq{}}\PY{p}{)}\PY{p}{,}\PY{l+m+mi}{1}\PY{p}{,}\PY{n}{rv}\PY{p}{)}
\end{Verbatim}

            \begin{Verbatim}[commandchars=\\\{\}]
{\color{outcolor}Out[{\color{outcolor}34}]:} array([[ 0.02462349,  0.57921451],
                [ 0.02313629,  0.65813139],
                [ 0.02690737,  0.46175665],
                [ 0.01927064,  0.85160859],
                [ 0.02455753,  0.58267687],
                [ 0.0551436 ,  0.00439168],
                [ 0.02582343,  0.51737188],
                [ 0.02793807,  0.41210544],
                [ 0.02163583,  0.73737606],
                [ 0.02397197,  0.61361516]])
\end{Verbatim}
        
    \begin{Verbatim}[commandchars=\\\{\}]
{\color{incolor}In [{\color{incolor}35}]:} \PY{n+nb}{map}\PY{p}{(}\PY{k}{lambda} \PY{n}{x}\PY{p}{:} \PY{n}{kstest}\PY{p}{(}\PY{n}{x}\PY{p}{,}\PY{l+s}{\PYZsq{}}\PY{l+s}{norm}\PY{l+s}{\PYZsq{}}\PY{p}{)}\PY{p}{,}\PY{n}{rv}\PY{p}{)}
\end{Verbatim}

            \begin{Verbatim}[commandchars=\\\{\}]
{\color{outcolor}Out[{\color{outcolor}35}]:} [(0.024623490897107081, 0.57921450514349404),
          (0.02313628625770614, 0.65813139144955546),
          (0.026907370554564158, 0.46175664577782749),
          (0.019270644257675884, 0.85160859135592293),
          (0.02455753392874805, 0.58267687044366556),
          (0.055143604679474989, 0.0043916809490187614),
          (0.025823431496813432, 0.51737188246826193),
          (0.027938071951015719, 0.41210543959431489),
          (0.021635830354071017, 0.73737606358434737),
          (0.023971974347128389, 0.61361516398657845)]
\end{Verbatim}
        
    \begin{Verbatim}[commandchars=\\\{\}]
{\color{incolor}In [{\color{incolor}36}]:} \PY{o}{\PYZpc{}\PYZpc{}}\PY{k}{timeit}
         \PY{n}{np}\PY{o}{.}\PY{n}{apply\PYZus{}along\PYZus{}axis}\PY{p}{(}\PY{k}{lambda} \PY{n}{x}\PY{p}{:} \PY{n}{kstest}\PY{p}{(}\PY{n}{x}\PY{p}{,}\PY{l+s}{\PYZsq{}}\PY{l+s}{norm}\PY{l+s}{\PYZsq{}}\PY{p}{)}\PY{p}{,}\PY{l+m+mi}{1}\PY{p}{,}\PY{n}{rv}\PY{p}{)}
\end{Verbatim}

    \begin{Verbatim}[commandchars=\\\{\}]
100 loops, best of 3: 6.61 ms per loop
    \end{Verbatim}

    \begin{Verbatim}[commandchars=\\\{\}]
{\color{incolor}In [{\color{incolor}37}]:} \PY{o}{\PYZpc{}\PYZpc{}}\PY{k}{timeit}
         \PY{n+nb}{map}\PY{p}{(}\PY{k}{lambda} \PY{n}{x}\PY{p}{:} \PY{n}{kstest}\PY{p}{(}\PY{n}{x}\PY{p}{,}\PY{l+s}{\PYZsq{}}\PY{l+s}{norm}\PY{l+s}{\PYZsq{}}\PY{p}{)}\PY{p}{,}\PY{n}{rv}\PY{p}{)}
\end{Verbatim}

    \begin{Verbatim}[commandchars=\\\{\}]
100 loops, best of 3: 6.4 ms per loop
    \end{Verbatim}

    \begin{Verbatim}[commandchars=\\\{\}]
{\color{incolor}In [{\color{incolor}23}]:} \PY{n}{np}\PY{o}{.}\PY{n}{apply\PYZus{}along\PYZus{}axis}\PY{err}{?}\PY{err}{?}
\end{Verbatim}

    \begin{Verbatim}[commandchars=\\\{\}]
{\color{incolor}In [{\color{incolor}93}]:} \PY{n}{t\PYZus{}statistic}\PY{p}{,} \PY{n}{p\PYZus{}value} \PY{o}{=} \PY{n}{stats}\PY{o}{.}\PY{n}{ttest\PYZus{}ind}\PY{p}{(}\PY{n}{X}\PY{o}{.}\PY{n}{rvs}\PY{p}{(}\PY{n}{size}\PY{o}{=}\PY{l+m+mi}{1000}\PY{p}{)}\PY{p}{,} \PY{n}{X}\PY{o}{.}\PY{n}{rvs}\PY{p}{(}\PY{n}{size}\PY{o}{=}\PY{l+m+mi}{1000}\PY{p}{)}\PY{p}{)}
         \PY{k}{print} \PY{n}{p\PYZus{}value}
\end{Verbatim}

    \begin{Verbatim}[commandchars=\\\{\}]
0.433472011446
    \end{Verbatim}

    \begin{Verbatim}[commandchars=\\\{\}]
{\color{incolor}In [{\color{incolor}90}]:} 
\end{Verbatim}

    \begin{Verbatim}[commandchars=\\\{\}]
{\color{incolor}In [{\color{incolor}}]:} 
\end{Verbatim}

    \begin{Verbatim}[commandchars=\\\{\}]
{\color{incolor}In [{\color{incolor}}]:} 
\end{Verbatim}

    \begin{Verbatim}[commandchars=\\\{\}]
{\color{incolor}In [{\color{incolor}}]:} 
\end{Verbatim}

    \begin{Verbatim}[commandchars=\\\{\}]
{\color{incolor}In [{\color{incolor}}]:} 
\end{Verbatim}

    \begin{Verbatim}[commandchars=\\\{\}]
{\color{incolor}In [{\color{incolor}}]:} 
\end{Verbatim}

    \begin{Verbatim}[commandchars=\\\{\}]
{\color{incolor}In [{\color{incolor}25}]:} \PY{k+kn}{from} \PY{n+nn}{os} \PY{k+kn}{import} \PY{n}{urandom}
\end{Verbatim}

    \begin{Verbatim}[commandchars=\\\{\}]
{\color{incolor}In [{\color{incolor}26}]:} \PY{n}{ur} \PY{o}{=} \PY{n}{urandom}\PY{p}{(}\PY{l+m+mi}{16}\PY{p}{)}
\end{Verbatim}

    \begin{Verbatim}[commandchars=\\\{\}]
{\color{incolor}In [{\color{incolor}27}]:} \PY{n}{ur}\PY{o}{.}\PY{n}{encode}\PY{p}{(}\PY{l+s}{\PYZsq{}}\PY{l+s}{hex}\PY{l+s}{\PYZsq{}}\PY{p}{)}
\end{Verbatim}

            \begin{Verbatim}[commandchars=\\\{\}]
{\color{outcolor}Out[{\color{outcolor}27}]:} 'a745c7bc678df931b1ac1945baf2c7d1'
\end{Verbatim}
        
    \begin{Verbatim}[commandchars=\\\{\}]
{\color{incolor}In [{\color{incolor}28}]:} \PY{k+kn}{import} \PY{n+nn}{random}
\end{Verbatim}

    \begin{Verbatim}[commandchars=\\\{\}]
{\color{incolor}In [{\color{incolor}29}]:} \PY{n}{random}\PY{o}{.}\PY{n}{random}\PY{p}{(}\PY{p}{)}
\end{Verbatim}

            \begin{Verbatim}[commandchars=\\\{\}]
{\color{outcolor}Out[{\color{outcolor}29}]:} 0.9962189820806353
\end{Verbatim}
        
    What about choosing from some predetermined set?

    \begin{Verbatim}[commandchars=\\\{\}]
{\color{incolor}In [{\color{incolor}1}]:} \PY{k+kn}{from} \PY{n+nn}{os} \PY{k+kn}{import} \PY{n}{urandom}
\end{Verbatim}

    \begin{Verbatim}[commandchars=\\\{\}]
{\color{incolor}In [{\color{incolor}9}]:} \PY{n}{urandom}\PY{p}{(}\PY{l+m+mi}{2}\PY{p}{)}\PY{o}{.}\PY{n}{encode}\PY{p}{(}\PY{l+s}{\PYZsq{}}\PY{l+s}{hex}\PY{l+s}{\PYZsq{}}\PY{p}{)}
\end{Verbatim}

            \begin{Verbatim}[commandchars=\\\{\}]
{\color{outcolor}Out[{\color{outcolor}9}]:} '27b9'
\end{Verbatim}
        
    \begin{Verbatim}[commandchars=\\\{\}]
{\color{incolor}In [{\color{incolor}10}]:} \PY{n}{a} \PY{o}{=} \PY{n}{urandom}\PY{p}{(}\PY{l+m+mi}{64}\PY{p}{)}
         \PY{n}{a}\PY{o}{.}\PY{n}{encode}\PY{p}{(}\PY{l+s}{\PYZsq{}}\PY{l+s}{base\PYZhy{}64}\PY{l+s}{\PYZsq{}}\PY{p}{)}
\end{Verbatim}

            \begin{Verbatim}[commandchars=\\\{\}]
{\color{outcolor}Out[{\color{outcolor}10}]:} 'lLn5OJCxw1DmD87MiiHHZDvOxjNpcG/h0uiGS2Y4kG6h1gOqAoySBKYiGX31SxD1weDkaF0YkTwa\textbackslash{}nu1Akx92EHg==\textbackslash{}n'
\end{Verbatim}
        
    \begin{Verbatim}[commandchars=\\\{\}]
{\color{incolor}In [{\color{incolor}11}]:} \PY{k+kn}{from} \PY{n+nn}{base64} \PY{k+kn}{import} \PY{n}{b64encode}
         \PY{k+kn}{from} \PY{n+nn}{os} \PY{k+kn}{import} \PY{n}{urandom}
         
         \PY{n}{random\PYZus{}bytes} \PY{o}{=} \PY{n}{urandom}\PY{p}{(}\PY{l+m+mi}{64}\PY{p}{)}
         \PY{n}{token} \PY{o}{=} \PY{n}{b64encode}\PY{p}{(}\PY{n}{random\PYZus{}bytes}\PY{p}{)}\PY{o}{.}\PY{n}{decode}\PY{p}{(}\PY{l+s}{\PYZsq{}}\PY{l+s}{utf\PYZhy{}8}\PY{l+s}{\PYZsq{}}\PY{p}{)}
         \PY{n}{token}
\end{Verbatim}

            \begin{Verbatim}[commandchars=\\\{\}]
{\color{outcolor}Out[{\color{outcolor}11}]:} u'H+y/fHdAumBMVJr3CKG6PJX/dg1su8WEc48z9I9MYECc0KDqiXz9+WoPS0/rlt/cJdyyZAvw68mPa+uFyEbpZw=='
\end{Verbatim}
        
    \begin{Verbatim}[commandchars=\\\{\}]
{\color{incolor}In [{\color{incolor}3}]:} \PY{o}{\PYZpc{}}\PY{k}{load\PYZus{}ext} \PY{n}{version\PYZus{}information}
        \PY{o}{\PYZpc{}}\PY{k}{version\PYZus{}information} \PY{n}{numpy}
\end{Verbatim}

    \begin{Verbatim}[commandchars=\\\{\}]
The version\_information extension is already loaded. To reload it, use:
  \%reload\_ext version\_information
    \end{Verbatim}
\texttt{\color{outcolor}Out[{\color{outcolor}3}]:}
    
    \begin{tabular}{|l|l|}\hline
{\bf Software} & {\bf Version} \\ \hline\hline
Python & 2.7.8 |Anaconda 2.1.0 (64-bit)| (default, Aug 21 2014, 18:22:21) [GCC 4.4.7 20120313 (Red Hat 4.4.7-1)] \\ \hline
IPython & 2.3.0 \\ \hline
OS & posix [linux2] \\ \hline
numpy & 1.9.1 \\ \hline
\hline \multicolumn{2}{|l|}{Thu Dec 04 14:48:14 2014 CET} \\ \hline
\end{tabular}


    

    \begin{Verbatim}[commandchars=\\\{\}]
{\color{incolor}In [{\color{incolor}1}]:} \PY{k+kn}{from} \PY{n+nn}{IPython.core.display} \PY{k+kn}{import} \PY{n}{HTML}
        \PY{k}{def} \PY{n+nf}{css\PYZus{}styling}\PY{p}{(}\PY{p}{)}\PY{p}{:}
            \PY{n}{styles} \PY{o}{=} \PY{n+nb}{open}\PY{p}{(}\PY{l+s}{\PYZdq{}}\PY{l+s}{./styles/custom.css}\PY{l+s}{\PYZdq{}}\PY{p}{,} \PY{l+s}{\PYZdq{}}\PY{l+s}{r}\PY{l+s}{\PYZdq{}}\PY{p}{)}\PY{o}{.}\PY{n}{read}\PY{p}{(}\PY{p}{)}
            \PY{k}{return} \PY{n}{HTML}\PY{p}{(}\PY{n}{styles}\PY{p}{)}
        \PY{n}{css\PYZus{}styling}\PY{p}{(}\PY{p}{)}
\end{Verbatim}

            \begin{Verbatim}[commandchars=\\\{\}]
{\color{outcolor}Out[{\color{outcolor}1}]:} <IPython.core.display.HTML at 0x7fc5082118d0>
\end{Verbatim}
        
    \begin{Verbatim}[commandchars=\\\{\}]
{\color{incolor}In [{\color{incolor}}]:} 
\end{Verbatim}


    % Add a bibliography block to the postdoc
    
    
    
    \end{document}


%\newpage
%
% Default to the notebook output style

    


% Inherit from the specified cell style.




    
\documentclass{article}

    
    
    \usepackage{graphicx} % Used to insert images
    \usepackage{adjustbox} % Used to constrain images to a maximum size 
    \usepackage{color} % Allow colors to be defined
    \usepackage{enumerate} % Needed for markdown enumerations to work
    \usepackage{geometry} % Used to adjust the document margins
    \usepackage{amsmath} % Equations
    \usepackage{amssymb} % Equations
    \usepackage[mathletters]{ucs} % Extended unicode (utf-8) support
    \usepackage[utf8x]{inputenc} % Allow utf-8 characters in the tex document
    \usepackage{fancyvrb} % verbatim replacement that allows latex
    \usepackage{grffile} % extends the file name processing of package graphics 
                         % to support a larger range 
    % The hyperref package gives us a pdf with properly built
    % internal navigation ('pdf bookmarks' for the table of contents,
    % internal cross-reference links, web links for URLs, etc.)
    \usepackage{hyperref}
    \usepackage{longtable} % longtable support required by pandoc >1.10
    \usepackage{booktabs}  % table support for pandoc > 1.12.2
    

    
    
    \definecolor{orange}{cmyk}{0,0.4,0.8,0.2}
    \definecolor{darkorange}{rgb}{.71,0.21,0.01}
    \definecolor{darkgreen}{rgb}{.12,.54,.11}
    \definecolor{myteal}{rgb}{.26, .44, .56}
    \definecolor{gray}{gray}{0.45}
    \definecolor{lightgray}{gray}{.95}
    \definecolor{mediumgray}{gray}{.8}
    \definecolor{inputbackground}{rgb}{.95, .95, .85}
    \definecolor{outputbackground}{rgb}{.95, .95, .95}
    \definecolor{traceback}{rgb}{1, .95, .95}
    % ansi colors
    \definecolor{red}{rgb}{.6,0,0}
    \definecolor{green}{rgb}{0,.65,0}
    \definecolor{brown}{rgb}{0.6,0.6,0}
    \definecolor{blue}{rgb}{0,.145,.698}
    \definecolor{purple}{rgb}{.698,.145,.698}
    \definecolor{cyan}{rgb}{0,.698,.698}
    \definecolor{lightgray}{gray}{0.5}
    
    % bright ansi colors
    \definecolor{darkgray}{gray}{0.25}
    \definecolor{lightred}{rgb}{1.0,0.39,0.28}
    \definecolor{lightgreen}{rgb}{0.48,0.99,0.0}
    \definecolor{lightblue}{rgb}{0.53,0.81,0.92}
    \definecolor{lightpurple}{rgb}{0.87,0.63,0.87}
    \definecolor{lightcyan}{rgb}{0.5,1.0,0.83}
    
    % commands and environments needed by pandoc snippets
    % extracted from the output of `pandoc -s`
    \DefineVerbatimEnvironment{Highlighting}{Verbatim}{commandchars=\\\{\}}
    % Add ',fontsize=\small' for more characters per line
    \newenvironment{Shaded}{}{}
    \newcommand{\KeywordTok}[1]{\textcolor[rgb]{0.00,0.44,0.13}{\textbf{{#1}}}}
    \newcommand{\DataTypeTok}[1]{\textcolor[rgb]{0.56,0.13,0.00}{{#1}}}
    \newcommand{\DecValTok}[1]{\textcolor[rgb]{0.25,0.63,0.44}{{#1}}}
    \newcommand{\BaseNTok}[1]{\textcolor[rgb]{0.25,0.63,0.44}{{#1}}}
    \newcommand{\FloatTok}[1]{\textcolor[rgb]{0.25,0.63,0.44}{{#1}}}
    \newcommand{\CharTok}[1]{\textcolor[rgb]{0.25,0.44,0.63}{{#1}}}
    \newcommand{\StringTok}[1]{\textcolor[rgb]{0.25,0.44,0.63}{{#1}}}
    \newcommand{\CommentTok}[1]{\textcolor[rgb]{0.38,0.63,0.69}{\textit{{#1}}}}
    \newcommand{\OtherTok}[1]{\textcolor[rgb]{0.00,0.44,0.13}{{#1}}}
    \newcommand{\AlertTok}[1]{\textcolor[rgb]{1.00,0.00,0.00}{\textbf{{#1}}}}
    \newcommand{\FunctionTok}[1]{\textcolor[rgb]{0.02,0.16,0.49}{{#1}}}
    \newcommand{\RegionMarkerTok}[1]{{#1}}
    \newcommand{\ErrorTok}[1]{\textcolor[rgb]{1.00,0.00,0.00}{\textbf{{#1}}}}
    \newcommand{\NormalTok}[1]{{#1}}
    
    % Define a nice break command that doesn't care if a line doesn't already
    % exist.
    \def\br{\hspace*{\fill} \\* }
    % Math Jax compatability definitions
    \def\gt{>}
    \def\lt{<}
    % Document parameters
    \title{X-FFT}
    
    
    

    % Pygments definitions
    
\makeatletter
\def\PY@reset{\let\PY@it=\relax \let\PY@bf=\relax%
    \let\PY@ul=\relax \let\PY@tc=\relax%
    \let\PY@bc=\relax \let\PY@ff=\relax}
\def\PY@tok#1{\csname PY@tok@#1\endcsname}
\def\PY@toks#1+{\ifx\relax#1\empty\else%
    \PY@tok{#1}\expandafter\PY@toks\fi}
\def\PY@do#1{\PY@bc{\PY@tc{\PY@ul{%
    \PY@it{\PY@bf{\PY@ff{#1}}}}}}}
\def\PY#1#2{\PY@reset\PY@toks#1+\relax+\PY@do{#2}}

\expandafter\def\csname PY@tok@gd\endcsname{\def\PY@tc##1{\textcolor[rgb]{0.63,0.00,0.00}{##1}}}
\expandafter\def\csname PY@tok@gu\endcsname{\let\PY@bf=\textbf\def\PY@tc##1{\textcolor[rgb]{0.50,0.00,0.50}{##1}}}
\expandafter\def\csname PY@tok@gt\endcsname{\def\PY@tc##1{\textcolor[rgb]{0.00,0.27,0.87}{##1}}}
\expandafter\def\csname PY@tok@gs\endcsname{\let\PY@bf=\textbf}
\expandafter\def\csname PY@tok@gr\endcsname{\def\PY@tc##1{\textcolor[rgb]{1.00,0.00,0.00}{##1}}}
\expandafter\def\csname PY@tok@cm\endcsname{\let\PY@it=\textit\def\PY@tc##1{\textcolor[rgb]{0.25,0.50,0.50}{##1}}}
\expandafter\def\csname PY@tok@vg\endcsname{\def\PY@tc##1{\textcolor[rgb]{0.10,0.09,0.49}{##1}}}
\expandafter\def\csname PY@tok@m\endcsname{\def\PY@tc##1{\textcolor[rgb]{0.40,0.40,0.40}{##1}}}
\expandafter\def\csname PY@tok@mh\endcsname{\def\PY@tc##1{\textcolor[rgb]{0.40,0.40,0.40}{##1}}}
\expandafter\def\csname PY@tok@go\endcsname{\def\PY@tc##1{\textcolor[rgb]{0.53,0.53,0.53}{##1}}}
\expandafter\def\csname PY@tok@ge\endcsname{\let\PY@it=\textit}
\expandafter\def\csname PY@tok@vc\endcsname{\def\PY@tc##1{\textcolor[rgb]{0.10,0.09,0.49}{##1}}}
\expandafter\def\csname PY@tok@il\endcsname{\def\PY@tc##1{\textcolor[rgb]{0.40,0.40,0.40}{##1}}}
\expandafter\def\csname PY@tok@cs\endcsname{\let\PY@it=\textit\def\PY@tc##1{\textcolor[rgb]{0.25,0.50,0.50}{##1}}}
\expandafter\def\csname PY@tok@cp\endcsname{\def\PY@tc##1{\textcolor[rgb]{0.74,0.48,0.00}{##1}}}
\expandafter\def\csname PY@tok@gi\endcsname{\def\PY@tc##1{\textcolor[rgb]{0.00,0.63,0.00}{##1}}}
\expandafter\def\csname PY@tok@gh\endcsname{\let\PY@bf=\textbf\def\PY@tc##1{\textcolor[rgb]{0.00,0.00,0.50}{##1}}}
\expandafter\def\csname PY@tok@ni\endcsname{\let\PY@bf=\textbf\def\PY@tc##1{\textcolor[rgb]{0.60,0.60,0.60}{##1}}}
\expandafter\def\csname PY@tok@nl\endcsname{\def\PY@tc##1{\textcolor[rgb]{0.63,0.63,0.00}{##1}}}
\expandafter\def\csname PY@tok@nn\endcsname{\let\PY@bf=\textbf\def\PY@tc##1{\textcolor[rgb]{0.00,0.00,1.00}{##1}}}
\expandafter\def\csname PY@tok@no\endcsname{\def\PY@tc##1{\textcolor[rgb]{0.53,0.00,0.00}{##1}}}
\expandafter\def\csname PY@tok@na\endcsname{\def\PY@tc##1{\textcolor[rgb]{0.49,0.56,0.16}{##1}}}
\expandafter\def\csname PY@tok@nb\endcsname{\def\PY@tc##1{\textcolor[rgb]{0.00,0.50,0.00}{##1}}}
\expandafter\def\csname PY@tok@nc\endcsname{\let\PY@bf=\textbf\def\PY@tc##1{\textcolor[rgb]{0.00,0.00,1.00}{##1}}}
\expandafter\def\csname PY@tok@nd\endcsname{\def\PY@tc##1{\textcolor[rgb]{0.67,0.13,1.00}{##1}}}
\expandafter\def\csname PY@tok@ne\endcsname{\let\PY@bf=\textbf\def\PY@tc##1{\textcolor[rgb]{0.82,0.25,0.23}{##1}}}
\expandafter\def\csname PY@tok@nf\endcsname{\def\PY@tc##1{\textcolor[rgb]{0.00,0.00,1.00}{##1}}}
\expandafter\def\csname PY@tok@si\endcsname{\let\PY@bf=\textbf\def\PY@tc##1{\textcolor[rgb]{0.73,0.40,0.53}{##1}}}
\expandafter\def\csname PY@tok@s2\endcsname{\def\PY@tc##1{\textcolor[rgb]{0.73,0.13,0.13}{##1}}}
\expandafter\def\csname PY@tok@vi\endcsname{\def\PY@tc##1{\textcolor[rgb]{0.10,0.09,0.49}{##1}}}
\expandafter\def\csname PY@tok@nt\endcsname{\let\PY@bf=\textbf\def\PY@tc##1{\textcolor[rgb]{0.00,0.50,0.00}{##1}}}
\expandafter\def\csname PY@tok@nv\endcsname{\def\PY@tc##1{\textcolor[rgb]{0.10,0.09,0.49}{##1}}}
\expandafter\def\csname PY@tok@s1\endcsname{\def\PY@tc##1{\textcolor[rgb]{0.73,0.13,0.13}{##1}}}
\expandafter\def\csname PY@tok@kd\endcsname{\let\PY@bf=\textbf\def\PY@tc##1{\textcolor[rgb]{0.00,0.50,0.00}{##1}}}
\expandafter\def\csname PY@tok@sh\endcsname{\def\PY@tc##1{\textcolor[rgb]{0.73,0.13,0.13}{##1}}}
\expandafter\def\csname PY@tok@sc\endcsname{\def\PY@tc##1{\textcolor[rgb]{0.73,0.13,0.13}{##1}}}
\expandafter\def\csname PY@tok@sx\endcsname{\def\PY@tc##1{\textcolor[rgb]{0.00,0.50,0.00}{##1}}}
\expandafter\def\csname PY@tok@bp\endcsname{\def\PY@tc##1{\textcolor[rgb]{0.00,0.50,0.00}{##1}}}
\expandafter\def\csname PY@tok@c1\endcsname{\let\PY@it=\textit\def\PY@tc##1{\textcolor[rgb]{0.25,0.50,0.50}{##1}}}
\expandafter\def\csname PY@tok@kc\endcsname{\let\PY@bf=\textbf\def\PY@tc##1{\textcolor[rgb]{0.00,0.50,0.00}{##1}}}
\expandafter\def\csname PY@tok@c\endcsname{\let\PY@it=\textit\def\PY@tc##1{\textcolor[rgb]{0.25,0.50,0.50}{##1}}}
\expandafter\def\csname PY@tok@mf\endcsname{\def\PY@tc##1{\textcolor[rgb]{0.40,0.40,0.40}{##1}}}
\expandafter\def\csname PY@tok@err\endcsname{\def\PY@bc##1{\setlength{\fboxsep}{0pt}\fcolorbox[rgb]{1.00,0.00,0.00}{1,1,1}{\strut ##1}}}
\expandafter\def\csname PY@tok@mb\endcsname{\def\PY@tc##1{\textcolor[rgb]{0.40,0.40,0.40}{##1}}}
\expandafter\def\csname PY@tok@ss\endcsname{\def\PY@tc##1{\textcolor[rgb]{0.10,0.09,0.49}{##1}}}
\expandafter\def\csname PY@tok@sr\endcsname{\def\PY@tc##1{\textcolor[rgb]{0.73,0.40,0.53}{##1}}}
\expandafter\def\csname PY@tok@mo\endcsname{\def\PY@tc##1{\textcolor[rgb]{0.40,0.40,0.40}{##1}}}
\expandafter\def\csname PY@tok@kn\endcsname{\let\PY@bf=\textbf\def\PY@tc##1{\textcolor[rgb]{0.00,0.50,0.00}{##1}}}
\expandafter\def\csname PY@tok@mi\endcsname{\def\PY@tc##1{\textcolor[rgb]{0.40,0.40,0.40}{##1}}}
\expandafter\def\csname PY@tok@gp\endcsname{\let\PY@bf=\textbf\def\PY@tc##1{\textcolor[rgb]{0.00,0.00,0.50}{##1}}}
\expandafter\def\csname PY@tok@o\endcsname{\def\PY@tc##1{\textcolor[rgb]{0.40,0.40,0.40}{##1}}}
\expandafter\def\csname PY@tok@kr\endcsname{\let\PY@bf=\textbf\def\PY@tc##1{\textcolor[rgb]{0.00,0.50,0.00}{##1}}}
\expandafter\def\csname PY@tok@s\endcsname{\def\PY@tc##1{\textcolor[rgb]{0.73,0.13,0.13}{##1}}}
\expandafter\def\csname PY@tok@kp\endcsname{\def\PY@tc##1{\textcolor[rgb]{0.00,0.50,0.00}{##1}}}
\expandafter\def\csname PY@tok@w\endcsname{\def\PY@tc##1{\textcolor[rgb]{0.73,0.73,0.73}{##1}}}
\expandafter\def\csname PY@tok@kt\endcsname{\def\PY@tc##1{\textcolor[rgb]{0.69,0.00,0.25}{##1}}}
\expandafter\def\csname PY@tok@ow\endcsname{\let\PY@bf=\textbf\def\PY@tc##1{\textcolor[rgb]{0.67,0.13,1.00}{##1}}}
\expandafter\def\csname PY@tok@sb\endcsname{\def\PY@tc##1{\textcolor[rgb]{0.73,0.13,0.13}{##1}}}
\expandafter\def\csname PY@tok@k\endcsname{\let\PY@bf=\textbf\def\PY@tc##1{\textcolor[rgb]{0.00,0.50,0.00}{##1}}}
\expandafter\def\csname PY@tok@se\endcsname{\let\PY@bf=\textbf\def\PY@tc##1{\textcolor[rgb]{0.73,0.40,0.13}{##1}}}
\expandafter\def\csname PY@tok@sd\endcsname{\let\PY@it=\textit\def\PY@tc##1{\textcolor[rgb]{0.73,0.13,0.13}{##1}}}

\def\PYZbs{\char`\\}
\def\PYZus{\char`\_}
\def\PYZob{\char`\{}
\def\PYZcb{\char`\}}
\def\PYZca{\char`\^}
\def\PYZam{\char`\&}
\def\PYZlt{\char`\<}
\def\PYZgt{\char`\>}
\def\PYZsh{\char`\#}
\def\PYZpc{\char`\%}
\def\PYZdl{\char`\$}
\def\PYZhy{\char`\-}
\def\PYZsq{\char`\'}
\def\PYZdq{\char`\"}
\def\PYZti{\char`\~}
% for compatibility with earlier versions
\def\PYZat{@}
\def\PYZlb{[}
\def\PYZrb{]}
\makeatother


    % Exact colors from NB
    \definecolor{incolor}{rgb}{0.0, 0.0, 0.5}
    \definecolor{outcolor}{rgb}{0.545, 0.0, 0.0}



    
    % Prevent overflowing lines due to hard-to-break entities
    \sloppy 
    % Setup hyperref package
    \hypersetup{
      breaklinks=true,  % so long urls are correctly broken across lines
      colorlinks=true,
      urlcolor=blue,
      linkcolor=darkorange,
      citecolor=darkgreen,
      }
    % Slightly bigger margins than the latex defaults
    
    \geometry{verbose,tmargin=1in,bmargin=1in,lmargin=1in,rmargin=1in}
    
    

    \begin{document}
    
    
    \maketitle
    
    

    
    \section{X-\href{http://www.ams.org/journals/mcom/1965-19-090/S0025-5718-1965-0178586-1/}{FFT}
- Fast Fourier Transform}\label{x-fft---fast-fourier-transform}

    \begin{Verbatim}[commandchars=\\\{\}]
{\color{incolor}In [{\color{incolor}8}]:} \PY{k+kn}{from} \PY{n+nn}{IPython.display} \PY{k+kn}{import} \PY{n}{HTML}
\end{Verbatim}

    \begin{Verbatim}[commandchars=\\\{\}]
{\color{incolor}In [{\color{incolor}9}]:} \PY{n}{HTML}\PY{p}{(}\PY{l+s}{\PYZsq{}}\PY{l+s}{http://www.ams.org/journals/mcom/1965\PYZhy{}19\PYZhy{}090/S0025\PYZhy{}5718\PYZhy{}1965\PYZhy{}0178586\PYZhy{}1/}\PY{l+s}{\PYZsq{}}\PY{p}{)}
\end{Verbatim}

            \begin{Verbatim}[commandchars=\\\{\}]
{\color{outcolor}Out[{\color{outcolor}9}]:} <IPython.core.display.HTML at 0x7f27b419fa90>
\end{Verbatim}
        
    There are several FFT implementations in python. We will use the NumPy
and SciPy ones. The fastest is
\href{https://pypi.python.org/pypi/pyFFTW}{PyFFTW}.

    \begin{Verbatim}[commandchars=\\\{\}]
{\color{incolor}In [{\color{incolor}9}]:} \PY{k+kn}{import} \PY{n+nn}{numpy} \PY{k+kn}{as} \PY{n+nn}{np}
        \PY{k+kn}{import} \PY{n+nn}{scipy} \PY{k+kn}{as} \PY{n+nn}{scp}
        \PY{o}{\PYZpc{}}\PY{k}{pylab}
\end{Verbatim}

    \begin{Verbatim}[commandchars=\\\{\}]
Using matplotlib backend: Qt4Agg
Populating the interactive namespace from numpy and matplotlib
    \end{Verbatim}

    \begin{Verbatim}[commandchars=\\\{\}]
{\color{incolor}In [{\color{incolor}2}]:} \PY{n}{help}\PY{p}{(}\PY{n}{np}\PY{o}{.}\PY{n}{fft}\PY{p}{)}
\end{Verbatim}

    \begin{Verbatim}[commandchars=\\\{\}]
Help on package numpy.fft in numpy:

NAME
    numpy.fft

FILE
    /home/jpsilva/anaconda/lib/python2.7/site-packages/numpy/fft/\_\_init\_\_.py

DESCRIPTION
    Discrete Fourier Transform (:mod:`numpy.fft`)
    =============================================
    
    .. currentmodule:: numpy.fft
    
    Standard FFTs
    -------------
    
    .. autosummary::
       :toctree: generated/
    
       fft       Discrete Fourier transform.
       ifft      Inverse discrete Fourier transform.
       fft2      Discrete Fourier transform in two dimensions.
       ifft2     Inverse discrete Fourier transform in two dimensions.
       fftn      Discrete Fourier transform in N-dimensions.
       ifftn     Inverse discrete Fourier transform in N dimensions.
    
    Real FFTs
    ---------
    
    .. autosummary::
       :toctree: generated/
    
       rfft      Real discrete Fourier transform.
       irfft     Inverse real discrete Fourier transform.
       rfft2     Real discrete Fourier transform in two dimensions.
       irfft2    Inverse real discrete Fourier transform in two dimensions.
       rfftn     Real discrete Fourier transform in N dimensions.
       irfftn    Inverse real discrete Fourier transform in N dimensions.
    
    Hermitian FFTs
    --------------
    
    .. autosummary::
       :toctree: generated/
    
       hfft      Hermitian discrete Fourier transform.
       ihfft     Inverse Hermitian discrete Fourier transform.
    
    Helper routines
    ---------------
    
    .. autosummary::
       :toctree: generated/
    
       fftfreq   Discrete Fourier Transform sample frequencies.
       rfftfreq  DFT sample frequencies (for usage with rfft, irfft).
       fftshift  Shift zero-frequency component to center of spectrum.
       ifftshift Inverse of fftshift.
    
    
    Background information
    ----------------------
    
    Fourier analysis is fundamentally a method for expressing a function as a
    sum of periodic components, and for recovering the function from those
    components.  When both the function and its Fourier transform are
    replaced with discretized counterparts, it is called the discrete Fourier
    transform (DFT).  The DFT has become a mainstay of numerical computing in
    part because of a very fast algorithm for computing it, called the Fast
    Fourier Transform (FFT), which was known to Gauss (1805) and was brought
    to light in its current form by Cooley and Tukey [CT]\_.  Press et al. [NR]\_
    provide an accessible introduction to Fourier analysis and its
    applications.
    
    Because the discrete Fourier transform separates its input into
    components that contribute at discrete frequencies, it has a great number
    of applications in digital signal processing, e.g., for filtering, and in
    this context the discretized input to the transform is customarily
    referred to as a *signal*, which exists in the *time domain*.  The output
    is called a *spectrum* or *transform* and exists in the *frequency
    domain*.
    
    Implementation details
    ----------------------
    
    There are many ways to define the DFT, varying in the sign of the
    exponent, normalization, etc.  In this implementation, the DFT is defined
    as
    
    .. math::
       A\_k =  \textbackslash{}sum\_\{m=0\}\^{}\{n-1\} a\_m \textbackslash{}exp\textbackslash{}left\textbackslash{}\{-2\textbackslash{}pi i\{mk \textbackslash{}over n\}\textbackslash{}right\textbackslash{}\}
       \textbackslash{}qquad k = 0,\textbackslash{}ldots,n-1.
    
    The DFT is in general defined for complex inputs and outputs, and a
    single-frequency component at linear frequency :math:`f` is
    represented by a complex exponential
    :math:`a\_m = \textbackslash{}exp\textbackslash{}\{2\textbackslash{}pi i\textbackslash{},f m\textbackslash{}Delta t\textbackslash{}\}`, where :math:`\textbackslash{}Delta t`
    is the sampling interval.
    
    The values in the result follow so-called "standard" order: If ``A =
    fft(a, n)``, then ``A[0]`` contains the zero-frequency term (the mean of
    the signal), which is always purely real for real inputs. Then ``A[1:n/2]``
    contains the positive-frequency terms, and ``A[n/2+1:]`` contains the
    negative-frequency terms, in order of decreasingly negative frequency.
    For an even number of input points, ``A[n/2]`` represents both positive and
    negative Nyquist frequency, and is also purely real for real input.  For
    an odd number of input points, ``A[(n-1)/2]`` contains the largest positive
    frequency, while ``A[(n+1)/2]`` contains the largest negative frequency.
    The routine ``np.fft.fftfreq(n)`` returns an array giving the frequencies
    of corresponding elements in the output.  The routine
    ``np.fft.fftshift(A)`` shifts transforms and their frequencies to put the
    zero-frequency components in the middle, and ``np.fft.ifftshift(A)`` undoes
    that shift.
    
    When the input `a` is a time-domain signal and ``A = fft(a)``, ``np.abs(A)``
    is its amplitude spectrum and ``np.abs(A)**2`` is its power spectrum.
    The phase spectrum is obtained by ``np.angle(A)``.
    
    The inverse DFT is defined as
    
    .. math::
       a\_m = \textbackslash{}frac\{1\}\{n\}\textbackslash{}sum\_\{k=0\}\^{}\{n-1\}A\_k\textbackslash{}exp\textbackslash{}left\textbackslash{}\{2\textbackslash{}pi i\{mk\textbackslash{}over n\}\textbackslash{}right\textbackslash{}\}
       \textbackslash{}qquad m = 0,\textbackslash{}ldots,n-1.
    
    It differs from the forward transform by the sign of the exponential
    argument and the normalization by :math:`1/n`.
    
    Real and Hermitian transforms
    -----------------------------
    
    When the input is purely real, its transform is Hermitian, i.e., the
    component at frequency :math:`f\_k` is the complex conjugate of the
    component at frequency :math:`-f\_k`, which means that for real
    inputs there is no information in the negative frequency components that
    is not already available from the positive frequency components.
    The family of `rfft` functions is
    designed to operate on real inputs, and exploits this symmetry by
    computing only the positive frequency components, up to and including the
    Nyquist frequency.  Thus, ``n`` input points produce ``n/2+1`` complex
    output points.  The inverses of this family assumes the same symmetry of
    its input, and for an output of ``n`` points uses ``n/2+1`` input points.
    
    Correspondingly, when the spectrum is purely real, the signal is
    Hermitian.  The `hfft` family of functions exploits this symmetry by
    using ``n/2+1`` complex points in the input (time) domain for ``n`` real
    points in the frequency domain.
    
    In higher dimensions, FFTs are used, e.g., for image analysis and
    filtering.  The computational efficiency of the FFT means that it can
    also be a faster way to compute large convolutions, using the property
    that a convolution in the time domain is equivalent to a point-by-point
    multiplication in the frequency domain.
    
    Higher dimensions
    -----------------
    
    In two dimensions, the DFT is defined as
    
    .. math::
       A\_\{kl\} =  \textbackslash{}sum\_\{m=0\}\^{}\{M-1\} \textbackslash{}sum\_\{n=0\}\^{}\{N-1\}
       a\_\{mn\}\textbackslash{}exp\textbackslash{}left\textbackslash{}\{-2\textbackslash{}pi i \textbackslash{}left(\{mk\textbackslash{}over M\}+\{nl\textbackslash{}over N\}\textbackslash{}right)\textbackslash{}right\textbackslash{}\}
       \textbackslash{}qquad k = 0, \textbackslash{}ldots, M-1;\textbackslash{}quad l = 0, \textbackslash{}ldots, N-1,
    
    which extends in the obvious way to higher dimensions, and the inverses
    in higher dimensions also extend in the same way.
    
    References
    ----------
    
    .. [CT] Cooley, James W., and John W. Tukey, 1965, "An algorithm for the
            machine calculation of complex Fourier series," *Math. Comput.*
            19: 297-301.
    
    .. [NR] Press, W., Teukolsky, S., Vetterline, W.T., and Flannery, B.P.,
            2007, *Numerical Recipes: The Art of Scientific Computing*, ch.
            12-13.  Cambridge Univ. Press, Cambridge, UK.
    
    Examples
    --------
    
    For examples, see the various functions.

PACKAGE CONTENTS
    fftpack
    fftpack\_lite
    helper
    info
    setup

DATA
    absolute\_import = \_Feature((2, 5, 0, 'alpha', 1), (3, 0, 0, 'alpha', 0\ldots
    division = \_Feature((2, 2, 0, 'alpha', 2), (3, 0, 0, 'alpha', 0), 8192\ldots
    print\_function = \_Feature((2, 6, 0, 'alpha', 2), (3, 0, 0, 'alpha', 0)\ldots
    using\_mklfft = True
    \end{Verbatim}

    \begin{Verbatim}[commandchars=\\\{\}]
{\color{incolor}In [{\color{incolor}3}]:} \PY{n}{help}\PY{p}{(}\PY{n}{scp}\PY{o}{.}\PY{n}{fft}\PY{p}{)}
\end{Verbatim}

    \begin{Verbatim}[commandchars=\\\{\}]
Help on function fft in module mklfft.fftpack:

fft(a, n=None, axis=-1)
    Compute the one-dimensional discrete Fourier Transform.
    
    This function computes the one-dimensional *n*-point discrete Fourier
    Transform (DFT) with the efficient Fast Fourier Transform (FFT)
    algorithm [CT].
    
    Parameters
    ----------
    a : array\_like
        Input array, can be complex.
    n : int, optional
        Length of the transformed axis of the output.
        If `n` is smaller than the length of the input, the input is cropped.
        If it is larger, the input is padded with zeros.  If `n` is not given,
        the length of the input along the axis specified by `axis` is used.
    axis : int, optional
        Axis over which to compute the FFT.  If not given, the last axis is
        used.
    
    Returns
    -------
    out : complex ndarray
        The truncated or zero-padded input, transformed along the axis
        indicated by `axis`, or the last one if `axis` is not specified.
    
    Raises
    ------
    IndexError
        if `axes` is larger than the last axis of `a`.
    
    See Also
    --------
    numpy.fft : for definition of the DFT and conventions used.
    ifft : The inverse of `fft`.
    fft2 : The two-dimensional FFT.
    fftn : The *n*-dimensional FFT.
    rfftn : The *n*-dimensional FFT of real input.
    fftfreq : Frequency bins for given FFT parameters.
    
    Notes
    -----
    FFT (Fast Fourier Transform) refers to a way the discrete Fourier
    Transform (DFT) can be calculated efficiently, by using symmetries in the
    calculated terms.  The symmetry is highest when `n` is a power of 2, and
    the transform is therefore most efficient for these sizes.
    
    The DFT is defined, with the conventions used in this implementation, in
    the documentation for the `numpy.fft` module.
    
    References
    ----------
    .. [CT] Cooley, James W., and John W. Tukey, 1965, "An algorithm for the
            machine calculation of complex Fourier series," *Math. Comput.*
            19: 297-301.
    
    Examples
    --------
    >>> np.fft.fft(np.exp(2j * np.pi * np.arange(8) / 8))
    array([ -3.44505240e-16 +1.14383329e-17j,
             8.00000000e+00 -5.71092652e-15j,
             2.33482938e-16 +1.22460635e-16j,
             1.64863782e-15 +1.77635684e-15j,
             9.95839695e-17 +2.33482938e-16j,
             0.00000000e+00 +1.66837030e-15j,
             1.14383329e-17 +1.22460635e-16j,
             -1.64863782e-15 +1.77635684e-15j])
    
    >>> import matplotlib.pyplot as plt
    >>> t = np.arange(256)
    >>> sp = np.fft.fft(np.sin(t))
    >>> freq = np.fft.fftfreq(t.shape[-1])
    >>> plt.plot(freq, sp.real, freq, sp.imag)
    [<matplotlib.lines.Line2D object at 0x\ldots>, <matplotlib.lines.Line2D object at 0x\ldots>]
    >>> plt.show()
    
    In this example, real input has an FFT which is Hermitian, i.e., symmetric
    in the real part and anti-symmetric in the imaginary part, as described in
    the `numpy.fft` documentation.
    \end{Verbatim}

    Let's obtain the FFT of \[e^{\frac{i2\pi k}{8}}\] for $k=1,...,8$

    \begin{Verbatim}[commandchars=\\\{\}]
{\color{incolor}In [{\color{incolor}15}]:} \PY{n}{np}\PY{o}{.}\PY{n}{fft}\PY{o}{.}\PY{n}{fft}\PY{p}{(}\PY{n}{np}\PY{o}{.}\PY{n}{exp}\PY{p}{(}\PY{l+m+mi}{2j} \PY{o}{*} \PY{n}{np}\PY{o}{.}\PY{n}{pi} \PY{o}{*} \PY{n}{np}\PY{o}{.}\PY{n}{arange}\PY{p}{(}\PY{l+m+mi}{8}\PY{p}{)} \PY{o}{/} \PY{l+m+mi}{8}\PY{p}{)}\PY{p}{)}
\end{Verbatim}

            \begin{Verbatim}[commandchars=\\\{\}]
{\color{outcolor}Out[{\color{outcolor}15}]:} array([ -3.44505240e-16 +1.14383329e-17j,
                  8.00000000e+00 -8.52057261e-16j,
                  2.33482938e-16 +1.22460635e-16j,
                  0.00000000e+00 +1.22460635e-16j,
                  9.95839695e-17 +2.33482938e-16j,
                 -8.88178420e-16 +1.17293449e-16j,
                  1.14383329e-17 +1.22460635e-16j,   0.00000000e+00 +1.22460635e-16j])
\end{Verbatim}
        
    \begin{Verbatim}[commandchars=\\\{\}]
{\color{incolor}In [{\color{incolor}16}]:} \PY{n}{x} \PY{o}{=} \PY{n}{np}\PY{o}{.}\PY{n}{exp}\PY{p}{(}\PY{l+m+mi}{2j} \PY{o}{*} \PY{n}{np}\PY{o}{.}\PY{n}{pi} \PY{o}{*} \PY{n}{np}\PY{o}{.}\PY{n}{arange}\PY{p}{(}\PY{l+m+mi}{9}\PY{p}{)} \PY{o}{/} \PY{l+m+mi}{8}\PY{p}{)}
         \PY{n}{xt} \PY{o}{=} \PY{n}{np}\PY{o}{.}\PY{n}{fft}\PY{o}{.}\PY{n}{fft}\PY{p}{(}\PY{n}{x}\PY{p}{)}
\end{Verbatim}

    \begin{Verbatim}[commandchars=\\\{\}]
{\color{incolor}In [{\color{incolor}17}]:} \PY{n}{plot}\PY{p}{(}\PY{n}{xt}\PY{p}{)}
\end{Verbatim}

            \begin{Verbatim}[commandchars=\\\{\}]
{\color{outcolor}Out[{\color{outcolor}17}]:} [<matplotlib.lines.Line2D at 0x7f4897a5dc50>]
\end{Verbatim}
        
    \begin{Verbatim}[commandchars=\\\{\}]
{\color{incolor}In [{\color{incolor}}]:} 
\end{Verbatim}

    \begin{Verbatim}[commandchars=\\\{\}]
{\color{incolor}In [{\color{incolor}}]:} 
\end{Verbatim}

    \begin{Verbatim}[commandchars=\\\{\}]
{\color{incolor}In [{\color{incolor}}]:} 
\end{Verbatim}

    \begin{Verbatim}[commandchars=\\\{\}]
{\color{incolor}In [{\color{incolor}}]:} 
\end{Verbatim}

    \begin{Verbatim}[commandchars=\\\{\}]
{\color{incolor}In [{\color{incolor}}]:} 
\end{Verbatim}

    \begin{Verbatim}[commandchars=\\\{\}]
{\color{incolor}In [{\color{incolor}}]:} 
\end{Verbatim}

    \begin{Verbatim}[commandchars=\\\{\}]
{\color{incolor}In [{\color{incolor}}]:} 
\end{Verbatim}

    \begin{Verbatim}[commandchars=\\\{\}]
{\color{incolor}In [{\color{incolor}}]:} 
\end{Verbatim}

    \begin{Verbatim}[commandchars=\\\{\}]
{\color{incolor}In [{\color{incolor}}]:} 
\end{Verbatim}

    \begin{Verbatim}[commandchars=\\\{\}]
{\color{incolor}In [{\color{incolor}}]:} 
\end{Verbatim}

    \begin{Verbatim}[commandchars=\\\{\}]
{\color{incolor}In [{\color{incolor}}]:} 
\end{Verbatim}

    \begin{Verbatim}[commandchars=\\\{\}]
{\color{incolor}In [{\color{incolor}}]:} 
\end{Verbatim}

    \begin{Verbatim}[commandchars=\\\{\}]
{\color{incolor}In [{\color{incolor}}]:} 
\end{Verbatim}

    \begin{Verbatim}[commandchars=\\\{\}]
{\color{incolor}In [{\color{incolor}}]:} 
\end{Verbatim}

    \begin{Verbatim}[commandchars=\\\{\}]
{\color{incolor}In [{\color{incolor}}]:} 
\end{Verbatim}

    \begin{Verbatim}[commandchars=\\\{\}]
{\color{incolor}In [{\color{incolor}}]:} 
\end{Verbatim}

    \begin{Verbatim}[commandchars=\\\{\}]
{\color{incolor}In [{\color{incolor}}]:} 
\end{Verbatim}

    \begin{Verbatim}[commandchars=\\\{\}]
{\color{incolor}In [{\color{incolor}}]:} 
\end{Verbatim}

    \begin{Verbatim}[commandchars=\\\{\}]
{\color{incolor}In [{\color{incolor}}]:} 
\end{Verbatim}

    \begin{Verbatim}[commandchars=\\\{\}]
{\color{incolor}In [{\color{incolor}}]:} 
\end{Verbatim}


    \subsubsection{Optional}


    \begin{itemize}
\item
  Install \href{http://www.fftw.org/download.html}{FFTW} sudo apt-get
  install fftw3 libfftw3-dev
\item
  Install \href{https://pypi.python.org/pypi/pyFFTW}{PyFFTW} pip install
  PyFFTW
\end{itemize}

    \begin{Verbatim}[commandchars=\\\{\}]
{\color{incolor}In [{\color{incolor}1}]:} \PY{k+kn}{import} \PY{n+nn}{pyfftw}
\end{Verbatim}

    \begin{Verbatim}[commandchars=\\\{\}]
{\color{incolor}In [{\color{incolor}}]:} 
\end{Verbatim}

    \begin{Verbatim}[commandchars=\\\{\}]
{\color{incolor}In [{\color{incolor}}]:} 
\end{Verbatim}

    \begin{Verbatim}[commandchars=\\\{\}]
{\color{incolor}In [{\color{incolor}7}]:} \PY{o}{\PYZpc{}}\PY{k}{load\PYZus{}ext} \PY{n}{version\PYZus{}information}
        \PY{o}{\PYZpc{}}\PY{k}{version\PYZus{}information} \PY{n}{numpy}\PY{p}{,}\PY{n}{scipy}\PY{p}{,}\PY{n}{pyfftw}
\end{Verbatim}

    \begin{Verbatim}[commandchars=\\\{\}]
The version\_information extension is already loaded. To reload it, use:
  \%reload\_ext version\_information
    \end{Verbatim}
\texttt{\color{outcolor}Out[{\color{outcolor}7}]:}
    
    \begin{tabular}{|l|l|}\hline
{\bf Software} & {\bf Version} \\ \hline\hline
Python & 2.7.8 |Anaconda 2.1.0 (64-bit)| (default, Aug 21 2014, 18:22:21) [GCC 4.4.7 20120313 (Red Hat 4.4.7-1)] \\ \hline
IPython & 2.3.1 \\ \hline
OS & posix [linux2] \\ \hline
numpy & 1.9.1 \\ \hline
scipy & 0.14.0 \\ \hline
pyfftw & 0.9.2 \\ \hline
\hline \multicolumn{2}{|l|}{Fri Dec 05 14:20:07 2014 CET} \\ \hline
\end{tabular}


    

    \begin{Verbatim}[commandchars=\\\{\}]
{\color{incolor}In [{\color{incolor}2}]:} \PY{k+kn}{from} \PY{n+nn}{IPython.core.display} \PY{k+kn}{import} \PY{n}{HTML}
        \PY{k}{def} \PY{n+nf}{css\PYZus{}styling}\PY{p}{(}\PY{p}{)}\PY{p}{:}
            \PY{n}{styles} \PY{o}{=} \PY{n+nb}{open}\PY{p}{(}\PY{l+s}{\PYZdq{}}\PY{l+s}{./styles/custom.css}\PY{l+s}{\PYZdq{}}\PY{p}{,} \PY{l+s}{\PYZdq{}}\PY{l+s}{r}\PY{l+s}{\PYZdq{}}\PY{p}{)}\PY{o}{.}\PY{n}{read}\PY{p}{(}\PY{p}{)}
            \PY{k}{return} \PY{n}{HTML}\PY{p}{(}\PY{n}{styles}\PY{p}{)}
        \PY{n}{css\PYZus{}styling}\PY{p}{(}\PY{p}{)}
\end{Verbatim}

            \begin{Verbatim}[commandchars=\\\{\}]
{\color{outcolor}Out[{\color{outcolor}2}]:} <IPython.core.display.HTML at 0x7fcfb8ffd550>
\end{Verbatim}
        
    \begin{Verbatim}[commandchars=\\\{\}]
{\color{incolor}In [{\color{incolor}}]:} 
\end{Verbatim}


    % Add a bibliography block to the postdoc
    
    
    
    \end{document}


\end{document}
